The non-linear, coupled set of equations (\ref{fluideq:cont} -
\ref{fluideq:mom}) consist of one continuity equation per species and one
momentum equation per direction per species.
With one electron fluid and one ion fluid, that totals $8$ partial differential
equations (PDEs). Resolving all the details in these equations are
computationally heavy, and we will here seek ways of simplifying the equations
to lessen the computational demand.

% Fixme:
FIXME: Not necessarily true in our case where we have sheaths

The computational demand can be lessened by exploiting the difference in the
parallel and the perpendicular dynamics. Due of the gyration of the particles,
the dynamics parallel to the magnetic field are much faster than the dynamics
perpendicular to the magnetic field. As a result the gradients perpendicular to
the magnetic field tends to be much larger than the gradients parallel to the
magnetic field. We can exploit this by using a courser grid in the parallel
direction as compared to the perpendicular direction. By these two arguments, we
have a good motivation to split our equations into perpendicular and parallel
parts.

Another way to lessen the demand is through the so called
drift ordering of the perpendicular velocities. This lessen the details we can
get out of the set of equations by neglecting the timescales which are faster
than the ion cyclotron frequency in the perpendicular direction (see section
\ref{sec:drift_order}). To derive the equations with the drift order
approximation, we start out by decomposing the equations in parallel and
perpendicular parts.

\section{Decomposition}
We start the derivation by rearranging equation (\ref{fluideq:mom}) in
the following way
%
\begin{align*}
 n_\a m_\a \d_{t,\a} \ve{u}_{\a} &=
 -\grad p_\a - \div\te{\pi}_\a +
 q_\a n_\a (\ve{E}+\ve{u}_{\a}\times\ve{B})
 + \ve{R}_{\beta \to \a}
 + \ve{R}_{n \to \a}
 - S_{\a,n}m_\a\ve{u}_\a
 \\
%
 \frac{
   n_\a m_\a \d_{t,\a} \ve{u}_{\a}
 }{n_\a q_\a B}
 &=
 -
 \frac{
   \grad p_\a + \div\te{\pi}_\a
 }{n_\a q_\a B}
 +
 \frac{
   q_\a n_\a \ve{E}
 }{n_\a q_\a B}
 +
 \frac{
     q_\a n_\a \ve{u}_{\a}\times\ve{B}
 }{n_\a q_\a B}
 +
 \frac{
   \ve{R}_{\beta \to \a}
 }{n_\a q_\a B}
 +
 \frac{
   \ve{R}_{n \to \a}
 }{n_\a q_\a B}
 -
 \frac{
 S_{\a,n}m_\a\ve{u}_\a
 }{n_\a q_\a B}
 %
 \note{$\ve{b} \defined \ve{B}/B$, where
             $B=\|\ve{B}\|$}\\
 %
 %
 \frac{1}{\om_{c\a}}\d_{t,\a} \ve{u}_{\a}
 &=
 -
 \frac{
   \grad p_\a
 }{n_\a  q_\a B}
 +
 \frac{\ve{E}}{B}
 +
 \ve{u}_{\a}\times\ve{b}
 -
  \frac{
   \div\te{\pi}_\a
 }{n_\a  q_\a B}
 +
 \frac{
   \ve{R}_{\beta \to \a}
 }{n_\a q_\a B}
 +
 \frac{
   \ve{R}_{n \to \a}
 }{n_\a q_\a B}
 -
 \frac{
   S_{\a,n}\ve{u}_\a
 }{n_\a \om_{c\a}},
 \label{eq:no_assumptions_momentum}
 \numberthis
\end{align*}
%
where $\om_{c\a} = \frac{q_{\a}B}{m_\a}$ is the cyclotron frequency for species
$\a$.%
\footnote{Note that the ``frequency'' can be negative because of the $q_{\a}$
          using this definition. However, this is just a ``remnant'' of the
          vector version of this quantity: The angular velocity $\ve{\om}_{c\a}
          = \frac{q_{\a}\ve{B}}{m_\a}$, where the $\pm$ sign from $q_{\a}$
          tells us if a particle is rotating clockwise or counter-clockwise.}%
%

In general, an arbitrary vector $\ve{a}$ can be written $\ve{a} =
\ve{a}\cdot\te{I}$, where $\te{I}$ is the identity tensor of rank $2$. We
introduce the rank-2 tensor $\ve{b}\ve{b}$, which is the outer product with the
unity vectors along $\ve{B}$. Thus
%
\begin{align}
 \ve{a} = \ve{a}\cdot\L(\te{I}+\ve{b}\ve{b}-\ve{b}\ve{b}\R)
        = \underbrace{\ve{a}\cdot\L(\te{I}-\ve{b}\ve{b}\R)}_{\ve{a}_{\perp}}
          + \underbrace{\L(\ve{a}\cdot\ve{b}\R)\ve{b}}_{\ve{a}_{\|}}
        \label{eq:perp_par}
\end{align}
%
In other words, we can find the parallel component (with respect to the
magnetic field) by taking the dot product with $\ve{b}$ and use the result
to scale $\ve{b}$. We further observe that
%
\begin{align*}
    -\ve{b}\times\L(\ve{b}\times\ve{a}\R)
    &=
    -
    \ve{b}\L(\ve{b}\cdot\ve{a}\R)
    +
    \ve{a}\L(\ve{b}\cdot\ve{b}\R)
    \note{
        $
        \ve{a}\times\L(\ve{b}\times\ve{c}\R)
        =
        \ve{b}\L(\ve{a}\cdot\ve{c}\R)
        -
        \ve{c}\L(\ve{a}\cdot\ve{c}\R)
        $
        }
    \\
    &=
    \L(
    \te{I}
    -
    \ve{b}\ve{b}
    \R)
    \cdot\ve{a}
\end{align*}
%
Generally we have
%
\begin{align*}
 \L(\L[\d_{t,\a} \ve{a}\R]\cdot\ve{b}\R)\ve{b}
 %
 =& \L(\L[\parti{}{t} \ve{a}\R]\cdot\ve{b}\R)\ve{b} +
 \L(\L[\ve{u}_{\a} \cdot\nabla \ve{a}\R]\cdot\ve{b}\R)\ve{b}
 \note{\hspace{-2cm}Assume $\partial_t \ve{b} = 0$}
 \\
 %
 =& \parti{}{t}\L(\L[\ve{a}\cdot\ve{b}\R]\ve{b}\R) +
 \L(\L[\ve{u}_{\a} \cdot\nabla \ve{a}\R]\cdot\ve{b}\R)\ve{b}
 \note{\hspace{-2cm}$\ve{v}\cdot\ve{w}=\ve{w}\cdot\ve{v}$}
 \\
 %
 =& \parti{}{t}\L(\L[\ve{a}\cdot\ve{b}\R]\ve{b}\R) +
 \L(\ve{u}_{\a} \cdot \L[\ve{b} \cdot\nabla \ve{a}\R]\R)\ve{b}
 \note{\hspace{-2cm}$\div(\ve{v}\ve{w}) = \ve{v}\div\ve{w} +
\ve{w}\div\ve{v}$}
 \\
 %
 =& \parti{}{t}\L(a_\|\ve{b}\R) +
 \L(\ve{u}_{\a} \cdot \L[\nabla \L(\ve{a} \cdot \ve{b}\R)
  - \ve{a} \cdot\nabla \ve{b}\R]\R)\ve{b}
 \note{\hspace{-2cm}Assume $\nabla \ve{b}$ is negligible}
 \\
 %
 =& \parti{}{t}\ve{a}_\| +  \L(\ve{u}_{\a} \cdot
 \nabla a_\|\R)\ve{b}\\
 %
 =& \parti{}{t}\ve{a}_\| +  \ve{u}_{\a} \cdot
 \L(\nabla \L[a_\|\R]\ve{b}\R)
 \note{\hspace{-2cm}$\grad(v\ve{w})=v\grad(\ve{w})+\grad(v)\ve{w}$}
 \\
 %
 =& \parti{}{t}\ve{a}_\| +  \ve{u}_{\a} \cdot
 \L(\nabla \L[a_\|\ve{b}\R] - a_\|\nabla \L[\ve{b}\R]\R)
 \note{\hspace{-2cm}Assumed $\nabla \ve{b}$ is negligible}
 \\
 %
 =& \parti{}{t}\ve{a}_\| +  \ve{u}_{\a} \cdot \nabla \ve{a}_\|
 \\
 %
 =& \d_{t,\a} \ve{a}_\|,
\end{align*}
%
where we have used the electrostatic approximation $\d_t \ve{B} \simeq 0$ (see
appendix \ref{app:elstat} for details).

Further we have that
%
%
\begin{align*}
 \L(\div\te{A}\R)_\|
 &\defined
 \L(\L[\div\te{A}\R]\cdot\ve{b}\R)\ve{b}
\end{align*}
%
%
and that
%
\begin{align*}
 \L(\L[\grad\ve{a}\R]\cdot\ve{b}\R)\ve{b}
 &= \L(\ve{b}\cdot\L[\grad\ve{a}\R]\R)\ve{b}
 \note{$c\ve{v} = \ve{v}c$ in $\mathbb{R}$}
 \\
 %
 &= \ve{b}\L(\ve{b}\cdot\L[\grad\ve{a}\R]\R)
 \note{$\partial_\| \defined \ve{b}\cdot \nabla$}
 \\
 %
 &= \ve{b}\L(\partial_\|\ve{a}\R)
 \note{$\nabla_\| \defined \ve{b}\ve{b}\cdot \nabla$}
 \\
 %
  &= \grad_\|\ve{a}
\end{align*}
%
For further references, we also define all the gradient operators
in this thesis here
%
\begin{empheq}[box=\tcbhighmath]{align}
    &\partial_\| \defined \ve{b}\cdot \nabla&
    &\nabla_\| \defined \ve{b}\ve{b}\cdot \nabla&
    &\nabla_\perp \defined \nabla - \nabla_\|&
    \\
    &\grad^2 = \div \grad&
    &\grad_\|^2 = \div \grad_\|&
    &\grad_\perp^2
    = \div\grad_\perp
    = \div\L(\grad - \grad_\|\R)
    = \grad^2 - \grad_\|^2
\end{empheq}
%
%
%https://www.physicsforums.com/threads/why-is-scalar-multiplication-on-vector-sp
%a ces-not-commutative.94783/
% http://en.wikipedia.org/wiki/Scalar_multiplication
%
This means that if we right dot equation (\ref{eq:no_assumptions_momentum})
with $\ve{b}\ve{b}$, we get
%
\begin{align*}
  \L(\frac{1}{\om_{c\a}}\d_{t,\a} \ve{u}_{\a}\R)\cdot\ve{b}\ve{b}
 &=
 \L(
 -
 \frac{
   \grad p_\a
 }{n_\a  q_\a B}
 +
 \frac{\ve{E}}{B}
 +
 \ve{u}_{\a}\times\ve{b}
 -
  \frac{
   \div\te{\pi}_\a
 }{n_\a  q_\a B}
 +
 \frac{
   \ve{R}_{\beta \to \a}
 }{n_\a q_\a B}
 +
 \frac{
   \ve{R}_{n \to \a}
 }{n_\a q_\a B}
 -
 \frac{
     S_{\a,n}\ve{u}_{\a,\|}
 }{n_\a \om_{c\a}}
 \R)\cdot\ve{b}\ve{b}
\end{align*}
%
\begin{empheq}[box=\tcbhighmath]{align}
 \frac{1}{\om_{c\a}}\d_{t,\a} \ve{u}_{\a,\|}
 &=
 -
 \frac{
   \grad_\| p_\a
 }{n_\a  q_\a B}
 +
 \frac{\ve{E_\|}}{B}
 -
  \frac{
   \L(\div\te{\pi}_\a\R)_\|
 }{n_\a  q_\a B}
 +
 \frac{
   \ve{R}_{\beta \to \a,\|}
 }{n_\a q_\a B}
 +
 \frac{
   \ve{R}_{n \to \a}
 }{n_\a q_\a B}
 -
 \frac{
     S_{\a,n}\ve{u}_{\a,\|}
 }{n_\a \om_{c\a}}
 \label{eq:material_dot_bb}
\end{empheq}
%
If we subtract equation (\ref{eq:material_dot_bb}) from equation
(\ref{eq:no_assumptions_momentum}), and use that  $\nabla_\perp =
\nabla - \nabla_\|$ and $\ve{a}\times\ve{b}=\ve{a}_\perp\times\ve{b}$, we get
%
\begin{align}
 \frac{1}{\om_{c\a}}\d_{t,\a} \ve{u}_{\a,\perp}
 &=
 -
 \frac{
   \grad_\perp p_\a
 }{n_\a  q_\a B}
 +
 \frac{\ve{E}_\perp}{B}
 +
 \ve{u}_{\a,\perp}\times\ve{b}
 -
  \frac{
   \L(\div\te{\pi}_\a\R)_\perp
 }{n_\a  q_\a B}
 +
 \frac{
   \ve{R}_{\beta \to \a,\perp}
 }{n_\a q_\a B}
 +
 \frac{
   \ve{R}_{n \to \a,\perp}
 }{n_\a q_\a B}
 -
 \frac{
     S_{\a,n}\ve{u}_{\a,\perp}
 }{n_\a \om_{c\a}}
 \label{eq:perp_mom_start}
\end{align}

\section{Drift ordering}\label{sec:drift_order}
% NOTE: As Aske Olsen pointed out, it would maybe be more stringent to
%       normalize first, then checking the orders of magnitude
%       Example: High aspect ratio:
%       r<<R => r/R<<1
%       Normalize R_tilde = R/R0
%       Normalize r_tilde = r/r0
%       But must have r0 << R0 in order to say that r_tilde << R_tilde
The goal of the drift ordering is to split the equations in
(\ref{eq:perp_mom_start}) in a different orders, yielding algebraic
equations for each order of $\ve{u}_{\a,\perp}$. In order to do the drift
ordering we will use that $\ve{R}_{\g \to \a} \propto
\nu_\g n_\a m_\a$, where $\nu_\g$ is the collision frequency. Thus
%
\begin{align*}
 \frac{\ve{R}_{\g \to \a}}{n_\a q_\a B}
 \propto \frac{\nu_\g n_\a m_\a}{n_\a q_\a B}
 = \frac{\nu_\g}{\om_{c\a}}
\end{align*}
%
We assume that the collision frequency is much smaller than the slowest
cyclotron frequency (that is the ion cyclotron frequency), so that
%
\begin{align*}
 \frac{\nu_\g}{\om_{ci}} \ll 1
\end{align*}
%
The same argument can be used for the $\L(\div \te{\pi}_\a\R)_\perp$ term in
equation (\ref{eq:perp_mom_start}).
Further on, we will make an assumption on the timescale. By writing out the
left hand side of equation (\ref{eq:perp_mom_start}) explicitly, we obtain
%
\begin{align*}
 \frac{1}{\om_{c\a}}\d_{t,\a} \ve{u}_{\a,\perp} &=
 \L(\frac{\partial_t}{\om_{c\a}} +
 \frac{\ve{u}_{\a}\cdot\nabla}{\om_{c\a}}\R) \ve{u}_{\a,\perp}
\end{align*}
%
Order of magnitude estimates gives
%
\begin{align*}
 &\frac{\L|\partial_t\R|}{\L|\om_{c\a}\R|}  \sim
 \frac{\L|\om_{\text{char}}\R|}{\L|\om_{c\a}\R|}&
%
 &\frac{\L|\ve{u}_{\a}\cdot\nabla\R|}{\L|\om_{c\a}\R|} \sim
 \frac{\L|\ve{u}_{\a}\cdot\ve{k}_{\text{char}}\R|}{\L|\om_{c\a}\R|}
\end{align*}
%
We now assume that these are small, so that
%
\begin{align*}
 &\frac{\L|\om_{\text{char}}\R|}{\L|\om_{c\a}\R|} \ll 1&
%
 &\frac{\L|\ve{u}_{\a}\cdot\ve{k}_{\text{char}}\R|}{\L|\om_{c\a}\R|} \ll 1
\end{align*}
%
In addition, as $S_{\a,n}$ represents the creation of particles per time per
unit volume, $\frac{S_{\a,n}}{n_\a}$ can be interpreted as the creation of particles per
time, which we will denote $\nu_{S,\a}$. We have that
%
\begin{align*}
 \frac{
     S_{\a,n}\ve{u}_{\a,\perp}
 }{n_\a \om_{c\a}}
 &=
 \frac{
     \nu_{S,\a}
 }{ \om_{c\a}}
\ve{u}_{\a,\perp}
\end{align*}
%
We will now assume that the creation of particles happens on a much slower time
scale than the ion cyclotron motion. In other words that
%
\begin{align*}
 \frac{
     \nu_{S,\a}
 }{ \om_{c\a}}
 \ll
 1
\end{align*}
%
Finally, we assume that
%
\begin{align*}
 \nu_\a \simeq
 \om_{\text{char}} \simeq
 \ve{u}_{\a}\cdot\ve{k}_{\text{char}}\simeq
 \nu_{S,\a}
\end{align*}
%
With this, we can now introduce the \emph{label} $\e$ which indicates
the order of the term. That is, $\e^0$ in front of a term indicates that it is
of zeroth order, $\e^1$ is of the first order and so on. We then have
$\e^0 f_a \gg \e^1 f_b \gg \e^2 f_c \gg \ldots$, where the different $f_i$
represent different terms (not to be confused with the distribution function)

% FIXME: Write that these are not assumptions at all, but in fact use typical
% values
If we assume that $-\frac{\grad_\perp p_\a}{n_\a  q_\a B} +
\frac{\ve{E}_\perp}{B} + \ve{u}_{\a}\times\ve{b}$ is of the zeroth order, we
can write equation (\ref{eq:perp_mom_start}) in the label notation as
%
\begin{align*}
  \e\frac{1}{\om_{c\a}}\d_{t,\a} \ve{u}_{\a,\perp}
 &=
 \e^0\L(
 -
 \frac{\grad_\perp p_\a}{n_\a  q_\a B}
 +
 \frac{\ve{E_\perp}}{B}
 +
 \ve{u}_{\a,\perp}\times\ve{b}
 \R)
 +
 \e\L(
  \frac{\ve{R}_{\beta \to \a,\perp}}{n_\a q_\a B}
  +
  \frac{\ve{R}_{n \to \a,\perp}}{n_\a q_\a B}
  -
  \frac{\L[\div\te{\pi}_\a\R]_\perp}{n_\a  q_\a B}
  -
  \frac{ S_{\a,n}\ve{u}_{\a,\perp} }{n_\a \om_{c\a}}
 \R)
\label{eq:first_assumptions_momentum_before_split}
\numberthis
\end{align*}
%
By splitting the velocities in order of magnitudes, we can perform the drift
ordering in an iterative manner (as shown below). We get
%
\begin{align*}
 \ve{u}_{\a,\perp} &= \ve{u}_{\a,0,\perp} + \e\ve{u}_{\a,1,\perp} +
                      \e^2\ve{u}_{\a,2,\perp} + \ldots\\
 \ve{u}_{\a,\perp} &\simeq \ve{u}_{\a,0,\perp} + \e\ve{u}_{\a,1,\perp},
 \label{eq:u_first_orders}
 \numberthis
\end{align*}
%
which is correct to the first iteration. If we substitute equation
(\ref{eq:u_first_orders}) into equation
(\ref{eq:first_assumptions_momentum_before_split}), we obtain
%
\begin{align}
  \e\frac{1}{\om_{c\a}}\d_{t,\a} \L(\ve{u}_{\a,0,\perp} +
  \e\ve{u}_{\a,1,\perp}\R)
 =&
 \e^0\L(
 -
 \frac{\grad_\perp p_\a}{n_\a  q_\a B}
 +
 \frac{\ve{E}_\perp}{B}
 +
 \L[\ve{u}_{\a,0,\perp} + \e\ve{u}_{\a,1,\perp}\R]\times\ve{b}
 \R)\notag\\
 &+
 \e\L(
  \frac{\ve{R}_{\beta \to \a,\perp}}{n_\a q_\a B}
  +
  \frac{\ve{R}_{n \to \a,\perp}}{n_\a q_\a B}
  -
  \frac{\L[\div\te{\pi}_\a\R]_\perp}{n_\a  q_\a B}
  -
  \frac{
      S_{\a,n}\L[\ve{u}_{\a,0,\perp} + \e\ve{u}_{\a,1,\perp}\R]
      }{
      n_\a \om_{c\a}}
 \R)
\label{eq:first_assumptions_momentum}
\end{align}
%
The left hand side of equation (\ref{eq:first_assumptions_momentum}) can now be
written
%
\begin{align*}
 \frac{1}{\om_a}\e\d_{t,\a} (\ve{u}_{\a,0,\perp} + \e\ve{u}_{\a,1,\perp}) =
 %
 &\frac{1}{\om_a}\e\parti{}{t}(\ve{u}_{\a,0,\perp} + \e\ve{u}_{\a,1,\perp})\\
 &+
 (\ve{u}_{\a,0,\perp} + \e\ve{u}_{\a,1,\perp} + \ve{u}_{\a,\|})\cdot
 \frac{1}{\om_a} \e\nabla (\ve{u}_{\a,0,\perp} + \e\ve{u}_{\a,1,\perp})
\end{align*}
%
Further, we can order $\frac{1}{\om_a}\e\d_{t,\a} \ve{u}_\a$. By introducing
the notation $\d_{t,\a}^n$, where $n$ denotes the order we have that
%
\begin{align*}
 \d_{t,\a}^0 \ve{u}_{\a,\perp} &= 0
 \note{No $\e^0$ terms in LHS of \ref{eq:first_assumptions_momentum}}\\
 %
 \d_{t,\a}^1 \ve{u}_{\a,\perp} &= \parti{}{t} \ve{u}_{\a,0,\perp} +
 \ve{u}_{\a,0,\perp} \cdot\nabla \ve{u}_{\a,0,\perp}\\
 %
 \d_{t,\a}^2 \ve{u}_{\a,\perp} &= \parti{}{t}\ve{u}_{\a,1,\perp} +
 \ve{u}_{\a,0} \cdot\nabla \ve{u}_{\a,1,\perp} +
 \ve{u}_{\a,1} \cdot\nabla \ve{u}_{\a,0,\perp}\\
 &\;\, \vdots \notag
\end{align*}
%
As there seem not to be any natural ordering of
parallel velocities, these are not ordere in the same way as the perpendicular
velocities. However, as the general material derivative can be written
$\d_{t,\a}=\partial_t + \ve{u}_{\a}\cdot\nabla$, and since
$\ve{u}_\a = \ve{u}_{\a,\|} + \ve{u}_{\a,0,\perp} + \ve{u}_{\a,1,\perp} +
\ldots$, the parallel advection needs to be accounted for. To simplify the
coming notation, we will include the parallel advection in $\d_{t,\a}^1$, such
that from this point on, we will write
%
\begin{align*}
 \d_{t,\a}^1 &= \parti{}{t} + (\ve{u}_{\a,0,\perp} + \ve{u}_{\a,\|}) \cdot\nabla
\end{align*}
%
although the parallel velocity is not a part of the formal ordering.

We also note that
%
\begin{align*}
  \e
  \frac{
      S_{\a,n}\ve{u}_{\a,0,\perp}
      }{
      n_\a \om_{c\a}}
\end{align*}
%
is formally a first order term, and
%
\begin{align*}
  \e^2
  \frac{
       S_{\a,n}\ve{u}_{\a,1,\perp}
      }{
      n_\a \om_{c\a}}
\end{align*}
%
is formally a second order term.

\section{Zeroth order perpendicular terms}
The zeroth order term of equation (\ref{eq:first_assumptions_momentum}) can
then be written
%
\begin{align}
 0 &=
 - \frac{\grad_\perp p_\a}{n_\a  q_\a B}
 + \frac{\ve{E}_\perp}{B}
 + \ve{u}_{\a,0,\perp}\times\ve{b}
 \label{eq:zeroth_start}
\end{align}
%
In general, we have that $-\ve{b}\times\L(\ve{b}\times\ve{a}\R) =
\ve{a} - \ve{a}_\|=\ve{a}_\perp$. Thus, we can solve equation
(\ref{eq:zeroth_start}) for $\ve{u}_{\a,0,\perp}$ by crossing the equation with
$\ve{b}$
%
\begin{align*}
 0 &=
 - \frac{\grad_\perp p_\a}{n_\a  q_\a B}\times\ve{b}
 + \frac{\ve{E}_\perp}{B}\times\ve{b}
 + \L(\ve{u}_{\a,0,\perp}\times\ve{b}\R)\times\ve{b}
 \\
 %
 &=
 - \frac{\grad_\perp p_\a\times\ve{b}}{n_\a  q_\a B}
 + \frac{\ve{E}_\perp\times\ve{b}}{B}
 + \ve{b}\times\L(\ve{b}\times\ve{u}_{\a,0,\perp}\R)
 \\
 %
 - \ve{b}\times\L(\ve{b}\times\ve{u}_{\a,0,\perp}\R)
 &=
 - \frac{\grad_\perp p_\a\times\ve{b}}{n_\a  q_\a B}
 + \frac{\ve{E}_\perp\times\ve{b}}{B}
 \\
  %
 \ve{u}_{\a,0,\perp}
 &=
 - \frac{\grad_\perp p_\a\times\ve{b}}{n_\a  q_\a B}
 + \frac{\ve{E}_\perp\times\ve{b}}{B}
 \numberthis
 \label{eq:u_0_balance}
\end{align*}
%
As mentioned in appendix \ref{app:elstat}, a low $\beta$ plasmas justifies the
electrostatic approximation. This means that $\ve{E} = -\grad\phi$, so
$\ve{E}_\perp = -\grad_\perp\phi$ and $\ve{E}_\| = -\grad_\|\phi$. Thus can we
rewrite equation (\ref{eq:u_0_balance}) as
%
\begin{empheq}[box=\tcbhighmath]{align*}
 \ve{u}_{\a,0,\perp} =&
 %
  \underbrace{
    %
    -\frac{\grad_\perp p_\a \times\ve{b}}{q_\a n_\a  B}
    %
   }
   _{\ve{u}_{\a,d}}
   %
 %
 \underbrace{
    %
    -  \frac{\grad_\perp \phi \times \ve{b}}{B}
    %
   }
   _{\ve{u}_{E}}
   %
 %
 \label{eq:u_0}
 \numberthis
\end{empheq}

\section{First order perpendicular terms}
We obtain the algebraic equation of order $1$ of equation
(\ref{eq:first_assumptions_momentum}) by collecting terms of order $1$. We get
%
\begin{align}
  \frac{1}{\om_{c\a}}\d^1_{t,\a} \ve{u}_{\a,0,\perp}
 =&
  \ve{u}_{\a,1,\perp}\times\ve{b}
  +
   \frac{\ve{R}_{\beta \to \a,\perp}}{n_\a q_\a B}
  +
   \frac{\ve{R}_{n \to \a,\perp}}{n_\a q_\a B}
  -
  \frac{\L(\div\te{\pi}_\a\R)_\perp}{n_\a  q_\a B}
  -
  \frac{ S_{\a,n}\ve{u}_{\a,0,\perp} }{n_\a \om_{c\a}}
  \label{eq:u_1_clean}
\end{align}
%
We see that the first order depends on the zeroth order, and hence we need to
solve for the orders iteratively. The algebraic equation for
$\ve{u}_{\a,1,\perp}$ can be found by crossing equation (\ref{eq:u_1_clean})
with $\ve{b}$. Generally we have
%
\begin{align*}
 \L(\d_{t,\a} \ve{a} \R)\times\ve{b}
 &= \L(\partial_t \ve{a} + \ve{u}_{\a}\cdot\nabla\ve{a} \R)\times\ve{b}
 \note{\hspace{-2.5cm}Assumed electrostatic}
 \\
 %
 &= \partial_t \L(\ve{a}\times\ve{b}\R) +
 \L(\L[\ve{u}_{\a}\cdot\nabla\ve{a}\R]\times\ve{b}\R)
 \note{\hspace{-2.5cm}
       $\grad\L(\ve{v}\times\ve{w}\R) = \L(\grad
        \ve{v}\R)\times\ve{w} - \L(\grad \ve{w}\R)\times\ve{v}$}
 \\
 %
 &= \partial_t \L(\ve{a}\times\ve{b}\R) +
 \ve{u}_{\a}\cdot\L(\nabla\L[\ve{a}\times\ve{b}\R] +
 \L[\nabla\ve{b}\R]\times\ve{a}\R)
 \note{\hspace{-2.5cm}Assumed electrostatic}
 \\
 %
 &= \partial_t \L(\ve{a}\times\ve{b}\R) +
 \ve{u}_{\a}\cdot\L(\nabla\L[\ve{a}\times\ve{b}\R]\R)
 \\
 %
 &= \d_{t,\a} \L(\ve{a} \times\ve{b} \R)
\end{align*}
%
This gives
%
\begin{align*}
  \L(\frac{1}{\om_{c\a}}\d^1_{t,\a} \ve{u}_{\a,0,\perp}\R)\times\ve{b}
 =&
  \L(\ve{u}_{\a,1,\perp}\times\ve{b}\R)\times\ve{b}
  +
  \frac{\ve{R}_{\beta \to \a,\perp}}{n_\a q_\a B}\times\ve{b}
  +
  \frac{\ve{R}_{n \to \a,\perp}}{n_\a q_\a B}\times\ve{b}
  \\
  &-
  \frac{\L(\div\te{\pi}_\a\R)_\perp}{n_\a  q_\a B}\times\ve{b}
  -
  \frac{ S_{\a,n}\ve{u}_{\a,0,\perp} }{n_\a \om_{c\a}}\times\ve{b}
  \\
 %
 -\ve{b}\times\L(\ve{b}\times\ve{u}_{\a,1,\perp}\R)
 =&
 -
 \frac{1}{\om_{c\a}}\d^1_{t,\a}\L( \ve{u}_{\a,0,\perp}\times\ve{b}\R)
  +
  \frac{\ve{R}_{\beta \to \a,\perp}\times\ve{b}}{n_\a q_\a B}
  +
  \frac{\ve{R}_{n \to \a,\perp}\times\ve{b}}{n_\a q_\a B}
  \\
  &-
  \frac{\L(\div\te{\pi}_\a\R)_\perp\times\ve{b}}{n_\a  q_\a B}
  -
  \frac{ S_{\a,n}\ve{u}_{\a,0,\perp} }{n_\a \om_{c\a}}\times\ve{b}
  \\
 %
 \ve{u}_{\a,1,\perp}
 =&
 -
 \frac{1}{\om_{c\a}}\d^1_{t,\a}\L( \ve{u}_{\a,0,\perp}\times\ve{b}\R)
  +
  \frac{\ve{R}_{\beta \to \a,\perp}\times\ve{b}}{n_\a q_\a B}
  +
  \frac{\ve{R}_{n \to \a,\perp}\times\ve{b}}{n_\a q_\a B}
  \\
  &-
  \frac{\L(\div\te{\pi}_\a\R)_\perp\times\ve{b}}{n_\a  q_\a B}
  -
  \frac{ S_{\a,n}\ve{u}_{\a,0,\perp} }{n_\a \om_{c\a}}\times\ve{b}
 \label{eq:u_1_u_0}
 \numberthis
\end{align*}
%
%
By substituting equation (\ref{eq:u_0}) in equation (\ref{eq:u_1_u_0}), we
obtain
%
% FIXME: Consider not to expand u_s as this disappears
\begin{align*}
 \ve{u}_{\a,1,\perp}
 =&
 -
 \frac{1}{\om_{c\a}}\d^1_{t,\a}\L(\L[
  - \frac{\grad_\perp p_\a\times\ve{b}}{n_\a  q_\a B}
  - \frac{\grad_\perp \phi \times \ve{b}}{B}
 \R]\times\ve{b}\R)
  +
  \frac{\ve{R}_{\beta \to \a,\perp}\times\ve{b}}{n_\a q_\a B}
  +
  \frac{\ve{R}_{n \to \a,\perp}\times\ve{b}}{n_\a q_\a B}
  \\
  &-
  \frac{\L(\div\te{\pi}_\a\R)_\perp\times\ve{b}}{n_\a  q_\a B}
  -
  \frac{ S_{\a,n}}{n_\a \om_{c\a}}
  \L(
  - \frac{\grad_\perp p_\a\times\ve{b}}{n_\a  q_\a B}
  - \frac{\grad_\perp \phi \times \ve{b}}{B}
  \R)
  \times\ve{b}
  \\
 %
 =&
 -
 \frac{1}{\om_{c\a}}\d^1_{t,\a}\L(
  - \frac{\ve{b}\times\L[\ve{b}\times\grad_\perp p_\a\R]}{n_\a  q_\a B}
  - \frac{\ve{b}\times\L[\ve{b}\times\grad_\perp \phi\R]}{B}
  \R)
  +
  \frac{\ve{R}_{\beta \to \a,\perp}\times\ve{b}}{n_\a q_\a B}
  +
  \frac{\ve{R}_{n \to \a,\perp}\times\ve{b}}{n_\a q_\a B}
  \\
  &-
  \frac{\L(\div\te{\pi}_\a\R)_\perp\times\ve{b}}{n_\a  q_\a B}
  -
  \frac{ S_{\a,n}}{n_\a \om_{c\a}}
  \L(
  - \frac{\ve{b}\times\L[\ve{b}\times\grad_\perp p_\a\R]}{n_\a  q_\a B}
  - \frac{\ve{b}\times\L[\ve{b}\times\grad_\perp \phi\R]}{B}
  \R)
  \note{Definition of perp. vectors}
  \\
 %
 =&
 -
 \frac{1}{\om_{c\a}}\d^1_{t,\a}\L(
    \frac{\grad_\perp p_\a}{n_\a  q_\a B}
  + \frac{\grad_\perp \phi}{B}
  \R)
  +
  \frac{\ve{R}_{\beta \to \a,\perp}\times\ve{b}}{n_\a q_\a B}
  +
  \frac{\ve{R}_{n \to \a,\perp}\times\ve{b}}{n_\a q_\a B}
  \\
  &-
  \frac{\L(\div\te{\pi}_\a\R)_\perp\times\ve{b}}{n_\a  q_\a B}
  -
  \frac{ S_{\a,n}}{n_\a \om_{c\a}}
  \L(
  \frac{\grad_\perp p_\a}{n_\a  q_\a B}
  + \frac{\grad_\perp \phi}{B}
  \R)
\end{align*}
%
Hence
%
\begin{empheq}[box=\tcbhighmath]{align*}
 \ve{u}_{\a,1,\perp} =&
 %
  \underbrace{
  \frac{1}{\om_{c\a}}\d^1_{t,\a}\L(
  - \frac{\grad_\perp p_\a}{n_\a  q_\a B}
  - \frac{\grad_\perp \phi}{B}
  \R)
   }
  _{\ve{u}_{\a,p}}
  \underbrace{
   + \frac{\ve{R}_{\beta \to \a,\perp}\times \ve{b}}{n_\a q_\a B}
   }
  _{\ve{u}_{\a,R}}
  \underbrace{
   + \frac{\ve{R}_{n \to \a,\perp}\times\ve{b}}{n_\a q_\a B}
   }
   _{\ve{u}_{\a,\text{Ped}}}
  \nonumber
  \\
  &
  \underbrace{
   - \frac{\L(\div\te{\pi}_\a\R)_\perp\times\ve{b}}{n_\a q_\a B}
   }
  _{\ve{u}_{\a,\nu}}
  \underbrace{
  -
  \frac{ S_{\a,n}}{n_\a \om_{c\a}}
  \L(
  \frac{\grad_\perp p_\a}{n_\a  q_\a B}
  + \frac{\grad_\perp \phi}{B}
  \R)
   }
  _{\ve{u}_{\a,S}}
  \label{eq:first_order}
  \numberthis
\end{empheq}
%
We note that even though the material derivative contains parallel derivatives,
the resulting vector is purely perpendicular.
