\subsection{The Boussinesq approximation}
\label{sec:boussinesq}
%
A somewhat common approximation is the so-called Boussinesq approximation.
Starting from equation (\ref{eq:start_of_boussinesq}), one do the splitting
%
\begin{align*}
    n = \overline{n} + \delta_n
\end{align*}
%
where $\overline{n}$ is the background profile, and $\delta_n$ is the fluctuation.
We now assume that
%
\begin{align*}
    &\grad \overline{n} \propto \frac{\overline{n}}{L}&
    &\grad \delta_n \propto \delta_n (k_\perp + k_\|) \simeq \delta_n k_\perp&
\end{align*}
%
where we have assumed small parallel gradients.  By normalizing the densities
with $n_0$ and the gradients with $\rho_s$, and writing the normalized units
with a tilde ( $ \wt{ } $ ), we find find
%
\begin{align*}
    &\wt{\grad} \wt{\overline{n}} \propto
    \frac{\overline{n}}{L}\frac{\rho_s}{n_0}
    \simeq \frac{\rho_s}{L}
    &
    %
    &\wt{\grad} \wt{\delta_n} \propto \delta_n k_\perp\frac{\rho_s}{n_0}
    = \frac{\delta_n}{n_0}k_\perp\rho_s
    &
\end{align*}
%
One of the underlying approximations is to say that $\frac{\delta_n}{n_0} \ll
1$. This is not always a good approximation as the fluctuations can be larger
than the background (especially when looking at
blobs).

% FIXME: Show that the last approximation is already taken
The second assumption is that $\frac{\rho_s}{L} \ll 1$. The final assumption is
that $k_\perp\rho_s \ll 1$, which is something we have already assumed
when making the drift ordering approximation.

Using the assumptions, we find that
%
\begin{align*}
    &\grad \overline{n} \propto \wt{\grad} \wt{\overline{n}} \ll 1
    &
    %
    &\grad \delta_n \propto \wt{\grad} \wt{\delta_n} \ll 1&
\end{align*}
%
Inserting this in equation (\ref{eq:start_of_boussinesq}), and by assuming that
$\delta_n$ is of order $\e^1$ in the fluid ordering, we find that
%
\begin{align*}
 \div\L( n
  \frac{1}{\om_{ci}}
  \L[
      \d_t + \ve{u}_{i,\|}\cdot\nabla
  \R]
  \L[
     \frac{\grad_\perp \phi}{B}
  \R]
 \R)
 \simeq&
 %
 (\overline{n} + \delta_n)\div\L(
  \frac{1}{\om_{ci}}
  \L[
      \d_t + \ve{u}_{i,\|}\cdot\nabla
  \R]
  \L[
     \frac{\grad_\perp \phi}{B}
  \R]
  \R)\\
  %
 \simeq&
 \overline{n}\div\L(
  \frac{1}{\om_{ci}}
  \L[
      \d_t + \ve{u}_{i,\|}\cdot\nabla
  \R]
  \L[
     \frac{\grad_\perp \phi}{B}
  \R]
 \R)
\end{align*}
%
where the second order term in $\e$ between $\delta_n$ and $\frac{1}{\om_{ci}}$ has
been neglected. If we assume relatively flat background profiles, we will have
$n_0 \simeq \overline{n}$.  This means that equation
(\ref{eq:from_gyroviscous}) with the Boussinesq approximation yields
%
\begin{align*}
 \div\L( n
  \frac{1}{\om_{ci}}
  \L[ \partial_t + (\ve{u}_{E} + \ve{u}_{i,\|})\cdot\nabla \R]
  \L[ - \frac{\grad_\perp \phi}{B} \R] \R)
 &\simeq
 - \frac{n_0}{\om_{ci}} \div\L(
     \d_t \L[ \frac{\grad_\perp \phi}{B} \R] \R)
- \frac{n_0}{\om_{ci}} \div\L(
     \ve{u}_{i,\|}\cdot\nabla
 \L[ \frac{\grad_\perp \phi}{B} \R]
 \R)
 \\
 %
 &=
 - \frac{n_0}{\om_{ci}} \div\L(
     \d_t \L[ \frac{\grad_\perp \phi}{B} \R] \R)
- \frac{n_0}{\om_{ci}} \div\L(
u_{i,\|}\partial_\| \L[ \frac{\grad_\perp \phi}{B} \R] \R)
\numberthis
\label{eq:boussinesq_vort}
\end{align*}
%
Notice that the approximation is only done here, and that we keep $\grad n$
elsewhere, although they according to the Boussinesq approximation are small.
