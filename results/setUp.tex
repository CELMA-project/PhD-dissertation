\section{Standard parameters for the runs}
%
In this thesis, we have chosen the following shape of the source
%
\begin{align*}
    S = AH(\rho,s_{\text{profile}},c_{\text{profile}},w_{\text{profile}})
\end{align*}
%
where $H$ is defined in \cref{eq:hat}, and $A$ is an amplitude chosen so that the maximum of the normalized $n$ is around $1$.
Note that the source is uniform in along the parallel direction.
A more probable scenario would probably be that the source would decrease along the parallel direction, until it is increased around the SE due to recycling from the wall.
In any case, finding the proper shape of the source seems to be challenging, and is probably only accurate if detailed atomic physics is taken into account.
Although interesting and relevant, such atomic processes is outside the scope of this thesis.
If nothing else is mentioned, the standard input for the simulations done are given in \cref{tb:grid,tb:input,tb:source}.
%

\begin{table}[!htb]
    \begin{minipage}{.3\linewidth}
      \centering
      \caption{Grid size used for the simulations.}
        \colorme
        \begin{tabular}{c|ll}
        \hline\hline
        %
        Variable & Value \\
        %
        \hline
        %
        $n_\rho$   & $32$  \\
        $n_z$      & $66$  \\
        $n_\theta$ & $256$ \\
        \hline\hline
        \end{tabular}
        \label{tb:grid}
    \end{minipage}%
    \hfill
    \begin{minipage}{.3\linewidth}
      \centering
        \caption{Standard parameters used in the simulations.}
            \colorme
            \begin{tabular}{c|ll}
            \hline\hline
            %
            Variable & Value & Units\\
            %
            $L_\rho$  & $0.0$            & $\m$     \\
            $L_z$     & $2.80$           & $ \m$    \\
            $n_0$     & $1\cdot 10^{19}$ & $\m^{-3}$\\
            $T_{e,0}$ & $2.5$            & $\eV$    \\
            $B_0$     & $0.1$            & $\T$     \\
            $n_n$     & $0.0$            & $\m^{-3}$\\
            Gas       & Ar               &          \\
            \hline\hline
            \end{tabular}
            \label{tb:input}
    \end{minipage}
    \hfill
    \begin{minipage}{.3\linewidth}
      \centering
        \caption{Source parameters used in the simulations.}
            \colorme
            \begin{tabular}{c|ll}
            \hline\hline
            %
            Variable & Value \\
            %
            \hline
            $s_{\text{profile}}$ & $5/L_\rho$\\
            $c_{\text{profile}}$ & $0$       \\
            $w_{\text{profile}}$ & $L_\rho$  \\
            \hline\hline
            \end{tabular}
            \label{tb:source}
    \end{minipage}
\end{table}
%
Input parameters from the implementation is found in the tables in \cref{chap:implBOUT++,chap:additionalImplementation}.

\section{Running the simulations}
\label{sec:initRun}
%
From the following initial conditions of the normalized evolved quantities%
%
\footnote{$\Om$ is used rather than $\Om^D$ when simulating using the Boussinesq approximation}%
%
\begin{align*}
    \ln(n)    =& 0\\
    j_\|      =& 0\\
    nu_{i,\|} =& \frac{z}{L_z}\\
    \Om^D     =& 0
\end{align*}
%
the simulations are performed in three following steps
%
\begin{enumerate}[noitemsep]
    \item The simulation is allowed to evolve freely to a steady state condition (that is, when there is a minimal difference between the time steps) using only one point in $\theta$.
        Usually this takes around $4000t\om_{ci}$.
    \item The simulation is expanded to $n_\theta = 256$, and run for additional $50t\om_{ci}$ in order to ensure that the system is still in a steady state.
    \item A perturbation in the form of white noise with an amplitude of $1\cdot10^{-6}$ is added to $\Om^D$%
        \footnote{Different approaches has been used for the perturbation.
            It should be noted that some of these gives a different route to the turbulent state (i.e. some perturbations dies out, while others have some intermediate non-linear steps before the linear growth rates), but gives (with very little difference) the same growth rates and the same statistical behavior of the saturated turbulence as the perturbation method stated.}%
        %
    \item If the system is unstable to small perturbations, and if there are no "crashes" in the simulation, a saturated turbulence state is eventually reached.
\end{enumerate}
%
All the simulations presented here are run on the \texttt{A1} (Broadwell) partition of the \texttt{Marconi} super cluster located at CINECA at Casalecchio di Reno (Bologna).
At the time of writing the cluster operated with the following specifications \cite{Marconi2016Web}
%
\begin{enumerate}[noitemsep]
    \item Model: Lenovo NeXtScale
    \item Architecture: Intel OmniPath Cluster
    \item Nodes: $1512$
    \item Processors: $2\times18$-cores Intel Xeon E$5-2697$ v$4$ (Broadwell) at $2.30$ GHz
    \item Cores: $36$ cores/node, $54432$ cores in total
    \item RAM: $128$ GB/node, $3.5$ GB/core
    \item Internal Network: Intel OmniPath Architecture $2$:$1$
    \item Disk Space: $17$PB (raw) of local storage
    \item Peak Performance: $2$ PFlop/s
\end{enumerate}
%
For each simulation, $2$ nodes has been allocated using $48$ cores.
This has been found to give a good speed-up (as compared to use one node) with a good trade off between the ration of simulation time to communication time per core.

\section{Resolution}
\label{sec:resolution}
%
% FIXME: Say something that this comes from  mu DDY(phi-lnn),
%        if because catastrophic cancellation, a higher mu would make it worse
%        in fact a higher mu helps as better resolution
%        Hypervisc will only alevate problem if visc coeff is in the order of the fields
In order to have a properly resolved grid, we choose to have a ratio at least $\frac{n_\rho}{L_\rho/\rho_s}>1$, but preferably above $3$.
As mentioned in (FIXME)
% FIXME: Add reference where say can have much coarser resolution in parallel direction
the resolution in the parallel direction can be less due to the separation of scales.
Note however, that we have a sheath boundary condition in the $z$-direction, which seems to limit the coarseness of the resolution.
As mentioned in (FIXME)
% FIXME: Add reference to where the coarseness parameter scan is done
we should require that $\frac{n_z}{L_z/\rho_s}\gtrsim0.2$.
We observe that
%
\begin{align*}
    \frac{n_z}{\frac{L_z}{\rho_s}}
    =& \frac{n_zc_s}{L_z \om_{ci}}
    \\
    =& \frac{n_z\sqrt{T_e}m_i}{\sqrt{m_i} L_z ZeB}
    \\
    =& \frac{n_z\sqrt{T_em_i}}{L_z ZeB}
\end{align*}
%
We would like to have as low as possible $n_z$ in order to have low computational times.
In addition, for a fixed machine, $L_z$ remains fixed.
As a consequence, we see that an increase in $B$ gives poorer resolution, whereas an increase in the electron temperature or the ion mass gives a better resolution.
On this argument, running simulations with singly ionized Ar gives a better resolution of $\sqrt{m_{\text{Ar}}/m_H}\approx6.3$ as compared with hydrogen.
