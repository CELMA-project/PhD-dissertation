%
The linear phase is succeeded by a transition phase to the saturated phase.
In this phase, the linear modes becomes large enough for non-linear effects to affect the dynamics of the system.
Through the non-linear terms in the equation (in particular the advective terms), mode coupling, and energy from the unstable, growing modes spreads to neighbouring modes through the non-linearities.
This can be seen around $t=0.0175\s$ in \cref{fig:fourierUnstable}, where $m_\theta=1$ suddenly shows a growth with a higher growth rate than the rest of the modes.
The transfer of energy can by studied through three-wave coupling under the assumptions that only neighbouring modes in the $k$-spectrum interacts (the weak turbulence assumption), and that four-wave and higher wave couplings are negligible.
This has been done in for example \cite{Ritz1989,Knorr1990}.

The cascading of energy through the different modes is what eventually brings the system into a saturated turbulence phase.
In an attempt to describe how this happens, an idealized case of fluid turbulence was considered by Kolmogorov and Oboukhov in their 1941 theory of turbulence \cite{Kolmogorov1962}.
The main assumptions of the paper is that the energy fed into the system in a small range of $k$, and that the dissipation of energy only happens at the smallest scales, well separated from the injection of energy.
In other words, there will be a mechanism feeding the turbulence with structures of a certain size.
These structures break up into smaller structures (the energy is cascading to higher $k$), which again break up into smaller structures, all the way until a viscous sink removes the energy from the system.
A decay rate $\propto k^{-5/3}$ is then found for the energy through dimensionality arguments.

Turbulence in plasmas has a more complex turbulent behavior as there exsits more degrees of freedom for the turbulence through the electromagnetic forces.
In addition, there is as mentioned a big separation the perpendicular and parallel scales in a strongly magnetized plasma.
The perpendicular displacement of fluid parcels is much more restricted than the displacement along the field lines due to the magnetic field.
Consequently, the turbulence have a $2$-dimensional character rather than a $3$-dimensional character.

In $2$-D, inviscid, turbulence enstrophy (global integrated vorticity) is conserved.
If the enstrophy is conserved, or at least almost conserved, there can be an inverse cascade of energy (i.e. migration of energy from small scales to large scale), as vortex stretching cannot occur \cite{Fjortoft1953}.
The main part of the energy, however, is still cascading towards the smaller structures in $2$-D turbulence.
In other words, the eddies will tend to merge together to larger coherent structures.
Such cascades has been observed in Earth's atmosphere \cite{Smith2002} in neutral fluid flows, but also arises naturally from the $2$-D Charney-Hasegawa-Mima model \cite{Boffetta2002}, and in the Hasegawa-Wakatani \cite{Manz2009} models in plasma phycis.





FOO
Energy is injected from the gradients into a few $k$s, and dissipated at a few $k$s.


The turbulence will reach a steady state once the input of energy through the source is balanced by the dissipation of energy.
On the transition from the linear phase to the turbulent phase a energy overshot is observed as seen in \cref{fig:energyTrace008}.



It is important to note that although there migth be one dominating instability which causes the onset to turbulence, the characteristic of the turbulence is more or less inependent of the onsetting instability.
In other words, one cannot easily extract the cause of the turbulence by looking at the turbulence alone.





growth continues until a sudden burst of the plasma column,
increase of energy (written below)
In order to exclude the effects of the transients, we define the start of the saturated turbulence after the










%
\begin{figure}[htb]
    \centering
    \includegraphics[width=1.0\textwidth]{fig/results/energyTrace/energyTraceB008}
    \caption{Time trace of the energy for $B=0.08$.}
    \label{fig:energyTrace008}
\end{figure}

As a consequence the eddies evolve at a faster phase at the transition as compared with the saturated turbulent state where eddies evolve at a slower rate.
In the saturated state the energy stays closer to the temporal mean, shown in \cref{fig:energyTrace008}.
It is also important to observe that the fluctuations can be big enough to push the plasma off center as observed in \cref{fig:turbEv}.
%
\begin{figure}[htbp]
    \centering
    \begin{subfigure}[h]{1.00\textwidth}
        \centering
        \includegraphics[width=1.0\textwidth]{fig/results/evolution/n-perpPar-2D-0}
    \end{subfigure}%
    \\
    \begin{subfigure}[h]{1.00\textwidth}
        \centering
        \includegraphics[width=1.0\textwidth]{fig/results/evolution/n-perpPar-2D-1}
    \end{subfigure}
    \\
    \begin{subfigure}[h]{1.00\textwidth}
        \centering
        \includegraphics[width=1.0\textwidth]{fig/results/evolution/n-perpPar-2D-2}
    \end{subfigure}
    \caption{Evolution of the plasma in the saturated turbulence phase.
        Here shown for $B=0.08\T$}
    \label{fig:turbEv}
\end{figure}
%
In the saturated turbulence state, the fluctuations are no longer in an ordered pattern as they were in the linear phase.
\Cref{fig:2DFluct} shows this.
%
\begin{figure}[htb]
    \centering
    \includegraphics[width=1.0\textwidth]{fig/results/2DTurbulence/fluct}
    \caption{Fluctuations in the turbulent state for $B=0.1\T$}
    \label{fig:2DFluct}
\end{figure}
%
\subsection{Fluxes}
FIXME: Move figures and describe
%
\begin{figure}[htb]
    \centering
    \includegraphics{fig/results/totalFlux/flux0008}
    \caption{Integrated flux for $B=0.08\T$}
    \label{fig:flux0008}
\end{figure}
%
\begin{figure}[htb]
    \centering
    \includegraphics{fig/results/totalFlux/flux0006}
    \caption{Integrated flux for $B=0.06\T$}
    \label{fig:flux0006}
\end{figure}
