NOTE: Considering to skip this part
% NOTE: Could maybe redo the x-axis to 80 60 40 20 pct ionization.
% NOTE: Could add 0 pct?

We scan for $B_0=0.08\T$ the degree of ionization $d$ in $80\%$, $60\%$, $40\%$, $20\%$, $1\%$, where $d$ is given by
%
\begin{align*}
    d = \frac{n_i}{n_i+n_n}
\end{align*}
%
In experiments, ionization degrees from $0.1\%$ up to $100\%$ has been observed in the experiments \cite{Schroder2003Phd}.

We take $n_i=n_0$ based on our quasi-neutral assumption.


This corresponds to neutral densities of
$2.5\cdot10^{18}\m^{-3}$
$6.7\cdot10^{18}\m^{-3}$
$1.5\cdot10^{19}\m^{-3}$
$4.0\cdot10^{19}\m^{-3}$
$9.9\cdot10^{20}\m^{-3}$




Collisionality calculated from \cref{eq:elArColl}, and we observe that $\nu_{en}\propto n_n$.

With our parameters, we the electron-ion collisions are constant $\nu_{ei}=7.25\cdot10^7 \s^{-1}$
only for $1\%$ ionization the neutral collisions start to dominate with $\nu_{en}=7.38\cdot10^7\s^{-1}$

$80\%$ gives $\nu_{ei}=1.87\cdot10^{5}$.


To keep consistency $\nu_{in}=0$ as $T_i=0$.
The physical justification for this is questionable as we assume that ions are streaming against stationary ions (see \cref{app:collisions} for details).
However, if all the ions are misaligned with respect to the neutrals, no collision will take place.

In any case, we usually have $\nu_{in}\ll\nu_{en}$ \cite{Schroder2003Phd}.
and that $\nu_{in}$ terms in \cref{eq:celma_vortD_evolution} is small.


Why shear flips opposite


In the steady state $\Om^D$ is maintained by a balance between $\partial_\|\div\L(u_{i,\|}n\frac{\grad_\perp}{B}\R)$.
Flip is changed as $j_\|$ changes.
$j_\|$ changes as the $\nu_{en}$ dominates.

Consequence $u_{e,\|}$, $j_\|$, $\Om$ and $\phi$ changes.





THIS IS NOT NEUTRAL BUT GOES TO STEADY STATE....OR MAYBE IT GOES TO SHEAR POLOIDAL FLOW?
Sharp gradient in $\Om$ because $j_\|$ sets up a boundary layer.
$j_\|$ froms a boundary layer as $\mu n \partial_\|\L(T_e\ln[n]-\phi\R)$ diverges due to difference in boundary conditions between $\phi$ and $n$.


Position of fluctuation only maintained until last

Variation in fluxes


Growth rates...oh yeah!
