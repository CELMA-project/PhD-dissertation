\section{Artificial viscosity}
\label{sec:art_visc}
%
As mentioned in \cref{sec:SOE}, we have neglected the viscosities (as we assume that these are small).
Instead we add some artificial viscosities, which is needed for numerical stability (in addition to those added through the upwinding schemes).
If there are no viscosities in the system, there is no mechanism (other than diffusion) which can remove the small scales from the system.
Scales which are smaller than what can be resolved by the grid will lead to problems like aliasing.
These small scales can then build up and make the simulation crash%
\footnote{
    We will here use the word "crash" meaning that the simulation stops because of bad behavior the numerics.
    The simulation stops if a field reaches $\pm$\texttt{inf} or \texttt{nan} values (this can cause division by zero problems) or if the solver takes too small time steps over a range of time because it cannot resolve unphysically large gradients.
}
%
.

For this reason, we add viscosities on the form
%
\begin{align*}
    D_{f, \|, \text{art}} \nabla_{\|}^2 f
    + D_{f, \perp, \text{art}} \grad_\perp^2 f
    &=
    D_{f, \|, \text{art}} \div \L(\ve{b}\ve{b}\cdot\grad\R) f
    + D_{f, \perp, \text{art}} \grad_\perp^2 f
    \note{$\partial_i \ve{b} = 0$}
    \\
    %
    &=
    D_{f, \|, \text{art}} \ve{b}\cdot\grad \L(\ve{b}\cdot\grad\R) f
    + D_{f, \perp, \text{art}} \grad_\perp^2 f
    \\
    %
    &=
    D_{f, \|, \text{art}} \partial_\|^2 f
    + D_{f, \perp, \text{art}} \grad_\perp^2 f,
    \numberthis
    \label{eq:art_vort}
\end{align*}
%
to all the evolved fields $f$.

In addition, a hyper-viscous term is added in the azimuthal direction for the $\Om^D$ and $\Om$ equations.
That is a term
%
\begin{align*}
    - D_{f, \theta, \text{art}}^H \partial_{\theta}^4 f.
\end{align*}
%
The reason for this is that small scales grow up in the vorticity equations despite the filtering.
To see that this term actually dampens higher modes, consider the example
%
\begin{align*}
    \partial_t f = (-1)^{n+1}D \partial_x^{2n} f \qquad n\in\mathbb{N},
\end{align*}
%
which we for simplicity assumes is periodical in $x$.
For a certain wave number $k$, we now have (see \cref{app:deriv_of_FT} for details)
%
\begin{align*}
    \partial_t f_k = (-1)^{n+1} (ik)^{2n} D f_k,
\end{align*}
%
where the $(-1)^{n+1}$ factor ensures that the time evolution of wave number $k$ is decreasing for every choice of $n$.
The artificial viscosity coefficients used in this thesis are given in table \cref{tb:artVisc}.
%
\begin{table}[h!]
{\footnotesize \centerline{
\colorme
\begin{tabular}{c|ccc}
\hline\hline
%
Variable & $\qquad D_\|\qquad$ & $\qquad D_\perp\qquad$ & $\qquad D^{\text{Hyper}}_\theta\qquad$ \\
%
\hline
%
$\ln(n)$    & $ 40$ & $3.0\cdot10^{-3}$ & $0$\\
$nu_{i,\|}$ & $ 40$ & $3.0\cdot10^{-3}$ & $0$\\
$j_\|$      & $ 40$ & $3.0\cdot10^{-3}$ & $0$\\
$\Om^D$     & $0.1$ & $2.4\cdot10^{-4}$ & $5.81\cdot10^{-6}$\\
$\Om$       & $0.1$ & $2.4\cdot10^{-4}$ & $5.81\cdot10^{-6}$\\
%
\hline\hline
\end{tabular}
}}
\caption[]{\textit{
        Artificial viscosities used in the CELMA code.
        Artificial viscosities to $\Om^D$ is only added for the case where the Boussinesq approximation is \textbf{NOT} used.
        Artificial viscosities to $\Om$ is only added for the case where the Boussinesq approximation is used.
    }}
\protect\label{tb:artVisc}
\end{table}
%
Higher coefficients are used in the parallel direction as the background gradients in the parallel direction are much more gradual than the background gradients in the perpendicular gradients.
In the end, both $D_{f, \|, \text{art}} \partial_\|^2 f$ and $D_{f, \perp, \text{art}} \grad_\perp^2 f$ are small compared to the other terms.
% FIXME: Would be nice to showing a plot showing that the artificial terms are small
Notice that the artificial viscosities in $\Om^D$ and $\Om$ are smaller than for the other fields.
This is because the vorticity is usually quite filamented, and large artificial viscosities will not only remove the small scales, but also alter the physical behavior of the system.
