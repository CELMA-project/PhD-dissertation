\section{Spectral filtering}
\label{sec:filt}
%
In order to ensure that no aliasing will occur and thereby create a numerical instability like the one mentioned in \cite{Phillips1959}, we must use a spectral filter in the periodic direction.
Although sufficiently high viscosities or diffusion can prevent aliasing (such that all the higher modes are damped out), they typically also damp modes which we would like to include in our simulation.
One way to get around this is to use spectral filters.

\subsection{Orszag's 2/3 rule}
As mentioned in \cite{Orszag1971}, only the $2/3$ of the topmost modes leads to aliasing.
To see this, recall that only mode numbers equal to or less than $N/2$ can be represented exactly on a grid discretized with $N$ points \cite{Bracewell2000book}.
Next, consider two mode numbers $m_1$ and $m_2$ which adds up to a mode $m_3$.
If $m_3=m_1+m_2>N/2$, the mode will be interpreted as $m_1+m_2 - N$ (i.e. it will be aliased to the negative frequencies).
If we call $M$ the highest mode we can have which would not give aliasing, we must require that
%
\begin{align*}
    m_1+m_2 - N &< -M\\
    M + M - N &< -M\\
    2M - N &< -M\\
    3M &<  N\\
    M &< \frac{N}{3}.
\end{align*}
%
In order words, modes with mode number less than $N/3$ does not contribute to aliasing.
Therefore, the name "$2/3$-rule" as $M$ is $2/3$ of the Nyquist frequency $M < (2/3)(N/2)$.

Thus, aliasing in this case is prevented if we set all modes with a mode number equal or above $N/3$ to zero.
As we see in the next section, this does not completely eliminate the aliasing in a cylinder, due to radial coupling.

\subsection{Radial coupling}
%
As terms like $\{\phi, f\}$ effectively advects modes of $f$ radially, we must ask ourselves what the smallest allowed wave length in the periodic direction is.
Since the circumference is given by $C=2\pi\rho$, and since the number of points in the periodic $\theta$-direction is constant, we see that resolution is limited by the shortest allowed wave length at the outermost radius $L_\rho$.
Let us now consider sinusoids on the form $\sin(kx)$ living on the circumference $C$ at radius $\rho$.
The wave number is then given by $k=\frac{2\pi m}{C}$, which means that the wavelength is
%
\begin{align}
    \lambda = \frac{C}{m}.
    \label{eq:wavelen}
\end{align}
%
The smallest resolved wavelength on $\rho=L_\rho$, where $L_\rho$ is the outermost radius, and is given by the Nyquist frequency.
This gives the wavelength $\lambda_{\text{Nyquist}} = \frac{C_{L_\rho}}{n_\theta/2}$, where $n_\theta$ is the number of points in the $\theta$ direction.
However, the smallest wavelength which does not give aliasing is given by the $2/3$-rule, so that
%
\begin{align*}
    \lambda_\min = \frac{C_{L_\rho}}{\frac{2}{3}\frac{n_\theta}{2}} = \frac{3C_{L_\rho}}{n_\theta}.
\end{align*}
%
By rearranging \cref{eq:wavelen}, we find that the largest allowed mode number for the circumference $C$ at radius $\rho$ is
%
\begin{align*}
    m_\max = \L\lfloor \frac{C}{\lambda_\min}\R\rfloor,
\end{align*}
%
where $\lfloor \cdot \rfloor$ denotes the floor function.
Note that we take the floor as we are looking for the maximum allowed integer.
