The CELMA code is implemented using the BOUT++ framework \cite{Dudson2009,Dudson2014a,Dudson2016}.
Historically, BOUT++ builds on the BOUT code \cite{Xu1998} (which again is build on the UEDGE code \cite{Rognlien1996}).
However, BOUT and BOUT++ have diverged, and should not be mixed.
BOUT is a code designed to solve a specific set of equations, wheras BOUT++ is a framework containing tools specifically designed to solve plasma equations, and is not bound to a specific model.
In BOUT++ the user can her or himself specify what set of equations to be solved, and choose what built in implemetation to do so.
The BOUT++ framework is open source, and available at \href{https://github.com/boutproject/BOUT-dev}{\texttt{https://github.com/boutproject/BOUT-dev}}.
For this thesis BOUT++ version $3.0$ with checksum number \texttt{11e8f23624b90cbbc67f797ac73eccb9e855c9c4} has been used%
%
\footnote{Note though that in the time of writing version $4.0.0$ is under developement.}%
%
.
