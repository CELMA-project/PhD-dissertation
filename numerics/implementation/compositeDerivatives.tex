\section{Composite derivatives}
\label{sec:compDeriv}
%
In \cref{eq:celma_dens,eq:celma_mom_dens,eq:celma_j_par,eq:celma_vortD_evolution} and \cref{eq:celma_vort_boussinesq} there are some terms which are written as a combination of two or more FDs.
These are
%
\begin{align*}
    &\{\phi,\cdot\},&
    &\{\ve{u}_E^2,n\},&
    &\partial_\|\div \L( u_{i,\|}n \frac{\grad_\perp \phi}{B}\R)&
    &\text{and}&
    &\div \L( S_n \L[ \frac{ \grad_\perp \phi }{ B } \R] \R)&
\end{align*}
%
These can either be calculated by applying two (or more) different FDs consecutively to a field $f$, or by making a new FD stencil specifically for the operator under consideration.

Special care must be taken at the ghost points if one choose to apply two (or more) different FDs consecutively.
To see this, we can call $g$ the result of calculating the FD of a field $f$.
As we are using centred stencils (as described in
% FIXME: Add reference to the FD stencils used
), then the ghost points of $g$ is not calculated by the FD%
%
\footnote{There is by definition no way to apply a centred stencil to the last grid point.}%
%
\footnote{One could in principle use a one-sided stencil on the ghost points $f$ when calculating the FD of $f$, so that the ghost point of $g$ would be known.
This is however not done here.}
%
.

\subsection{Arakawa's stencil}
%
As already noted in \cref{sec:vecAdvTerm}, we can write $\ve{E}\times\ve{B}$-advective terms (that is the $\{\phi,\cdot\}$ terms) as Poisson brackets.
The proof is found in \cref{app:poisson}.
The benefits of writing terms on Poisson brackets are presented in Arakawa's paper from 1966 \cite{Arakwa1966}.
In short, the paper shows that a na\"ive FD discretization of the Poisson bracket does not conserve energy and enstrophy.
At the same time it gives an alternative way of discretize in orthogonal curvilinear coordinates in order to keep these quantities conserved.
If the energy and enstrophy is not conserved, fake generation of these quantities occur, which eventually will lead to a blow up of the simulation (in a way described by Phillips in \cite{Phillips1959}).

\subsection{Advection by \texorpdfstring{$\ve{u}_E^2$}{the squared E cross B drift}}
\label{sec:ExBadv}
%
It is also possible to discretize the term $\{\ve{u}_E^2, n\}$ using Arakawa's method.
We observe that in cylindrical coordinates, we have
%
\begin{align*}
    \{\ve{u}_E^2, n\} &= \L\{\L(\frac{\grad_\perp \phi}{B}\R)^2, n\R\}
    \note{Constant $B$}
    \\
    %
    &= \L(\frac{1}{B}\R)^2
    \L\{\L(\L[\ve{e}^{\rho}\partial_{\rho} + \ve{e}^{\theta}\partial_{\theta}\R] \phi\R)
        \cdot
        \L(\L[\ve{e}^{\rho}\partial_{\rho} + \ve{e}^{\theta}\partial_{\theta}\R] \phi\R)
        , n\R\}
    \note{Orthogonality}
    \\
    %
    &= \L(\frac{1}{B}\R)^2
    \L\{g^{\rho\rho}\L(\partial_{\rho} \phi\R)^2+
        g^{\theta\theta}\L(\partial_{\theta} \phi\R)^2
        , n\R\}
    \\
    %
    &= \L(\frac{1}{B}\R)^2
    \L\{\L(\partial_{\rho} \phi\R)^2+ \frac{1}{\rho^2}\L(\partial_{\theta} \phi\R)^2
        , n\R\}
\end{align*}
%
Here, we must take care when we treat the ghost points.
No ghost points is needed in the $\theta$ direction, as this direction is periodic.
Thus, for $\partial_{\theta} \phi$, we only need to make sure that we take the $\theta$ derivatives at the ghost points in $\rho$.

For $\partial_{\rho} \phi$, we must re-apply the values in the $\rho$ ghost points as the derivative is not calculated there.
For the inner ghost point, the same procedure as used in \cref{sec:ghostRhoDeriv} can be used.
For the outer ghost point, we can use \cref{eq:extraPolUp} for extrapolation to the ghost point.

This way of discretizing is second order accurate, as indicated in MES
%FIXME: Refer to table or so in MES


\subsection{\texorpdfstring{$\div(g\grad_\perp f)$}{Divergence of g times the perpendicular gradient of f} terms}
%
We have that
%
\begin{align*}
    \div(f\nabla_\perp g)
    =& f\grad_\perp^2g + \grad f\cdot       \grad_\perp g
    \\
    =& f\grad_\perp^2g + \grad_\perp f\cdot \grad_\perp g
    \note{See \cref{sec:perpLapl}}
    \\
    =&
    f\L(g^{ij} \partial_i \partial_j + G^j \partial_j -\frac{1}{J} \partial_z \L(\frac{J}{g_{zz}} \partial_2\R)\R)g
    +
    \L(e^\rho \partial_\rho  + e^\theta \partial_\theta \R)f\cdot \L(e^\rho \partial_\rho  + e^\theta \partial_\theta \R)g
    \\
    =&
    f\L(g^{ij} \partial_i \partial_j + G^j \partial_j -\frac{1}{J} \partial_z \L(\frac{J}{g_{zz}} \partial_2\R)\R)g
    +
    g^{\rho\rho} \partial_\rho f \partial_\rho g
    +
    g^{\rho\theta} \partial_\rho f  \partial_\theta g
    +
    g^{\theta\rho} \partial_\theta f  \partial_\rho g
    +
    g^{\theta\theta} \partial_\theta f \partial_\theta g
    \numberthis
    \label{eq:perpPerpDeriv}
\end{align*}
%
As BOUT++ includes a numerical opertor for $\grad_\perp^2g$ (as mentioned in \cref{sec:perpLapl}), we could have used this operator for calculating $f\grad_\perp^2g$.
However, from \cref{app:coord}, we that
%
\begin{align*}
  G^\rho =& \frac{1}{J}\\
  G^z =& 0\\
  G^\theta =& 0\\
  g^{ij} =& 0 \qquad i\neq j\\
  g^{\rho\rho} =& 1\\
  g^{\theta\theta} =& \frac{1}{\rho^2}\\
  g^{zz}\partial_z^2 - \frac{1}{J}\partial_z\left(\frac{J}{g^{zz}}\partial_z\right) =& 0,
\end{align*}
%
in cylindrical coordinates.
Thus \cref{eq:perpPerpDeriv} can be rewritten to
%
\begin{align*}
    &f\L(g^{\rho\rho} \partial_\rho \partial_\rho + g^{\theta\theta} \partial_\theta \partial_\theta + G^\rho \partial_\rho \R)g
    + g^{\rho\rho} \partial_\rho f \partial_\rho g
    + g^{\theta\theta} \partial_\theta f \partial_\theta g
    \\
    =&
    f \partial^2_\rho g
    + f \frac{1}{\rho^2}\partial^2_\theta g
    + f\frac{1}{\rho}\partial_\rho g
    +  \partial_\rho f \partial_\rho g
    +  \frac{1}{\rho^2} \partial_\theta f \partial_\theta g
\end{align*}
%
%FIXME: Has this been MESed?
%
which is what is implemented for this operator in the CELMA code.

\subsection{Parallel derivative of the divergence of the cross term}
%
As $\partial_\|\div \L( u_{i,\|}n \frac{\grad_\perp \phi}{B}\R)$ can be rewritten to $\partial_\|\div(g\grad_\perp f)$, we just have to take care of the parallel ghost points of $\div \L( u_{i,\|}n \frac{\grad_\perp \phi}{B}\R)$ before taking the parallel derivate.
We can use \cref{eq:extraPolUp} for calculation of the upper ghost point, and \cref{eq:extraPolDown}  for calculation of the lower ghost point.
