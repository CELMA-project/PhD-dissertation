Up until know we have derived the drift-fluid equation from the kinetic
equation using the drift fluid approximation. We would now insert the fluid
drifts in the fluid equation without making assumption of the topology of the
$\ve{B}$ field. We will, on the other hand make the assumption that the plasma
itself consists of only one type of atomic element which is fully ionized.
%

\section{The curvature operator}
%
To simplify further calculations, we will introduce the curvature operator.
Assuming we are working in a field aligned Clebsch coordinate system (see
appendix \ref{app:Clebsch}), where
$\ve{b}$ is parallel to one of the basis vectors, we have that
%
\begin{align*}
\grad_\perp f \times \ve{b}= - \ve{b} \times \grad_\perp f
= - \ve{b} \times (\grad - \grad_\|) f
= - \ve{b} \times (\grad - \ve{b}\ve{b}\cdot\grad) f = -\grad f \times \ve{b}
\end{align*}
%
as a vector crossed with a parallel vector yields
$0$.  Therefore, we can write
%
\begin{align*}
- \div \L( \frac{\grad_\perp f \times \ve{b}}{B} \R)
 &= - \div \L( \frac{\grad f \times \ve{b}}{B} \R)
 %
 %
 \\
 &= \div \L(\frac{\ve{b}}{B}  \times \grad f \R)
 %
 \note{\hspace{-3cm}$\div(\ve{a}\times\ve{b}) =
       \ve{b} \cdot (\curl \ve{a}) - \ve{a} \cdot (\curl \ve{b}) $}
 \\
 %
 %
 &= \grad f \cdot \L(\curl \frac{\ve{b}}{B}\R) -
    \frac{\ve{b}}{B} \cdot (\curl  [\grad f])
 %
 \note{\hspace{-2.5cm}$\curl \grad f = 0$}\\
 %
 %
 &= \grad f \cdot \L(\curl \frac{\ve{b}}{B}\R)
 %
 \note{\hspace{-2.5cm}$\curl(f\ve{a}) = f(\curl\ve{a}) - \ve{a}\times(\grad
       f)$}\\
 %
 %
 &= \grad f \cdot
    \L(
    \frac{1}{B}\L[\curl\ve{b}\R]- \ve{b}\times\L[\grad\frac{1}{B}\R]
    \R)
 \\
 %
 %
 &= \frac{1}{B}\grad f\cdot\L[\curl\ve{b}\R] -
    \L(\ve{b}\times\L[\grad\frac{1}{B}\R]\R) \cdot \grad f\\
 %
 &= \frac{1}{B}\grad f\cdot\L[\curl\ve{b}\R] +
    \L(\L[\grad\frac{1}{B}\R] \times \ve{b}\R) \cdot \grad f
 %
 \note{$\ve{a}\cdot(\ve{b}\times\ve{c}) =
        \ve{b}\cdot(\ve{c}\times\ve{a}) =
        \ve{c}\cdot(\ve{a}\times\ve{b})$}\\
 %
 %
 &= \frac{1}{B}\grad f\cdot\L[\curl\ve{b}\R]  +
    \L[\grad\frac{1}{B}\R] \cdot \L( \ve{b} \times \grad f\R)\\
 %
 &\defined \mathcal{C}(f)
 \label{eq:curv_op}
 \numberthis
\end{align*}
%
which is non-zero only if the $\ve{B}$ field curves.

\section{The electron continuity equation}
From equation (\ref{fluideq:cont}), and the drift ordering presented in
chapter \ref{chap:drift-order}, we have that
%
\begin{align*}
    \partial_t n_\a + \div (n_\a \ve{u}_\a) &= S_{n,\a}
 \note{\hspace{-1cm}Quasi-neutrality}
 \\
 %
 \partial_t n + \div (n [
 \ve{u}_{\a,d} + \ve{u}_E + \ve{u}_{\a,p} + \ve{u}_{\a,R}
 + \ve{u}_{\a,\text{Ped}}
 + \ve{u}_{\a,\nu}
 + \ve{u}_{\a,S} + \ve{u}_{\a,\|}
 ]) &= S_{n, \a}
 \numberthis
 \label{eq:cont_eq}
\end{align*}
%
Based on our assumptions, the evolution of the density can be
described by both the electron continuity equation and the ion continuity
equation due to quasi-neutrality. The two should only differ slightly. To show
this, we can do an order of magnitude estimate comparison of $n_e$ and $n_i -
n_e$. We find that
%
\begin{align*}
    \frac{n_i - n_e}{n_e} =&
    \frac{e(n_i - n_e)}{en_e}
    =
    \frac{\frac{\div\ve{E}}{\e_0}}{en_e}
    =
    \frac{\frac{\div\ve{E}}{\e_0}}{en_e}
    \to
    \frac{\L|\frac{kE}{\e_0}\R|}{\L|en_e\R|}
    =
    \frac{\L|\frac{B}{\om\e_0}\R|}{\L|en_e\R|}
    =
    \L|\frac{eB}{m_e\om\e_0}\R|\L|\frac{m_e}{e^2n_e}\R|
    =
    \L|\frac{\om_{ce}}{\om\om_{pe}}\R|
\end{align*}
%
Where we have used the order of magnitude estimate $\curl\ve{E} = -\partial_t B
\to |E| \simeq |B/\om k|$, where $k$ is a characteristic length scale, and
$\om$ is a characteristic time scale. Since we are interested in phenomena
slower than the ion cyclotron frequency, we set $\om\to\om_{ci}$, and get
%
\begin{align*}
    \frac{\L|n_i - n_e\R|}{\L|n_e\R|}
    \simeq
    \L|\frac{\frac{m_i}{m_e}}{\om_{pe}}\R|
\end{align*}
%
% NOTE: Consider adding numbers
which for all practical purposes is a quantity much smaller than $1$.

We choose to use the electron continuity equation to calculate the
evolution of the density.  The reason for this, as will soon be demonstrated,
is that certain terms can be neglected due to the mass ratio between electrons
and ions. This gives
%
\begin{align}
 \partial_t n + \div (n [
 \ve{u}_{\a,d} + \ve{u}_E + \ve{u}_{\a,p} + \ve{u}_{\a,R}
 + \ve{u}_{\a,\text{Ped}}
 + \ve{u}_{\a,\nu}
 + \ve{u}_{\a,S} + \ve{u}_{\a,\|}
 ]) &= S_{n, \a}
 \label{eq:el_cont}
\end{align}
%
We will now inspect the drifts in equation (\ref{eq:el_cont}) term by term.

\subsection{First order perpendicular terms}
By using the curvature operator from equation (\ref{eq:el_cont}), we find that
the $\ve{E}\times\ve{B}$ term gives
%
\begin{align*}
    \div \L(n \ve{u}_{E}\R)
    &=
    n \div \ve{u}_{E}
    + \ve{u}_{E} \cdot \grad n
    \\
    &=
    n \div \frac{-\grad_\perp \phi \times \ve{b}}{B}
    + \ve{u}_{E} \cdot \grad n\\
    &=
    n \mathcal{C}(\phi)
    + \ve{u}_{E} \cdot \grad n
    \numberthis
    \label{eq:div_ExB}
\end{align*}
%
for the diamagnetic term in (\ref{eq:el_cont}) we find that
%
\begin{align*}
 \div\L( n \ve{u}_{e,d} \R) &=
 -\div\L( n
   \frac{\grad_\perp p_e \times\ve{b}}{q_e n  B}
  \R)\\
  %
  &=
 \frac{1}{e}\div\L(
   \frac{\grad_\perp p_e \times\ve{b}}{B}
  \R)\\
  &=
  -\frac{1}{e}\mathcal{C}(p_e)
  %
 \numberthis
 \label{eq:div_n_ued}
\end{align*}
%
in other words, we find that $n \div \ve{u}_{\a,d}$ cancels $\ve{u}_{\a,d}
\cdot \grad n$ in the absence of magnetic field inhomogeneities. This cancellation
is referred to as \emph{diamagnetic cancellation} in the literature.

\subsection{Electron polarization}
\label{sec:no_e_pol}
%
As a good approximation, one can neglect the electron polarization drift
$\ve{u}_{e,p}$. As we are interested in low frequency dynamics
$\om_{\text{char}}\ll\om_{\text{ci}}$, we can write $\ve{u}_{e,p}$ as
%
\begin{align*}
    \ve{u}_{e,p}
    &=
      \frac{1}{\om_{ce}}\d^1_{e,t}\L(
      - \frac{\grad_\perp p_e}{n_e  q_e B}
      - \frac{\grad_\perp \phi}{B}
      \R)\\
%
    &=
      \frac{m_e}{q_eB}\d^1_{e,t}\L(
      - \frac{\grad_\perp p_e}{n_e  q_e B}
      - \frac{\grad_\perp \phi}{B}
      \R)\\
%
    &=
      \frac{m_i}{q_eB}\frac{m_e}{m_i}\d^1_{e,t}\L(
      - \frac{\grad_\perp p_e}{n_e  q_e B}
      - \frac{\grad_\perp \phi}{B}
      \R)\\
%
    &=
    \frac{1}{\om_{ci}}\frac{m_e}{m_i}\d^1_{e,t}\L(
      - \frac{\grad_\perp p_e}{n_e  q_e B}
      - \frac{\grad_\perp \phi}{B}
      \R)
\end{align*}
%
As this term is first order, and since $\frac{m_e}{m_i} \ll 1$, this term will
be much smaller than for example $\ve{u}_E$ (which in addition is of order $\e$
larger), and can therefore be neglected in equation (\ref{eq:el_cont}).

\subsection{Electron-ion resistivity}
For the electron-ion resistivity term in equation (\ref{eq:el_cont}), we find
that
%
\begin{align*}
 \div\L( n \ve{u}_{e,R} \R) &=
 \div\L( n \frac{\ve{R}_{i \to e,\perp}\times \ve{b}}{nq_e B} \R)
 \\
  %
  &=
  -\frac{1}{e}
  \div\L( \frac{\ve{R}_{i \to e,\perp}\times \ve{b}}{B} \R)
 \numberthis
 \label{eq:div_nur}
  %
\end{align*}
%
From equation (2.6) in \cite{Braginskii1965}, we have that
%
\begin{align*}
    \ve{R}_{i \to e}
    =
    \ve{R}_{u} + \ve{R}_{T}
\end{align*}
%
where
%
\begin{align*}
    \ve{R}_{u}
    =
    -\frac{m_en_e}{\tau_e}\L(0.51\L[\ve{u}_{e,\|}-\ve{u}_{i,\|}\R] +
    \L[\ve{u}_{i,\perp}-\ve{u}_{e,\perp}\R]\R)
\end{align*}
%
and
%
\begin{align*}
    \ve{R}_{T}
    =
    -0.71n_e\grad T_e -
    \frac{3}{2}\frac{n_e}{\om_e\tau_e}\ve{b}\times\grad T_e
\end{align*}
%
where $\tau_e$ in the inverse of the electron-ion collision frequency $\nu_{ei}$
assuming constant $T_e$, we get that
%
\begin{align}
    \ve{R}_{i \to e}
    = \ve{R}_{u}
   &= -0.51m_en_e\nu_{ei}\L(\L[\ve{u}_{e,\|}-\ve{u}_{i,\|}\R] +
      \L[\ve{u}_{e,\perp}-\ve{u}_{i,\perp}\R]\R)
   \label{eq:res_term_full}
\end{align}
%
As $\ve{u}_{e,R}$ is already a first order drift, we substitute only the
zeroth order drifts into the perpendicular velocities in equation
(\ref{eq:res_term_full}). This yields
%
\begin{align*}
    \ve{R}_{i \to e}
    =&
    -0.51m_en_e\nu_{ei}
   \L(\L[\ve{u}_{e,\|}-\ve{u}_{i,\|}\R] +
      \L[
         \L(
           \ve{u}_{e,d} + \ve{u}_E
          \R)
          -
         \L(
          \ve{u}_{i,d} + \ve{u}_E
         \R)
      \R]
   \R)
   \\
%
%
   =&
   -0.51m_en_e\nu_{ei}
   \L(\L[\ve{u}_{e,\|}-\ve{u}_{i,\|}\R] +
      \L[
         \L(
            -
            \frac{\grad_\perp p_e \times \ve{b}}{q_inB}
            -
            \frac{\grad_\perp \phi \times \ve{b}}{B}
          \R)
          -
         \L(
            -
            \frac{\grad_\perp p_i \times \ve{b}}{q_enB}
            -
          \frac{\grad_\perp \phi \times \ve{b}}{B}
         \R)
      \R]
   \R)
   \\
%
%
   =&
   -0.51m_en_e\nu_{ei}
   \L(\L[\ve{u}_{e,\|}-\ve{u}_{i,\|}\R] +
      \L[
            -
            \frac{\grad_\perp \L(p_e + p_i\R) \times \ve{b}}{enB}
      \R]
   \R)
\end{align*}
%
Inserting this into equation (\ref{eq:div_nur}) yields
%
\begin{align*}
    \div(n\ve{u}_{e,R})
  =&
  -\frac{1}{e}
 \div\L(
 -\frac{0.51m_en_e\nu_{ei}}{B}
   \L[\L(\ve{u}_{e,\|}-\ve{u}_{i,\|}\R) +
      \L( - \frac{\grad_\perp \L(p_e + p_i\R) \times \ve{b}}{enB} \R)
   \R]
   \times\ve{b}
 \R)
 \note{$\ve{a}_{\|}\cdot\ve{b} = 0$}
 \\
 %
  =&
  -\frac{1}{e}
 \div\L(
 \frac{0.51m_en_e\nu_{ei}}{B}
   \ve{b}\times
 \L[- \frac{\grad_\perp \L(p_e + p_i\R) \times\ve{b} }{enB} \R]
 \R)
 \\
 %
  =&
  -\frac{1}{e}
 \div\L(
   -\frac{0.51m_en_e\nu_{ei}}{B}
    \ve{b}\times
   \L[ \ve{b} \times \frac{\grad_\perp \L(p_e + p_i\R)}{enB}
   \R]
 \R)
 \note{$-\ve{b}\times\ve{b}\times\ve{a}=\ve{a}_\perp$}
 \\
 %
  =&
  -\frac{0.51m_e\nu_{ei}}{e^2}
 \div\L(
 \frac{n_e}{B}
 \frac{\grad_\perp \L(p_e + p_i\R)}{nB}
 \R)
 \note{Quasi-neutrality}
 \\
 %
  =&
  -\frac{0.51m_e\nu_{ei}}{e^2}
 \div\L( \frac{\grad_\perp \L(p_e + p_i\R)}{B^2} \R)
 \\
 %
  =&
  -\frac{m_i}{m_i}\frac{0.51m_e\nu_{ei}}{e^2}
 \div\L( \frac{\grad_\perp \L(p_e + p_i\R)}{B^2} \R)
 \\
 %
  =&
  -\frac{1}{\mu}\frac{0.51m_i\nu_{ei}}{e^2}
 \div\L( \frac{\grad_\perp \L(p_e + p_i\R)}{B^2} \R)
   % \numberthis
   % \label{eq:div_nur_calc}
\end{align*}
%
We see that we could have neglected this term as well on the basis of the mass
ratio. However, we choose to keep this term as we see that it will contribute
to perpendicular dissipation of $n$ through the divergence of the gradient of
the pressures. This dissipation can help removing accumulation of small wave
numbers which results from discretization of the grid.

\subsection{The Pedersen drift}
%
There will also be a drift associated with collisions between charged particles
and neutrals. This drift is often referred to as the Pedersen drift, due to the
related quantity Pedersen conductivity (see for example
\cite{Baumjohann1997book}). To address the Pedersen drift, we start by assuming
that
%
\begin{align*}
    \ve{R}_{n\to\a} =
    -
     m_\a n \nu_{\a n}\L(
        \L[\ve{u}_{\a,\|}-\ve{u}_{n,\|} \R]
        +
        \L[\ve{u}_{\a,\perp}-\ve{u}_{n,\perp} \R]
        \R)
\end{align*}
%
As we here want to model the neutrals as a static background, we get that
%
\begin{align*}
    \ve{R}_{n\to\a} =
    -
     m_\a n \nu_{\a n}\L(
        \ve{u}_{\a,\|}
        +
        \ve{u}_{\a,\perp}
        \R)
\end{align*}
%
Again, since $\ve{u}_{\a,R}$ is already a first order drift, we substitute only the
zeroth order drifts into the perpendicular velocities. We get that
%
\begin{align}
    \ve{R}_{n\to\a} =
    - m_\a n \nu_{\a n}\L( \ve{u}_{\a,\|}
    - \frac{\grad_\perp p_\a \times \ve{b}}{q_\a nB}
    - \frac{\grad_\perp \phi \times \ve{b}}{B} \R)
    \label{eq:neut_res}
\end{align}
%
Inserting equation (\ref{eq:neut_res}) into the Pedersen drift of equation
(\ref{eq:first_order}), assuming quasi-neutrality yields
%
\begin{align*}
    \ve{u}_{\a,\text{Ped}}
    =&
    - \frac{ m_\a n \nu_{\a n} }{n q_\a B} \L( \ve{u}_{\a,\|}
    - \frac{\grad_\perp p_\a \times \ve{b}}{q_\a nB}
    - \frac{\grad_\perp \phi \times \ve{b}}{B} \R) \times\ve{b}
    \note{Definition of perp. vectors}
    \\
    =&
    - \frac{\nu_{\a n} }{\om_{c\a}}
    \L(\frac{\grad_\perp p_\a}{q_\a nB}
    + \frac{\grad_\perp \phi}{B} \R)
\end{align*}
%
For the electron Pedersen drift, we find that
%
\begin{align*}
    \ve{u}_{e,\text{Ped}}
    =&
    \frac{m_i}{m_i}\frac{ m_e \nu_{e n} }{e B}
    \L(\frac{\grad_\perp p_e}{q_e nB}
    + \frac{\grad_\perp \phi}{B} \R)
    \\
    =&
    \frac{m_e}{m_i}\frac{\nu_{e n} }{\om_{ci}}
    \L(\frac{\grad_\perp p_e}{q_e nB}
    + \frac{\grad_\perp \phi}{B} \R)
\end{align*}
%
We see that this term can be neglected using the same arguments as in section
\ref{sec:no_e_pol}.

\subsection{Electron viscosity}
%
% FIXME: Should maybe explain a bit better?
Typically we have that the magnitude of electron-electron collisions
frequency $\nu_{ee}$ are around $\sqrt{\frac{m_e}{m_i}}$ as frequent as
$\nu_{ii}$ collisions. Thus, we have that $\te{\pi}_e \simeq
\sqrt{\frac{m_e}{m_i}}\te{\pi}_i$. If we substitute this into equation
(\ref{eq:el_cont}), we see that the $\sqrt{\frac{m_e}{m_i}}$ factor in
$\ve{u}_{e, \nu}$ makes this term much smaller than for example $\ve{u}_E$
(which in addition is of order $\e$ larger).

\subsection{Electron source drift}
\label{sec:no_e_source}
%
By using the same argument as given in section \ref{sec:no_e_pol}, we see that
this term can also be neglected as we from the source drift of equation
(\ref{eq:first_order}) have
%
\begin{align*}
    \ve{u}_{e,S}
    =&
  - \frac{ S_{e,n}}{n_e \om_{ce}}
  \L( \frac{\grad_\perp p_e}{n_e q_e B}
  + \frac{\grad_\perp \phi}{B} \R)
  \\
%
  =&
  - \frac{m_i}{m_i}
  \frac{ S_{e,n} m_e}{n_e e B} \L(
   \frac{\grad_\perp p_e}{n_e e B}
  + \frac{\grad_\perp \phi}{B} \R)
  \\
%
  =&
  - \frac{m_e}{m_i} \frac{ S_{e,n}}{n_e \om_{ci}} \L(
  \frac{\grad_\perp p_e}{n_e e B}
  + \frac{\grad_\perp \phi}{B}
  \R)
\end{align*}

\subsection{Resulting electron continuity equation}
Using what we found above, we can write equation
(\ref{eq:cont_eq}) for electrons as
%
\begin{align*}
    \partial_t n + \div\L(n\ve{u}_e\R)
    &= S_{e,n}
    \\
%
    \partial_t n
    &=
    - \div\L(n\L[ \ve{u}_E
    + \ve{u}_{e,D}
    + \ve{u}_{e,p}
    + \ve{u}_{e,R}
    + \ve{u}_{e,\text{Ped}}
    + \ve{u}_{e,S}
    + \ve{u}_{e,\|} \R]\R)
    + S_{e,n}
    \\
%
    &\simeq
    - \div\L(n\L[\ve{u}_E + \ve{u}_{e,D} + \ve{u}_{e,R} + \ve{u}_{e,\|}\R]\R)
    + S_{e,n}
    \\
%
    &=
    - \ve{u}_E\nabla\cdot n
    - n\div\ve{u}_E
    - \div\L(n \ve{u}_{e,D}\R)
    - \div\L(n \ve{u}_{e,R}\R)
    - \div\L(n \ve{u}_{e,\|}\R)
    + S_{e,n}
\end{align*}
%
If we now introduce the notation $\d_t = \partial_t + \ve{u}_E\cdot\nabla$, we
find that
%
\begin{align}
    \d_t n
    &=
    - n\mathcal{C}(\phi)
    + \frac{1}{e}\mathcal{C}(\phi)
    +\frac{1}{\mu}\frac{0.51m_i\nu_{ei}}{e^2}
    \div\L( \frac{\grad_\perp \L(p_e + p_i\R)}{B^2} \R)
    - \div\L(n \ve{u}_{e,\|}\R)
    + S_{e,n}
    \label{eq:dens_evol_gen}
\end{align}

\section{Current conservation and vorticity}
%
If we multiply the two continuity equations of equation (\ref{eq:cont_eq}) with
$q_\a$, and add them, we get by applying quasi-neutrality that
%
\begin{align*}
    q_i\partial_t n + q_i\div (n \ve{u}_i)
    + q_e\partial_t n + q_e\div (n \ve{u}_e)
    =&
    q_iS_{i,n} + q_eS_{e,n}
    \\
    %
    e\partial_t n + e\div (n \ve{u}_i)
    - e\partial_t n - e\div (n \ve{u}_e)
    =&
    eS_{i,n} - eS_{e,n}
    \\
    %
    \div (en \ve{u}_i - en \ve{u}_e) =&
    e\L(S_{i,n} - S_{e,n}\R)
\end{align*}
%
We will now use that the source produces equal amount of positive and
negative charges, which means that $S_{n,e}=S_{n,i}=S_n$. If we in addition
write $\ve{J} = \sum_{\a}q_\a n_\a\ve{u}_\a$, we get that
%
\begin{align*}
    \\
    %
    \div \ve{J} =&
    e\L(S_{n} - S_{n}\R)
    \\
    %
    \div \L( \ve{J}_\perp + \ve{J}_\|\R)=& 0
    \\
    %
    \div \ve{J}_\perp =& -\div \ve{J}_\|
\end{align*}
%
This gives
%
\begin{align*}
    \div \L( n [\ve{u}_{i,\perp} - \ve{u}_{e,\perp}] \R) =&
    -\div \L(n [\ve{u}_{i,\|} - \ve{u}_{e,\|}]\R)
\end{align*}
%
As explained in section \ref{sec:no_e_pol}, the electron polarization drift,
the electron drift and the electron source drift can be neglected.  However, we
retain the ion terms up to order of $\e$, which due to the mass ratio is larger
than the electron terms of order $\e$. This gives
%
\begin{align}
    \div \L( n [
   \ve{u}_{E} - \ve{u}_{E}
  +\ve{u}_{i,d} - \ve{u}_{e,d}
  +\ve{u}_{i,p}
  +\ve{u}_{i,R} - \ve{u}_{e,R}
  +\ve{u}_{i,\text{Ped}}
  +\ve{u}_{i,\nu}
  +\ve{u}_{i,S}
  ] \R) =&
  -\div \L(n [\ve{u}_{i,\|} - \ve{u}_{e,\|}]\R)
  \label{eq:current_drifts}
\end{align}
%
We will now inspect equation (\ref{eq:current_drifts}) term by term.


\subsection{First order perpendicular terms}
The $\ve{E}\times\ve{B}$ drift for the two species cancels as it is equal for
both species.  The diamagnetic terms yields
%
\begin{align*}
 \div\L( n_e \ve{u}_{e,d} \R) - \div\L( n_i \ve{u}_{i,d} \R)
 &=
  \frac{1}{q_e}\mathcal{C}(p_e) - \frac{1}{q_i}\mathcal{C}(p_i)
 \\
 %
 &=
  -\frac{1}{e}\L[\mathcal{C}(p_e) + \mathcal{C}(p_i)\R]
  \note{$\mathcal{C}(f) + \mathcal{C}(g) = \mathcal{C}(f+g)$}
 \\
 %
 &=
  -\frac{1}{e}\L[\mathcal{C}(p_e+p_i)\R]
\end{align*}
%


\subsection{Polarization and viscosity}
% FIXME: Gyroviscous cancellation in parallel
As apparent from equation (\ref{eq:first_order}), there is a kind of material
derivative appearing in the polarization drift. This material derivate, and
more general, the other material derivatives appearing in the equations must be
treated with care. Although it appears that all the drifts can contribute to
the advective part of the material derivative, it is only the parallel velocity
and the $\ve{E}\times\ve{B}$-drift that can contribute to the advection. The
reason for this is not so trivial in the drift-fluid picture, but comes as a
consequence of what is being referred to as gyroviscous cancellation
\cite{Smolyakov1998}. The cancellation comes as a consequence of parts of the
viscosity tensor canceling the drifts. Note that this also applies for the
parallel momentum equation. This means that
% Remember that the u_e looking part in the polarization is not really u_e, but
% u_exb
%
% Previously tried:
% 1. Split n first (Volker)
% 2. Split parallel and put n into ddt
\begin{align*}
    &\div\L( n \ve{u}_{i,p} + n \ve{u}_{i,\nu}\R)
 \\
 \simeq&
 \div\L( n \frac{1}{\om_{ci}}
  \L[ \partial_t + (\ve{u}_{E} + \ve{u}_{i,\|})\cdot\nabla \R]
  \L[ - \frac{\grad_\perp \phi}{B} \R]
 \R)
 \\
 %
 =&
 - \div\L( \frac{n}{\om_{ci}}
  \L[ \d_t + \ve{u}_{i,\|}\cdot\nabla \R]
  \frac{\grad_\perp \phi}{B}
 \R)
 \numberthis
 \label{eq:start_of_boussinesq}
 \\
 %
 =&
 - \div\L( \frac{1}{\om_{ci}}\L[
 \L( \d_t + \ve{u}_{i,\|}\cdot\nabla \R)
 \L(\frac{\grad_\perp \phi}{B}n\R)
 -\frac{\grad_\perp \phi}{B}
 \L( \d_t + \ve{u}_{i,\|}\cdot\nabla \R)
 n
 \R]
 \R)
 \\
 %
 =&
 - \div\L( \frac{1}{\om_{ci}}
 \L[ \d_t + \ve{u}_{i,\|}\cdot\nabla \R]
 \L[\frac{\grad_\perp \phi}{B}n\R]
 \R)
 +
 \div\L( \frac{1}{\om_{ci}}
 \frac{\grad_\perp \phi}{B}
 \L[ \d_t + \ve{u}_{i,\|}\cdot\nabla \R]
 n
 \R)
\numberthis
\label{eq:from_gyroviscous_n_not_subs}
\end{align*}
%
We see that we can insert the ion continuity equation in the last term of
equation (\ref{eq:from_gyroviscous_n_not_subs}). We note that this term is of
order $\e$, which means that when inserted the order $\e$ terms from the ion
continuity equation (that is $\ve{u}_{i,p}$, $\ve{u}_{i,R}$,
$\ve{u}_{i,\text{Ped}}$, $\ve{u}_{i,\nu}$, $\ve{u}_{i,S}$) will be of order
$\e^2$, and will therefore be neglected. The ion continuity equation to order
$\e^0$ reads
%
\begin{align*}
 \partial_t n + \div (n [ \ve{u}_{i,d} + \ve{u}_E + \ve{u}_{i,\|} ])
 &= S_{n}
 \\
 \partial_t n + \div (n \ve{u}_{i,d}) + \div (n\ve{u}_E) + \div (n\ve{u}_{i,\|} )
 &= S_{n}
 \note{Eq (\ref{eq:div_ExB}) and (\ref{eq:div_n_ued}) for ions}
 \\
 \partial_t n
 + \frac{1}{e}\mathcal{C}(p_i)
 + n \mathcal{C}(\phi)
 + \ve{u}_E \cdot \grad n
 + n \div \ve{u}_{i,\|}
 + \ve{u}_{i,\|} \cdot \grad n
 &= S_{n}
 \\
 \partial_t n
 + \ve{u}_E \cdot \grad n
 + \ve{u}_{i,\|} \cdot \grad n
 &=
 S_{n}
 - \frac{1}{e}\mathcal{C}(p_i)
 - n \mathcal{C}(\phi)
 - n \div \ve{u}_{i,\|}
 \\
 \L(\d_t + \ve{u}_{i,\|} \cdot \grad\R) n
 &=
 S_{n}
 - \frac{1}{e}\mathcal{C}(p_i)
 - n \mathcal{C}(\phi)
 - n \div \ve{u}_{i,\|}
 \numberthis
 \label{eq:i_cont}
\end{align*}
%
Inserting equation (\ref{eq:i_cont}) into equation
(\ref{eq:from_gyroviscous_n_not_subs}) yields
%
\begin{align*}
    &\div\L( n \ve{u}_{i,p} + n \ve{u}_{i,\nu}\R)
 \\
 \simeq&
 - \div\L( \frac{1}{\om_{ci}}
 \L[ \d_t + \ve{u}_{i,\|}\cdot\nabla \R]
 \L[\frac{\grad_\perp \phi}{B}n\R]
 \R)
 +
 \div\L( \frac{1}{\om_{ci}}
 \frac{\grad_\perp \phi}{B}
 \L[
 S_{n}
 - \frac{1}{e}\mathcal{C}(p_i)
 - n \mathcal{C}(\phi)
 - n \div \ve{u}_{i,\|}
 \R]
 \R)
\numberthis
\label{eq:from_gyroviscous}
\end{align*}


\subsection{The resistivity term}
The resistivity drifts for the two species cancels as
%
\begin{align*}
 n\ve{u}_{i,R} - n\ve{u}_{e,R} &=
 n
   \frac{\ve{R}_{e \to i,\perp}\times \ve{b}}{nq_i B}
 -
 n
   \frac{\ve{R}_{i \to e,\perp}\times \ve{b}}{nq_e B}
 \\
  %
  &=
 \frac{1}{e}\frac{\ve{R}_{e \to i,\perp}\times \ve{b}}{B}
 +
 \frac{1}{e}\frac{\ve{R}_{i \to e,\perp}\times \ve{b}}{B}
 \note{\hspace{-2cm} $\ve{R}_{e \to i,\perp} = - \ve{R}_{i \to e,\perp}$}
 \\
  %
 &=
 \frac{1}{e}\frac{\ve{R}_{e \to i,\perp}\times \ve{b}}{B}
 -
 \frac{1}{e}\frac{\ve{R}_{e \to i,\perp}\times \ve{b}}{B}
 \\
  %
 &= 0
\end{align*}
%



\subsection{The neutral term}
The contribution from the Pedersen drift yields
%
\begin{align*}
    \div\L(n\ve{u}_{i,\text{Ped}} \R)
    =&
    -
    \div\L(\frac{n\nu_{in}}{\om_{ci}}
        \L[
            \frac{\grad_\perp p_i}{enB}
            +
            \frac{\grad_\perp \phi}{B}
        \R]
        \R)\\
    %
    =&
    -
    n
    \div\L(\frac{\nu_{in}}{\om_{ci}}
        \L[
            \frac{\grad_\perp p_i}{enB}
            +
            \frac{\grad_\perp \phi}{B}
        \R]
        \R)
    -
    \L(\frac{\nu_{in}}{\om_{ci}}
        \L[
            \frac{\grad_\perp p_i}{enB}
            +
            \frac{\grad_\perp \phi}{B}
        \R]
        \R)
        \cdot\grad
        n
    \numberthis
    \label{eq:div_ped}
\end{align*}
%



\subsection{The source term}
Finally we find that the divergence of the source drift multiplied with density
gives
%
\begin{align*}
    \div\L(n\ve{u}_{\a,S}\R)
    =&
    -
    \div
    \L(
      \frac{ S_{\a,n}}{\om_{c\a}}
      \L[
        \frac{\grad_\perp p_\a}{n_\a q_\a B}
        + \frac{\grad_\perp \phi}{B}
      \R]
    \R)
\end{align*}
%


\subsection{Summing terms}
By summing the different contribution, equation (\ref{eq:current_drifts}) can be
now written as
%
\begin{align*}
  -\div \L(n [\ve{u}_{i,\|} - \ve{u}_{e,\|}]\R)
  =&
    \div \L( n [
   \ve{u}_{i,d} - \ve{u}_{e,d}
   +\ve{u}_{i,\text{Ped}}
  +\ve{u}_{i,p}
  +\ve{u}_{i,\nu}
  +\ve{u}_{i,S}
  ] \R)
  \\
%
%
=&
  -\frac{1}{e}\L[\mathcal{C}(p_e+p_i)\R]
  \\
  &
  - n \div\L(\frac{\nu_{in}}{\om_{ci}}
        \L[ \frac{\grad_\perp p_i}{enB} + \frac{\grad_\perp \phi}{B} \R] \R)
  - \L(\frac{\nu_{in}}{\om_{ci}}
        \L[ \frac{\grad_\perp p_i}{enB} + \frac{\grad_\perp \phi}{B} \R] \R)
        \cdot\grad n
  \\
  %
  &
 - \div\L( \frac{1}{\om_{ci}}
 \L[ \d_t + \ve{u}_{i,\|}\cdot\nabla \R]
 \L[\frac{\grad_\perp \phi}{B}n\R] \R)
 \\&
 +
 \div\L( \frac{1}{\om_{ci}}
 \frac{\grad_\perp \phi}{B}
 \L[S_{n} - \frac{1}{e}\mathcal{C}(p_i) - n \mathcal{C}(\phi)
 - n \div \ve{u}_{i,\|} \R] \R)
 \\
%
 &
    - \div \L( \frac{ S_{i,n}}{\om_{ci}}
      \L[ \frac{\grad_\perp p_i}{n_i q_i B} + \frac{\grad_\perp \phi}{B} \R]
    \R)
  \\
%
%
\div \L(n [\ve{u}_{i,\|} - \ve{u}_{e,\|}]\R)
=&
  \frac{1}{e}\L[\mathcal{C}(p_e+p_i)\R]
  \\
  &
  +  n \div\L(\frac{\nu_{in}}{\om_{ci}}
        \L[ \frac{\grad_\perp p_i}{enB} + \frac{\grad_\perp \phi}{B} \R] \R)
  + \L(\frac{\nu_{in}}{\om_{ci}}
        \L[ \frac{\grad_\perp p_i}{enB} + \frac{\grad_\perp \phi}{B} \R] \R)
        \cdot\grad n
  \\
  %
  &
 + \div\L( \frac{1}{\om_{ci}}
 \L[ \d_t + \ve{u}_{i,\|}\cdot\nabla \R]
 \L[\frac{\grad_\perp \phi}{B}n\R] \R)
 \\&
 -
 \div\L( \frac{1}{\om_{ci}}
 \frac{\grad_\perp \phi}{B}
 \L[S_{n} - \frac{1}{e}\mathcal{C}(p_i) - n \mathcal{C}(\phi)
 - n \div \ve{u}_{i,\|} \R] \R)
  \\
%
  &
    + \div \L( \frac{ S_{i,n}}{\om_{ci}}
      \L[ \frac{\grad_\perp p_i}{n_i q_i B} + \frac{\grad_\perp \phi}{B} \R]
    \R)
  \numberthis
  \label{eq:full_vort_eq}
\end{align*}
%
