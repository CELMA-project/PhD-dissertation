By using the definition of the Fourier transformed, we have
%
\begin{align*}
    F(x,y,\xi) = \defi{-\infty}{\infty}{f(x,y,z)\exp\L(-2\pi iz\xi\R)}{z}.
\end{align*}
%
This gives
%
\begin{align*}
    &\defi{-\infty}{\infty}{\L(\partial_zf[x,y,z]\R)\exp\L(-2\pi iz\xi\R)}{z}\\
%
    =& \defi{-\infty}{\infty}{\partial_z\L(f[x,y,z]\exp\L[-2\pi iz\xi\R]\R)}{z}
    - \defi{-\infty}{\infty}{f(x,y,z)\partial_z\exp\L(-2\pi iz\xi\R)}{z}\\
%
    =& \L(f[x,y,z]\exp\L[-2\pi iz\xi\R]\R)\bigg|_{-\infty}^{\infty} - \L(-2\pi
    i\xi\R)\defi{-\infty}{\infty}{f(x,y,z)\exp\L(-2\pi iz\xi\R)}{z}\\
%
=& 2\pi i\xi F(x,y,\xi),
    \numberthis
\label{eq:f_derivative}
%
\end{align*}
%
where we have used that $f(x,y,\pm\infty)=0$ in order to have a well defined
Fourier transform.
This means that
%
\begin{align*}
    \partial_z^n F(x,y,\xi) = (2\pi i \xi)^n F(x,y,\xi).
\end{align*}
%
In our case, we are dealing with periodic boundary conditions.
Strictly speaking, the Fourier transform does not exist in such cases, but it is possible to define a Fourier transform in the limit which in the end leads to the Fourier series (see \cite{Bracewell2000book} for details).

By discretizing the spatial domain, it is no longer possible to represent the infinite amount of Fourier modes, but only $N+1$ number of modes, where $N$ is the number of points (this includes the modes with negative frequencies, and the zeroth offset mode).
For the discrete Fourier transform, we have
%
\begin{align}
    F(x,y)_{k} = \frac{1}{N}\sum_{Z=0}^{N-1}f(x,y)_{Z}\exp\L(\frac{-2\pi i k
        Z}{N}\R),
%
\label{eq:DFT}
%
\end{align}
%
where $k$ is the mode number, $N$ is the number of points in $z$.
If we call the sampling points of $z$ for $z_Z$, where $Z = 0, 1 \ldots N-1$, we have that $z_Z = Z \text{d}z$.
As our domain goes from $[0, 2\pi[$, we have that (since we have one less line segments than points) $\text{d}z (N-1) = L_z = 2\pi - \text{d}z$, which gives $\text{d}z = \frac{2\pi}{N}$.
Inserting this is equation (\ref{eq:DFT}) yields
%
\begin{align*}
    F(x,y)_{k} = \frac{1}{N}\sum_{Z=0}^{N-1}f(x,y)_{Z}\exp\L( - i k
    Z\text{d}z\R) = \frac{1}{N}\sum_{Z=0}^{N-1}f(x,y)_{Z}\exp\L( - i k z_Z\R).
\end{align*}
%
The discrete version of equation (\ref{eq:f_derivative}) thus gives
%
\begin{align*}
    \partial_z^n F(x,y)_k = (i k)^n F(x,y)_k.
\end{align*}
%
