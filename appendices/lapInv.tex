We would here like to explain how $\Om = \frac{\grad_\perp^2\phi}{B}$ can be sovled numerically using the fourier transform.
This is a special case of the equation
%
\begin{align}
    d&\nabla_\perp^2f + \frac{1}{c}(\nabla_\perp c)\cdot\nabla_\perp f + af = b,
\label{eq:to_invert_tri}
\end{align}
%
which the BOUT++ framework has an own class of inverting for.

In order to explain the numerical implementation, we must first look at how the Laplacian operator looks like Clebsch coordinates.
This section is also included in the user manual and coordinates manual of the BOUT++ version mentioned in
% FIXME: Add either BOUT++ checksum, or refer to a place where it is

\section{The Laplacian}
%
The Laplacian operator is defined
%
\begin{align*}
    \grad^2f \defined \div \grad f
\end{align*}
%
In general we have (using equation (2.6.39) in D'Haeseleer \cite{Dhaeseleer1991book})
%
\begin{align}
    \div \ve{A} = \frac{1}{J} \partial_i \L(JA^i\R)
    \label{eq:divA}
\end{align}
%
and that
%
\begin{align*}
    A^i = \ve{A}\cdot \ve{e}^i
\end{align*}
%
In our case $A \to \grad$, so that
%
\begin{align*}
    \grad^i = \L(\grad\R)\cdot \ve{e}^i = \ve{e}^i \cdot \L(\grad\R) = \ve{e}^i
    \cdot \L(\ve{e}^j \partial_j\R) = g^{ij} \partial_j
\end{align*}
%
Thus
%
\begin{align*}
    \grad^2 =& \frac{1}{J} \partial_i \L(J g^{ij} \partial_j\R)\\ =&
    \frac{1}{J} g^{ij} J \partial_i \partial_j + \frac{1}{J} \partial_i \L(J
    g^{ij} \R) \partial_j\\ =& g^{ij} \partial_i \partial_j + G^j \partial_j\\
\end{align*}
%
where we have defined
%
\footnote{Notice that $G^i$ is \textbf{not} the same as the \emph{Christoffel symbols of second kind} (also known as the \emph{connection coefficients} or $\Gamma^i_{jk}=\ve{e}^i\cdot\partial_k \ve{e}_j$), although the derivation of the two are quite similar.}
%
\begin{align*}
    G^j =& \frac{1}{J} \partial_i \L(J g^{ij} \R)\\ =& \frac{1}{J} \L(
    \partial_1 \L[J g^{1j} \R] + \partial_2 \L[J g^{2j} \R] + \partial_3 \L[J
    g^{3j} \R] \R)
\end{align*}
%
By writing the terms out, we get
%
\begin{align*}
    \grad^2 =& g^{ij} \partial_i \partial_j + G^j \partial_j\\
%
            =& \L(  g^{1j} \partial_1 \partial_j + g^{2j} \partial_2 \partial_j
    + g^{3j} \partial_3 \partial_j\R) + \L(G^j \partial_j\R)\\
%
            =& \quad \, \L(  g^{11} \partial_1^2 + g^{21} \partial_2 \partial_1
    + g^{31} \partial_3 \partial_1\R) + \L(G^1 \partial_1\R)\\ &+ \L(  g^{12}
    \partial_1 \partial_2 + g^{22} \partial_2^2 + g^{32} \partial_3
    \partial_2\R) + \L(G^2 \partial_2\R)\\ &+ \L(  g^{13} \partial_1 \partial_3
    + g^{23} \partial_2 \partial_3 + g^{33} \partial_3^2\R) + \L(G^3
    \partial_3\R)
\end{align*}
%
We now use that the metric tensor is symmetric (by definition), so that $g^{ij}=g^{ji}$, and $g_{ij}=g_{ji}$, and that the partial derivatives commutes for smooth functions $\partial_i\partial_j=\partial_j\partial_i$.
This gives
%
\begin{align*}
    \grad^2 =&\quad \, \L(g^{11} \partial_1^2 \R) + \L(G^1 \partial_1\R)\\ &+
    \L(g^{22} \partial_2^2 \R) + \L(G^2 \partial_2\R)\\ &+ \L(g^{33}
    \partial_3^2\R) + \L(G^3 \partial_3\R)\\ &+ 2\L( g^{12} \partial_1
    \partial_2 + g^{13} \partial_1 \partial_3 + g^{23} \partial_2 \partial_3
    \R)\\
%
           =&\quad \, \L(g^{11} \partial_1^2\R) + \L( \frac{1}{J} \L[
\partial_1 \L\{J g^{11} \R\} + \partial_2 \L\{J g^{21} \R\} + \partial_3 \L\{J
g^{31} \R\} \R] \partial_1\R)\\ &+ \L(g^{22} \partial_2^2\R) + \L( \frac{1}{J}
    \L[ \partial_1 \L\{J g^{12} \R\} + \partial_2 \L\{J g^{22} \R\} +
    \partial_3 \L\{J g^{32} \R\} \R] \partial_2\R)\\ &+ \L(g^{33}
        \partial_3^2\R) + \L( \frac{1}{J} \L[ \partial_1 \L\{J g^{13} \R\} +
        \partial_2 \L\{J g^{23} \R\} + \partial_3 \L\{J g^{33} \R\} \R]
        \partial_3\R)\\ &+ 2\L( g^{12} \partial_1 \partial_2 + g^{13}
        \partial_1 \partial_3 + g^{23} \partial_2 \partial_3 \R)
\end{align*}
%
Notice that $G^i$ does not operate on $\partial_i$, but rather that the two are
multiplied together.


\section{The parallel Laplacian}
%
We have that
%
\begin{align*}
    \grad_\| =& \L(\ve{b} \cdot \grad\R) \ve{b}\ = \ve{b} \ve{b} \cdot \grad =
    \frac{\ve{e}_2 \ve{e}_2}{g_{22}} \cdot \grad = \frac{\ve{e}_2
    \ve{e}_2}{g_{22}} \cdot \ve{e}^i \partial_i = \frac{\ve{e}_2}{g_{22}}
    \partial_2
\end{align*}
%
and that
%
\begin{align*}
    \grad_\|^i =& \L(\frac{\ve{e}_2}{g_{22}} \partial_2\R)\cdot \ve{e}^i =
    \ve{e}^i \cdot \L(\frac{\ve{e}_2}{g_{22}} \partial_2\R)
\end{align*}
%
so that by equation (\ref{eq:divA}),
%
\begin{align*}
    \grad_\|^2 =& \div\L(\ve{b} \ve{b} \cdot \grad\R)\\ =&
    \div\L(\frac{\ve{e}_2}{g_{22}} \cdot \partial_2\R)\\ =& \frac{1}{J}
    \partial_i \L( J\ve{e}^i \cdot \L[\frac{\ve{e}_2}{g_{22}} \partial_2\R]
    \R)\\ =& \frac{1}{J} \partial_2 \L(\frac{J}{g_{22}} \partial_2\R)
\end{align*}


\section{The perpendicular Laplacian}
%
For the perpendicular Laplacian, we have that
%
\begin{align*}
    \grad_\perp^2 =& \grad^2 - \grad_\|^2\\ =& g^{ij} \partial_i \partial_j +
    G^j \partial_j -\frac{1}{J} \partial_2 \L(\frac{J}{g_{22}} \partial_2\R)\\
%
            =& \quad \, \L(g^{11} \partial_1^2\R) + \L( \frac{1}{J} \L[
\partial_1 \L\{J g^{11} \R\} + \partial_2 \L\{J g^{21} \R\} + \partial_3 \L\{J
g^{31} \R\} \R] \partial_1\R)\\ &+ \L(g^{22} \partial_2^2\R) + \L( \frac{1}{J}
    \L[ \partial_1 \L\{J g^{12} \R\} + \partial_2 \L\{J g^{22} \R\} +
    \partial_3 \L\{J g^{32} \R\} \R] \partial_2\R)\\ &+ \L(g^{33}
        \partial_3^2\R) + \L( \frac{1}{J} \L[ \partial_1 \L\{J g^{13} \R\} +
        \partial_2 \L\{J g^{23} \R\} + \partial_3 \L\{J g^{33} \R\} \R]
        \partial_3\R)\\ &+ 2\L( g^{12} \partial_1 \partial_2 + g^{13}
        \partial_1 \partial_3 + g^{23} \partial_2 \partial_3 \R)\\ &-
        \frac{1}{J} \partial_2 \L(\frac{J}{g_{22}} \partial_2\R)
\end{align*}
%


\subsection{The perpendicular Laplacian in Laplacian inversion}
%
Notice that BOUT++ currently assumes small parallel gradients in the dependent
variable in Laplacian inversion if $g_{12}$ and $g_{23}$ are non-zero (if these
are zero, the derivation can be done directly from equation
(\ref{eq:reduced_grad_perp}) instead), so that
%
\begin{align*}
    \grad_\perp^2 \simeq& \quad \, \L(g^{11} \partial_1^2\R) + \L( \frac{1}{J}
    \L[ \partial_1 \L\{J g^{11} \R\} + \partial_2 \L\{J g^{21} \R\} +
    \partial_3 \L\{J g^{31} \R\} \R] \partial_1\R)\\ &+ \L(g^{33}
        \partial_3^2\R) + \L( \frac{1}{J} \L[ \partial_1 \L\{J g^{13} \R\} +
        \partial_2 \L\{J g^{23} \R\} + \partial_3 \L\{J g^{33} \R\} \R]
        \partial_3\R)\\ &+ 2\L(g^{13} \partial_1 \partial_3\R)\\
%
           =& \L(g^{11} \partial_1^2\R) + G^1\partial_1 + \L(g^{33}
        \partial_3^2\R) + G^3 \partial_3 + 2\L(g^{13} \partial_1 \partial_3\R)
\end{align*}
%


\section{Numerical implementation}
\label{sec:num_laplace}
%
We will here go through the implementation of the laplacian inversion algorithm, as it is performed in BOUT++.
We would like to solve the following equation for $f$
%
\begin{align}
    d&\nabla_\perp^2f + \frac{1}{c_1}(\nabla_\perp c_2)\cdot\nabla_\perp f + af
    = b
%
\label{eq:to_invert}
%
\end{align}
%
BOUT++ is neglecting the $y$-parallel derivatives if $g_{12}$ and $g_{23}$ are no-zero when using the solvers \texttt{Laplacian} and \texttt{LaplaceXZ}.
For these two solvers, equation (\ref{eq:to_invert}) becomes (see \texttt{coordinates} manual for derivation)
%
\begin{align*}
    \, &d \L(    g^{11} \partial_1^2 + G^1 \partial_1 + g^{33} \partial_3^2 +
    G^3 \partial_3 + 2g^{13} \partial_1 \partial_3 \R) f \\
%
    +& \frac{1}{c_1}\L( \ve{e}^1 \partial_1 +  \ve{e}^3 \partial_3 \R) c_2
    \cdot \L( \ve{e}^1 \partial_1 +  \ve{e}^3 \partial_3 \R) f \\
%
    +& af = b \numberthis
%
\label{eq:invert_expanded}
%
\end{align*}
%


\subsection{Using tridiagonal solvers}
%
When using the tridiagonal solvers, $c_1 = c_2$ in equation (\ref{eq:to_invert}), hence, it is rather solving
%
\begin{align}
    d&\nabla_\perp^2f + \frac{1}{c}(\nabla_\perp c)\cdot\nabla_\perp f + af = b
%
\label{eq:to_invert_tri}
%
\end{align}
%
Since there are no parallel $y$-derivatives if $g_{12}=g_{23}=0$ (or if they are neglected), equation (\ref{eq:to_invert_tri}) will only contain derivatives of $x$ and $z$ for the dependent variable.
The hope is that the modes in the periodic $z$ direction will decouple, so that we in the end only have to invert for the $x$ coordinate.

If the modes decouples when Fourier transforming equation (\ref{eq:invert_expanded}), we can use a tridiagonal solver to solve the equation for each Fourier mode.

Using the discrete Fourier transform
%
\begin{align}
    F(x,y)_{k} = \frac{1}{N}\sum_{Z=0}^{N-1}f(x,y)_{Z}\exp\L(\frac{-2\pi i k
    Z}{N}\R)
\end{align}
%
we see that the modes will not decouple if a term consist of a product of two terms which depends on $z$, as this would give terms like
%
\begin{align*}
    \frac{1}{N}\sum_{Z=0}^{N-1} a(x,y)_3 f(x,y)_3 \exp\L(\frac{-2\pi i k
    Z}{N}\R)
\end{align*}
%
Thus, in order to use a tridiagonal solver, $a$, $c$ and $d$ cannot be functions of $z$.
Because of this, the $\ve{e}^3 \partial_3 c$ term in equation (\ref{eq:invert_expanded}) is zero.
In principle the modes would still decouple if the $\ve{e}^3 \partial_3 f$ part of equation (\ref{eq:invert_expanded}) was kept, but currently this part is also neglected in solvers using a tridiagonal matrix.
Thus the tridiagonal solvers are solving equations on the form
%
\begin{align*}
    \, &d(x,y) \L(    g^{11}(x,y) \partial_1^2 + G^1(x,y) \partial_1 +
    g^{33}(x,y) \partial_3^2 + G^3(x,y) \partial_3 + 2g^{13}(x,y) \partial_1
    \partial_3 \R) f(x,y,z) \\
%
    +& \frac{1}{c(x,y)}\L( \ve{e}^1 \partial_1 \R) c(x,y) \cdot \L( \ve{e}^1
    \partial_1 \R) f(x,y,z) \\
%
   +& a(x,y)f(x,y,z) = b(x,y,z)
\end{align*}
%
after using the discrete Fourier transform (see section \ref{sec:deriv_of_FT}), we get
%
\begin{align*}
    \, &d \L(    g^{11} \partial_1^2F_3 + G^1 \partial_1F_3 + g^{33} [i k]^2F_3
    + G^3 [i k]F_3 + 2g^{13} \partial_1[i k]F_3 \R) \\
%
    +& \frac{1}{c}\L( \ve{e}^1 \partial_1 \R) c \cdot \L( \ve{e}^1
    \partial_1F_3 \R) \\
%
    +& aF_3 = B_3
\end{align*}
%
which gives
%
\begin{align*}
    \, &d \L(    g^{11} \partial_1^2 + G^1 \partial_1 - k^2 g^{33} + i kG^3 + i
    k2g^{13} \partial_1 \R)F_3 \\
%
    +& \frac{g^{11}}{c} \L( \partial_1 c \R) \partial_1F_3 \\
%
    +& aF_3 = B_3 \numberthis
%
\label{eq:FT_laplace_inversion}
%
\end{align*}
%
As nothing in equation (\ref{eq:FT_laplace_inversion}) couples points in $y$ together (since we neglected the $y$-derivatives if $g_{12}$ and $g_{23}$ were non-zero).
Also, as the modes are decoupled, we may solve equation (\ref{eq:FT_laplace_inversion})  $k$ mode by $k$ mode in addition to $y$-plane by $y$-plane.

The second order centred approximation of the first and second derivatives in $x$ reads
%
\begin{align*}
    &&\partial_1 f \simeq \frac{-f_{n-1} + f_{n+1}}{2\text{d}x}&&
    &&\partial_1^2 f \simeq \frac{f_{n-1} - f_{n} + f_{n+1}}{\text{d}x^2}&&
\end{align*}
%
This gives
%
\begin{align*}
    \, &d \L(    g^{11} \frac{F_{z,n-1} - 2F_{z,n} + F_{z, n+1}}{\text{d}x^2} +
    G^1 \frac{-F_{z,n-1} + F_{z,n+1}}{2\text{d}x} - k^2 g^{33}F_{z,n} \R.\\
    &\quad\L.  + i kG^3F_{z,n} + i k2g^{13} \frac{-F_{z,n-1} +
F_{z,n+1}}{2\text{d}x} \R) \\
%
    +& \frac{g^{11}}{c} \L( \frac{-c_{n-1} + c_{n+1}}{2\text{d}x} \R)
\frac{-F_{z,n-1} + F_{z,n+1}}{2\text{d}x} \\
%
    +& aF_{z,n} = B_{z,n}
\end{align*}
%
collecting point by point
%
\begin{align*}
    &\L( \frac{dg^{11}}{\text{d}x^2} - \frac{dG^1}{2\text{d}x} -
    \frac{g^{11}}{c_{n}} \frac{-c_{n-1} + c_{n+1}}{4\text{d}x^2} - i\frac{d
    k2g^{13}}{2\text{d}x} \R) F_{z,n-1} \\
    %
    +&\L( - \frac{ dg^{11} }{\text{d}x^2} - dk^2 g^{33} + a + idkG^3 \R)
    F_{z,n} \\
    %
    +&\L( \frac{dg^{11}}{\text{d}x^2} + \frac{dG^1}{2\text{d}x} +
    \frac{g^{11}}{c_{n}} \frac{-c_{n-1} + c_{n+1}}{4\text{d}x^2} +
    i\frac{dk2g^{13}}{2\text{d}x} \R) F_{z, n+1} \\
%
     =& B_{z,n} \numberthis
%
\label{eq:discretized_laplace}
%
\end{align*}
%
We now introduce
%
\begin{align*}
    &c_1 = \frac{dg^{11}}{\text{d}x^2}& &c_2 = dg^{33}& &c_3 =
    \frac{2dg^{13}}{2\text{d}x}& && \\ &c_4 = \frac{dG^1 + g^{11}\frac{-c_{n-1}
    + c_{n+1}}{2c_n\text{d}x}}{2\text{d}x}& &c_5 = dG^3& &&
\end{align*}
%
which inserted in equation (\ref{eq:discretized_laplace}) gives
%
\begin{align*}
    &\L( c_1 - c_4 -ikc_3 \R) F_{z,n-1} \\
    %
    +&\L( -2c_1 - k^2c_2 +ikc_5 + a \R) F_{z,n} \\
    %
    +&\L( c_1 + c_4 + ikc_3 \R) F_{z, n+1} \\
%
     =& B_{z,n}
\end{align*}
%
This can be formulated as the matrix equation
%
\begin{align*}
    AF_3=B_3
\end{align*}
%
where the matrix $A$ is tridiagonal. The boundary conditions are set by setting
the first and last rows in $A$ and $B_3$.
