We will here address the electrostatic approximation.
Note that in all cases, we assume that the magnetic field perturbation from the plasma is negligible as compared with the background magnetic field.
The approximation states that $\partial_t \ve{B} \sim 0$, which through the Maxwell-Faraday equation states that $\curl\ve{E} \sim 0$.
In general, we have that the $\ve{E}$ field can be expressed through the potentials
%
\begin{align}
    \ve{E} = -\grad\phi - \partial_t \ve{A}
    \label{eq:potentialE}
\end{align}
%
(see for example \cite{Griffiths2013book,Fitzpatrick2008book}).
The magnetic potetial carries with it a degree of freedom, and we are free to use the Coloumb gauge without loss of generality.
That is
%
\begin{align*}
\div\ve{A}=0.
\end{align*}
%
We note that \cref{eq:potentialE} fulfills Gauss' law and Maxwell-Faraday's law of induction, and that $\partial_t \ve{A}=0$ implies an electrostatic condition.
With the Coloumb gauge, we have that Amp{\`e}re's circuit law reads
%
\begin{align*}
    \curl\curl\ve{A}=&\curl\ve{B}
    \note{Low frequency}
    \\
    \grad^2\ve{A} - \grad(\div \ve{A})=&\mu_0\ve{j}
    \note{Coloumb gauge}
    \\
    \frac{\grad^2\ve{A}}{\mu_0}=&\ve{j}
    \numberthis
    \label{eq:coloumbGauge}
\end{align*}
%
in the low frequent case.
Notice that $\grad^2\ve{A}=\ve{j}\mu_0$ is nothing but three Poisson equations, so that the solution can be written in terms of Green's functions (assuming the currents vanishes at infinity) as
%
\begin{align}
    \ve{A} (\ve{r} ,t) =& \frac{\mu_{0}}{4\pi} \int_{V} \frac{\ve{j}(\ve {r}',t')}{|\ve {r} -\ve {r} '|}\,\d ^{3}\ve {r}',\\
    \partial_t \ve{A} (\ve{r} ,t) =& \frac{\mu_{0}}{4\pi} \int_{V} \partial_t\L(\frac{\ve{j}(\ve {r}',t')}{|\ve {r} -\ve {r} '|}\R)\,\d ^{3}\ve {r}',
    \numberthis
    \label{eq:solA}
\end{align}
%
where $\ve{r}$ is the general position vector, $\ve{r'}$ is the position of the current distribution and $t'=t-\frac{|\ve{r}-\ve{r}'|}{c}$ is the retarded time.
We now see that $\partial_t \ve{A}$ is big whenever the time derivative of the current is big.

\section{Small time derivatives of the perpendicular magnetic potential}
%
In the drift ordering, we have stated assumed that the evolution of the perpendicular ion and electron velocity are of the same order of magnitude, and small compared with $\om_{ci}$.
Assuming also that the evolution of the density is also restricted by this time evolution%
%
\footnote{Note that if this was not the case, the fast evolution of $n$ would still couple to the evolution of $\ve{u}$ through the set of equations.}%
%
, the perpendicular current would also be restricted.
This would lead to a small $\partial_t \ve{A}_\perp$, which then could be neglected in the perpendicular set of equation as the ordering would imply that $-\grad_\perp\phi$ would be of a higher order of magnitude.

\section{Small time derivatives of the parallel magnetic potential}
%
Next, we will check if $\partial_t \ve{A}_\|$ is small by comparing it with the other terms of \cref{eq:parElNonNorm}%
%
\footnote{A similar approach is done in \cite{Schroder2003Phd}.}%
%
.
Writing out the $E_\|$-term in \cref{eq:parElNonNorm} gives
%
\begin{align*}
    \partial_t j_{\|}
 =&
 -\frac{1}{JB}\L\{\phi, j_{\|}\R\}
   \\&
 -e u_{i,\|}\partial_\|\L( n \L[u_{i,\|}+ u_{e,\|} \R]\R)
 + 2e u_{e,\|}\partial_\|\L( n  u_{e,\|} \R)
   \\&
 - 0.51 \nu_{ei}j_\|
   \\&
   +\frac{e}{m_e} T_e \partial_\| n
   + \frac{e^2}{m_e}n \L(-\partial_\| \phi - \partial_\| A_\|\R)
 + en \L(\nu_{e n}u_{e,\|} - \nu_{i n}u_{i,\|} \R)
 - \frac{4e\eta_{e,0}}{3m_e} \partial_z^2 u_{e,\|}
 \numberthis
 \label{eq:curWithAPerp}
\end{align*}
%
From \cref{eq:coloumbGauge}, we get that
%
\begin{align*}
    \frac{\grad^2\ve{A}}{\mu_0}=&\ve{j}
    \\
    \ve{b}\cdot\frac{\grad^2\ve{A}}{\mu_0}=&\ve{b}\cdot\ve{j}
    \note{Assuming $\partial_i\ve{b}=0$}
    \\
    \frac{\grad^2 A_\|}{\mu_0}=&j_\|
    \numberthis
    \label{eq:parCurAsA}
\end{align*}
%
Which means that the LHS of \cref{eq:curWithAPerp} can be written
%
\begin{align*}
    \partial_t j_\|=& \partial_t\frac{\grad^2 A_\|}{\mu_0}
\end{align*}
%
this means that we readily can compare $\partial_t A_\|$ with $\partial_t j_\|$ or $\nu_{ei} j_\|$ to check if the term is small, as these three terms can be expressed in terms of $A_\|$.
Thus, the $\partial_t A_\|$ term can be neglected if
%
\begin{align*}
    \partial_t \frac{\grad^2_\perp A_\|}{\mu_0}
    \gg &
    \frac{ne^2}{m_e}\partial_t A_\|
\end{align*}
%
which using order of magnitude estimates gives
%
\begin{align*}
    \frac{1}{\tau_{A_\|}} \frac{A_\|}{L_{A_\|}\mu_0}
    \gg &
    \frac{ne^2}{m_e}\frac{1}{\tau_{A_\|}} A_\|
    \\
    \frac{1}{L^2_{A_\|}\mu_0}
    \gg &
    \frac{ne^2}{m_e}
    \\
    \frac{1}{L^2_{A_\|}}
    \gg &
    \frac{2}{2} \frac{m_i^2}{m_i^2}\frac{B^2}{B^2}\frac{T_e}{T_e}
    \frac{e^2}{m_e}n\mu_0
    \\
    \frac{1}{L^2_{A_\|}}
    \gg &
    \frac{1}{2}\frac{m_i}{m_e}\frac{m_i}{T_e}
    \frac{e^2B^2}{m_i^2}
    \frac{2n\mu_0T_e}{B^2}
    \\
    \frac{1}{L^2_{A_\|}}
    \gg &
    \frac{1}{2}\mu
    \frac{\om^2_{ci}}{c_s^2}
    \b
    \\
    \frac{\rho_s^2}{L^2_{A_\|}}
    \gg &
    \frac{1}{2}\mu
    \b
    \numberthis
    \label{eq:firstNeglect}
\end{align*}
%
where the plasma beta $\b=\frac{nT_e}{\frac{B^2}{2\mu_0}}$ can be interpreted as the electron kinetic pressure over the magnetic pressure.
Also, the $\partial_t A_\|$ term can be neglected if
%
\begin{align*}
    0.51\nu_{ei}j_\|
    \gg &
    \frac{ne^2}{m_e}\partial_t A_\|
\end{align*}
%
By using orders of magnitude estimates and \cref{eq:parCurAsA}, we get
%
\begin{align*}
    0.51\nu_{ei}\frac{\grad^2 A_\|}{\mu_0}
    \gg &
    \frac{ne^2}{m_e}\partial_t A_\|
    \\
    0.51\nu_{ei}\frac{A_\|}{L_{A_\|}\mu_0}
    \gg &
    \frac{ne^2}{m_e}\frac{1}{\tau_{A_{\|}}} A_\|
    \\
    \frac{1}{L^2_{A_\|}}
    \gg &
    \frac{ne^2}{m_e}\frac{\mu_0}{0.51\nu_{ei}}\om_{A_{\|}}
    \note{As \cref{eq:firstNeglect}}
    \\
    \frac{\rho_s^2}{L^2_{A_\|}}
    \gg &
    \frac{1}{2}\frac{\om_{A_{\|}}}{0.51\nu_{ei}}\mu
    \b
\end{align*}
%
