We will in this section look at big and small terms in the perpendicular momentum equation in \cref{eq:perp_mom_start}.
The motivation for this is to make a drift ordering similar to what is done in \cite{Fitzpatrick2014book}, and from this get algebraic equations for each order of the perpendicular velocities.
We will do so by looking at characteristic scales of the system.
Before starting, we will have a brief look at the definition of the gradient length scales and of the quasi-neutrality of the system.

\section{Gradient length scale}
When doing order of magnitude estimates%
\footnote{
    Easy, approximate ways to estimate some numbers within the same orders of magnitudes as we would have reached by doing a more correct and rigorous study.
}%
, and will therefore introduce the \emph{gradient length scale}.
The gradient scale length serves as an estimate for the size of $\grad$.
That is, it tells us over how large distances there are sharp gradients for a bounded, smooth function.
For a field $f$, the gradient length scale $L_f$ is defined as%
\footnote{
    Note that the inverse gradient length scale is often denoted $k$ in the literature.
    This make sense for for example plane wave perturbations, which happens to have $\frac{1}{L_f}=k$, where $k$ is the inverse wave number.
    However, in order to avoid ambiguity, we will in this thesis use $L_f$ for the gradient scale length.
}%
%
\begin{align}
    \frac{1}{L_f} \defined \frac{\|\max\{\operatorname{abs}(\grad f)\}\|}{\operatorname{abs}\L(f\bigg|_{\|\max\{\operatorname{abs}(\grad f)\}\|}\R)}
    \label{eq:lenScale}
\end{align}
%
i.e. short and sharp gradients in $f$ have a short $L_f$.
We can similarly define the temporal scale as
%
\begin{align}
    \om_f = \frac{1}{\tau_f} \defined \frac{\max\{\operatorname{abs}(\partial_t f)\}}{\operatorname{abs}\L(f\bigg|_{\max\{\operatorname{abs}(\partial_t f)\}}\R)}
    \label{eq:timeScale}
\end{align}
%
The definitions in \cref{eq:lenScale,eq:timeScale} are quite strict, and we will in this thesis use a more approximate estimate for the gradient length scale.
Thus, when referring to gradient length scales in this thesis, we will mean "typical" values for the gradient scale lengths, so that
%
\begin{align*}
    \grad \sim \frac{1}{L},
\end{align*}
%
where $\sim$ denotes "of same order".

\section{Quasi-neutrality}
\label{sec:qn}
To get the condition of whether the system is quasi-neutral or not, we can do an order of magnitude estimate comparison of $Zn_i$ and $Zn_i - n_e$.
We find that
%
\begin{align}
    \frac{Zn_i - n_e}{Zn_i} =
    \frac{e(Zn_i - n_e)}{eZn_i}
    =
    \frac{\e_0\div\ve{E}}{eZn_i}
    \sim
    \frac{\e_0\L|E\R|}{\L|L_E\R|eZn_i}
    \label{eq:quasiNeutral}
\end{align}
%
We can find an approximate expression for $E$ through Faraday's induction law:
%
\begin{align*}
    \curl\ve{E} =& -\partial_t B\\
    \frac{|E|}{|L_E|} \sim& \frac{|B|}{\tau_B}\\
    |E| \sim& \frac{|L_E||B|}{\tau_B}
\end{align*}
%
Inserting this in \cref{eq:quasiNeutral} yields
%
\begin{align*}
    \frac{Zn_i - n_e}{Zn_i}
    \sim
    \frac{\e_0 |L_E||B|}{|L_E| e Zn_i \tau_B}\frac{m_i Ze}{m_i Ze}
    =
    \frac{\om_{ci}\e_0}{Z e n_i\tau_B}\frac{m_i}{Z e}
    =
    \frac{\om_{ci}}{\tau_B\om_{pi}^2},
\end{align*}
%
where $\om_{pi}$ denotes the ion plasma frequency%
\footnote{This can be interpreted as something like the typical frequency the ions would oscillate with if the ions where perturbed in a completely quiescent plasma.}
%
.
We will now assume that the fastest time scales which can occur in our system is much slower than the ion cyclotron frequency.
If we therefore set $1/\tau_B\to\om_{ci}$, we get
%
\begin{align}
    \frac{Zn_i - n_e}{Zn_i}
    \simeq
    \frac{\om_{ci}^2}{\om_{pi}^2}
    =
    \frac{\om_{ci}^2}{\om_{pi}^2}
    \label{eq:quasiNCompare}
\end{align}
%
which for our interest is a quantity much smaller than $1$.
Equivalently, if we introduce the \emph{normalizing} ion sound speed%
%
\footnote{Although the real ion sound speed is given by $c_s=\sqrt{\frac{T_e+\gamma' T_i}{m_i}}\;\gamma'=\frac{N+2}{N}$ for $N$ degrees of freedom, we will in this thesis use the symbol $c_s$ for $\sqrt{\frac{T_e}{m_i}}$, as this term frequently pops up in the derivations.}
%
\begin{align*}
    c_s = \sqrt{\frac{T_e}{m_i}},
\end  {align*}
%
and the ion hybrid radius (the ion gyro radius at the electron temperature)
%
\begin{align*}
   \rho_s=\frac{c_s}{\om_{ci}},
\end  {align*}
%
\cref{eq:quasiNCompare} can be stated as
%
\begin{align*}
    \frac{Zn_i - n_e}{Zn_i}
    \simeq
    \frac{\om_{ci}^2}{\om_{pi}^2}
    =
    \frac{c_s^2\om_{ci}^2}{c_s^2\om_{pi}^2}
    =
    \frac{c_s^2}{\rho_s^2\om_{pi}^2}
    =
    \frac{\frac{T_e}{m_i}}{\rho_s^2\frac{n_iZ^2e^2}{m_i\e_0}}
    =
    \frac{\frac{T_e\e_0}{n_iZ^2e^2}}{\rho_s^2}
    =
    \frac{\lambda_D^2}{\rho_s^2},
\end{align*}
%
where $\lambda_D$ is the Debye length, which tells at what radius an isolated charge is effectively electrically shielded by surrounding charged particles.

In other words, the quasi-neutrality is just a statement of what scales we are looking at.
Note that this does not imply that there cannot be large electrical field throughout the plasma, rather that the left hand side of
%
\begin{align*}
    \frac{\div \ve{E}}{Zn_i} = \frac{e}{\e_0} \frac{Zn_i-n_e}{Zn_i}
\end{align*}
%
is small (as $e$ and $\e_0$ is of the same order of magnitude).
We therefore have
%
\begin{align*}
    n\simeq Zn_i \simeq n_e
\end{align*}

\section{The inertia term}
\label{sec:doInert}
The left hand side of \cref{eq:perp_mom_start} reads
%
\begin{align*}
&
 \frac{1}{\om_{c\a}}
 \L(
 \partial_t \ve{u}_{\a,\perp}
 + \L[\ve{u}_{\a,\perp}
 + \ve{u}_{\a,\|}\R]\cdot\grad\ve{u}_{\a,\perp}
 \R)
\\
 %
 %
 %
 =&
 \frac{1}{\om_{c\a}}
 \L(
 \partial_t \ve{u}_{\a,\perp}
 + \L[\ve{u}_{\a,\perp}
 + \ve{u}_{\a,\|}\R]\cdot\grad\ve{u}_{\a,\perp}
 \R)
 \\
 %
 %
 %
 =&
 \frac{1}{\om_{c\a}}
 \L(
 \partial_t \ve{u}_{\a,\perp}
 + \ve{u}_{\a,\perp}\cdot\grad_\perp\ve{u}_{\a,\perp}
 + \ve{u}_{\a,\|}\cdot\grad_\|\ve{u}_{\a,\perp}
 \R)
 \numberthis
 \label{eq:do_inertia}
\end{align*}
%
From this, we can withdraw a characteristic timescale of change of the perpendicular velocity, a characteristic gradient length scale and a characteristic perpendicular velocity.
We will use the notation for a field $f$
%
\begin{align*}
    f = f^c \wt{f}
\end{align*}
%
where the superscript $f^c$ denotes the characteristic size of $f$ so that $\wt{f}$ is of $\mathcal{O}(1)$.
If we now apply this on \cref{eq:do_inertia}, we get
%
\begin{align*}
    &
 \frac{1}{\om_{c\a}}
 \L(
 \om^c_{\a,\perp}u^c_{\a,\perp}
 \partial_{\wt{t}} \wt{\ve{u}}_{\a,\perp}
 + \frac{u_{\a,\perp}^{c}u_{\a,\perp}^{c}}{L_{\perp, u_{\a,\perp}}}
 \wt{\ve{u}}_{\a,\perp}\cdot\wt{\grad}_\perp\wt{\ve{u}}_{\a,\perp}
 + \frac{u^c_{\a,\perp}u^c_{\a,\|}}{L_{\|, u_{\a,\perp}}}
 \wt{\ve{u}}_{\a,\|}\cdot\wt{\grad}_\|\wt{\ve{u}}_{\a,\perp}
 \R)
 \\
 %
 %
 %
 =&
 \frac{u^c_{\a,\perp}}{\om_{c\a}}
 \L(
 \om^c_{\a,\perp}
 \partial_{\wt{t}} \wt{\ve{u}}_{\a,\perp}
 + \frac{u^c_{\a,\perp}}{L_{\perp, u_{\a,\perp}}}
 \wt{\ve{u}}_{\a,\perp}\cdot\wt{\grad}_\perp\wt{\ve{u}}_{\a,\perp}
 + \frac{u^c_{\a,\|}}{L_{\|, u_{\a,\perp}}}
 \wt{\ve{u}}_{\a,\|}\cdot\wt{\grad}_\|\wt{\ve{u}}_{\a,\perp}
 \R)
\end{align*}
%
We now relate the velocities to $c_s$, so that
%
\begin{align*}
    \lambda \defined \frac{u^c}{c^c_s}.
\end{align*}
%
Further, by using the ion hybrid radius, we get
%
\begin{align*}
    &
 \lambda_{u_{\a,\perp}}
 \frac{c^c_s}{\om_{c\a}}
 \L(
 \om^c_{\a,\perp}
 \partial_{\wt{t}} \wt{\ve{u}}_{\a,\perp}
 +
 \lambda_{u_{\a,\perp}}
 \frac{ c^c_s }{L_{\perp, u_{\a,\perp}}}
 \wt{\ve{u}}_{\a,\perp}\cdot\wt{\grad}_\perp\wt{\ve{u}}_{\a,\perp}
 +
 \lambda_{u_{\a,\|}}
 \frac{L_{\perp, u_{\a,\perp}}}{L_{\perp, u_{\a,\perp}}}
 \frac{c^c_s}{L_{\|, u_{\a,\perp}}}
 \wt{\ve{u}}_{\a,\|}\cdot\wt{\grad}_\|\wt{\ve{u}}_{\a,\perp}
 \R)
 \\
 %
 %
 %
 =&
 \lambda_{u_{\a,\perp}}
 \frac{c^c_s}{\om_{c\a}}
 \L(
 \om^c_{\a,\perp}
 \partial_{\wt{t}} \wt{\ve{u}}_{\a,\perp}
 +
 \lambda_{u_{\a,\perp}}
 \frac{ \rho^c_s }{L_{\perp, u_{\a,\perp}}}
 \om^c_{ci}
 \wt{\ve{u}}_{\a,\perp}\cdot\wt{\grad}_\perp\wt{\ve{u}}_{\a,\perp}
 +
 \lambda_{u_{\a,\|}}
 \frac{L_{\perp, u_{\a,\perp}}}{L_{\|, u_{\a,\perp}}}
 \frac{\rho^c_s}{L_{\perp, u_{\a,\perp}}}
 \om^c_{ci}
 \wt{\ve{u}}_{\a,\|}\cdot\wt{\grad}_\|\wt{\ve{u}}_{\a,\perp}
 \R)
\end{align*}
%
we will now assume that the scales for the ions and electrons are the same.
This is, we are constraining the system in the following way
%
\begin{align*}
    \lambda              &\sim \lambda_{u_{e,\perp}}  \sim \lambda_{u_{i,\perp}} \\
    \om^c                &\sim \om^c_{e,\perp}        \sim \om^c_{i,\perp}       \\
    L_{\perp, u_{\perp}} &\sim L_{\perp, u_{e,\perp}} \sim L_{\perp, u_{i,\perp}}\\
    L_{\|, u_{\perp}}    &\sim L_{\|, u_{e,\perp}}    \sim L_{\|, u_{i,\perp}}
\end{align*}
%
If we at the same time introduce
%
\begin{align*}
    \frac{ \rho^c_s }{L_{\perp, u_{\perp}}} \defined \gamma
\end{align*}
%
we get
%
\begin{align}
 \lambda
 \frac{c^c_s}{\om_{c\a}}
 \L(
 \om^c
 \partial_{\wt{t}} \wt{\ve{u}}_{\a,\perp}
 +
 \lambda
 \gamma
 \om^c_{ci}
 \wt{\ve{u}}_{\a,\perp}\cdot\wt{\grad}_\perp\wt{\ve{u}}_{\a,\perp}
 +
 \lambda_{u_{\a,\|}}
 \frac{L_{\perp, u_{\a,\perp}}}{L_{\|, u_{\a,\perp}}}
 \gamma
 \om^c_{ci}
 \wt{\ve{u}}_{\a,\|}\cdot\wt{\grad}_\|\wt{\ve{u}}_{\a,\perp}
 \R)
 \label{eq:doLHSAfterFirstConstraint}
\end{align}
%
In order for these terms to  be of the same order, we must have that
%
\begin{align*}
 \om^c                    &\sim \lambda \gamma \om^c_{ci}  \sim \lambda_{u_{\a,\|}} \frac{L_{\perp}}{L_{\|, u_{\perp}}} \gamma \om^c_{ci}
 \\
 \frac{\om^c}{\om^c_{ci}} &\sim \lambda \gamma             \sim \lambda_{u_{\a,\|}} \frac{L_{\perp}}{L_{\|, u_{\perp}}} \gamma
 \numberthis
 \label{eq:firstOrdering}
\end{align*}
%
If we assume low frequent turbulence, we must have that
%
\begin{align*}
    \frac{\om^c}{\om^c_{ci}} \defined \e \ll 1
\end{align*}
%
Comparing the two first terms in \cref{eq:firstOrdering} gives
%
\begin{align*}
 \frac{\om^c}{\om^c_{ci}}
 \sim&
 \lambda
 \gamma
 \\
 \e
 \sim&
 \lambda
 \gamma
\end{align*}
%
in order to have a balanced restriction between gradient scale lengths and velocities, we can set
%
\begin{align*}
 \lambda
 \sim&
 \sqrt{\e}
 \\
 \gamma
 \sim&
 \sqrt{\e}
\end{align*}
%
note that if we would have started our constraint by saying $\sqrt{\e}\ll 1$ instead, the non-linear terms would be negligible.

Finally, we put the restriction on the parallel velocity and scale lengths.
We now define
%
\begin{align*}
\frac{L_{\perp}}{L_{\|, u_{\a,\perp}}}\defined \zeta
\end{align*}
%
By assuming $u^c_{\a,\|}\sim c^c_s$, and that
%
\begin{align*}
\zeta\sim \sqrt{\e},
\end{align*}
%
we find that
%
\begin{align*}
 \lambda_{u_{\a,\|}}
 \frac{L_{\perp}}{L_{\|, u_{\a,\perp}}}
 \gamma
 =
 \lambda_{u_{\a,\|}}
 \zeta
 \gamma
 \sim
 1\sqrt{\e}^2
 =
 \e
\end{align*}
%
This means that the left hand side of \cref{eq:perp_mom_start} can in an order of magnitude estimate be wirtten as (assuming that $\om_{c\a}\sim\om^c_{c\a}$
%
\begin{align*}
 &
 \lambda
 \frac{m_i}{m_i}
 \frac{c^c_s}{\om^c_{c\a}}
 \L(
 \om^c
 \partial_{\wt{t}} \wt{\ve{u}}_{\a,\perp}
 +
 \lambda
 \gamma
 \om^c_{ci}
 \wt{\ve{u}}_{\a,\perp}\cdot\wt{\grad}_\perp\wt{\ve{u}}_{\a,\perp}
 +
 \lambda_{u_{\a,\|}} \frac{L_{\perp, u_{\a,\perp}}}{L_{\|, u_{\a,\perp}}}
 \gamma
 \om^c_{ci}
 \wt{\ve{u}}_{\a,\|}\cdot\wt{\grad}_\|\wt{\ve{u}}_{\a,\perp}
 \R)
 \\
 %
 %
 &=
 \lambda
 \frac{m_\a}{m_i}
 \frac{c^c_s}{\om^c_{ci}}
 \L(
 \om^c
 \partial_{\wt{t}} \wt{\ve{u}}_{\a,\perp}
 +
 \e
 \om^c_{ci}
 \wt{\ve{u}}_{\a,\perp}\cdot\wt{\grad}_\perp\wt{\ve{u}}_{\a,\perp}
 +
 \e
 \om^c_{ci}
 \wt{\ve{u}}_{\a,\|}\cdot\wt{\grad}_\|\wt{\ve{u}}_{\a,\perp}
 \R)
 \\
 %
 %
 &=
 \lambda
 \frac{m_\a}{m_i}
 c^c_s
 \L(
 \frac{\om^c}{\om_{ci}}
 \partial_{\wt{t}} \wt{\ve{u}}_{\a,\perp}
 +
 \e
 \wt{\ve{u}}_{\a,\perp}\cdot\wt{\grad}_\perp\wt{\ve{u}}_{\a,\perp}
 +
 \e
 \wt{\ve{u}}_{\a,\|}\cdot\wt{\grad}_\|\wt{\ve{u}}_{\a,\perp}
 \R)
 \\
 %
 %
 &=
 \lambda
 c^c_s
 \frac{m_\a}{m_i}
 \e
 \L(
 \partial_{\wt{t}} \wt{\ve{u}}_{\a,\perp}
 +
 \wt{\ve{u}}_{\a,\perp}\cdot\wt{\grad}_\perp\wt{\ve{u}}_{\a,\perp}
 +
 \wt{\ve{u}}_{\a,\|}\cdot\wt{\grad}_\|\wt{\ve{u}}_{\a,\perp}
 \R)
 \\
 %
 %
 &=
 \lambda c^c_s \frac{m_\a}{m_i}
 \e
 \wt{\d}^\a_t\wt{\ve{u}}_{\a,\perp}
 \numberthis
 \label{eq:doInertiaEps}
\end{align*}
%

\section{Pressure, electric field and perpendicular velocities}
\label{sec:pep}
%
We will now group the three next terms in \cref{eq:perp_mom_start}, and we will in the end see that these have the same order under the right assumptions.
We have that
%
\begin{align*}
- \frac{ \grad_\perp p_\a }{n_\a  q_\a B}
+ \frac{\ve{E}_\perp}{B}
+ \ve{u}_{\a,\perp}\times\ve{b}
=&
- \frac{n^c T^c_\a}{L_{\perp, p_\a}n^cq_\a B^c}
\frac{ \wt{\grad}_\perp \wt{p}_\a }{\wt{n}_\a \wt{B}}
+ \frac{E_\perp^c}{B^c}
\frac{\wt{\ve{E}}_\perp}{\wt{B}}
+ \ve{u}^c_{\a,\perp}
\wt{\ve{u}}_{\a,\perp}\times\ve{b}
\\
%
%
=&
- \frac{T^c_\a}{L_{\perp, p_\a}q_\a B^c}
\frac{ \wt{\grad}_\perp \wt{p}_\a }{\wt{n}_\a \wt{B}}
+ \frac{E_\perp^c}{B^c}
\frac{\wt{\ve{E}}_\perp}{\wt{B}}
+\lambda c^c_s
\wt{\ve{u}}_{\a,\perp}\times\ve{b}
\numberthis
\label{eq:doTriple}
\end{align*}
%
We will now constrain the system further by saying that
%
\begin{align*}
L_{\perp} \sim& L_{\perp, p_\a} \sim L_{\perp, u_{\perp}}\\
T^c       \sim& T^c_e           \sim T^c_i
\end{align*}
%
This means that $c^c_s = \sqrt{\frac{T^c}{m_i}}$.
Using this in \cref{eq:doTriple} gives
%
\begin{align*}
-
\frac{m_i}{m_i}
\frac{T^c}{q_\a B^c}
\frac{1}{L_{\perp}}
\frac{ \wt{\grad}_\perp \wt{p}_\a }{\wt{n}_\a \wt{B}}
+ \frac{E_\perp^c}{B^c}
\frac{\wt{\ve{E}}_\perp}{\wt{B}}
+\lambda c^c_s
\wt{\ve{u}}_{\a,\perp}\times\ve{b}
%
%
=&
-
(c^c_s)^2
\frac{q_i}{q_\a}
\frac{1}{\om^c_{ci}}
\frac{1}{L_{\perp}}
\frac{ \wt{\grad}_\perp \wt{p}_\a }{\wt{n}_\a \wt{B}}
+ \frac{E_\perp^c}{B^c}
\frac{\wt{\ve{E}}_\perp}{\wt{B}}
+\lambda c^c_s
\wt{\ve{u}}_{\a,\perp}\times\ve{b}
\\
%
%
=&
-
c^c_s
\frac{q_i}{q_\a}
\frac{c^c_s}{\om^c_{ci}}
\frac{1}{L_{\perp}}
\frac{ \wt{\grad}_\perp \wt{p}_\a }{\wt{n}_\a \wt{B}}
+ \frac{E_\perp^c}{B^c}
\frac{\wt{\ve{E}}_\perp}{\wt{B}}
+\lambda c^c_s
\wt{\ve{u}}_{\a,\perp}\times\ve{b}
\\
%
%
=&
-
c^c_s
\frac{q_i}{q_\a}
\frac{\rho^c_s}{L_{\perp}}
\frac{ \wt{\grad}_\perp \wt{p}_\a }{\wt{n}_\a \wt{B}}
+ \frac{E_\perp^c}{B^c}
\frac{\wt{\ve{E}}_\perp}{\wt{B}}
+\lambda c^c_s
\wt{\ve{u}}_{\a,\perp}\times\ve{b}
\\
%
%
=&
-
c^c_s
\frac{q_i}{q_\a}
\gamma
\frac{ \wt{\grad}_\perp \wt{p}_\a }{\wt{n}_\a \wt{B}}
+ \frac{E_\perp^c}{B^c}
\frac{\wt{\ve{E}}_\perp}{\wt{B}}
+\lambda c^c_s
\wt{\ve{u}}_{\a,\perp}\times\ve{b}
\numberthis
\label{eq:doTripleToEps}
\end{align*}
%
For these to be of the same order, we must have
%
\begin{align*}
    \gamma c^c_s &\sim \frac{E_\perp^c}{B^c} \sim \lambda c^c_s \\
    \gamma     &\sim \frac{E_\perp^c}{c^c_sB^c} \sim \lambda
\end{align*}
%
this gives
%
\begin{align*}
    \Xi \defined \frac{E_\perp^c}{c^c_sB^c} \sim \sqrt{\e}
\end{align*}
%
Inserting this in \cref{eq:doTripleToEps} yields
%
\begin{align*}
-
c^c_s
\frac{q_i}{q_\a}
\gamma
\frac{ \wt{\grad}_\perp \wt{p}_\a }{\wt{n}_\a \wt{B}}
+ \frac{c^c_s}{c^c_s}\frac{E_\perp^c}{B^c}
\frac{\wt{\ve{E}}_\perp}{\wt{B}}
+\lambda c^c_s
\wt{\ve{u}}_{\a,\perp}\times\ve{b}
=&
-
c^c_s
\L(
\gamma
\frac{q_i}{q_\a}
\frac{ \wt{\grad}_\perp \wt{p}_\a }{\wt{n}_\a \wt{B}}
+ \frac{1}{c^c_s}\frac{E_\perp^c}{B^c}
\frac{\wt{\ve{E}}_\perp}{\wt{B}}
+\lambda
\wt{\ve{u}}_{\a,\perp}\times\ve{b}
\R)
\\
%
%
=&
-
c^c_s
\sqrt{\e}
\L(
\frac{q_i}{q_\a}
\frac{ \wt{\grad}_\perp \wt{p}_\a }{\wt{n}_\a \wt{B}}
+
\frac{\wt{\ve{E}}_\perp}{\wt{B}}
+
\wt{\ve{u}}_{\a,\perp}\times\ve{b}
\R)
\end{align*}

\section{Collisionalities and sources}
%
Next, we look at the collisionalities and sources of \cref{eq:perp_mom_start}.
We would like these to be (at maximum) of the same order of magnitude as the inertia terms of \cref{sec:doInert}.
Before we start, we will assume quasi-neutrality%
\footnote{This is discussed further in \cref{sec:qn}}%
%
, i.e.
%
\begin{align*}
    n\sim n_e \sim Zn_i
\end{align*}
%

\subsection{Coulomb collisions}
%
For the electron-ion collisionalities, we find
%
\begin{align*}
\frac{ \ve{R}_{\beta \to \a,\perp} }{n_\a q_\a B}
=&
\frac{ m_en_e\nu_{ei}\L(\ve{u}_{e,\perp}-\ve{u}_{i,\perp}\R) }{n_\a q_\a B}
\\
%
%
=&
\frac{m_e\nu^c_{ei}\lambda c^c_sn^c}{q_\a n^cB^c}
\frac{ \wt{n}_\a \wt{\nu}_{ei}\L(\wt{\ve{u}}_{e,\perp}-\wt{\ve{u}}_{i,\perp}\R) }{\wt{n}_\a \wt{B}}
\\
%
%
=&
\lambda c^c_s
\frac{m_iq_i}{m_iq_i}\frac{ m_e}{q_\a B^c} \nu^c_{ei}\frac{\wt{\ve{R}}_{\beta \to \a,\perp} }{\wt{n}_\a \wt{B}}
\\
%
%
=&
\lambda c^c_s
\frac{q_i}{q_\a}
\frac{m_e}{m_i}
\frac{\nu^c_{ei}}{\om^c_{ci}}
\frac{\wt{\ve{R}}_{\beta \to \a,\perp} }{\wt{n}_\a \wt{B}}
\\
%
%
=&
\lambda c^c_s
\frac{q_i}{q_\a}
\frac{1}{\mu}
\frac{\nu^c_{ei}}{\om^c_{ci}}
\frac{\wt{\ve{R}}_{\beta \to \a,\perp} }{\wt{n}_\a \wt{B}}
\numberthis
\label{eq:doRes}
\end{align*}
%
We will now set the ordering condition according to the ions.
If the ion equation of \cref{eq:doRes} is to be at the same order as the ion equation of \cref{eq:doInertiaEps} we get the condition
%
\begin{align*}
 \frac{1}{\mu}\frac{\nu^c_{ei}}{\om^c_{ci}}\lambda c^c_s \sim& \lambda c^c_s \e
 \\
 \xi\defined\frac{1}{\mu} \frac{\nu^c_{ei}}{\om^c_{ci}}           \sim& \e
\end{align*}
%
Where we have used that the masses in \cref{eq:doInertiaEps} cancels, and as all the terms with tilde are of order $\mathcal{O}(1)$
%
\cref{eq:doRes} can therefore be written as
%
\begin{align*}
\frac{ \ve{R}_{\beta \to \a,\perp} }{n_\a q_\a B}
\sim&
\lambda c^c_s
\frac{q_i}{q_\a}
\xi
\frac{\wt{\ve{R}}_{\beta \to \a,\perp} }{\wt{n}_\a \wt{B}}
\end{align*}

\subsection{Neutral collisions}
%
For the neutral collisions, we find that
%
\begin{align*}
\frac{ \ve{R}_{n \to \a,\perp} }{n_\a q_\a B}
=&
\frac{ m_\a n_\a \nu_{\a n} \ve{u}_{\a,\perp}}{n_\a q_\a B}
\note{Quasi-neutrality}
\\
=&
\frac{ m_\a n^c }{n^c q_\a B^c}\nu^c_{\a n} \lambda c^c_s
\frac{\wt{n}_\a \wt{\nu}_{\a n}\wt{\ve{u}}_{\a,\perp} }{\wt{n}_\a  \wt{B}}
\\
=&
\frac{m_iq_i}{m_iq_i}
\frac{ m_\a }{ q_\a B^c}\nu^c_{\a n} \lambda c^c_s
\frac{\wt{\ve{R}}_{n \to \a,\perp}  }{\wt{n}_\a  \wt{B}}
\\
=&
\frac{q_i}{q_\a}
\frac{m_\a}{m_i}
\frac{\nu^c_{\a n}}{\om^c_{ci}} \lambda c^c_s
\frac{\wt{\ve{R}}_{n \to \a,\perp} }{\wt{n}_\a  \wt{B}}
\end{align*}
%
We will now try to relate $\nu^c_{e n}$ to $\nu^c_{i n}$.
From \cref{app:collisions} we have that
%
\begin{align*}
    &\nu_{en} \propto \frac{n_n a_0^2 \sqrt{T_e}}{\sqrt{m_e}}&
    &\nu_{in} \propto \frac{n_n a_0^2 \sqrt{T_i}}{\sqrt{m_i}}&
\end{align*}
%
As we have that $T_e \sim T_i$, we get
%
\begin{align*}
    \nu_{en} \propto& \frac{\sqrt{m_i}}{\sqrt{m_i}}\frac{n_n a_0^2 \sqrt{T_e}}{\sqrt{m_e}}\\
    \propto& \frac{\sqrt{m_i}}{\sqrt{m_e}}\frac{n_n a_0^2 \sqrt{T_e}}{\sqrt{m_i}}\\
    \sim   & \frac{\sqrt{m_i}}{\sqrt{m_e}}\nu_{in}
\end{align*}
%
This gives
%
\begin{align*}
    \frac{q_i}{q_\a}
    \frac{m_\a}{m_i}
    \frac{\sqrt{m_i}}{\sqrt{m_i}}
    \frac{\nu^c_{\a n}}{\om^c_{ci}} \lambda c^c_s
    \frac{\wt{\ve{R}}_{n \to \a,\perp} }{\wt{n}_\a  \wt{B}}
    \sim
    \frac{q_i}{q_\a}
    \frac{m_\a}{m_i}
    \frac{\sqrt{m_i}}{\sqrt{m_\a}}
    \frac{\nu^c_{i n}}{\om^c_{ci}} \lambda c^c_s
    \frac{\wt{\ve{R}}_{n \to \a,\perp} }{\wt{n}_\a  \wt{B}}
    \\
    %
    =
    \frac{q_i}{q_\a}
    \frac{\sqrt{m_\a}}{\sqrt{m_i}}
    \frac{\nu^c_{i n}}{\om^c_{ci}} \lambda c^c_s
    \frac{\wt{\ve{R}}_{n \to \a,\perp} }{\wt{n}_\a  \wt{B}}
    \numberthis
    \label{eq:doNeut}
\end{align*}
%
Again, we can do an ordering condition according to the ions.
The ion equation of \cref{eq:doNeut} is of the same order as the ion equation of \cref{eq:doInertiaEps} when
%
\begin{align*}
    \lambda c^c_s \e \sim& \frac{\nu^c_{i n}}{\om^c_{ci}} \lambda c^c_s\\
    \e             \sim& \frac{\nu^c_{i n}}{\om^c_{ci}} \defined \Theta
\end{align*}
%
as the charges and masses of equation \cref{eq:doNeut} cancels for ions.
%
Thus, the order of magnitude estimate of the neutral collisions can be written
%
\begin{align*}
\frac{ \ve{R}_{n \to \a,\perp} }{n_\a q_\a B}
\sim
    \lambda c^c_s\frac{q_i}{q_\a} \frac{\sqrt{m_\a}}{\sqrt{m_i}} \Theta \frac{\wt{\ve{R}}_{n \to \a,\perp} }{\wt{n}_\a  \wt{B}}
\end{align*}

\subsection{Source terms}
%
By using \cref{eq:kinSource}, the source term can be written
%
\begin{align*}
\frac{ S_{\a,n}\ve{u}_{\a,\perp} }{n_\a \om_{c\a}}
=&
\frac{|q_\a|}{e}
\frac{m_i}{m_i}
\frac{ S_{i,n}\ve{u}_{\a,\perp} }{n_\a \om_{c\a}}
\\
%
%
=&
\frac{|q_\a|}{e}
\frac{m_i}{m_\a}
\frac{ S_{i,n}\ve{u}_{\a,\perp} }{n_\a \om^c_{ci}}
\note{Quasi-neutrality}
\\
%
%
=&
\frac{|q_\a|}{e}
\frac{m_i}{m_\a}
\frac{S^c_{i,n}c^c_s}{n^c \om^c_{ci}}
\frac{ \wt{S}_{i,n}\wt{\ve{u}}_{\a,\perp} }{\wt{n}_\a}
\\
%
%
=&
\frac{|q_\a|}{e}
\frac{m_i}{m_\a}
\frac{\nu^c_{S_{i,n}}c^c_s}{\om^c_{ci}}
\frac{ \wt{S}_{i,n}\wt{\ve{u}}_{\a,\perp} }{\wt{n}_\a}
\numberthis
\label{eq:doSource}
\end{align*}
%
The ion equation of \cref{eq:doSource} will be at the same order as the ion equation of \cref{eq:doInertiaEps} if
%
\begin{align*}
    \lambda c^c_s \e \sim& \frac{\nu^c_{S_{i,n}}c^c_s}{\om^c_{ci}}\\
    \e               \sim& \frac{1}{\lambda }\frac{\nu^c_{S_{i,n}}}{\om^c_{ci}} \defined \sigma
\end{align*}
%
Hence, we have that the source terms give
%
\begin{align*}
\frac{ S_{\a,n}\ve{u}_{\a,\perp} }{n_\a \om_{c\a}}
    \sim
    \lambda c^c_s \frac{|q_\a|}{e} \frac{m_i}{m_\a} \sigma \frac{ \wt{S}_{i,n}\wt{\ve{u}}_{\a,\perp} }{\wt{n}_\a}
\end{align*}

\section{Viscosities}
%
Finally, we deal with the viscosities in our drift ordering.
If we assume constant viscosity coefficients $\eta_{\a,N}$, where $N\in\{1,2,3,4\}$, and that $\eta_{\a,0}$ is dominating (see \cref{app:piTensor}), we get%
%
\footnote{
    Note that although we use drift ordering in \cref{app:piTensor}, the results should still be approximately valid as we already have constrained our system in a way so that the terms in \cref{sec:pep} are of leading order.
}%
%
\begin{align}
 \L(\div\te{\pi}_\a\R)_\perp \simeq
 \frac{2}{3}\eta_{\a,0}
 \L(
  \ve{e}_x\partial_x\partial_\|u_{\a,\|}
  +
  \ve{e}_y\partial_y\partial_\|u_{\a,\|}
 \R)
\label{eq:doRefVisc}
\end{align}
%
We will now assume that
%
\begin{align*}
    \partial_x u_{\a,\|}\sim \partial_y u_{\a,\|} \sim \frac{1}{L_{\perp, u_{\a,\|}}}u^c_{\a,\|},
\end{align*}
%
and also that
%
\begin{align*}
    L_{\perp, u_{\a,\|}} \sim L_\perp
    L_{\|   , u_{\a,\|}} \sim L_{\|, u_{\a,\perp}} \sim L_\|
\end{align*}
%
This means that
%
\begin{align*}
  \ve{e}_x\partial_x\partial_\|u_{\a,\|} +
  \ve{e}_y\partial_y\partial_\|u_{\a,\|}
  \sim&
  \frac{c^c_s}{L_\perp L_{\|}}
  \L(
  \ve{e}_x\partial_{\wt{x}}\partial_{\wt{\|}}\wt{u}_{\a,\|} +
  \ve{e}_y\partial_{\wt{y}}\partial_{\wt{\|}}\wt{u}_{\a,\|}
  \R)
  \\
  %
  %
  =&
  c^c_s
  \frac{L_\perp}{L_\perp}
  \frac{1}{L_\perp L_{\|}}
  \L(
  \ve{e}_x\partial_{\wt{x}}\partial_{\wt{\|}}\wt{u}_{\a,\|} +
  \ve{e}_y\partial_{\wt{y}}\partial_{\wt{\|}}\wt{u}_{\a,\|}
  \R)
  \\
  %
  %
  =&
  \rho^c_s\om^c_{ci}
  \frac{1}{L_\perp^2}
  \zeta
  \L(
  \ve{e}_x\partial_{\wt{x}}\partial_{\wt{\|}}\wt{u}_{\a,\|} +
  \ve{e}_y\partial_{\wt{y}}\partial_{\wt{\|}}\wt{u}_{\a,\|}
  \R)
  \\
  %
  %
  =&
  \om^c_{ci}
  \frac{1}{L_\perp}
  \gamma
  \zeta
  \L(
  \ve{e}_x\partial_{\wt{x}}\partial_{\wt{\|}}\wt{u}_{\a,\|} +
  \ve{e}_y\partial_{\wt{y}}\partial_{\wt{\|}}\wt{u}_{\a,\|}
  \R)
  \numberthis
  \label{eq:doViscDeriv}
\end{align*}
%
If we insert \cref{eq:doViscDeriv} into \cref{eq:doRefVisc}, we obtain
%
\begin{align*}
 \L(\div\te{\pi}_\a\R)_\perp \simeq&
 \frac{2}{3}\eta_{\a,0}
 \L(
  \ve{e}_x\partial_x\partial_\|u_{\a,\|}
  +
  \ve{e}_y\partial_y\partial_\|u_{\a,\|}
 \R)
 \\
 %
 %
 \sim&
 \frac{2}{3}
  \om^c_{ci}
  \frac{1}{L_\perp}
  \gamma
  \zeta
  \eta^c_{\a,0}
  \wt{\eta}_{\a,0}
 \L(
  \ve{e}_x\partial_{\wt{x}}\partial_{\wt{\|}}\wt{u}_{\a,\|} +
  \ve{e}_y\partial_{\wt{y}}\partial_{\wt{\|}}\wt{u}_{\a,\|}
 \R)
 \\
 %
 %
 =&
 \frac{2}{3}
  \om^c_{ci}
  \frac{1}{L_\perp}
  \gamma
  \zeta
  \eta^c_{\a,0}
 \L(\wt{\grad}\cdot\wt{\te{\pi}}_\a\R)_\perp
\end{align*}
%
From \cref{app:piTensor} we have that $\eta_{\a,0}=\frac{C_{\eta_{\a,0}}n_\a T_\a}{\nu_{\a i}}$, where $C_{\eta_{e,0}}=0.73$ and $C_{\eta_{i,0}}=0.96\sqrt{2}$.
As we have assumed $T_e \sim T_i$, and since $\nu_{e i}\propto \frac{1}{\sqrt{m_e}}$ and $\nu_{\a i}\propto \frac{1}{\sqrt{m_i}}$ we have that $\nu_{\a i}\sim \frac{\sqrt{m_e}}{\sqrt{m_\a}}\nu_{ei}$
This gives
%
\begin{align*}
 \frac{2}{3}
  \om^c_{ci}
  \frac{1}{L_\perp}
  \gamma
  \zeta
  \eta^c_{\a,0}
 \L(\wt{\grad}\cdot\wt{\te{\pi}}_\a\R)_\perp
 =&
 \frac{2}{3}
  \om^c_{ci}
  \frac{1}{L_\perp}
  \gamma
  \zeta
  \frac{C_{\eta_{\a,0}} n^c T^c_\a}{\frac{\sqrt{m_e}}{\sqrt{m_\a}}\nu^c_{ei}}
 \L(\wt{\grad}\cdot\wt{\te{\pi}}_\a\R)_\perp
 \\
 %
 %
 =&
 \frac{2}{3}
 C_{\eta_{\a,0}}
  \om^c_{ci}
  \frac{1}{L_\perp}
  \gamma
  \zeta
  \frac{\sqrt{m_\a}}{\sqrt{m_e}}
  \frac{n^c T^c_\a}{\nu^c_{ei}}
 \L(\wt{\grad}\cdot\wt{\te{\pi}}_\a\R)_\perp
\end{align*}
%
Thus
%
\begin{align*}
    \frac{\L(\div\te{\pi}_\a\R)_\perp}{n_\a q_\a B}
    \sim&
    \frac{2}{3}
    C_{\eta_{\a,0}}
     \om^c_{ci}
     \frac{1}{L_\perp}
     \gamma
     \zeta
     \frac{1}{n^c q_\a B^c}
     \frac{\sqrt{m_\a}}{\sqrt{m_e}}
     \frac{n^c T^c_\a}{\nu^c_{ei}}
     \frac{ \L(\wt{\grad}\cdot\wt{\te{\pi}}_\a\R)_\perp }{\wt{n}_\a \wt{B}}
     \\
     %
     %
     =&
    \frac{2}{3}
    C_{\eta_{\a,0}}
     \om^c_{ci}
     \frac{1}{L_\perp}
     \gamma
     \zeta
     \frac{m_i}{m_i}
     \frac{q_i}{q_i}
     \frac{1}{q_\a B^c}
     \frac{\sqrt{m_\a}}{\sqrt{m_e}}
     \frac{m_e}{m_e}
     \frac{T^c_\a}{\nu^c_{ei}}
     \frac{ \L(\wt{\grad}\cdot\wt{\te{\pi}}_\a\R)_\perp }{\wt{n}_\a \wt{B}}
     \\
     %
     %
     =&
    \frac{2}{3}
    C_{\eta_{\a,0}}
     \om^c_{ci}
     \frac{1}{L_\perp}
     \gamma
     \zeta
     \frac{m_e}{m_i}
     \frac{q_i}{q_\a}
     \frac{1}{\om^c_{ci}}
     \frac{\sqrt{m_\a}}{\sqrt{m_e}}
     \L(c^c_s\R)^2
     \frac{1}{\nu^c_{ei}}
     \frac{ \L(\wt{\grad}\cdot\wt{\te{\pi}}_\a\R)_\perp }{\wt{n}_\a \wt{B}}
     \\
     %
     %
     =&
    \frac{2}{3}
    C_{\eta_{\a,0}}
    \L(c^c_s\R)^2
     \frac{1}{L_\perp}
     \gamma
     \zeta
     \frac{1}{\mu}
     \frac{q_i}{q_\a}
     \frac{\sqrt{m_\a}}{\sqrt{m_e}}
     \frac{1}{\nu^c_{ei}}
     \frac{ \L(\wt{\grad}\cdot\wt{\te{\pi}}_\a\R)_\perp }{\wt{n}_\a \wt{B}}
     \\
     %
     %
     =&
    \frac{2}{3}
    C_{\eta_{\a,0}}
     c^c_s
     \rho^c_s
     \om^c_{ci}
     \frac{q_i}{q_\a}
     \frac{\sqrt{m_\a}}{\sqrt{m_e}}
     \frac{1}{L_\perp}
     \gamma
     \zeta
     \frac{1}{\mu}
     \frac{}{\nu^c_{ei}}
     \frac{ \L(\wt{\grad}\cdot\wt{\te{\pi}}_\a\R)_\perp }{\wt{n}_\a \wt{B}}
     \\
     %
     %
     =&
    \frac{2}{3}
    C_{\eta_{\a,0}}
     c^c_s
     \frac{q_i}{q_\a}
     \frac{\sqrt{m_\a}}{\sqrt{m_e}}
     \frac{\rho^c_s}{L_\perp}
     \gamma
     \zeta
     \frac{1}{\mu}
     \frac{\om^c_{ci}}{\nu^c_{ei}}
     \frac{ \L(\wt{\grad}\cdot\wt{\te{\pi}}_\a\R)_\perp }{\wt{n}_\a \wt{B}}
     \\
     %
     %
     =&
    \frac{2}{3}
    C_{\eta_{\a,0}}
     c^c_s
     \frac{q_i}{q_\a}
     \frac{\sqrt{m_\a}}{\sqrt{m_e}}
     \gamma^2
     \zeta
     \xi
     \frac{ \L(\wt{\grad}\cdot\wt{\te{\pi}}_\a\R)_\perp }{\wt{n}_\a \wt{B}}
\end{align*}
%

\section{The order of the terms}
%
Before we order the terms in \cref{eq:perp_mom_start}, let us briefly recapitulate the size of the non-dimensional terms
%
\begin{empheq}[box={\tcbhighmath[colback=yellow!5!white]}]{align*}
    &\e      \defined \frac{\om^c}{\om^c_{ci}}                    \ll 1  &
    &\xi     \defined \frac{1}{\mu} \frac{\nu^c_{ei}}{\om^c_{ci}} \sim \e&
    &\Theta  \defined \frac{\nu^c_{i n}}{\om^c_{ci}}              \sim \e&
    &\sigma  \defined \frac{\nu^c_{S_{i,n}}}{\om^c_{ci}}          \sim \e&
    \\
    &\lambda \defined \frac{u^c}{c^c_s}             \sim \sqrt{\e}&
    &\gamma  \defined \frac{ \rho^c_s }{L_{\perp }} \sim \sqrt{\e}&
    &\zeta   \defined \frac{L_{\perp}}{L_{\|}}      \sim \sqrt{\e}&
    &\Xi     \defined \frac{E_\perp^c}{c^c_sB^c}    \sim \sqrt{\e}&
\end{empheq}
%
where we have assumed
%
\begin{empheq}[box={\tcbhighmath[colback=yellow!5!white]}]{align*}
    &T^c_e                 \sim T^c_i                 &
    &\om^c_{u_{e,\perp}}   \sim \om^c_{u_{i,\perp}}   &
    &n_e                   \sim Zn_i                  &
    &u_{e,\perp}           \sim u_{i,\perp}           &
    \\
    &L_{\perp,u_{e,\perp}} \sim L_{\perp,u_{i,\perp}} &
    &L_{\perp,p_\a}        \sim L_{\perp,u_{\a,\perp}}&
    &u_{e,\|}              \sim u_{i,\|}              &
\end{empheq}
%
The order of magnitude estimate of \cref{eq:perp_mom_start} now yields
%
\begin{align*}
 \lambda c^c_s \frac{m_\a}{m_i} \e \wt{\d}^\a_t\wt{\ve{u}}_{\a,\perp}
 =&
 - c^c_s \sqrt{\e} \L( \frac{q_i}{q_\a} \frac{ \wt{\grad}_\perp \wt{p}_\a }{\wt{n}_\a \wt{B}} + \frac{\wt{\ve{E}}_\perp}{\wt{B}} + \wt{\ve{u}}_{\a,\perp}\times\ve{b} \R)
 \\&
 - \frac{2}{3} C_{\eta_{\a,0}} c^c_s \frac{q_i}{q_\a} \frac{\sqrt{m_\a}}{\sqrt{m_e}} \gamma^2 \zeta \xi \frac{ \L(\wt{\grad}\cdot\wt{\te{\pi}}_\a\R)_\perp }{\wt{n}_\a \wt{B}}
 \\&
 + \lambda c^c_s \frac{q_i}{q_\a} \xi \frac{\wt{\ve{R}}_{\beta \to \a,\perp} }{\wt{n}_\a \wt{B}}
 + \lambda c^c_s\frac{q_i}{q_\a} \frac{\sqrt{m_\a}}{\sqrt{m_i}} \Theta \frac{\wt{\ve{R}}_{n \to \a,\perp} }{\wt{n}_\a  \wt{B}}
 - \lambda c^c_s \frac{|q_\a|}{e} \frac{m_i}{m_\a} \sigma \frac{ \wt{S}_{i,n}\wt{\ve{u}}_{\a,\perp} }{\wt{n}_\a}
 \\
 %
 %
 \frac{m_\a}{m_i} \e \wt{\d}^\a_t\wt{\ve{u}}_{\a,\perp}
 =&
 - \frac{\sqrt{\e}}{\lambda} \L( \frac{q_i}{q_\a} \frac{ \wt{\grad}_\perp \wt{p}_\a }{\wt{n}_\a \wt{B}} + \frac{\wt{\ve{E}}_\perp}{\wt{B}} + \wt{\ve{u}}_{\a,\perp}\times\ve{b} \R)
 \\&
 - \frac{2}{3} \frac{1}{\lambda} C_{\eta_{\a,0}} \frac{q_i}{q_\a} \frac{\sqrt{m_\a}}{\sqrt{m_e}} \gamma^2 \zeta \xi \frac{ \L(\wt{\grad}\cdot\wt{\te{\pi}}_\a\R)_\perp }{\wt{n}_\a \wt{B}}
 \\&
 + \frac{q_i}{q_\a} \xi \frac{\wt{\ve{R}}_{\beta \to \a,\perp} }{\wt{n}_\a \wt{B}}
 + \frac{q_i}{q_\a} \frac{\sqrt{m_\a}}{\sqrt{m_i}} \Theta \frac{\wt{\ve{R}}_{n \to \a,\perp} }{\wt{n}_\a  \wt{B}}
 - \frac{|q_\a|}{e} \frac{m_i}{m_\a} \sigma \frac{ \wt{S}_{i,n}\wt{\ve{u}}_{\a,\perp} }{\wt{n}_\a}
 \\
 %
 %
 \frac{m_\a}{m_i} \e \wt{\d}^\a_t\wt{\ve{u}}_{\a,\perp}
 =&
 - \frac{\sqrt{\e}}{\sqrt{\e}} \L( \frac{q_i}{q_\a} \frac{ \wt{\grad}_\perp \wt{p}_\a }{\wt{n}_\a \wt{B}} + \frac{\wt{\ve{E}}_\perp}{\wt{B}} + \wt{\ve{u}}_{\a,\perp}\times\ve{b} \R)
 \\&
 - \frac{2}{3} \frac{\e^2 \sqrt{\e}}{\sqrt{\e}} C_{\eta_{\a,0}} \frac{q_i}{q_\a} \frac{\sqrt{m_\a}}{\sqrt{m_e}} \frac{ \L(\wt{\grad}\cdot\wt{\te{\pi}}_\a\R)_\perp }{\wt{n}_\a \wt{B}}
 \\&
 +\e \frac{q_i}{q_\a} \frac{\wt{\ve{R}}_{\beta \to \a,\perp} }{\wt{n}_\a \wt{B}}
 +\e \frac{q_i}{q_\a} \frac{\sqrt{m_\a}}{\sqrt{m_i}}  \frac{\wt{\ve{R}}_{n \to \a,\perp} }{\wt{n}_\a  \wt{B}}
 -\e \frac{|q_\a|}{e} \frac{m_i}{m_\a} \frac{ \wt{S}_{i,n}\wt{\ve{u}}_{\a,\perp} }{\wt{n}_\a}
 \\
 %
 %
 =&
 - \L( \frac{q_i}{q_\a} \frac{ \wt{\grad}_\perp \wt{p}_\a }{\wt{n}_\a \wt{B}} + \frac{\wt{\ve{E}}_\perp}{\wt{B}} + \wt{\ve{u}}_{\a,\perp}\times\ve{b} \R)
 \\&
 - \e^2 \frac{2}{3} C_{\eta_{\a,0}} \frac{q_i}{q_\a} \frac{\sqrt{m_\a}}{\sqrt{m_e}} \frac{ \L(\wt{\grad}\cdot\wt{\te{\pi}}_\a\R)_\perp }{\wt{n}_\a \wt{B}}
 \\&
 \\&
 + \e
 \L(
  \frac{q_i}{q_\a} \frac{\wt{\ve{R}}_{\beta \to \a,\perp} }{\wt{n}_\a \wt{B}}
 + \frac{q_i}{q_\a} \frac{\sqrt{m_\a}}{\sqrt{m_i}}  \frac{\wt{\ve{R}}_{n \to \a,\perp} }{\wt{n}_\a  \wt{B}}
 - \frac{|q_\a|}{e} \frac{m_i}{m_\a} \frac{ \wt{S}_{i,n}\wt{\ve{u}}_{\a,\perp} }{\wt{n}_\a}
 \R)
 \numberthis
 \label{eq:DO}
\end{align*}
%
Notice that the quantities in the tildes are of order $\mathcal{O}(1)$, and so that the only big or small terms appear in front of the terms in tilde.
From this we can for example see that the electron inertia term is small compare to most other terms, and can probably be neglegted.
We also note that altough the ion visocsity is small, it is questionable if it should be neglected.
