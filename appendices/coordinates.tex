The coordinates in a cylindrical geometry are written\\
%
\begin{minipage}{0.4\textwidth}
\begin{align*}
    x=&\rho \cos\theta\\
    y=&\rho \sin\theta\\
    z=& z
\end{align*}
\end{minipage}
\hfill
\begin{minipage}{0.4\textwidth}
\begin{align*}
    \rho=& \sqrt{x^2+y^2}\\
    \theta=&\atan\L(\frac{y}{x}\R)\\
    z=& z
\end{align*}
\end{minipage}

\section{The metrics}
\label{sec:metr}
%
We have that
%
\begin{align*}
    &\ve{e}_i = \partial_i&
    &\ve{e}^i = \d u_i&
\end{align*}
%
where $u_i$ is the set of the coordinate curves

To coordinate transform a covariant basis vector, we can consider an arbitrary line $f$ passing through the point under consideration written in the new set of coordinates.
We then use the chain rule to determine how the basis vector is written in the new set of coordinates.
We have
%
\begin{align*}
    \parti{f(\rho, \theta, z)}{x_i}
    =
    \parti{f}{\rho} \parti{\rho}{x_i}
    + \parti{f}{\theta} \parti{\theta}{x_i}
    + \parti{f}{z} \parti{z}{x_i}
\end{align*}
%
as the line $f$ is arbitrary, we have
%
\begin{align*}
    \ve{e}_i
    =
    \parti{}{x_i}
    =
    %
    \parti{\rho}{x_i} \parti{}{\rho}
    + \parti{\theta}{x_i} \parti{}{\theta}
    + \parti{z}{x_i} \parti{}{z}
    %
    =
    %
    \parti{\rho}{x_i} \ve{e}_{\rho}
    + \parti{\theta}{x_i} \ve{e}_{\theta}
    + \parti{z}{x_i} \ve{e}_{z}
    %
\end{align*}
%
where in our case $x_i \in \{x,y,z\}$.

To coordinate transform a contravariant basis vector, we can apply the chain rule directly to determine how the basis vector is written in the new set of coordinates.
We have
%
\begin{align*}
    \ve{e}^i
    =
    \d u^i(x,y,z)
    =
    %
    \parti{u_i}{x}\d x
    + \parti{u_i}{y}\d y
    + \parti{u_i}{z}\d z
    %
    =
    %
    \parti{u_i}{x}\ve{e}^x
    + \parti{u_i}{y}\ve{e}^y
    + \parti{u_i}{z}\ve{e}^z
    %
\end{align*}
%
At this point we note that there are no difference between co and contravariant basis vectors in a Cartesian coordinate system.

In the following, we are going to make use of the following relations
\\
%
\begin{minipage}{0.4\textwidth}
\begin{align*}
    \partial_x \rho
    =&
    \partial_x \sqrt{x^2+y^2}
    =
    \frac{x}{\rho}
    =
    \cos\theta\\
    %
    \partial_y \rho
    =&
    \partial_y \sqrt{x^2+y^2}
    =
    \frac{y}{\rho}
    =
    \sin\theta
    \\
    %
    \partial_z \rho
    =&
    \partial_z \sqrt{x^2+y^2}
    =
    0
    \\
    %
    %
    \partial_x \theta
    =&
    \partial_x \atan\L(\frac{y}{x}\R)
    =
    -\frac{y}{\rho^2}
    =
    -\frac{1}{\rho}\sin\theta
    \\
    %
    \partial_y \theta
    =&
    \partial_y \atan\L(\frac{y}{x}\R)
    =
    \frac{x}{\rho^2}
    =
    \frac{1}{\rho}\cos\theta
    \\
    %
    \partial_z \theta
    =&
    \partial_z \atan\L(\frac{y}{x}\R)
    =
    0
    \\
\end{align*}
\end{minipage}
%
\hfill
%
\begin{minipage}{0.4\textwidth}
\begin{align*}
    \partial_\rho x
    =&
    \partial_\rho \rho\cos\theta
    =
    \cos\theta\\
    %
    \partial_\rho y
    =&
    \partial_\rho \rho\sin\theta
    =
    \sin\theta
    \\
    %
    \partial_\rho z
    =&
    0
    \\
    %
    %
    \partial_\theta x
    =&
    \partial_\theta \rho\cos\theta
    =
    -\rho\sin\theta
    \\
    %
    \partial_\theta y
    =&
    \partial_\theta \rho\sin\theta
    =
    \rho\cos\theta
    \\
    %
    \partial_\theta z
    =&
    0
\end{align*}
\end{minipage}\\
%
This means that a basis vector written in a Cartesian basis can be written with a covariant basis vector using cylindrical coordinates as
%
\begin{align*}
    \ve{e}_x
    &=
    \parti{\rho}{x} \ve{e}_{\rho}
    + \parti{\theta}{x} \ve{e}_{\theta}
    + \parti{z}{x} \ve{e}_{z}
    %
    =
    \cos\theta \ve{e}_\rho
    - \frac{1}{\rho} \sin\theta \ve{e}_\theta
    \\
%
%
    \ve{e}_y
    &=
    \parti{\rho}{y} \ve{e}_{\rho}
    + \parti{\theta}{y} \ve{e}_{\theta}
    + \parti{z}{y} \ve{e}_{z}
    %
    =
    \sin\theta \ve{e}_\rho
    + \frac{1}{\rho} \cos\theta \ve{e}_\theta
    \\
%
%
    \ve{e}_z
    &=
    \parti{\rho}{z} \ve{e}_{\rho}
    + \parti{\theta}{z} \ve{e}_{\theta}
    + \parti{z}{z} \ve{e}_{z}
    %
    =
    \ve{e}_z
\end{align*}
%
For the back transformation we have
%
\begin{align*}
    \ve{e}_\rho
    &=
    \parti{x}{\rho} \ve{e}_{x}
    + \parti{y}{\rho} \ve{e}_{y}
    + \parti{z}{\rho} \ve{e}_{z}
    %
    =
    \cos\theta \ve{e}_x
    + \sin\theta \ve{e}_y
    \\
%
%
    \ve{e}_\theta
    &=
    \parti{x}{\theta} \ve{e}_{x}
    + \parti{y}{\theta} \ve{e}_{y}
    + \parti{z}{\theta} \ve{e}_{z}
    %
    =
    -\rho\sin\theta \ve{e}_x
    + \rho\cos\theta \ve{e}_y
    \\
%
%
    \ve{e}_z
    &=
    \parti{x}{z} \ve{e}_{x}
    + \parti{y}{z} \ve{e}_{y}
    + \parti{z}{z} \ve{e}_{z}
    %
    =
    \ve{e}_z
\end{align*}
%
Further, a basis vector written in a Cartesian basis can be written with a contravariant basis vector using cylindrical coordinates as
%
\begin{align*}
    \ve{e}^\rho
    &=
    \parti{\rho}{x} \ve{e}^{x}
    + \parti{\rho}{y} \ve{e}^{y}
    + \parti{\rho}{z} \ve{e}^{z}
    %
    =
    \cos\theta \ve{e}^x
    + \sin\theta \ve{e}^y
    \\
%
%
    \ve{e}^\theta
    &=
    \parti{\theta}{x} \ve{e}^{x}
    + \parti{\theta}{y} \ve{e}^{y}
    + \parti{\theta}{z} \ve{e}^{z}
    %
    =
    -\frac{1}{\rho}\sin\theta \ve{e}^x
    + \frac{1}{\rho} \cos\theta \ve{e}^y
    \\
%
%
    \ve{e}^z
    &=
    \parti{z}{x} \ve{e}^{x}
    + \parti{z}{y} \ve{e}^{y}
    + \parti{z}{z} \ve{e}^{z}
    %
    =
    \ve{e}^z
\end{align*}
%
The covariant metric tensor $g^{ij}=\ve{e}^i\cdot\ve{e}^j$ and the contravariant metric tensor $g_{ij}=\ve{e}_i\cdot\ve{e}_j$ can now be computed.
For the contravariant components, we get
%
\begin{align*}
    g^{\rho\rho}
    =&
    \L(
      \cos\theta\ve{e}^x
      + \sin\theta\ve{e}^y
    \R)
    \cdot
    \L(
      \cos\theta\ve{e}^x
      + \sin\theta\ve{e}^y
    \R)
    =
      \cos^2\theta
      + \sin^2\theta
    =
    1
    \\
    %
    g^{\rho\theta}
    =&
    g^{\theta\rho}
    =
    \L(
      \cos\theta\ve{e}^x
      + \sin\theta\ve{e}^y
    \R)
    \cdot
    \L(
      -\frac{1}{\rho}\cos\theta\ve{e}^x
      + \frac{1}{\rho}\sin\theta\ve{e}^y
    \R)
    =
      -\frac{1}{\rho}\cos\theta\sin\theta
      + \frac{1}{\rho}\sin\theta\cos\theta
    =
    0
    \\
    %
    g^{\rho z}
    =&
    g^{z \rho}
    =
    \L(
      \cos\theta\ve{e}^x
      + \sin\theta\ve{e}^y
    \R) \cdot \ve{e}^z = 0
    \\
    %
    %
    %
    g^{\theta\theta}
    =&
    \L(
      -\frac{1}{\rho}\cos\theta\ve{e}^x
      + \frac{1}{\rho}\sin\theta\ve{e}^y
    \R)
    \cdot
    \L(
      -\frac{1}{\rho}\cos\theta\ve{e}^x
      + \frac{1}{\rho}\sin\theta\ve{e}^y
    \R)
    =
      \frac{1}{\rho^2}\cos^2\theta
      + \frac{1}{\rho^2}\sin^2\theta
    =
    \frac{1}{\rho^2}
    \\
    %
    g^{z\theta}
    =&
    g^{\theta z}
    =
    \L(
      -\frac{1}{\rho}\cos\theta\ve{e}^x
      + \frac{1}{\rho}\sin\theta\ve{e}^y
    \R) \cdot \ve{e}^z = 0
    \\
    %
    %
    %
    g^{z z}
    =&
    \ve{e}^z \cdot \ve{e}^z =1
\end{align*}
%
And for the covariant components, we get
%
\begin{align*}
    g_{\rho\rho}
    =&
    \L(
      \cos\theta\ve{e}_x
      + \sin\theta\ve{e}_y
    \R)
    \cdot
    \L(
      \cos\theta\ve{e}_x
      + \sin\theta\ve{e}_y
    \R)
    =
      \cos^2\theta
      + \sin^2\theta
    =
    1
    \\
    %
    g_{\rho\theta}
    =&
    g_{\theta\rho}
    =
    \L(
      \cos\theta\ve{e}_x
      + \sin\theta\ve{e}_y
    \R)
    \cdot
    \L(
      -\rho\sin\theta\ve{e}_x
      + \rho\cos\theta\ve{e}_y
    \R)
    =
      -\rho\cos\theta\sin\theta
      + \rho\sin\theta\cos\theta
    =
    0
    \\
    %
    g_{\rho z}
    =&
    g_{z \rho}
    =
    \L(
      \cos\theta\ve{e}_x
      + \sin\theta\ve{e}_y
    \R) \cdot \ve{e}_z = 0
    \\
    %
    %
    %
    g_{\theta\theta}
    =&
    \L(
      -\rho\sin\theta\ve{e}_x
      + \rho\cos\theta\ve{e}_y
    \R)
    \cdot
    \L(
      -\rho\sin\theta\ve{e}_x
      + \rho\cos\theta\ve{e}_y
    \R)
    =
      \rho^2\cos^2\theta
      + \rho^2\sin^2\theta
    =
    \rho^2
    \\
    %
    g_{z\theta}
    =&
    g_{\theta z}
    =
    \L( -\rho\sin\theta\ve{e}_x
      + \rho\cos\theta\ve{e}_y \R) \cdot \ve{e}_z
    =
    0
    \\
    %
    g_{z z} =&
    \ve{e}_z \cdot \ve{e}_z =1
\end{align*}
%
The Jacobian
%
\begin{align*}
    J=\sqrt{\det(g_{ij})}=\rho
\end{align*}
%
Finally, we calculate the derivatives of the contravariant basis vectors.
From what is calculated above, we see that $\partial_z e^i=0$ and $\partial_i e^z=0$.
The other basis vectors gives
%
\begin{align*}
    \partial_\rho \ve{e}^\theta
    &=&
    \partial_\rho
    \L( - \frac{1}{\rho} \sin\theta \ve{e}^x
        + \frac{1}{\rho} \cos\theta \ve{e}^y \R)
    =
        \frac{\sin\theta}{\rho^2} \ve{e}^x
        - \frac{\cos\theta}{\rho^2} \ve{e}^y
    =
    - \frac{1}{\rho}
    \L( - \frac{1}{\rho} \sin\theta \ve{e}^x
        + \frac{1}{\rho} \cos\theta \ve{e}^y \R)
    &=&
    - \frac{1}{\rho} \ve{e}^\theta
    \\
    %
    \partial_\rho \ve{e}^\rho
    &=&
    \partial_\rho
    \L( \cos\theta \ve{e}^x
        + \sin\theta \ve{e}^y \R)
    &=&
    0
    \\
    %
    \partial_\theta \ve{e}^\theta
    &=&
    \partial_\theta
    \L( - \frac{1}{\rho} \sin\theta \ve{e}^x
        + \frac{1}{\rho} \cos\theta \ve{e}^y \R)
    =
    \L( - \frac{1}{\rho} \cos\theta \ve{e}^x
        - \frac{1}{\rho} \sin\theta \ve{e}^y \R)
    =
    - \frac{1}{\rho}
    \L( \cos\theta \ve{e}^x
        + \sin\theta \ve{e}^y \R)
    &=&
    - \frac{1}{\rho}
    \ve{e}^\rho
    \\
    %
    \partial_\theta \ve{e}^\rho
    &=&
    \partial_\theta
    \L( \cos\theta \ve{e}^x
        + \sin\theta \ve{e}^y \R)
    =
        - \sin\theta \ve{e}^x
        + \cos\theta \ve{e}^y
    =
    \rho
    \L( - \frac{1}{\rho} \sin\theta \ve{e}^x
        + \frac{1}{\rho} \cos\theta \ve{e}^y \R)
    &=&
    \rho
    \ve{e}^\theta
\end{align*}
%

\subsection{Summary}
\label{app:cylSummary}
%
Basis vector transformations\\
%
\begin{minipage}{0.3\textwidth}
\begin{align*}
    \ve{e}_x
    &=
    \cos\theta \ve{e}_\rho
    - \frac{1}{\rho} \sin\theta \ve{e}_\theta
    \\
%
%
    \ve{e}_y
    &=
    \sin\theta \ve{e}_\rho
    + \frac{1}{\rho} \cos\theta \ve{e}_\theta
    \\
%
%
    \ve{e}_z &= \ve{e}_z
\end{align*}
\end{minipage}
%
\hfill
%
\begin{minipage}{0.3\textwidth}
    \begin{align*}
        \ve{e}_\rho
        &=
        \cos\theta \ve{e}_x
        + \sin\theta \ve{e}_y
        \\
    %
    %
        \ve{e}_\theta
        &=
        -\rho\sin\theta \ve{e}_x
        + \rho\cos\theta \ve{e}_y
        \\
    %
    %
        \ve{e}_z &= \ve{e}_z
    \end{align*}
\end{minipage}
%
\hfill
%
\begin{minipage}{0.3\textwidth}
    \begin{align*}
        \ve{e}^\rho
        &=
        \cos\theta \ve{e}^x
        + \sin\theta \ve{e}^y
        \\
    %
    %
        \ve{e}^\theta
        &=
        -\frac{1}{\rho}\sin\theta \ve{e}^x
        + \frac{1}{\rho} \cos\theta \ve{e}^y
        \\
    %
    %
        \ve{e}^z &= \ve{e}^z
    \end{align*}
\end{minipage}
\\
%
Metric tensors
%
\begin{align*}
    &
    g^{\rho\theta} = g^{\theta\rho}
    = g^{\rho z} = g^{z \rho}
    = g^{z\theta} = g^{\theta z}
    = 0
    &
    &
    g_{\rho\theta} = g_{\theta\rho}
    = g_{\rho z} = g_{z \rho}
    = g_{z\theta} = g_{\theta z}
    = 0
    &
    \\
    &
    g^{\rho\rho} = g^{z z} = 1
    &
    &
    g_{\rho\rho} = g_{z z} = 1
    &
    \\
    %
    &
    g^{\theta\theta} = \frac{1}{\rho^2}
    &
    &
    g_{\theta\theta} = \rho^2
    &
\end{align*}
%
The Jacobian
%
\begin{align*}
    J=\rho
\end{align*}
%
The derivatives of the contravariant basis vectors.
%
\begin{align*}
    \partial_\rho \ve{e}^\rho =&
    \partial_\rho \ve{e}^z =
    \partial_\theta \ve{e}^z =
    \partial_z \ve{e}^\rho =
    \partial_z \ve{e}^\theta =
    \partial_z \ve{e}^z = 0
    \\
    %
    \partial_\rho \ve{e}^\theta =& - \frac{1}{\rho} \ve{e}^\theta
    \\
    %
    \partial_\theta \ve{e}^\theta =& - \frac{1}{\rho} \ve{e}^\rho
    \\
    %
    \partial_\theta \ve{e}^\rho =& \rho \ve{e}^\theta
\end{align*}

\section{Alignment with the Clebsch formalism}
%
As most of the numerical differentiation operators in BOUT++ is only valid for a coordinate system written on the Clebsch form (at least at the time of writing), it makes sense to align our coordinates with the Clebsch coordinates.
Note as whereas the Clebsch coordinate system gives
%
\begin{align*}
    \ve{B}_{\text{Clebsch}}
    =&\ve{e}^3 \times \ve{e}^1\\
    J^{-1}\ve{e}_2=&\ve{e}^3 \times \ve{e}^1
\end{align*}
%
so that
%
\begin{align*}
    B_{\text{Clebsch}}\defined & \sqrt{\ve{B}_{\text{Clebsch}}\cdot\ve{B}_{\text{Clebsch}}}
    = \sqrt{J^{-1}\ve{e}_2\cdot J^{-1}\ve{e}_2}
    = \sqrt{J^{-2}g_{22}}
    = J^{-1}\sqrt{g_{22}}
\end{align*}
%
and
%
\begin{align*}
    \ve{B}_{\text{Clebsch}}=&B_{\text{Clebsch}}\ve{b}_{\text{Clebsch}}\\
    \ve{b}_{\text{Clebsch}}
    =&\frac{\ve{B}_{\text{Clebsch}}}{B_{\text{Clebsch}}}
    =\frac{J^{-1}\ve{e}_2}{J^{-1}\sqrt{g_{22}}}
    =\frac{\ve{e}_2}{\sqrt{g_{22}}},
\end{align*}
%
the $B$-field in our case is constant. This can be obtained if we let%
%
\footnote{
    BOUT++ uses the indices $\{x,y,z\}$ for $\{1,2,3\}$.
    Be aware that this can be a source of confusion as $y$ in BOUT++ coordinates maps to $z$ in cylindrical coordinates as shown below
    %
    \begin{center}
        \begin{tabular}{ccccc}
            Generic &     & BOUT++ indices &     & Cylindrical coordinates\\
            $1$     &$\to$& $x$            &$\to$& $\rho$                 \\
            $2$     &$\to$& $y$            &$\to$& $z$                    \\
            $3$     &$\to$& $z$            &$\to$& $\theta$
        \end{tabular}
    \end{center}
}%
%
\footnote{
    Note that this system is left-handed.
}
%
\begin{align*}
    1 &\to \rho\\
    2 &\to z\\
    3 &\to \theta
\end{align*}
%
We now have that $\ve{B}_{\text{Cylinder}}=B_0J\ve{B}_{\text{Clebsch}}$, where $B_0$ is a constant value, which means that
%
\begin{align*}
    B_{\text{Cylinder}}\defined &
    \sqrt{B_0J\ve{B}_{\text{Clebsch}}\cdot B_0J\ve{B}_{\text{Clebsch}}}
    = \sqrt{B_0\ve{e}_\theta\cdot B_0\ve{e}_\theta}
    = B_0\sqrt{g_{\theta\theta}}
    = B_0
\end{align*}
%
and
%
\begin{align*}
    \ve{B}_{\text{Cylinder}}=&B_{\text{Cylinder}}\ve{b}_{\text{Cylinder}}\\
    \ve{b}_{\text{Cylinder}}
    =&\frac{\ve{B}_{\text{Cylinder}}}{B_{\text{Cylinder}}}
    =\frac{B_0J\ve{B}_{\text{Clebsch}}}{B_0}
    =\frac{B_0JJ^{-1}\ve{e}_\theta}{B_0}
    =\ve{e}_\theta
\end{align*}
%
In other words, we see that the cylindrical coordinate system overlaps with the Clebsch coordinate system, in the sense that $\ve{b}_{\text{Cylinder}}\propto\ve{b}_{\text{Clebsh}}$, but that they are not equal since $\ve{b}_{\text{Cylinder}}\neq\ve{b}_{\text{Clebsh}}$.
Care must therefore be taken when using BOUT++ operators which explicitly uses ${B}_{\text{Clebsch}}$.
In the scope of this thesis, it means that care must be taken whenever using the Poisson bracket, as explained in \cref{app:poisson}.
