% NOTE: The integrals here can be checked with symbolab.com
% NOTE: Derivation does not directly compare with
%       \nu_{\a\b} = n_\b\expt{\sigma_{\a\b} v}_\a
%       as extra v^2 by considering drifting Maxwellians. Without this the
%       nu_ei frequency becomes infinite
We will here derive an estimate for the elastic electron-neutral and ion-neutral collision frequencies, in the same way as the electron-ion and the ion-ion collision frequency is derived in \cite{Goldston1995book}.
In order to do so, we start by calculating the frictional force experienced by species $\a$ as it is drifting with respect to the stationary species $\b$.
We have
%
\begin{align*}
    \ve{F}_\a = - n_\a m_\a \expt{n_\b\sigma_{\a\b} v\ve{v}}_\a
\end{align*}
%
where $\expt{\cdot}_\a$ denotes the average over the drifting distribution function of species $\a$, and $\sigma_{\a\b}$ is the cross section of the process.
If we let the particles stream towards the stationary target along $z$, so that the fuid velocity $\ve{u}_\a = u_z \ve{e}_z$, we get
%
\begin{align*}
    f_\a
    =&
    \frac{n_\a}{(2\pi)^{3/2}v_{th,\a}^3}
    \exp\L(-\frac{\L[\ve{v}-\ve{u}\R]^2}{2v_{th,\a}^2}\R)
    \note{Assume $\ve{u} \ll v_{th,\a}^2$}
    \\
    %
    %
    \simeq&
    \frac{n_\a}{(2\pi)^{3/2}v_{th,\a}^3}
    \L(
    \L.\exp\L[-\frac{\L(\ve{v}-\ve{u}\R)^2}{2v_{th,\a}^2}\R]\R|_{\ve{u}=0}
    +
    \ve{u}\cdot
    \L[
    -2\frac{\L(\ve{v}-\ve{u}\R)}{2v_{th,\a}^2}(-1)
    \exp\L(-\frac{\L(\ve{v}-\ve{u}\R)^2}{2v_{th,\a}^2}\R)
    \R]_{\ve{u}=0}
    \R)
    \\
    %
    %
    =&
    \frac{n_\a}{(2\pi)^{3/2}v_{th,\a}^3}
    \L(
    \exp\L[-\frac{\ve{v}^2}{2v_{th,\a}^2}\R]
    +
    2\frac{\ve{u}\cdot\ve{v}}{2v_{th,\a}^2}
    \exp\L[-\frac{\ve{v}^2}{2v_{th,\a}^2}\R]
    \R)
    \\
    %
    %
    =&
    \frac{n_\a}{(2\pi)^{3/2}v_{th,\a}^3}
    \L( 1 + \frac{2u_zv_z}{2v_{th,\a}^2} \R)
    \exp\L(-\frac{\ve{v}^2}{2v_{th,\a}^2}\R)
    \\
    %
    %
    =&
    \L( 1 + \frac{u_zv_z}{v_{th,\a}^2} \R) f_{\a,0}
\end{align*}
%
where $\ve{v}$ denotes the particle velocity, $f_{\a,0}$ denotes the unshifted Maxwellian and
%
\begin{align*}
    v_{th,\a} \defined \sqrt{\frac{T_\a}{m_\a}} .
\end{align*}
%
Thus, the friction force in the direction of the drifting is
%
\begin{align*}
    F_{\a,z} =& - n_\a m_\a \expt{n_\b\sigma_{\a\b} v v_z}_\a
    \\
    %
    %
    \simeq&
    - n_\a m_\a
    \frac{n_\b}{n_\a}\iiint\L( 1 + \frac{u_zv_z}{v_{th,\a}^2} \R)f_{\a,0}\sigma_{\a\b} v v_z\d^3v
    \\
    %
    %
    =&
    - n_\a m_\a
    \frac{n_\b}{n_\a}
    \L(
    u_z\iiint\frac{v_z}{v_{th,\a}^2}f_{\a,0}\sigma_{\a\b} v v_z\d^3v
    +\iiint f_{\a,0}\sigma_{\a\b} v v_z\d^3v
    \R)
    \note{Second integral even in $v_z$}
    % Eventually: Can v_z cast to spherical coordinates and see it from there
    \\
    %
    %
    =&
    - n_\a m_\a
    \frac{n_\b}{n_\a}
    \L(
    u_z\iiint\frac{v_z^2}{v_{th,\a}^2}f_{\a,0}\sigma_{\a\b} v \d^3v
    \R)
    \note{Integral over $v_z^2$ is $1/3$ of integral over $v^2$ due to sperical
        symmetry}
    \\
    %
    %
    =&
    - n_\a m_\a
    \frac{n_\b}{n_\a}
    \frac{1}{3}
    \L(u_z\iiint\frac{1}{v_{th,\a}^2}f_{\a,0}\sigma_{\a\b} v^3\d^3v\R)
    \\
    %
    %
    =&
    - n_\a m_\a
    u_z
    \frac{n_\b}{n_\a}
    \frac{1}{3}
    \frac{1}{v_{th,\a}^2}
    \iiint f_{\a,0}\sigma_{\a\b} v^3\d^3v
    \note{Spherical coordinates}
    \\
    %
    %
    =&
    - n_\a m_\a
    u_z
    \frac{n_\b}{n_\a}
    \frac{1}{3}
    \frac{1}{v_{th,\a}^2}
    \int_0^\infty\int_0^{2\pi}\int_0^\pi
    f_{\a,0}\sigma_{\a\b} v^5
    \sin\theta \d\theta \d \phi\d v
    \\
    %
    %
    =&
    - n_\a m_\a
    u_z
    \frac{n_\b}{n_\a}
    \frac{1}{3}
    \frac{1}{v_{th,\a}^2}
    4\pi
    \int_0^\infty
    f_{\a,0}\sigma_{\a\b} v^5
    \d v
     \\
    %
    %
    =&
    - n_\a m_\a
    u_z
    \nu_{\a\b, \text{stationary target}}
\end{align*}
%
where we here have defined the averaged collision frequency
%
\begin{align*}
    \nu_{\a\b, \text{stationary target}}
    \defined&
    \frac{n_\b}{n_\a}
    \frac{1}{3}
    \frac{1}{v_{th,\a}^2}
    4\pi
    \int_0^\infty
    f_{\a,0}\sigma_{\a\b} v^5
    \d v
    \\
    %
    %
    =&
    \frac{n_\b}{n_\a}
    \frac{4\pi}{3v_{th,\a}^2}
    \int_0^\infty
    \frac{n_\a}{(2\pi)^{3/2}v_{th,\a}^3}
    \exp\L(-\frac{\ve{v}^2}{2v_{th,\a}^2}\R)
    \sigma_{\a\b} v^5
    \d v
    \\
    %
    %
    =&
    \frac{n_\b4\pi}{3(2\pi)^{3/2}v_{th,\a}^5}
    \int_0^\infty
    \exp\L(-\frac{\ve{v}^2}{2v_{th,\a}^2}\R)
    \sigma_{\a\b} v^5
    \d v
\end{align*}
%
The subscript $_\text{stationary target}$ will be dropped from here on.

For a $\sigma_{\a\b}$ constant in $v$, the integral reads
%
\begin{align*}
    \nu_{\a\b, \text{ Constant } \sigma}
    %
    =&
    \frac{n_\b4\pi}{3(2\pi)^{3/2}v_{th,\a}^5}
    \sigma_{\a\b, \text{ Constant}}
    \int_0^\infty
    \exp\L(-\frac{\ve{v}^2}{2v_{th,\a}^2}\R)
    v^5
    \d v
    \\
    %
    %
    &=
    \frac{n_\b4\pi}{3(2\pi)^{3/2}v_{th,\a}^5}
    \sigma_{\a\b, \text{ Constant}}
    8 v_{th,\a}^6
    \\
    %
    %
    &=
    \frac{8\sqrt{2}}{3}
    \frac{n_\b }{\sqrt{\pi}}
    v_{th,\a}
    \sigma_{\a\b, \text{ Constant}}
    \\
    %
    %
    &=
    \frac{8\sqrt{2}}{3}
    \frac{n_\b }{\sqrt{\pi}}
    \sqrt{ \frac{T_\a}{m_\a}}
    \sigma_{\a\b, \text{ Constant}}
    \numberthis
    \label{eq:constSigma}
\end{align*}
%

\section{Electron collisions}
\label{sec:nue}
\subsection{Electron ion collision}
Using the cross section for electron ion collisions
%
\begin{align*}
    &\sigma_{ei} = \frac{Z^2e^4\ln\Lambda}{4\pi\e_0^2m_e^2v^4}
    &&
    \ln\Lambda = \ln\L(\frac{12\pi n \lambda_D^3}{Z}\R)
    &&
    \lambda_D = \sqrt{\frac{\e_0 T_e}{n_e e^2}}
\end{align*}
%
yields
%
\begin{align*}
    \nu_{ei}
    =&
    \frac{n_i4\pi}{3(2\pi)^{3/2}v_{th,e}^5}
    \int_0^\infty
    \exp\L(-\frac{\ve{v}^2}{2v_{th,e}^2}\R)
    \frac{Z^2e^4\ln\Lambda}{4\pi\e_0^2m_e^2v^4} v^5
    \d v
    \\
    %
    %
    =&
    \frac{n_i}{2^{1/2}6\pi^{3/2}v_{th,e}^5}
    \frac{Z^2e^4\ln\Lambda}{\e_0^2m_e^2}
    \int_0^\infty
    \exp\L(-\frac{\ve{v}^2}{2v_{th,e}^2}\R)
    v \d v
    \\
    %
    %
    =&
    \frac{2}{2}
    \frac{n_i}{2^{1/2}6\pi^{3/2}v_{th,e}^5}
    \frac{Z^2e^4\ln\Lambda}{\e_0^2m_e^2}
    v_{th,e}^2
    \\
    %
    %
    =&
    \frac{2^{1/2}n_iZ^2e^4\ln\Lambda}{12\pi^{3/2}e_0^2m_e^2\L(\sqrt{\frac{T_e}{m_e}}\R)^3}
    \\
    %
    %
    =&
    \frac{2^{1/2}n_iZ^2e^4\ln\Lambda}{12\pi^{3/2}e_0^2m_e^{1/2}T_e^{3/2}}
\end{align*}
%

\subsection{Electron hydrogen collision}
A rough estimate for the hydrogen collision cross section can be obtained from the Bohr radius, and reads
%
\begin{align*}
    \sigma_{en_\text{H}} = \pi a_0^2
\end{align*}
%
inserting this in \cref{eq:constSigma} yields
%
\begin{align*}
    \nu_{en_\text{H}}
    =&
    \frac{8\sqrt{2}}{3}
    \frac{n_{n_\text{H}} }{\sqrt{\pi}}
    \sqrt{ \frac{T_e}{m_e}}
    \pi a_0^2
    =
    \frac{8\sqrt{2}}{3}
    \sqrt{\pi} n_{n_\text{H}}  a_0^2
    \sqrt{ \frac{T_e}{m_e}}
\end{align*}
%
% NOTE: Corresponds roughly to what is reported in NRL plasma formulary in weakly ionized plasmas

\subsection{Electron argon collisions}
%
A formula for the electron argon cross section is given in \cite{Hayashi1981}.
The cross section can be integrated numerically, as done in \cite{Schroder2003Phd}.
A sixth order polynomial which fits the integrated data in the range $0.1 - 10 \eV$ reads%
%
\footnote{Data from \cite{Schroder2003Phd} has been read using \cite{Ankit2016Web} to obtain these values.}
%
\begin{align*}
    \nu_{en_\text{Ar}}[\s^{-1}]
    =&
    \frac{n_{\text{Ar}}[\m^{-3}]}{2.5\cdot10^{19}[\m^{-3}]}
    \\
    &
    (
    \quad 33640.349990\cdot T_e[\eV]^0 -33174.059200\cdot T_e[\eV]^1\\
&+642273.100111\cdot T_e[\eV]^2 -188328.743082\cdot T_e[\eV]^3\\
&+25742.288823\cdot T_e[\eV]^4 -1784.118597\cdot T_e[\eV]^5 +50.336945\cdot T_e[\eV]^6)
\end{align*}

\section{Ion collisions}
\label{sec:nui}
%
\subsection{Ion-ion collisions}
The analysis done above was valid when the target was stationary with respect to the moving particles.
This is a fairly good approximation when the stationary particles are much heavier than the moving particles.
Hence, the derivation of the average ion-ion collision frequency or the average ion-neutral collision frequency is strictly not valid.
However, according to \cite{Goldston1995book}, the analysis yields the correct result within factors of orders of unity.

One could do the analysis by going to the center of mass frame (which in the end gives an additional factor $2^{-1/2}$) and use relative velocities one find velocities
%
\begin{align*}
    \nu_{ii}
    %
    =&
    \frac{n_iZ^2e^4\ln\Lambda}{12\pi^{3/2}e_0^2m_i^{1/2}T_i^{3/2}}
\end{align*}
%

\subsection{Ion-hydrogen collisions}
As the mass of the neutral atom is approximately the same as the ion mass (as we are considering neutrals of the same species as the plasma), we get for the ion-neutral collision
%
\begin{align*}
    \nu_{in_\text{H}}
    =&
    \frac{8}{3} \sqrt{\pi}
    n_{n_\text{H}} a_0^2
    \frac{\sqrt{T_i}}{\sqrt{m_H}}
\end{align*}

\subsection{Ion-argon collisions}
%
Although several atomic processes are involved when an Argon ion collides with an Argon atom in the ground state, most can be neglected under temperatures under $100 \eV$.
The charge exchange reaction and the elastic collisions dominated, and are approximately equally big \cite{Lieberman2005}.
The following formula for the cross section is given in \cite{Anders1990} for the charge exchange reaction $\text{Ar}+\text{Ar}^+ \to\text{Ar}^+ +\text{Ar}$ at low energies
%
\begin{align*}
    \sigma_{ce}[\m^2] = 4.8\cdot10^{-19}[\m^2]\L(1+0.14\ln\L[\frac{1[\eV]}{T_i[\eV]}\R]\R)^2
\end{align*}
%
so that
%
\begin{align*}
    \sigma_{in_\text{Ar}}[\m^2] \simeq 9.6\cdot10^{-19}[\m^2]\L(1+0.14\ln\L[\frac{1[\eV]}{T_i[\eV]}\R]\R)^2
\end{align*}
%
Inserting this into \cref{eq:constSigma} gives us
%
\begin{align*}
    \nu_{in_\text{Ar}}
    %
    \simeq&
    \frac{8\sqrt{2}}{3}
    \frac{n_{\text{Ar}}}{\sqrt{\pi}}
    \sqrt{ \frac{T_i[\J]}{m_{\text{Ar}}}}
    9.6\cdot10^{-19}[\m^2]\L(1+0.14\ln\L[\frac{1[\eV]}{T_i[\eV]}\R]\R)^2
\end{align*}
