We will here derive the advection of the modified vorticity
%
\begin{align}
    \frac{1}{\om_{ci}}
    \div\L(\ve{u}_E\cdot\grad\L[n\frac{\grad_\perp \phi}{B}\R]\R)
    \label{vortD:modVortAdv}
\end{align}
%
in cylindrical coordinates.

The first thing we notice is that equation (\ref{vortD:modVortAdv}) can only have perpendicular components.
As it is shown in equation (\ref{poi:def})
%
\begin{align*}
    \ve{u}_E\cdot\nabla
    = \frac{1}{JB}
      (\partial_\theta\phi\partial_\rho - \partial_\rho\phi\partial_\theta).
\end{align*}
%
When this term is acting on $\grad_\perp \phi$, no $\ve{e}_z$ or $\ve{e}^z$ terms will be created as seen from appendix \ref{app:cylSummary}.
The same holds when one takes the divergence of the resulting quantity.
Thus
%
\begin{align*}
    \frac{1}{\om_{ci}}
    \div\L(\ve{u}_E\cdot\grad\L[n\frac{\grad_\perp \phi}{B}\R]\R)
    =&
    \frac{1}{\om_{ci}}
    \grad_\perp\cdot
    \L(\frac{1}{JB}\L\{\phi, n\frac{\grad_\perp \phi}{B}\R\}\R)
    \\
    %
    =&
    \frac{1}{\om_{ci}}
    \L\{\phi, n\frac{\grad_\perp \phi}{B} \R\}\cdot\grad_\perp\frac{1}{JB}
    +
    \frac{1}{\om_{ci}}
    \frac{1}{JB}\grad_\perp\cdot\L\{\phi, n\frac{\grad_\perp \phi}{B} \R\}
    \numberthis
    \label{vortD:firstDeriv}
\end{align*}
%
Expansion of the first term of equation (\ref{vortD:firstDeriv}) gives
%
\begin{align*}
    \frac{1}{\om_{ci}}
    \L\{\phi, n\frac{\grad_\perp \phi}{B} \R\}\cdot\grad_\perp\frac{1}{JB}
    =&
    \frac{1}{\om_{ci}}
    \L\{\phi, n\frac{\grad_\perp \phi}{B} \R\}\cdot
    \L(\ve{e}^\rho\partial_\rho + \ve{e}^\theta\partial_\theta\R)
    \frac{1}{B\rho}
    \note{Constant $B$}
    \\
    %
    =&
    -
    \frac{1}{\om_{ci}}
    \L\{\phi, n\frac{\grad_\perp \phi}{B} \R\}\cdot
    \ve{e}^\rho \frac{1}{B\rho^2}
    \numberthis
    \label{vortD:firstTerm}
\end{align*}
%
When calculating the second term of equation (\ref{vortD:firstDeriv}), we will use that for a general scalar field $c$ and a general vector $\ve{d}$, we have that
%
\begin{align*}
    \grad_\perp\cdot\L\{c, \ve{d}\R\}
    =&
    \grad_\perp\cdot
    \L(
        \partial_\theta c \partial_\rho \ve{d}
      - \partial_\rho c   \partial_\theta \ve{d}
    \R)
    \\
    %
    =&
       \quad \L(\partial_\theta c\R) \grad_\perp \cdot \partial_\rho \ve{d}
      +  \grad_\perp\L(\partial_\theta c\R) \cdot \partial_\rho \ve{d}
      \\&
      -\L(
         \L[\partial_\rho c\R] \grad_\perp \cdot\partial_\theta \ve{d}
       + \grad_\perp \L[\partial_\rho c\R] \cdot \partial_\theta \ve{d}
      \R)
    \note{
        $\ve{e}^i\partial_jf
         =
           \partial_j\L(\ve{e}^i\partial_i f\R)
         - \partial_i f\partial_j\ve{e}^i
         $
         }
    \\
    %
    =&
       \quad \L(\partial_\theta c\R) \partial_\rho\L( \grad_\perp \cdot \ve{d}\R)
           - \L(\partial_\theta c\R) \partial_i \ve{d} \cdot\partial_\rho \ve{e}^i
      \\&
      +  \L(\partial_\theta \grad_\perp c\R) \cdot \partial_\rho \ve{d}
      - \L(\partial_i c\partial_\theta\ve{e}^i\R) \cdot \partial_\rho \ve{d}
      \\&
      -\L(
         \quad \L[\partial_\rho c\R] \partial_\theta\L[\grad_\perp \cdot \ve{d}\R]
             - \L[\partial_\rho c\R] \partial_i \ve{d} \cdot\partial_\theta \ve{e}^i
      \R.\\&\L.
     \quad\; + \L[\partial_\rho \grad_\perp c\R] \cdot \partial_\theta \ve{d}
            - \L[\partial_i c\partial_\rho\ve{e}^i\R] \cdot \partial_\theta \ve{d}
      \R)
    \\
    %
    =&
      \{c, \grad_\perp\cdot \ve{d}\} + \{\grad_\perp c; \ve{d}\}
      \\&
    - \L(\partial_\theta c\R) \partial_i \ve{d} \cdot\partial_\rho \ve{e}^i
    - \L(\partial_i c\partial_\theta\ve{e}^i\R) \cdot \partial_\rho \ve{d}
    + \L(\partial_\rho c\R) \partial_i \ve{d} \cdot\partial_\theta \ve{e}^i
    + \L(\partial_i c\partial_\rho\ve{e}^i\R) \cdot \partial_\theta \ve{d}
    \note{$;$ denotes the dot-product}
    \\
    %
    =&
      \{c, \grad_\perp\cdot \ve{d}\} + \{\grad_\perp c; \ve{d}\}
     + \mathcal{G}
\end{align*}
%
where $\mathcal{G}$ is the correction coming from the geometry.
We have
%
\begin{align*}
    \mathcal{G}
    =&
    - \L(\partial_\theta c\R) \partial_i \ve{d}\cdot \partial_\rho \ve{e}^i
    \\&
    - \L(\partial_i c\partial_\theta\ve{e}^i\R) \cdot \partial_\rho \ve{d}
    \\&
    + \L(\partial_\rho c\R) \partial_i \ve{d} \cdot\partial_\theta \ve{e}^i
    \\&
    + \L(\partial_i c\partial_\rho\ve{e}^i\R) \cdot \partial_\theta \ve{d}
    \\
    %
    =&
    - \L(\partial_\theta c\R) \partial_\rho \ve{d} \cdot\partial_\rho \ve{e}^\rho
    - \L(\partial_\theta c\R) \partial_\theta \ve{d} \cdot\partial_\rho \ve{e}^\theta
    \\&
    - \L(\partial_\rho c\partial_\theta\ve{e}^\rho\R) \cdot \partial_\rho \ve{d}
    - \L(\partial_\theta c\partial_\theta\ve{e}^\theta\R) \cdot \partial_\rho \ve{d}
    \\&
    + \L(\partial_\rho c\R) \partial_\rho \ve{d} \cdot\partial_\theta \ve{e}^\rho
    + \L(\partial_\rho c\R) \partial_\theta \ve{d} \cdot\partial_\theta \ve{e}^\theta
    \\&
    + \L(\partial_\rho c\partial_\rho\ve{e}^\rho\R) \cdot \partial_\theta \ve{d}
    + \L(\partial_\theta c\partial_\rho\ve{e}^\theta\R) \cdot \partial_\theta \ve{d}
    \\
    %
    =&
    - 0
    - \L(\partial_\theta c\R) \partial_\theta \ve{d} \cdot\L(-\frac{1}{\rho}\ve{e}^\theta\R)
    \\&
    - \rho\ve{e}^\theta\L(\partial_\rho c\R) \cdot \partial_\rho \ve{d}
    - \L(-\frac{1}{\rho}\ve{e}^\rho\R)\L(\partial_\theta c\R) \cdot \partial_\rho \ve{d}
    \\&
    + \L(\partial_\rho c\R) \partial_\rho \ve{d} \cdot\L(\rho \ve{e}^\theta\R)
    + \L(\partial_\rho c\R) \partial_\theta \ve{d} \cdot\L(-\frac{1}{\rho}\ve{e}^\rho\R)
    \\&
    + 0
    + \L(-\frac{1}{\rho}\ve{e}^\theta\R)\L(\partial_\theta c\R) \cdot \partial_\theta \ve{d}
    \\
    %
    =&
     \frac{1}{\rho}\ve{e}^\theta \cdot \L(\partial_\theta c\R) \partial_\theta \ve{d}
    -\frac{1}{\rho}\ve{e}^\theta \cdot \L(\partial_\theta c\R) \partial_\theta \ve{d}
    \\&
    - \rho \ve{e}^\theta \cdot \L(\partial_\rho c\R) \partial_\rho \ve{d}
    + \rho \ve{e}^\theta \cdot \L(\partial_\rho c\R) \partial_\rho \ve{d}
    \\&
    - \frac{1}{\rho}\ve{e}^\rho \cdot \L(\partial_\rho   c\R)\partial_\theta \ve{d}
    + \frac{1}{\rho}\ve{e}^\rho \cdot \L(\partial_\theta c\R)\partial_\rho   \ve{d}
    \\
    %
    =&
    \frac{1}{\rho}\ve{e}^\rho \cdot \L\{c, \ve{d}\R\}
\end{align*}
%
thus, expansion of the second term in equation (\ref{vortD:firstDeriv}) gives
%
\begin{align*}
    \frac{1}{\om_{ci}}
    \frac{1}{JB}\grad_\perp\cdot\L\{\phi, n\frac{\grad_\perp \phi}{B} \R\}
    &=
    \frac{1}{\om_{ci}}
    \frac{1}{B\rho}\L(
       \L\{\phi, \grad_\perp\cdot n\frac{\grad_\perp \phi}{B} \R\}
       + \L\{\grad_\perp \phi; n\frac{\grad_\perp \phi}{B} \R\}
     + \frac{1}{\rho}\ve{e}^\rho \cdot \L\{\phi, n\frac{\grad_\perp \phi}{B}\R\}
    \R)
    \\
    &=
    \frac{1}{\om_{ci}}
    \frac{1}{B\rho}\{\phi, \Om^D\}
    +
    \frac{1}{\om_{ci}}
    \frac{1}{B\rho}
    \L\{\grad_\perp \phi; n\frac{\grad_\perp \phi}{B}\R\}
    +
    \frac{1}{\om_{ci}}
    \frac{1}{B\rho^2}
    \ve{e}^\rho \cdot \L\{\phi, n\frac{\grad_\perp \phi}{B}\R\}
    \numberthis
    \label{vortD:secondTerm}
\end{align*}
%
Combining equation (\ref{vortD:firstTerm}) and equation (\ref{vortD:secondTerm}), we get that
%
\begin{align*}
    \frac{1}{\om_{ci}}
    \div\L(\ve{u}_E\cdot\grad\L[n\frac{\grad_\perp \phi}{B} \phi\R]\R)
    =&
    -
    \frac{1}{\om_{ci}}
    \L\{\phi, n\frac{\grad_\perp \phi}{B} \phi\R\}\cdot \ve{e}^\rho \frac{1}{B\rho^2}
    \\ &
    +
    \frac{1}{\om_{ci}}
    \frac{1}{B\rho}\{\phi, \Om^D\}
    +
    \frac{1}{\om_{ci}}
    \frac{1}{B\rho}\L\{\grad_\perp \phi; n\frac{\grad_\perp \phi}{B}\R\}
    \\ &
    +
    \frac{1}{\om_{ci}}
    \frac{1}{B\rho^2}\ve{e}^\rho \cdot \L\{\phi, n\frac{\grad_\perp \phi}{B} \R\}
    \\
    %
    =&
    \frac{1}{\om_{ci}}
    \frac{1}{B\rho}\{\phi, \Om^D\}
    +
    \frac{1}{\om_{ci}}
    \frac{1}{B\rho}\L\{\grad_\perp \phi; n\frac{\grad_\perp \phi}{B} \phi\R\}
    \note{Product rule}
    \\
    %
    =&
    \frac{1}{\om_{ci}}
    \frac{1}{B\rho}\{\phi, \Om^D\}
    +
    \frac{1}{\om_{ci}}
    n\frac{1}{B\rho}\L\{\grad_\perp \phi; \frac{\grad_\perp \phi}{B}\R\}
    \note{$\L\{a,\frac{a}{B}\R\}=\frac{1}{B}\L\{a,a\R\}=0$}
    \\ &
    +
    \frac{1}{\om_{ci}}
    \frac{1}{B\rho}
    \frac{\grad_\perp \phi}{B}
    \cdot\L\{\grad_\perp \phi, n\R\}
    \\
    %
    =&
    \frac{1}{\om_{ci}}
    \frac{1}{B\rho}\{\phi, \Om^D\}
    +
    \frac{1}{\om_{ci}}
    \frac{1}{B\rho}
    \frac{\grad_\perp \phi}{B}
    \cdot\L\{\grad_\perp \phi, n\R\}
    \note{$\partial_i (ff) = 2f\partial f$}
    \\
    %
    =&
    \frac{1}{\om_{ci}}
    \frac{1}{B\rho}\{\phi, \Om^D\}
    +
    \frac{1}{\om_{ci}}
    \frac{1}{2B^2\rho}\{(\grad_\perp \phi)^2, n\}
    \note{Constant $B$}
    \\
    %
    =&
    \frac{1}{\om_{ci}}
    \frac{1}{B\rho}\{\phi, \Om^D\}
    +
    \frac{1}{\om_{ci}}
    \frac{1}{2\rho}\L\{\L(\frac{\grad_\perp \phi}{B}\R)^2, n\R\}
    \\
    %
    =&
    \frac{1}{\om_{ci}}
    \frac{1}{B\rho}\{\phi, \Om^D\}
    +
    \frac{1}{\om_{ci}}
    \frac{1}{2\rho}\{\ve{u}_E^2, n\}
\end{align*}
%
% Checking that the dimensions are OK
% First term
% ----------
% 1/om_ci = T
% 1/rho = L^-1
% partial_rho phi/B = LT^-1
% vortD = T^-1L^-3
% Product:
% TL^-1LT^-1T^-1L^-3 = T^-1L^-3
%
% Second term
% ----------
% 1/om_ci = T
% 1/rho = L^-1
% partial_rho = L^-1
% ue^2 = L^2T^-2
% n = L^-3
% Product:
% TL^-1L^-1L^2T^-2L^-3 = T^-1L^-3
%
% This fits with 1/om_ci partial_t vortD
