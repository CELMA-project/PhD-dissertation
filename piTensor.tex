We have that in a cartesian coordinate system
CITE HELANDER AND SIGMAR
%
\begin{align*}
    W^{ij}_{\a} \defined \partial_j u^{i}_{\a}
                     + \partial_i u^{j}_{\a}
                     - \frac{2}{3} \delta_i^j \div \ve{u}_{\a}
\end{align*}
%
Using that to first order, only the $\ve{E}\times\ve{B}$ drift advects
particles perpendiculary, we have that
$\ve{u}_{\a} \simeq \ve{u}_{E} + \ve{u}_{\a, \|}$.
%
We will then have that
%
\begin{align*}
    \div \ve{u}_{\a}
    &\simeq \div \ve{u}_{E} + \div \ve{u}_{\a, \|}
    \note{Eq. \ref{eq:curv_op} when $\ve{B}=$ constant}
    \\
    &= 0 + \div \ve{u}_{\a, \|}
\end{align*}
%
In Clebsch coordinates, we have that $\ve{u}_{E}$ is given by equation
\ref{app:ExB}, which means that in cylindrical coordinates, we will have
CHECK HANDEDNESS!!!
%
\begin{align*}
    \ve{u}_E
    =&\frac{1}{JB}
           \L(
           - g_{zz}\ve{e}_y \partial_x
           + g_{zz}\ve{e}_x  \partial_y
           \R)
           \phi
   =\frac{1}{B}
           \L(
           - \ve{e}_y \partial_x
           + \ve{e}_x  \partial_y
           \R)
           \phi
    \\
    %
    u_{E}^{x} =& \frac{1}{B} \partial_y \phi
    \qquad
    u_{E}^{y} = -\frac{1}{B} \partial_x \phi
    \qquad
    u_{E}^{z} = 0
\end{align*}
%
and
%
\begin{align*}
    u_{\a, \|}^{x} = 0 \qquad
    u_{\a, \|}^{y} = 0 \qquad
    u_{\a, \|}^{z} = u_{\a, \|}^{z}
\end{align*}
%
This means that
%
\begin{align*}
W^{xx} + W^{yy}
=&
\partial_x u^{x}_{\a}
+ \partial_x u^{x}_{\a}
- \frac{2}{3} \div \ve{u}_{\a}
+ \partial_y u^{y}_{\a}
+ \partial_y u^{y}_{\a}
- \frac{2}{3} \div \ve{u}_{\a}
\\
%
=&
2\partial_x u^{x}_{\a}
+ 2\partial_y u^{y}_{\a}
- \frac{4}{3} \div \ve{u}_{\a}
\\
%
=&
2\L(\partial_x u^{x}_{\a}
    + \partial_y u^{y}_{\a}
    - \frac{2}{3} \L[\partial_x u^{x}_{\a}
          +\partial_y u^{y}_{\a}
          +\partial_z u^{z}_{\a}
                      \R]
\R)
\\
%
=&
\frac{2}{3}\L(\partial_x u^{x}_{\a}
          + \partial_y u^{y}_{\a}
          - 2\partial_z u^{z}_{\a}
\R)
\\
%
=&
\frac{2}{3B}
\L(\partial_x \partial_y \phi
- \partial_y \partial_x \phi
\R)
-
\frac{4}{3}
 \partial_z u^{z}_{\a,\|}
\\
%
=&
-
\frac{4}{3}
\partial_z u^{z}_{\a,\|}
\\
%
%
%
W^{xx} - W^{yy}
=&
\partial_x u^{x}_{\a}
+ \partial_x u^{x}_{\a}
- \frac{2}{3} \div \ve{u}_{\a}
- \partial_y u^{y}_{\a}
- \partial_y u^{y}_{\a}
+ \frac{2}{3} \div \ve{u}_{\a}
\\
%
=&
2\partial_x u^{x}_{\a} -2 \partial_y u^{y}_{\a}
\\
%
=&
2\L(\partial_x u^{x}_{\a} - \partial_y u^{y}_{\a} \R)
%
\\
=&
\frac{2}{B}
\L(\partial_x \partial_y \phi + \partial_y \partial_x \phi \R)
%
\\
=&
\frac{4}{B} \partial_x \partial_y \phi
\\
%
%
%
%
W^{zz}
=&
\partial_z u^{z}_{\a} + \partial_z u^{z}_{\a} - \frac{2}{3} \div \ve{u}_{\a}
\\
%
=&
2\partial_z u^{z}_{\a} - \frac{2}{3}
    \L(\partial_x u^{x}_{\a}
       +\partial_y u^{y}_{\a}
       +\partial_z u^{z}_{\a}
    \R)
\\
%
=&
\frac{4}{3} \partial_z u^{z}_{\a} - \frac{2}{3}
    \L(\partial_x u^{x}_{\a}
       +\partial_y u^{y}_{\a}
    \R)
\\
%
=&
\frac{2}{3}
2 \partial_z u^{z}_{\a,\|} -
    \frac{1}{B}
    \L(\partial_x \partial_y \phi
       - \partial_y \partial_x \phi
    \R)
\\
%
=& \frac{4}{3} \partial_z u^{z}_{\a,\|}
\\
%
%
%
%
W^{xy}
=&
\partial_x u^{y}_{\a} + \partial_y u^{x}_{\a}
\\
%
=&
\frac{1}{B}\L( - \partial_x^2 \phi + \partial_y^2 \phi \R)
\\
%
%
%
%
W^{xz}
=&
\partial_x u^{z}_{\a} + \partial_z u^{x}_{\a}
%
\\
=&
\partial_x u^{z}_{\a, \|} + \frac{1}{B}\partial_z \partial_y \phi
\\
%
%
%
%
W^{yz}
=&
\partial_y u^{z}_{\a} + \partial_z u^{y}_{\a}
%
\\
=&
\partial_y u^{z}_{\a,\|} - \frac{1}{B} \partial_z \partial_x \phi
\end{align*}
%
WAIT WAIT WAIT: WHAT ABOUT CALCULATING IN CARTESIAN SYSTEM, THEN CONVERT THE
TENSOR TO CYLINDRICAL

YES: CHECK EXAMPLE 4 IN DIFF GEOM NOTE
%
Although no difference between co and contravariant in Cartesian, we here write
components contravariant. Calculate the tensor in Cartesian coordinates, then
transform it to cylindrical coordinates
%
\begin{align*}
    \pi^{xx}_{\a}
    %
    =& -\frac{\eta_{0,\a}}{2}\L(W^{xx} + W^{yy}\R)
       -\frac{\eta_{1,\a}}{2}\L(W^{xx} - W^{yy}\R)
       -\eta_3 W^{xy}
    \\
    %
    =&  \frac{2\eta_{0,\a}}{3}\partial_z u^{z}_{\a,\|}
       -\frac{2\eta_{1,\a}}{B}\partial_x \partial_y \phi
       -\frac{ \eta_{3,\a}}{B}\L(-\partial_x^2 \phi
       + \partial_y^2 \phi \R)
    \\
    %
    %
    %
\pi^{yy}_{\a}
    =& -\frac{\eta_{0,\a}}{2}\L(W^{xx} + W^{yy}\R)
       -\frac{\eta_{1,\a}}{2}\L(W^{yy} - W^{xx}\R)
       +\eta_{3,\a}W^{xy}
    \\
    %
    =&  \frac{2\eta_{0,\a}}{3}\partial_z u^{z}_{\a,\|}
       +\frac{2\eta_{1,\a}}{B}\partial_x \partial_y \phi
       +\frac{ \eta_{3,\a}}{B}\L(-\partial_x^2 \phi + \partial_y^2 \phi \R)
    \\
    %
    %
    %
\pi^{zz}_{\a}
    =& -\eta_{0,\a}W^{zz}
    \\
    %
    =& - \frac{4\eta_{0,\a}}{3} \partial_z u^{z}_{\a,\|}
    \\
    %
    %
    %
\pi^{xy}_{\a} =& \pi^{yx}_{\a}
    = -\eta_{1,\a}W^{xy} -\frac{\eta_{3,\a}}{2} \L(W^{xx}-W^{yy}\R)
    \\
    %
    =& -\frac{\eta_{1,\a}}{B}\L(-\partial_x^2 \phi + \partial_y^2 \phi \R)
    -\frac{2\eta_{3,\a}}{B}\partial_x \partial_y \phi
    \\
    %
    %
    %
\pi^{xz}_{\a} =& \pi^{zx}_{\a}
    = -\eta_{2,\a}W_{xz} - \eta_{4,\a}W_{yz}
    \\
    %
    =& -\eta_{2,\a}\L(\partial_x u^{z}_{\a, \|} + \frac{1}{B}\partial_z \partial_y \phi\R)
        - \eta_{4,\a}\L(\partial_y u^{z}_{\a,\|} - \frac{1}{B} \partial_z \partial_x \phi\R)
    \\
    %
    %
    %
\pi^{yz}_{\a} =& \pi^{zy}_{\a}
    = -\eta_{2,\a}W_{yz} + \eta_{4,\a}W_{xz}
    \\
    %
    =& -\eta_{2,\a}\L(\partial_y u^{z}_{\a,\|} - \frac{1}{B} \partial_z \partial_x \phi\R)
    + \eta_{4,\a}\L(\partial_x u^{z}_{\a, \|} + \frac{1}{B}\partial_z \partial_y \phi\R)
\end{align*}
%
We have that
%
\begin{align*}
    &\eta_{0,i}=0.96n_iT_i\nu_{ei} &
    &\eta_{1,i}=\frac{3n_iT_i}{10\Om_{ci}^2 \sqrt{2}\nu_{ii}} &
    &\eta_{2,i}=\frac{12n_iT_i}{10\Om_{ci}^2\sqrt{2}\nu_{ii}} &
    &\eta_{3,i}=\frac{n_iT_i}{2\Om_{ci}} &
    &\eta_{4,i}=\frac{n_iT_i}{\Om_{ci}} &
    \\
    &\eta_{0,e}=0.73n_eT_e\nu_{ei} &
    &\eta_{1,e}=0.51\frac{n_eT_e}{\Om_{ce}^2\nu_{ei}} &
    &\eta_{2,e}=2.04\frac{n_eT_e}{\Om_{ce}^2\nu_{ei}} &
    &\eta_{3,e}=\frac{n_eT_e}{2\Om_{ce}} &
    &\eta_{4,e}=\frac{n_eT_e}{\Om_{ce}} &
\end{align*}
Q: How to transform the partial derivatives?
A: See diff geom note (use chain rule)

Q: How to transform the components?
A:


WAIT WAIT WAIT: CALCULATE EVERYTHING IN CARTESIAN
TRANSFORM ONLY DIVERGENCE OF PI TENSOR
%NO PROBLEM WITH COMPONENTS as u^x, AS f(x)=f(x(x'))

After calculated the pi tensor in cartesian: Components needs to be
recalculated
Q: How to transform the partial derivatives?
A: See diff geom note (use chain rule)

Q: How to transform the components?
A:


WAIT WAIT WAIT: CALCULATE EVERYTHING IN CARTESIAN
TRANSFORM ONLY DIVERGENCE OF PI TENSOR
%NO PROBLEM WITH COMPONENTS as u^x, AS f(x)=f(x(x'))

After calculated the pi tensor in cartesian: Components needs to be
recalculated
