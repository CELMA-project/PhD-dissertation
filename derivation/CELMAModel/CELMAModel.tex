% FIXME: RELATIVE ERROR RATHER THAN ABSOLUTE ERROR IN MES
We will now use the above derived equation in a homogeneous magnetic field, assuming cold ions and constant electron temperature in a cylindrical geometry.
The resulting model will from here be referred to as the CELMA model, standing for \textbf{C}onsistent \textbf{E}quations in a \textbf{L}inear \textbf{MA}chine

\section{The density equation}
%
We will now rewrite the electron density equation (\cref{eq:dens_evol_gen}) using that we are in a straight magnetic field (so that $B=\text{const}$ and $\mathcal{C}(f)=0$) using the assumptions that the electron temperature is constant and the ion temperature is negligible.
This yields
%
\begin{align*}
    \d_t n
    &=
    - n\mathcal{C}(\phi)
    + \frac{1}{e}\mathcal{C}(\phi)
    +\frac{1}{\mu}\frac{m_i\nu_{ei}}{e^2}
    \div\L( \frac{\grad_\perp \L(p_e + p_i\R)}{B^2} \R)
    - \div\L(n \ve{u}_{e,\|}\R)
    + S_{n}
    \note{$p_i=0$\\$B=\text{const}$}
    \\
%
    &=
  \frac{1}{\mu}
  \frac{m_iT_e\nu_{ei}}{B^2e^2}
   \grad_\perp^2 n
   - \div\L(n \ve{b}u_{e,\|}\R)
   + S_n
   \note{$\partial_i \ve{b} = 0$}
    \\
%
    &=
  \frac{1}{\mu}
  \frac{m_iT_e\nu_{ei}}{B^2e^2}
   \grad_\perp^2 n
   - \ve{b}\cdot\grad \L(n u_{e,\|}\R)
   + S_n
    \\
%
    &=
  \frac{1}{\mu}
  \frac{m_iT_e\nu_{ei}}{B^2e^2}
   \grad_\perp^2 n
   - \partial_\| \L(n u_{e,\|}\R)
   + S_n
\end{align*}
%
Using that
$\rho_s=\frac{c_s}{\om_{ci}}=\sqrt{\frac{T_e}{m_i}}\frac{m_i}{eB}
       =\sqrt{\frac{T_em_i}{e^2B^2}}$
we find that
%
\begin{align}
    \d_t n
    &=
  \frac{\rho_s^2\nu_{ei}}{\mu}
   \grad_\perp^2 n
   - \partial_\|\L(n u_{e,\|} \R)
   + S_n
    \label{eq:non_norm_dens}
\end{align}

\section{The vorticity equation}
%
We will now define the vorticity
%
\begin{align*}
    \frac{\grad^2_\perp \phi}{B} \defined \Om
\end{align*}
%
This is in principle not the full vorticity, but rather the parallel part of the advective vorticity, which has is analogue to what we find in fluid mechanics
%
\begin{align*}
    \L(\curl \ve{u}_E\R)\cdot\ve{b}
    &=
    \L(-\curl \frac{\grad \phi \times \ve{b}}{B}\R)\cdot\ve{b}
    \note{$\grad \frac{1}{B}\simeq 0$}
    \\
%
    &=
    \frac{1}{B}\L(\curl \ve{b} \times \grad \phi \R)\cdot\ve{b}
    \note{$\curl (\ve{A}\times\ve{B}) = \ve{A}(\div\ve{B}) - \ve{B}(\div\ve{A})
                        + (\ve{B}\cdot\grad)\ve{A} - (\ve{A}\cdot\grad)\ve{B}$}
    \\
%
    &=
    \frac{1}{B}\L(   \ve{b}\L[\div\grad\phi\R]
                   - \grad\phi\L[\div\ve{b}\R]
                   + \L[\grad\phi\cdot\grad\R]\ve{b}
                   - \L[\ve{b}\cdot\grad\R]\grad\phi
               \R)\cdot\ve{b}
    \\
%
    &=
    \frac{1}{B}\L(   \ve{b}\L[\div\grad\phi\R]
                   - \L[\ve{b}\cdot\grad\R]\grad\phi
               \R)\cdot\ve{b}
    \note{$\div \ve{b} \simeq 0$ and $\grad \ve{b} \simeq 0$}
    \\
%
    &=
    \frac{1}{B}\L( \ve{b} \cdot \ve{b}\L[\div\grad\phi\R]
                   - \ve{b} \cdot \L[\ve{b}\cdot\grad\R]\grad\phi \R)
    \\
%
    &=
    \frac{1}{B}\L( \grad^2\phi
                   - \ve{b} \cdot \L[\grad\ve{b}\cdot\R]\grad\phi \R)
               \\
%
    &=
    \frac{1}{B}\L( \grad^2\phi
                   - \L[ \ve{b} \cdot \grad \R]\ve{b}\cdot\grad\phi \R)
               \\
%
    &=
    \frac{1}{B}\L( \grad^2\phi
                   - \div \L[ \ve{b} \ve{b}\cdot\grad\phi \R] \R)
               \\
%
    &=
    \frac{1}{B}\L(\grad_\perp^2\phi \R)
\end{align*}
%
Where we in the last line have used that
$
\grad^2_\perp = \grad^2 - \grad_\|^2 = \grad^2 -
                   \div \L[ \ve{b} \ve{b}\cdot\grad \R]
$
%
We will in light of this, using the current magnetic field topology, investigate the left hand side of \cref{eq:full_vort_eq} term by term.

\subsection{The diamagnetic contribution}
%
\begin{align*}
    \frac{1}{e}\L[\mathcal{C}(p_e+p_i)\R]
\end{align*}
%
disappears as $\mathcal{C}(f)$ vanishes for a straight magnetic field.

\subsection{The neutral contribution}
%
The two next terms, origin from the Pedersen drift \cref{eq:div_ped}, can be rewritten using that we are dealing with cold ions.
We get
%
\begin{align*}
    \div\L(n\ve{u}_{i,\text{Ped}} \R)
    =&
    n \div\L(\frac{\nu_{in}}{\om_{ci}} \frac{\grad_\perp \phi}{B} \R)
    + \L(\frac{\nu_{in}}{\om_{ci}} \frac{\grad_\perp \phi}{B} \R) \cdot\grad n
    \note{Const $B$}
    \\
%
%
    =&
    n \frac{\nu_{in}}{\om_{ci}} \frac{\grad_\perp^2 \phi}{B}
    + \frac{\nu_{in}}{\om_{ci}} \frac{\grad_\perp \phi}{B} \cdot\grad n
    \\
%
%
  = &
 \frac{\nu_{in}}{\om_{ci}} \L(n\Om + \frac{\grad_\perp \phi}{B} \cdot \grad n\R)
\end{align*}
%

\subsection{The polarization contribution}
%
Further on, we will investigate the terms arising from the sum of the divergence of the ion polarization drift and ion viscosity drift multiplied with the density. The first term yields
%
\begin{align*}
 &
 \div\L( \frac{1}{\om_{ci}}
 \L[ \d_t + \ve{u}_{i,\|}\cdot\nabla \R]
 \L[\frac{\grad_\perp \phi}{B}n\R] \R)
    \\
    %
    =& \div\L( \frac{1}{\om_{ci}}\partial_t\L[\frac{\grad_\perp \phi}{B}n \R]\R)
    + \div\L(
    \frac{1}{\om_{ci}} \ve{u}_E\cdot\nabla \L[\frac{\grad_\perp \phi}{B} n\R]\R)
    + \div\L(
    \frac{1}{\om_{ci}} \ve{u}_{i,\|}\cdot\nabla\L[\frac{\grad_\perp \phi}{B}n\R]\R)
    \note{$B=\text{const}$}
    \\
    %
    =& \frac{1}{\om_{ci}}\div\L(\partial_t\L[\frac{\grad_\perp \phi}{B}n \R]\R)
    + \frac{1}{\om_{ci}} \div\L(
    \ve{u}_E\cdot\nabla \L[\frac{\grad_\perp \phi}{B}n \R]\R)
    + \frac{1}{\om_{ci}} \div\L(
    u_{i,\|}\ve{b}\cdot\nabla\L[\frac{\grad_\perp \phi}{B}n\R]\R)
    \note{Assume interchangibility of derivatives}
    \\
    %
    =& \frac{1}{\om_{ci}}\partial_t\L(\div\L[\frac{\grad_\perp \phi}{B}n \R]\R)
    + \frac{1}{\om_{ci}} \div\L(
    \ve{u}_E\cdot\nabla \L[\frac{\grad_\perp \phi}{B}n \R]\R)
    + \frac{1}{\om_{ci}} \div\L(
    u_{i,\|}\partial_\|\L[\frac{\grad_\perp \phi}{B}n\R]\R)
\end{align*}
%
We now define
%
\begin{align*}
    \Om^D \defined
    \div \L(n \frac{\grad_\perp \phi}{B} \R)
\end{align*}
%
which gives
%
\begin{align*}
    \frac{1}{\om_{ci}}\partial_t\Om^D
    + \frac{1}{\om_{ci}} \div\L(
    \ve{u}_E\cdot\nabla \L[\frac{\grad_\perp \phi}{B}n \R]\R)
    + \frac{1}{\om_{ci}} \div\L(
    u_{i,\|}\partial_\|\L[\frac{\grad_\perp \phi}{B}n\R]\R)
\end{align*}
%
As we have no curvature, the second term gives
%
\begin{align*}
    &
 - \div\L( \frac{1}{\om_{ci}}
 \frac{\grad_\perp \phi}{B}
 \L[S_{n} - \frac{1}{e}\mathcal{C}(p_i) - n \mathcal{C}(\phi)
 - n \div \ve{u}_{i,\|} \R] \R)
 \\
 =&
 - \frac{1}{\om_{ci}} \div\L(
 \frac{\grad_\perp \phi}{B}
 \L[S_{n} - n \div \L(\ve{b}u_{i,\|}\R) \R] \R)
 \note{$\partial_i \ve{b}=0$}
 \\
 =&
 - \frac{1}{\om_{ci}} \div\L(
 \frac{\grad_\perp \phi}{B}
 \L[S_{n} - n \ve{b}\cdot\grad u_{i,\|} \R] \R)
 \\
 =&
 - \frac{1}{\om_{ci}} \div\L(
 \frac{\grad_\perp \phi}{B}
 \L[S_{n} - n \partial_\| u_{i,\|} \R] \R)
 \\
 =&
 - \frac{1}{\om_{ci}} \div\L(
 \frac{\grad_\perp \phi}{B}
 S_{n} \R)
 + \frac{1}{\om_{ci}} \div\L(
 \frac{\grad_\perp \phi}{B}
 n \partial_\| u_{i,\|} \R)
\end{align*}
% TODO: Delete me if not included in thesis
%       Boussinesq part
This means that the polarization contribution from \cref{eq:full_vort_eq} can be written as
%
\begin{align*}
    &
    \frac{1}{\om_{ci}}\partial_t\Om^D
    + \frac{1}{\om_{ci}} \div\L(
    \ve{u}_E\cdot\nabla \L[\frac{\grad_\perp \phi}{B}n \R]\R)
    + \frac{1}{\om_{ci}} \div\L(
    u_{i,\|}\partial_\|\L[\frac{\grad_\perp \phi}{B}n\R]\R)
    \\
    -&
  \frac{1}{\om_{ci}} \div\L( \frac{\grad_\perp \phi}{B} S_{n} \R)
 + \frac{1}{\om_{ci}}
 \div\L( \frac{\grad_\perp \phi}{B} n \partial_\| u_{i,\|} \R)
\end{align*}

\subsection{The source contribution}
%
Further on, the contribution from the source can be rewritten.
Under the assumption of cold ions ($T_i = 0$), we can simplify the source term in \cref{eq:full_vort_eq} to be
%
\begin{align*}
    \div \L( \frac{ S_{i,n}}{\om_{ci}}
      \L[ \frac{\grad_\perp p_i}{n e B} + \frac{\grad_\perp \phi}{B} \R]
    \R)
    =&
    \div \L( \frac{ S_{i,n}}{\om_{ci}} \L[ \frac{\grad_\perp \phi}{B} \R] \R)
    \note{$T_i \simeq 0$\\ Const $B$\\ $S_{\a,n}=S_n$}
    \\
%
%
    =&
    \frac{1}{\om_{ci}} \div \L( S_n \L[ \frac{ \grad_\perp \phi }{ B } \R] \R)
\end{align*}
%

\subsection{The parallel contribution}
%
Next, the RHS of \cref{eq:full_vort_eq} reads
%
\begin{align*}
    \div \L(n [\ve{u}_{i,\|} - \ve{u}_{e,\|}]\R)
    &=
    \div \L(n [\ve{b}u_{i,\|} - \ve{b}u_{e,\|}]\R)
    \note{$\partial_i \ve{b} = 0$}
    \\
    %
    %
    &=
    \ve{b}\cdot\nabla \L(n [u_{i,\|} - u_{e,\|}]\R)
    \\
    %
    %
    &=
    \partial_\| \L(n [u_{i,\|} - u_{e,\|}]\R)
\end{align*}
%

\subsection{Collecting terms}
\label{sec:CELMACollect}
%
From the calculations above, \cref{eq:full_vort_eq} can be rewritten to
%
\begin{align*}
  %
  &
  \quad
 \frac{\nu_{in}}{\om_{ci}} \L(n\Om + \frac{\grad_\perp \phi}{B} \cdot \grad n\R)
  \\
 &
 + \frac{1}{\om_{ci}} \partial_t \Om^D
 + \frac{1}{\om_{ci}} \div
 \L(
 \ve{u}_E\cdot\nabla \L[\frac{\grad_\perp \phi}{B}n \R]
 + u_{i,\|}\partial_\|\L[\frac{\grad_\perp \phi}{B}n\R]
 + \frac{\grad_\perp \phi}{B} n \partial_\| u_{i,\|}
 \R)
 - \frac{1}{\om_{ci}} \div\L( \frac{\grad_\perp \phi}{B} S_{n} \R)
 \\
 %
 &
 + \frac{1}{\om_{ci}}
    \div \L( S_n \L[ \frac{ \grad_\perp \phi }{ B } \R] \R)
 \\
 %
 =&
 \partial_\| \L(n [u_{i,\|} - u_{e,\|}]\R)
\end{align*}
%
Rearranging yields
%
\begin{align*}
  %
  \frac{1}{\om_{ci}}
  \partial_t \Om^D
  =&
  - \frac{\nu_{in}}{\om_{ci}} \L(n\Om + \frac{\grad_\perp \phi}{B} \cdot \grad n\R)
  \\
  %
  &
  - \frac{1}{\om_{ci}} \div
 \L(
 \ve{u}_E\cdot\nabla \L[\frac{\grad_\perp \phi}{B}n \R]
 + u_{i,\|}\partial_\|\L[\frac{\grad_\perp \phi}{B}n\R]
 + \frac{\grad_\perp \phi}{B} n \partial_\| u_{i,\|}
 \R)
 \\
 &
 %
 + \partial_\| \L(n [u_{i,\|} - u_{e,\|}]\R)
 \numberthis
 \label{eq:non_norm_vort_1}
\end{align*}
%
We note that $\frac{1}{\om_{ci}} \div\L( \frac{\grad_\perp \phi}{B} S_{n} \R)$ arising from the time derivative of $n_i$ in the polarization term of the current conservation equation, cancels with the same term with opposite sign arising from the source term of the current conservation equation.

\subsection{Vector advective terms}
\label{sec:vecAdvTerm}
%
We can simplify \cref{eq:non_norm_vort_1} even further by first observing that
%
\begin{align*}
u_{i,\|}\partial_\|\L[\frac{\grad_\perp \phi}{B}n\R]
+ \frac{\grad_\perp \phi}{B} n \partial_\| u_{i,\|}
=
\partial_\| \L( u_{i,\|}\frac{\grad_\perp \phi}{B}n \R),
\end{align*}
%
and using the fact that in cylindrical coordinates we have that $\partial_z \ve{e}_i = \partial_z \ve{e}^i = 0$.
This yields
%
\begin{align*}
  %
  \frac{1}{\om_{ci}}
  \partial_t \Om^D
  =&
  - \frac{\nu_{in}}{\om_{ci}} \L(n\Om + \frac{\grad_\perp \phi}{B} \cdot \grad n\R)
  \\
  %
  &
  - \frac{1}{\om_{ci}} \div
 \L(
 \ve{u}_E\cdot\nabla \L[\frac{\grad_\perp \phi}{B}n \R]
 \R)
  - \frac{1}{\om_{ci}} \partial_\|\div
 \L( u_{i,\|}\frac{\grad_\perp \phi}{B}n \R)
 \\
 &
 %
 + \partial_\| \L(n [u_{i,\|} - u_{e,\|}]\R)
 \numberthis
 \label{eq:non_norm_vort_2}
\end{align*}
%
Note that the
%
$ - \frac{1}{\om_{ci}} \div
\L( u_{i,\|}\partial_\|\L[\frac{\grad_\perp \phi}{B}n\R] \R) $
%
term arises from the parallel advection in the polarization term, whereas the
%
$ - \frac{1}{\om_{ci}} \div
 \L( \frac{\grad_\perp \phi}{B} n \partial_\| u_{i,\|} \R) $
%
term arises from the ion continuity equation.

Next, we see from \cref{app:vortDAdv} that
%
\begin{align*}
 \frac{1}{\om_{ci}}
  \div
  \L( \ve{u}_E\cdot\nabla \L[\frac{\grad_\perp \phi}{B}n \R]\R)
  =&
  \frac{1}{\om_{ci}}
  \frac{1}{B\rho}\{\phi, \Om^D\}
  +
  \frac{1}{\om_{ci}}
  \frac{1}{2\rho}\{\ve{u}_E^2, n\}
\end{align*}
%
A similar expression is expected to be found for other geometries as well, at least as long as the $B$ field is constant.
When this is not the case, terms arising from $\grad \frac{1}{B}$ are expected.

\subsection{The complete vorticity equation}
%
The evolution of the modified vorticity can now be written as
%
\begin{align*}
  %
  \frac{1}{\om_{ci}}
  \partial_t \Om^D
  =&
  - \frac{\nu_{in}}{\om_{ci}} \L(n\Om + \frac{\grad_\perp \phi}{B} \cdot \grad n\R)
  \\
  %
  &
  -
 \frac{1}{\om_{ci}\rho}
 \L(
  \frac{1}{B}\{\phi, \Om^D\}
  +
  \frac{1}{2}\{\ve{u}_E^2, n\}
 \R)
  -
 \frac{1}{\om_{ci}} \partial_\|\div
 \L( u_{i,\|}\frac{\grad_\perp \phi}{B}n \R)
 \\
 &
 %
 + \partial_\| \L(n [u_{i,\|} - u_{e,\|}]\R)
 \numberthis
 \label{eq:complVort}
\end{align*}
