\section{Boundary conditions}
%
Setting correct boundary condition when calculating PDEs of plasma quantities turns out to be a challenge.
One of the reasons for this is the formation of plasma sheaths between the plasma and the material which is due to the difference in mobility between the ion species and the electron species.
This leads to a potential build-up on the material surface which affects the plasma upstream.

Unfortunately the fluid description of the plasma breaks down at the sheath as mentioned in \cite{Loizu2012a}.
Hence, a proper description in this area can only be accounted for with kinetic codes, solving for example the Fokker-Planck equation.
In other words, the fluid models are only valid up until the sheath entrance.
One should notice though that in reality it is hard, and maybe even impossible to tell where the bulk plasma end, and where does the sheath start.

%
\subsection{Boundary conditions at SE}
% NOTE: Consider to add calculation of sheath velocities
Following the calculations of Choudora and Bohm (see for example \cite{Stangeby2000book}), it is common practice to define the sheath entrance to be the place where ion velocity reaches the ion sound speed (that is where $u_{i,\|}=c_s$), and we will adopt this practice in this thesis.
Considering a quiescent plasma, one can calculate the equilibrium profiles which the plasma obtains in contact with materials, and use this to set the boundaries for $u_{i, \|}$ and $u_{e, \|}$ at the sheath entrance (SE).
One should note though, that even these boundary conditions are valid only in steady state.

\subsubsection{Ion velocity at SE}
For the ion velocity, we have from steady state calculations defined the sheath entrance to be the point where
%
\begin{align*}
    u_{i,\|} \bigg|_{L_z} = c_s
\end{align*}
%
Normalization gives
%
\begin{align*}
    u_{i,\|} \bigg|_{L_z} = 1
\end{align*}
%

\subsubsection{Electron velocity at SE}
%
Further, the steady state gives the following condition on the parallel electron velocity
%
\begin{align*}
    u_{e,\|} \bigg|_{L_z} = c_s \exp\L(\Lambda - \frac{e[\phi_0 + \phi]}{T_e}\R),
\end{align*}
%
where $\phi_0$ is an arbitrary potential%
%
\footnote{Due to the inversion algorithm $\phi_0$ has been set to $\Lambda$ for better numerical stability}%
%
we have defined $\phi_{\text{sheath entrance}} = 0$, and $\Lambda=\ln\L(\sqrt{\frac{\mu}{2\pi}}\R)$. Normalization gives
%
\begin{align*}
    u_{e,\|} \bigg|_{L_z} = \exp\L(\Lambda - \phi\R)
\end{align*}
%

\subsubsection{Density BC at SE}
%
% NOTE: Tried
%   1. From steady state electron continuity
%      This means that n cannot change at the boundary
%   2. Neumann = 0
%      Could be that this it has a gradient
%      Problem that Neumann both at SE and non-SE site
% NOTE: Idea
%   1. Could try robin BC at non-SE site
%
% FIXME: Find out what this should be
%
Although the "standard" calculation for the sheath gives us conditions for setting the velocities in steady state, non such conditions exists for the density.
Instead, one would have to rely on other arguments.
In \cite{Loizu2012a}, Loizu et al.  presents a boundary condition for the density in a field line geometry, where the field lines are allowed to be tilted with respect to the end-plate.
For a field line geometry where the field lines are perpendicular to the end-plate (which is the case in this thesis), the boundary condition reduces to
%
\begin{align*}
    \L. \partial_\| n \R|_{L_z}&= -\L.\frac{n}{c_s}\partial_\| u_{i.\|}\R|_{L_z}\\
    \note{$\L. u_{i.\|}\R|_{L_z} = c_s$}
    \L.u_{i.\|} \partial_\| n \R|_{L_z}&= -\L.n \partial_\| u_{i.\|}\R|_{L_z}\\
    \L.u_{i.\|} \partial_\| n \R|_{L_z}+\L.n \partial_\| u_{i.\|}\R|_{L_z}&= 0\\
    \L.\partial_\|\L(u_{i.\|}  n \R)\R|_{L_z}&= 0\\
    \L.\ve{b}\cdot\grad\L(u_{i.\|}  n \R)\R|_{L_z}&= 0
    \note{$\partial_i \ve{b} = 0$}
    \\
    \L.\div\L(\ve{b}u_{i,\|}  n \R)\R|_{L_z}&= 0\\
    \L.\div\L(\ve{u}_{i,\|}  n \R)\R|_{L_z}&= 0\\
    \L.\L(\grad_\perp + \grad_\|\R)\cdot\L(\ve{u}_{i,\|} n \R)\R|_{L_z}&= 0\\
    \L.\grad_\|\cdot\L(\ve{u}_{i,\|} n \R)\R|_{L_z}&= 0
    \note{$\grad_\| \cdot \ve{u}_{i,\perp} = 0$}
    \\
    \L.\div_\|\L(\ve{u}_{i} n \R)\R|_{L_z}&= 0
    \note{Continuity equation}
    \\
    \L.- \partial_tn \R|_{L_z}- \L.\grad_\perp\cdot\L(\ve{u}_{i} n \R)\R|_{L_z}&= 0\\
    \L.\partial_tn \R|_{L_z}&=  - \L.\grad_\perp\cdot\L(\ve{u}_{i} n \R)\R|_{L_z}
    \numberthis
    \label{eq:nBC}
\end{align*}
%
In other words, it says that the only change in the density at the SE can come from perpendicular outflux.
Another way to look at it is to consider
%
\begin{align*}
    \L.\L[\inde{\partial_\|\L(u_{i.\|}  n \R)}{z}\R]\R|_{L_z}&= 0\\
    \L.u_{i.\|}  n \R|_{L_z}&= C
\end{align*}
%
which means that the flux through the SE is constant in time, which may be a too big constraint on the system.

One alternative is to set
%
\begin{align*}
    \L. \partial_\| n \R|_{L_z} &= 0
\end{align*}
%
If we follow the derivation of equation \ref{eq:nBC} and use the ion continuity equation, this can be written as
%
\begin{align*}
    \L.- \partial_tn \R|_{L_z}- \L.\grad_\perp\cdot\L(\ve{u}_{i} n \R)\R|_{L_z}
    -\L.n \partial_\| u_{i.\|}\R|_{L_z}
    &= 0\\
    \L.\partial_tn \R|_{L_z}&=  - \L.\grad_\perp\cdot\L(\ve{u}_{i} n \R)\R|_{L_z}-
    \L.n \partial_\| u_{i.\|}\R|_{L_z}
\end{align*}
%
which means that the flux through the SE is no longer fixed.
One should note that if one has a Neumann condition in both ends of the machine, the PDE is formally ill-posed, and the solutions found are unique only up to some constant.
When numerically solving the system the solution found will be specified by the initial condition.
Although it is important to be aware of, it is not believed to change the physical behavior of the system as the dynamics are driven by its source and its sinks.

A third approach, which does not have the same problem, is to let the density be completely free at the sheath entrance and instead fix the value and the gradient at the stagnation point (where the velocities are $0$) using a Cauchy boundary condition (not to be confused with a Robin or "mixed" boundary condition).
This kind of boundary condition specifies the parallel dynamics fully.
The derivatives of $n$ at the last physical point before the SE can then be calculated numerically by either a one-sided stencil, or to extrapolate the solution to a ghost point and use this is a \emph{artificial boundary condition} \cite{Leveque2007book}.

A final possibility is to let the derived equation for the evolution of the density be valid at the boundary.

In the scope of this thesis the differences between $\L. \partial_\| n \R|_{L_z} = 0$ and the Cauchy BC has been investigated.

% FIXME: Add reference to where stuff is compared

\subsubsection{Potential BC at SE}
%
% NOTE: Tried
%   1. Fixed value
%      Bad idea, as difference in parallel velocities are not regulated by the
%      potential
%   2. Neumann 0
%      Bad idea as this may make an artificial constraint on the system
% FIXME: Maybe flawed?
The potential in our equations are not being evolved in time directly, but is calculated by inverting either $\Om$ or $\Om^D$ for each drift plane (that is for each perpendicular plane).
This inversion takes into account the outer radial boundary condition in $\phi$.
This outer radial boundary condition is set from the material properties of the wall.
On the other hand, we do not have any physical material constraint at the SE.
The only thing we impose it that $\phi$ on the plate very close to the SE is set to an arbitrary constant.
 It is not clear how this is reflected at the SE in a non-steady state plasma.

However, parallel derivatives of the potential is being taken in our set of equations.
In order to calculate the parallel derivative of $\phi$ in the last point before the SE, we need either to make a one sided stencil for this very point (as the value of the boundary, and thereby the value of the ghost point is unknown), or we can extrapolate the value of $\phi$ to the ghost point (as we anyway assume that $\phi$ is a smooth function in order to discretize the differential operators working on $\phi$).
The latter has been chosen in the current implementation.

\subsubsection{Vorticity BC at SE}
% NOTE: Tried
%   1. Neumann = 0
%      Could be that this it has a gradient
%      Problem that Neumann both at SE and non-SE site
% IDEA:
%   1. Should be 0, as the wall is slowing down? Maybe not as not necessarily
%      have a non-slip BC on the wall
%
% FIXME: Find out what this should be
%
% FIXME
% No good physical argument yet.
% ATM implemented as Neumann zero.

\subsection{BCs at non-SE drift plane}
% FIXME:
% Check that this is what you really want to do
The boundary conditions at the opposite site of the SE in the cylinder varies from experiment to experiment (as does also the source term in the distribution function itself).
In the following model, we want to model a cylinder which is mirrored at the opposite side of SE.
In other words, the model does not reflect any experimental linear devices, but can help to shed light on some of the features found in real world experiments.
The mirroring condition gives the following set of boundary conditions
%
\begin{align*}
    &\partial_\| n \bigg|_0    =0 &
    &u_{e,\|} \bigg|_0         =0 &
    &u_{i,\|} \bigg|_0         =0 &
    &\partial_\| \phi \bigg|_0 =0 &
    &\partial_\|\Om \bigg|_0   =0 &
\end{align*}

\subsection{Radial boundary conditions}
%
The radial boundary conditions can in principle determine the results of the experiments entirely.
This have been seen in experiments on for example the Mirabelle machine \cite{Schroder2001}.
However, to correctly asset the physics, one would have to take into account the neutral interaction and the material properties of the wall.
We will therefore in this thesis focus on a much simplified approximation of the radial boundary conditions.
In fact we will let the radial boundary condition be approximately where the neutrals are dominating, and thus dampening out the plasma dynamics.

\subsubsection{BC at the center}
%
The cylindrical coordinate system has a discontinuity at the center where $\rho=0$.
One way to avoid this problem is to put the grid points close to, but not in the very center.
In a cylinder, there is no real boundary condition in the center, but we anyway have to address $\rho$ derivatives for the grid points which lays close to the center.
A workaround for this problem is to put the boundary condition half way between a inner ghost point and the first inner grid point (both located $\frac{\Delta \rho}{2}$ from the center, diametrically opposite of each other).
This method was first suggested by Naulin, V. et al. in \cite{Naulin2008}.

% FIXME: Describe how this is done in Laplace inversion

\subsubsection{Density BC at outer radius}
The density will go toward $0$ as we are approaching the wall, thus we set
%
\begin{align*}
    n \bigg|_{L_\rho} = C,
\end{align*}
%
where $C$ is a small constant.
Setting this too small, however gives numerical problems due to loss of precision.
In normalized units, $C = \frac{1}{20}$ of the maximum value in the equilibrium.

\subsubsection{Velocity BCs at outer radius}
% FIXME:
% Not really physically explained
% If we let u_|_ =\= 0 we have no physical explanation why this should be 0
%FIXME: Not really physically explained

There should be no parallel velocity towards the edge at the outer boundary, thus
%
\begin{align*}
    u_{e,\|} \bigg|_{L_\rho} = u_{i,\|} \bigg|_{L_\rho} = 0
\end{align*}

\subsubsection{Potential BC at outer radius}
%
As there are no real sheath where the magnetic field lines are perpendicular to the material, thus the potential at outer wall should be the at the floating potential in an ideal conductor which we have assumed here.
Of course, there will be a small deviation from this as ions are lost a bit more rapidly radially due a larger Larmor radius of the ions as compared to the electrons.
 This is neglected here as finite Larmor radius (FLR) effects are not taken into account here.
For that reason we have that
%
\begin{align*}
    \phi \bigg|_{L_\rho} = 0
\end{align*}

\subsubsection{Vorticity velocity BC at outer radius}
%
% NOTE: Tried
%       1. Om = 0
%       This will give rise to a boundary layer which will try to oppose the
%       eventual solid body rotation in the center
% NOTE: The boundary here is set from either the non-slip or the free-slip
%       condition of v_theta
% FIXME: Write me, see blackboard photo: BC-rho-on-vort
%
% FIXME
%
\begin{align*}
    \Om \bigg|_{L_\rho} = 0
\end{align*}
