We have now derived the set of equations, referred to as the CELMA model, both with and without the Boussinesq approximation.
We have seen how our dynamic system can be described by four coupled equations derived from the Fokker-Planck equations.
These equations are used in a cylindrical domain, and we have specified a set of boundary conditions which are more or less appropriate.
In principle, we are ready to put the model into a code which performs the calcuation, and thereby simulate the system.
Before doing so, we need to address some caveats regarding the implementation.
It is also important to verify that the machinery we use to solve the equations are actually working.
We will do so in the next part, and we will in \cref{part:results} discuss the results of the simulations.
