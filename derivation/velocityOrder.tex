The non-linear, coupled set of equation (\ref{fluideq:cont}) and (\ref{fluideq:mom}) consist of one continuity equation per species and one momentum equation per direction per species.
 With one electron fluid and one ion fluid, that totals $8$ partial differential equations (PDEs).
Resolving all the details in these equations are computationally heavy, and we will here seek ways of simplifying the equations to lessen the computational demand.
% FIXME: Not necessarily true in our case where we have sheaths

The computational demand can be lessened by exploiting the difference in the parallel and the perpendicular dynamics.
Due of the gyration of the particles, the dynamics parallel to the magnetic field are much faster than the dynamics perpendicular to the magnetic field.
As a result the gradients perpendicular to the magnetic field tends to be much larger than the gradients parallel to the magnetic field.
We can exploit this by using a courser grid in the parallel direction as compared to the perpendicular direction.
By these two arguments, we have a good motivation to split our equations into perpendicular and parallel parts.

Another way to lessen the demand is through the so called drift ordering of the perpendicular velocities.
 This reduces the details we can get out of the set of equations, but has the advantage that the perpendicular velocities can be solved algebraically rather than by solving four PDEs (two for each species).
 To derive the equations with the drift order approximation, we start out by decomposing the equations in parallel and perpendicular parts.

\section{Decomposition}
We start the derivation by rearranging equation (\ref{fluideq:mom}) in
the following way
%
\begin{align*}
 n_\a m_\a \d_{t,\a} \ve{u}_{\a} &=
 -\grad p_\a - \div\te{\pi}_\a +
 q_\a n_\a (\ve{E}+\ve{u}_{\a}\times\ve{B})
 + \ve{R}_{\beta \to \a}
 + \ve{R}_{n \to \a}
 - S_{\a,n}m_\a\ve{u}_\a
 \\
%
 \frac{
   n_\a m_\a \d_{t,\a} \ve{u}_{\a}
 }{n_\a q_\a B}
 &=
 -
 \frac{
   \grad p_\a + \div\te{\pi}_\a
 }{n_\a q_\a B}
 +
 \frac{
   q_\a n_\a \ve{E}
 }{n_\a q_\a B}
 +
 \frac{
     q_\a n_\a \ve{u}_{\a}\times\ve{B}
 }{n_\a q_\a B}
 +
 \frac{
   \ve{R}_{\beta \to \a}
 }{n_\a q_\a B}
 +
 \frac{
   \ve{R}_{n \to \a}
 }{n_\a q_\a B}
 -
 \frac{
 S_{\a,n}m_\a\ve{u}_\a
 }{n_\a q_\a B}
 %
 \note{$\ve{b} \defined \ve{B}/B$, where
             $B=\|\ve{B}\|$}\\
 %
 %
 \frac{1}{\om_{c\a}}\d_{t,\a} \ve{u}_{\a}
 &=
 -
 \frac{
   \grad p_\a
 }{n_\a  q_\a B}
 +
 \frac{\ve{E}}{B}
 +
 \ve{u}_{\a}\times\ve{b}
 -
  \frac{
   \div\te{\pi}_\a
 }{n_\a  q_\a B}
 +
 \frac{
   \ve{R}_{\beta \to \a}
 }{n_\a q_\a B}
 +
 \frac{
   \ve{R}_{n \to \a}
 }{n_\a q_\a B}
 -
 \frac{
   S_{\a,n}\ve{u}_\a
 }{n_\a \om_{c\a}},
 \label{eq:no_assumptions_momentum}
 \numberthis
\end{align*}
%
where $\om_{c\a} = \frac{q_{\a}B}{m_\a}$ is the cyclotron frequency for species
$\a$.%
\footnote{Note that the ``frequency'' can be negative because of the $q_{\a}$ using this definition.
        However, this is just a ``remnant'' of the vector version of this quantity: The angular velocity $\ve{\om}_{c\a} = \frac{q_{\a}\ve{B}}{m_\a}$, where the $\pm$ sign from $q_{\a}$ tells us if a particle is rotating clockwise or counter-clockwise.}%
%

In general, an arbitrary vector $\ve{a}$ can be written $\ve{a} = \ve{a}\cdot\te{I}$, where $\te{I}$ is the identity tensor of rank $2$.
We introduce the rank-2 tensor $\ve{b}\ve{b}$, which is the outer product with the unity vectors along $\ve{B}$.
Thus
%
\begin{align}
 \ve{a} = \ve{a}\cdot\L(\te{I}+\ve{b}\ve{b}-\ve{b}\ve{b}\R)
        = \underbrace{\ve{a}\cdot\L(\te{I}-\ve{b}\ve{b}\R)}_{\ve{a}_{\perp}}
          + \underbrace{\L(\ve{a}\cdot\ve{b}\R)\ve{b}}_{\ve{a}_{\|}}
        \label{eq:perp_par}
\end{align}
%
In other words, we can find the parallel component (with respect to the magnetic field) by taking the dot product with $\ve{b}$ and use the result to scale $\ve{b}$.
We further observe that
%
\begin{align*}
    -\ve{b}\times\L(\ve{b}\times\ve{a}\R)
    &=
    -
    \ve{b}\L(\ve{b}\cdot\ve{a}\R)
    +
    \ve{a}\L(\ve{b}\cdot\ve{b}\R)
    \note{
        $
        \ve{a}\times\L(\ve{b}\times\ve{c}\R)
        =
        \ve{b}\L(\ve{a}\cdot\ve{c}\R)
        -
        \ve{c}\L(\ve{a}\cdot\ve{c}\R)
        $
        }
    \\
    &=
    \L(
    \te{I}
    -
    \ve{b}\ve{b}
    \R)
    \cdot\ve{a}
\end{align*}
%
Generally we have
%
\begin{align*}
 \L(\L[\d_{t,\a} \ve{a}\R]\cdot\ve{b}\R)\ve{b}
 %
 =& \L(\L[\parti{}{t} \ve{a}\R]\cdot\ve{b}\R)\ve{b} +
 \L(\L[\ve{u}_{\a} \cdot\nabla \ve{a}\R]\cdot\ve{b}\R)\ve{b}
 \note{\hspace{-2cm}Assume $\partial_t \ve{b} = 0$}
 \\
 %
 =& \parti{}{t}\L(\L[\ve{a}\cdot\ve{b}\R]\ve{b}\R) +
 \L(\L[\ve{u}_{\a} \cdot\nabla \ve{a}\R]\cdot\ve{b}\R)\ve{b}
 \note{\hspace{-2cm}$\ve{v}\cdot\ve{w}=\ve{w}\cdot\ve{v}$}
 \\
 %
 =& \parti{}{t}\L(\L[\ve{a}\cdot\ve{b}\R]\ve{b}\R) +
 \L(\ve{u}_{\a} \cdot \L[\ve{b} \cdot\nabla \ve{a}\R]\R)\ve{b}
 \note{\hspace{-2cm}$\div(\ve{v}\ve{w}) = \ve{v}\div\ve{w} +
\ve{w}\div\ve{v}$}
 \\
 %
 =& \parti{}{t}\L(a_\|\ve{b}\R) +
 \L(\ve{u}_{\a} \cdot \L[\nabla \L(\ve{a} \cdot \ve{b}\R)
  - \ve{a} \cdot\nabla \ve{b}\R]\R)\ve{b}
 \note{\hspace{-2cm}Assume $\nabla \ve{b}$ is negligible}
 \\
 %
 =& \parti{}{t}\ve{a}_\| +  \L(\ve{u}_{\a} \cdot
 \nabla a_\|\R)\ve{b}\\
 %
 =& \parti{}{t}\ve{a}_\| +  \ve{u}_{\a} \cdot
 \L(\nabla \L[a_\|\R]\ve{b}\R)
 \note{\hspace{-2cm}$\grad(v\ve{w})=v\grad(\ve{w})+\grad(v)\ve{w}$}
 \\
 %
 =& \parti{}{t}\ve{a}_\| +  \ve{u}_{\a} \cdot
 \L(\nabla \L[a_\|\ve{b}\R] - a_\|\nabla \L[\ve{b}\R]\R)
 \note{\hspace{-2cm}Assumed $\nabla \ve{b}$ is negligible}
 \\
 %
 =& \parti{}{t}\ve{a}_\| +  \ve{u}_{\a} \cdot \nabla \ve{a}_\|
 \\
 %
 =& \d_{t,\a} \ve{a}_\|,
\end{align*}
%
where we have used the electrostatic approximation $\d_t \ve{B} \simeq 0$ (see \cref{app:elstat} for details).

Further we have that
%
\begin{align*}
 \L(\div\te{A}\R)_\|
 &\defined
 \L(\L[\div\te{A}\R]\cdot\ve{b}\R)\ve{b}
\end{align*}
%
and that
%
\begin{align*}
 \L(\L[\grad\ve{a}\R]\cdot\ve{b}\R)\ve{b}
 &= \L(\ve{b}\cdot\L[\grad\ve{a}\R]\R)\ve{b}
 \note{$c\ve{v} = \ve{v}c$ in $\mathbb{R}$}
 \\
 %
 &= \ve{b}\L(\ve{b}\cdot\L[\grad\ve{a}\R]\R)
 \note{$\partial_\| \defined \ve{b}\cdot \nabla$}
 \\
 %
 &= \ve{b}\L(\partial_\|\ve{a}\R)
 \note{$\nabla_\| \defined \ve{b}\ve{b}\cdot \nabla$}
 \\
 %
  &= \grad_\|\ve{a}
\end{align*}
%
For further references, we also define all the gradient operators in this thesis here
%
\begin{empheq}[box=\tcbhighmath]{align*}
    &\partial_\| \defined \ve{b}\cdot \nabla&
    &\nabla_\| \defined \ve{b}\ve{b}\cdot \nabla&
    &\nabla_\perp \defined \nabla - \nabla_\|&
    \\
    &\grad^2 = \div \grad&
    &\grad_\|^2 = \div \grad_\|&
    &\grad_\perp^2
    = \div\grad_\perp
    = \div\L(\grad - \grad_\|\R)
    = \grad^2 - \grad_\|^2
\end{empheq}
%
%
%https://www.physicsforums.com/threads/why-is-scalar-multiplication-on-vector-sp
%a ces-not-commutative.94783/
% http://en.wikipedia.org/wiki/Scalar_multiplication
%
This means that if we right dot \cref{eq:no_assumptions_momentum} with $\ve{b}\ve{b}$, we get
%
\begin{align*}
  \L(\frac{1}{\om_{c\a}}\d_{t,\a} \ve{u}_{\a}\R)\cdot\ve{b}\ve{b}
 &=
 \L(
 -
 \frac{
   \grad p_\a
 }{n_\a  q_\a B}
 +
 \frac{\ve{E}}{B}
 +
 \ve{u}_{\a}\times\ve{b}
 -
  \frac{
   \div\te{\pi}_\a
 }{n_\a  q_\a B}
 +
 \frac{
   \ve{R}_{\beta \to \a}
 }{n_\a q_\a B}
 +
 \frac{
   \ve{R}_{n \to \a}
 }{n_\a q_\a B}
 -
 \frac{
     S_{\a,n}\ve{u}_{\a,\|}
 }{n_\a \om_{c\a}}
 \R)\cdot\ve{b}\ve{b}
\end{align*}
%
\begin{empheq}[box=\tcbhighmath]{align}
 \frac{1}{\om_{c\a}}\d_{t,\a} \ve{u}_{\a,\|}
 &=
 -
 \frac{
   \grad_\| p_\a
 }{n_\a  q_\a B}
 +
 \frac{\ve{E_\|}}{B}
 -
  \frac{
   \L(\div\te{\pi}_\a\R)_\|
 }{n_\a  q_\a B}
 +
 \frac{
   \ve{R}_{\beta \to \a,\|}
 }{n_\a q_\a B}
 +
 \frac{
   \ve{R}_{n \to \a}
 }{n_\a q_\a B}
 -
 \frac{
     S_{\a,n}\ve{u}_{\a,\|}
 }{n_\a \om_{c\a}}
 \label{eq:material_dot_bb}
\end{empheq}
%
If we subtract \cref{eq:material_dot_bb} from \cref{eq:no_assumptions_momentum}, and use that  $\nabla_\perp = \nabla - \nabla_\|$ and $\ve{a}\times\ve{b}=\ve{a}_\perp\times\ve{b}$, we get
%
\begin{empheq}[box=\tcbhighmath]{align}
 \frac{1}{\om_{c\a}}\d_{t,\a} \ve{u}_{\a,\perp}
 =&
 -
 \frac{
   \grad_\perp p_\a
 }{n_\a  q_\a B}
 +
 \frac{\ve{E}_\perp}{B}
 +
 \ve{u}_{\a,\perp}\times\ve{b}
 \notag
 \\&
 -
  \frac{
   \L(\div\te{\pi}_\a\R)_\perp
 }{n_\a  q_\a B}
 +
 \frac{
   \ve{R}_{\beta \to \a,\perp}
 }{n_\a q_\a B}
 +
 \frac{
   \ve{R}_{n \to \a,\perp}
 }{n_\a q_\a B}
 -
 \frac{
     S_{\a,n}\ve{u}_{\a,\perp}
 }{n_\a \om_{c\a}}
 \label{eq:perp_mom_start}
\end{empheq}

\section{Velocity drifts}
%
% NOTE: See http://www.cims.nyu.edu/~eve2/reg_pert.pdf for perturbation theory intro
%
The goal of the drift ordering is to split \cref{eq:perp_mom_start} in different orders, yielding algebraic equations for each order of $\ve{u}_{\a,\perp}$.
From \cref{eq:DO} in \cref{app:DO}, we have that \cref{eq:perp_mom_start} can be written in orders of $\e$ as%
%
\footnote{
Note that we write the viscosity part as order $\mathcal{O}(\e)$, as we for ions assume that $\mu\e^2$ at least is not bigger than $\e$.
In other words we assume that this term cannot be larger than $\e$.
}%
%
\begin{align*}
 \overbrace{
 \frac{1}{\om_{c\a}}\d_{t,\a} \ve{u}_{\a,\perp}
 }^{\e^1}
 =&
 \overbrace{
 - \frac{ \grad_\perp p_\a }{n_\a  q_\a B}
 + \frac{\ve{E}_\perp}{B}
 + \ve{u}_{\a,\perp}\times\ve{b}
 }^{\e^0}
 \\&
 \overbrace{
 - \frac{ \L(\div\te{\pi}_\a\R)_\perp }{n_\a  q_\a B}
 + \frac{ \ve{R}_{\beta \to \a,\perp} }{n_\a q_\a B}
 + \frac{ \ve{R}_{n \to \a,\perp} }{n_\a q_\a B}
 - \frac{ S_{\a,n}\ve{u}_{\a,\perp} }{n_\a \om_{c\a}}
 }^{\e^1},
 \numberthis
 \label{eq:uOrderStart}
\end{align*}
%
where the order of the term is indicated above the term.
We can now solve \cref{eq:uOrderStart} for $\ve{u}_{\a,\perp}$ by using perturbation theory.
To do this, we first assume that
%
\begin{align}
    \ve{u}_{\a,\perp} = \e^0\ve{u}_{\a,0,\perp} + \e^1\ve{u}_{\a,1,\perp} + \e^2\ve{u}_{\a,2,\perp} + \ldots
    \label{eq:uOrderConcept}
\end{align}
%
so that
%
\begin{align*}
    \e^0\ve{u}_{\a,0,\perp} \gg \e^1\ve{u}_{\a,1,\perp} \gg \e^2\ve{u}_{\a,2,\perp} \gg \ldots,
\end{align*}
%
since $\e\ll1$.

The basic idea is to split equation \cref{eq:uOrderStart} into one equation for each order (i.e. equation $n$ would only contain of $\mathcal{O}(\e^n)$ terms).
In this way, the solution to the $\mathcal{O}(\e^0)$ equation will be an approximate solution to the system.
The first order correction would be given by the solution of the $\mathcal{O}(\e^1)$ equation, which will depend on the $\mathcal{O}(\e^0)$ solution due to the non-linearities in $\ve{u}_{\a,\perp}$.
The second order correction will given by the solution of the $\mathcal{O}(\e^2)$ euqation, which depends on the  $\mathcal{O}(\e^1)$ solution, and so on.
We note that this is the same strategy as was used for system closure by Chapman, Enskog and Braginskii as stated in \cite{Brush1972book,Chapman1970book,Braginskii1965}.

We will in this thesis only be interested in an accuray in the order of $\mathcal{O}(\e^1)$.
Therefore, we truncate \cref{eq:uOrderConcept} after the first order an get that
%
\begin{align}
    \ve{u}_{\a,\perp} \simeq \e^0\ve{u}_{\a,0,\perp} + \e^1\ve{u}_{\a,1,\perp}
    \label{eq:uOrder}
\end{align}
%
If we insert \cref{eq:uOrder} into \cref{eq:uOrderStart}, we get
%
\begin{align*}
 \overbrace{
 \frac{1}{\om_{c\a}}\d_{t,\a} \e^0\ve{u}_{\a,0,\perp}
 }^{\e^1}
 +
 \overbrace{
 \frac{1}{\om_{c\a}}\d_{t,\a} \e^1\ve{u}_{\a,1,\perp}
 }^{\e^2}
 =&
 \overbrace{
 - \frac{ \grad_\perp p_\a }{n_\a  q_\a B}
 + \frac{\ve{E}_\perp}{B}
 + \e^0\ve{u}_{\a,0,\perp}\times\ve{b}
 }^{\e^0}
 +
 \overbrace{
 \ve{u}_{\a,1,\perp}\times\ve{b}
 }^{\e^1}
 \\&
 \overbrace{
 - \frac{ \L(\div\te{\pi}_\a\R)_\perp }{n_\a  q_\a B}
 + \frac{ \ve{R}_{\beta \to \a,\perp} }{n_\a q_\a B}
 + \frac{ \ve{R}_{n \to \a,\perp} }{n_\a q_\a B}
 - \frac{ S_{\a,n}\ve{u}_{\a,\perp} }{n_\a \om_{c\a}}
 }^{\e^1}
\end{align*}
%
As $\d_{t,\a}$ is a function of $\ve{u}_{\a,\perp}$, and should be accounted for in the ordering.
If we introduce the notation $\d_{t,\a}^n$, where $n$ denotes the order we have that
%
\begin{align*}
 \d_{t,\a}^0 \ve{u}_{\a,\perp} &= 0
 \note{No $\e^0$ terms in LHS of \cref{eq:uOrderStart}}\\
 %
 \d_{t,\a}^1 \ve{u}_{\a,\perp} &= \parti{}{t} \ve{u}_{\a,0,\perp}
 + \ve{u}_{\a,0,\perp} \cdot\nabla \ve{u}_{\a,0,\perp}
 + \ve{u}_{\a,\|} \cdot\nabla \ve{u}_{\a,0,\perp}\\
 %
 \d_{t,\a}^2 \ve{u}_{\a,\perp} &= \parti{}{t}\e\ve{u}_{\a,1,\perp} +
 \ve{u}_{\a,0} \cdot\nabla \e\ve{u}_{\a,1,\perp}
 + \e\ve{u}_{\a,1} \cdot\nabla \ve{u}_{\a,0,\perp}
 + \ve{u}_{\a,\|} \cdot\nabla \e\ve{u}_{\a,1,\perp}\\
 &\;\, \vdots \notag
\end{align*}
%
This gives the following set of equations
%
\begin{align}
&\mathcal{O}(\e^0): \qquad &
&
 0
 =
 \overbrace{
 - \frac{ \grad_\perp p_\a }{n_\a  q_\a B}
 + \frac{\ve{E}_\perp}{B}
 + \e^0\ve{u}_{\a,0,\perp}\times\ve{b}
 }^{\e^0}
&
\label{eq:uZero}
 \\
 %
 %
&\mathcal{O}(\e^1): \qquad &
&
 \overbrace{
 \frac{1}{\om_{c\a}}\d^1_{t,\a} \e^0\ve{u}_{\a,0,\perp}
 }^{\e^1}
 =
 \overbrace{
 \ve{u}_{\a,1,\perp}\times\ve{b}
 }^{\e^1}
 -
 \overbrace{
   \frac{ \L(\div\te{\pi}_\a\R)_\perp }{n_\a  q_\a B}
 + \frac{ \ve{R}_{\beta \to \a,\perp} }{n_\a q_\a B}
 + \frac{ \ve{R}_{n \to \a,\perp} }{n_\a q_\a B}
 - \frac{ S_{\a,n}\ve{u}_{\a,\perp} }{n_\a \om_{c\a}}
 }^{\e^1}
&
\label{eq:uFirst}
\\
&\quad\vdots&&&
\notag
\end{align}

\section{Zeroth order perpendicular terms}
%
We will now solve \cref{eq:uFirst} for $\ve{u}_{\a,0,\perp}$.
In general, we have that $-\ve{b}\times\L(\ve{b}\times\ve{a}\R) = \ve{a} - \ve{a}_\|=\ve{a}_\perp$.
Thus, in order to isolate $\ve{u}_{\a,0,\perp}$ from \cref{eq:uFirst}, we can cross the equation with $\ve{b}$.
This gives
%
\begin{align*}
 0 &=
 - \frac{\grad_\perp p_\a}{n_\a  q_\a B}\times\ve{b}
 + \frac{\ve{E}_\perp}{B}\times\ve{b}
 + \L(\ve{u}_{\a,0,\perp}\times\ve{b}\R)\times\ve{b}
 \\
 %
 &=
 - \frac{\grad_\perp p_\a\times\ve{b}}{n_\a  q_\a B}
 + \frac{\ve{E}_\perp\times\ve{b}}{B}
 + \ve{b}\times\L(\ve{b}\times\ve{u}_{\a,0,\perp}\R)
 \\
 %
 - \ve{b}\times\L(\ve{b}\times\ve{u}_{\a,0,\perp}\R)
 &=
 - \frac{\grad_\perp p_\a\times\ve{b}}{n_\a  q_\a B}
 + \frac{\ve{E}_\perp\times\ve{b}}{B}
 \\
  %
 \ve{u}_{\a,0,\perp}
 &=
 - \frac{\grad_\perp p_\a\times\ve{b}}{n_\a  q_\a B}
 + \frac{\ve{E}_\perp\times\ve{b}}{B}
 \numberthis
 \label{eq:u_0_balance}
\end{align*}
%
As mentioned in \cref{app:elstat}, a low $\beta$ plasmas justifies the electrostatic approximation.
This means that $\ve{E} = -\grad\phi$, so $\ve{E}_\perp = -\grad_\perp\phi$ and $\ve{E}_\| = -\grad_\|\phi$.
Thus can we rewrite \cref{eq:u_0_balance} as
%
\begin{empheq}[box=\tcbhighmath]{align*}
 \ve{u}_{\a,0,\perp} =&
 %
  \underbrace{
    %
    -\frac{\grad_\perp p_\a \times\ve{b}}{q_\a n_\a  B}
    %
   }
   _{\ve{u}_{\a,d}}
   %
 %
 \underbrace{
    %
    -  \frac{\grad_\perp \phi \times \ve{b}}{B}
    %
   }
   _{\ve{u}_{E}}
   %
 %
 \label{eq:u_0}
 \numberthis
\end{empheq}

\section{First order perpendicular terms}
%
Once $\ve{u}_{\a,0,\perp}$ is known, the algebraic equation for $\ve{u}_{\a,1,\perp}$ can be found by crossing \cref{eq:uFirst} with $\ve{b}$.
Generally we have
%
\begin{align*}
 \L(\d_{t,\a} \ve{a} \R)\times\ve{b}
 &= \L(\partial_t \ve{a} + \ve{u}_{\a}\cdot\nabla\ve{a} \R)\times\ve{b}
 \note{\hspace{-2.5cm}Assume electrostatic}
 \\
 %
 &= \partial_t \L(\ve{a}\times\ve{b}\R) +
 \L(\L[\ve{u}_{\a}\cdot\nabla\ve{a}\R]\times\ve{b}\R)
 \note{\hspace{-2.5cm}
       $\grad\L(\ve{v}\times\ve{w}\R) = \L(\grad
        \ve{v}\R)\times\ve{w} - \L(\grad \ve{w}\R)\times\ve{v}$}
 \\
 %
 &= \partial_t \L(\ve{a}\times\ve{b}\R) +
 \ve{u}_{\a}\cdot\L(\nabla\L[\ve{a}\times\ve{b}\R] +
 \L[\nabla\ve{b}\R]\times\ve{a}\R)
 \note{\hspace{-2.5cm}Assume electrostatic}
 \\
 %
 &= \partial_t \L(\ve{a}\times\ve{b}\R) +
 \ve{u}_{\a}\cdot\L(\nabla\L[\ve{a}\times\ve{b}\R]\R)
 \\
 %
 &= \d_{t,\a} \L(\ve{a} \times\ve{b} \R)
\end{align*}
%
This gives
%
\begin{align*}
  \L(\frac{1}{\om_{c\a}}\d^1_{t,\a} \ve{u}_{\a,0,\perp}\R)\times\ve{b}
 =&
  \L(\ve{u}_{\a,1,\perp}\times\ve{b}\R)\times\ve{b}
  +
  \frac{\ve{R}_{\beta \to \a,\perp}}{n_\a q_\a B}\times\ve{b}
  +
  \frac{\ve{R}_{n \to \a,\perp}}{n_\a q_\a B}\times\ve{b}
  \\
  &-
  \frac{\L(\div\te{\pi}_\a\R)_\perp}{n_\a  q_\a B}\times\ve{b}
  -
  \frac{ S_{\a,n}\ve{u}_{\a,0,\perp} }{n_\a \om_{c\a}}\times\ve{b}
  \\
 %
 -\ve{b}\times\L(\ve{b}\times\ve{u}_{\a,1,\perp}\R)
 =&
 -
 \frac{1}{\om_{c\a}}\d^1_{t,\a}\L( \ve{u}_{\a,0,\perp}\times\ve{b}\R)
  +
  \frac{\ve{R}_{\beta \to \a,\perp}\times\ve{b}}{n_\a q_\a B}
  +
  \frac{\ve{R}_{n \to \a,\perp}\times\ve{b}}{n_\a q_\a B}
  \\
  &-
  \frac{\L(\div\te{\pi}_\a\R)_\perp\times\ve{b}}{n_\a  q_\a B}
  -
  \frac{ S_{\a,n}\ve{u}_{\a,0,\perp} }{n_\a \om_{c\a}}\times\ve{b}
  \\
 %
 \ve{u}_{\a,1,\perp}
 =&
 -
 \frac{1}{\om_{c\a}}\d^1_{t,\a}\L( \ve{u}_{\a,0,\perp}\times\ve{b}\R)
  +
  \frac{\ve{R}_{\beta \to \a,\perp}\times\ve{b}}{n_\a q_\a B}
  +
  \frac{\ve{R}_{n \to \a,\perp}\times\ve{b}}{n_\a q_\a B}
  \\
  &-
  \frac{\L(\div\te{\pi}_\a\R)_\perp\times\ve{b}}{n_\a  q_\a B}
  -
  \frac{ S_{\a,n}\ve{u}_{\a,0,\perp} }{n_\a \om_{c\a}}\times\ve{b}
 \label{eq:u_1_u_0}
 \numberthis
\end{align*}
%
By substituting \cref{eq:u_0} in \cref{eq:u_1_u_0}, we obtain
%
% FIXME: Consider not to expand u_s as this disappears
\begin{align*}
 \ve{u}_{\a,1,\perp}
 =&
 -
 \frac{1}{\om_{c\a}}\d^1_{t,\a}\L(\L[
  - \frac{\grad_\perp p_\a\times\ve{b}}{n_\a  q_\a B}
  - \frac{\grad_\perp \phi \times \ve{b}}{B}
 \R]\times\ve{b}\R)
  +
  \frac{\ve{R}_{\beta \to \a,\perp}\times\ve{b}}{n_\a q_\a B}
  +
  \frac{\ve{R}_{n \to \a,\perp}\times\ve{b}}{n_\a q_\a B}
  \\
  &-
  \frac{\L(\div\te{\pi}_\a\R)_\perp\times\ve{b}}{n_\a  q_\a B}
  -
  \frac{ S_{\a,n}}{n_\a \om_{c\a}}
  \L(
  - \frac{\grad_\perp p_\a\times\ve{b}}{n_\a  q_\a B}
  - \frac{\grad_\perp \phi \times \ve{b}}{B}
  \R)
  \times\ve{b}
  \\
 %
 =&
 -
 \frac{1}{\om_{c\a}}\d^1_{t,\a}\L(
  - \frac{\ve{b}\times\L[\ve{b}\times\grad_\perp p_\a\R]}{n_\a  q_\a B}
  - \frac{\ve{b}\times\L[\ve{b}\times\grad_\perp \phi\R]}{B}
  \R)
  +
  \frac{\ve{R}_{\beta \to \a,\perp}\times\ve{b}}{n_\a q_\a B}
  +
  \frac{\ve{R}_{n \to \a,\perp}\times\ve{b}}{n_\a q_\a B}
  \\
  &-
  \frac{\L(\div\te{\pi}_\a\R)_\perp\times\ve{b}}{n_\a  q_\a B}
  -
  \frac{ S_{\a,n}}{n_\a \om_{c\a}}
  \L(
  - \frac{\ve{b}\times\L[\ve{b}\times\grad_\perp p_\a\R]}{n_\a  q_\a B}
  - \frac{\ve{b}\times\L[\ve{b}\times\grad_\perp \phi\R]}{B}
  \R)
  \note{Definition of perp. vectors}
  \\
 %
 =&
 -
 \frac{1}{\om_{c\a}}\d^1_{t,\a}\L(
    \frac{\grad_\perp p_\a}{n_\a  q_\a B}
  + \frac{\grad_\perp \phi}{B}
  \R)
  +
  \frac{\ve{R}_{\beta \to \a,\perp}\times\ve{b}}{n_\a q_\a B}
  +
  \frac{\ve{R}_{n \to \a,\perp}\times\ve{b}}{n_\a q_\a B}
  \\
  &-
  \frac{\L(\div\te{\pi}_\a\R)_\perp\times\ve{b}}{n_\a  q_\a B}
  -
  \frac{ S_{\a,n}}{n_\a \om_{c\a}}
  \L(
  \frac{\grad_\perp p_\a}{n_\a  q_\a B}
  + \frac{\grad_\perp \phi}{B}
  \R)
\end{align*}
%
Hence
%
\begin{empheq}[box=\tcbhighmath]{align*}
 \ve{u}_{\a,1,\perp} =&
 %
  \underbrace{
  \frac{1}{\om_{c\a}}\d^1_{t,\a}\L(
  - \frac{\grad_\perp p_\a}{n_\a  q_\a B}
  - \frac{\grad_\perp \phi}{B}
  \R)
   }
  _{\ve{u}_{\a,p}}
  \underbrace{
   + \frac{\ve{R}_{\beta \to \a,\perp}\times \ve{b}}{n_\a q_\a B}
   }
  _{\ve{u}_{\a,R}}
  \underbrace{
   + \frac{\ve{R}_{n \to \a,\perp}\times\ve{b}}{n_\a q_\a B}
   }
   _{\ve{u}_{\a,\text{Ped}}}
  \nonumber
  \\
  &
  \underbrace{
   - \frac{\L(\div\te{\pi}_\a\R)_\perp\times\ve{b}}{n_\a q_\a B}
   }
  _{\ve{u}_{\a,\nu}}
  \underbrace{
  -
  \frac{ S_{\a,n}}{n_\a \om_{c\a}}
  \L(
  \frac{\grad_\perp p_\a}{n_\a  q_\a B}
  + \frac{\grad_\perp \phi}{B}
  \R)
   }
  _{\ve{u}_{\a,S}}
  \label{eq:first_order}
  \numberthis
\end{empheq}
%
We note that even though the material derivative contains parallel derivatives, the resulting vector is purely perpendicular.
