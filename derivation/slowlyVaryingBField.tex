In the previous chapter, we derived the fluid drifts from the drift-fluid approximation.
We will now insert the fluid drifts into the continuity equations without making any assumption on the \emph{topology} of the $\ve{B}$ field.
We will, however, assume that the $\ve{B}$-field varies slowly in space and  time%
\footnote{Note that we already used this assumtion in \cref{eq:parMat}.}%
%
, and that the plasma consists of only one type of atomic element.

By inserting the fluid drifts into the electron continuity equation, we will obtain an eqation for the temporal evolution of $n$.
Then, we will sum the electron continuity equation with the ion continuity equation, which will yield an equation for the current balance.
These equations will be general for any topology assuming a slowly varying $B$-field, electrostatic conditions and that the drift ordering holds.

In \cref{chap:CELMA} we will constrain the system by specifying a straigth magnetic field, and from that assumption simplify the system further.
The equation for temporal evolution of the modified vorticity can then be extracted from the current balance equation.
Readers familiar with the with the topic can skip the derivation of the drift-continuity eqation, but should read \cref{sec:gyrovisc} and \cref{sec:curConserve} as this gives the startpoint of the derivation of the modified vorticity equation.
%

\section{The drift-continuity equation}
From \cref{fluideq:cont}, and the fluid drifts presented in \cref{chap:drift-order}, we have that
%
\begin{align*}
    \partial_t n_\a + \div (n_\a \ve{u}_\a) &= S_{n,\a}
 \\
 %
 \partial_t n_\a + \div (n_\a [
 \ve{u}_{\a,d} + \ve{u}_E + \ve{u}_{\a,p} + \ve{u}_{\a,R}
 + \ve{u}_{\a,\text{Ped}}
 + \ve{u}_{\a,\nu}
 + \ve{u}_{\a,S} + \ve{u}_{\a,\|}
 ]) &= S_{n, \a}.
 \numberthis
 \label{eq:cont_eq}
\end{align*}
%
In order to simplify this equation we will first introduce the curvature operator and the collisional frequencies, before evaluating the divergence terms one by one.

\subsection{The curvature operator}
%
To easen the calculations, we will introduce the curvature operator.
Assuming we are working in a field aligned Clebsch coordinate system (see \cref{app:coord}), where $\ve{b}$ is parallel to one of the basis vectors, we have that
%
\begin{align*}
\grad_\perp f \times \ve{b}= - \ve{b} \times \grad_\perp f
= - \ve{b} \times (\grad - \grad_\|) f
= - \ve{b} \times (\grad - \ve{b}\ve{b}\cdot\grad) f = \grad f \times \ve{b}
\end{align*}
%
as the crossed product of two parallel vectors yields $0$.
Therefore, we can write
%
\begin{align*}
- \div \L( \frac{\grad_\perp f \times \ve{b}}{B} \R)
 &= - \div \L( \frac{\grad f \times \ve{b}}{B} \R)
 %
 %
 \\
 &= \div \L(\frac{\ve{b}}{B}  \times \grad f \R)
 %
 \note{\hspace{-3cm}$\div(\ve{a}\times\ve{b}) =
       \ve{b} \cdot (\curl \ve{a}) - \ve{a} \cdot (\curl \ve{b}) $}
 \\
 %
 %
 &= \grad f \cdot \L(\curl \frac{\ve{b}}{B}\R) -
    \frac{\ve{b}}{B} \cdot (\curl  [\grad f])
 %
 \note{\hspace{-2.5cm}$\curl \grad f = 0$}\\
 %
 %
 &= \grad f \cdot \L(\curl \frac{\ve{b}}{B}\R)
 %
 \note{\hspace{-2.5cm}$\curl(f\ve{a}) = f(\curl\ve{a}) - \ve{a}\times(\grad
       f)$}\\
 %
 %
 &= \grad f \cdot
    \L(
    \frac{1}{B}\L[\curl\ve{b}\R]- \ve{b}\times\L[\grad\frac{1}{B}\R]
    \R)
 \\
 %
 %
 &= \frac{1}{B}\grad f\cdot\L[\curl\ve{b}\R] -
    \L(\ve{b}\times\L[\grad\frac{1}{B}\R]\R) \cdot \grad f\\
 %
 &= \frac{1}{B}\grad f\cdot\L[\curl\ve{b}\R] +
    \L(\L[\grad\frac{1}{B}\R] \times \ve{b}\R) \cdot \grad f
 %
 \note{$\ve{a}\cdot(\ve{b}\times\ve{c}) =
        \ve{b}\cdot(\ve{c}\times\ve{a}) =
        \ve{c}\cdot(\ve{a}\times\ve{b})$}\\
 %
 %
 &= \frac{1}{B}\grad f\cdot\L[\curl\ve{b}\R]  +
    \L[\grad\frac{1}{B}\R] \cdot \L( \ve{b} \times \grad f\R)\\
 %
 &\defined \mathcal{C}(f),
 \label{eq:curv_op}
 \numberthis
\end{align*}
%
which is non-zero only if the $\ve{B}$ field curves.

\subsection{Collisional drifts}
%
From equation (2.6) in \cite{Braginskii1965}, we have that
%
\begin{align*}
    \ve{R}_{i \to e}
    =
    \ve{R}_{u} + \ve{R}_{T},
\end{align*}
%
where
%
\begin{align*}
    \ve{R}_{u}
    =
    -\frac{m_en_e}{\tau_e}\L(0.51\L[\ve{u}_{e,\|}-\ve{u}_{i,\|}\R] +
    \L[\ve{u}_{e,\perp}-\ve{u}_{i,\perp}\R]\R)
\end{align*}
%
and
%
\begin{align*}
    \ve{R}_{T}
    =
    -0.71n_e\grad T_e -
    \frac{3}{2}\frac{n_e}{\om_e\tau_e}\ve{b}\times\grad T_e,
\end{align*}
%
where $\tau_e$ in the inverse of the electron-ion collision frequency $\nu_{ei}$ (given analytically in \cref{sec:nue}).
Assuming constant $T_e$, we get that
%
\begin{align}
    \ve{R}_{i \to e}
    = \ve{R}_{u}
   &= -m_en_e\nu_{ei}\L(0.51\L[\ve{u}_{e,\|}-\ve{u}_{i,\|}\R] +
      \L[\ve{u}_{e,\perp}-\ve{u}_{i,\perp}\R]\R).
   \label{eq:res_term_full}
\end{align}
%
Later, we will insert \cref{eq:res_term_full} into the resistive drift $\ve{u}_{\a,R}$.
As $\ve{u}_{\a,R}$ is already a first order drift, and since we are already interested in an accuracy of only $\mathcal{O}(\e)$, we substitute only the zeroth order drifts into the perpendicular velocities in \cref{eq:res_term_full}.
This yields
%
\begin{align*}
    \ve{R}_{i \to e}
    =&
    -m_en_e\nu_{ei}
   \L( 0.51 \L[\ve{u}_{e,\|}-\ve{u}_{i,\|}\R] +
      \L[
         \L(
           \ve{u}_{e,d} + \ve{u}_E
          \R)
          -
         \L(
          \ve{u}_{i,d} + \ve{u}_E
         \R)
      \R]
   \R)
   \\
%
%
   =&
   -m_en_e\nu_{ei}
   \L( 0.51 \L[\ve{u}_{e,\|}-\ve{u}_{i,\|}\R] +
      \L[
         \L(
            -
            \frac{\grad_\perp p_e \times \ve{b}}{q_en_eB}
            -
            \frac{\grad_\perp \phi \times \ve{b}}{B}
          \R)
          -
         \L(
            -
            \frac{\grad_\perp p_i \times \ve{b}}{q_in_iB}
            -
          \frac{\grad_\perp \phi \times \ve{b}}{B}
         \R)
      \R]
   \R)
   \\
%
%
   =&
   -m_en_e\nu_{ei}
   \L(0.51\L[\ve{u}_{e,\|}-\ve{u}_{i,\|}\R] + \frac{\grad_\perp \L(p_e + p_i\R) \times \ve{b}}{enB}
  \R)
\end{align*}
%
for the resistive force density acting on the electrons by the ions.
From momentum conservation we have that $\ve{R}_{i \to e}=-\ve{R}_{e \to i}$, which gives
%
\begin{align*}
    \frac{\ve{R}_{\b \to \a}}{q_\a} = - \frac{\ve{R}_{i \to e}}{\L|\frac{q_\a}{e}\R|e}.
\end{align*}
%
Inserting this into $\ve{u}_{\a,R}$ of \cref{eq:first_order} yields
%
\begin{align*}
  \ve{u}_{\a,R} =& \frac{\ve{R}_{\b\to\a}\times\ve{b}}{n_\a q_\a B}\\
  =& -\frac{\ve{R}_{i \to e,\perp}\times\ve{b}}{n_\a \L|\frac{q_\a}{e}\R|e B}\\
  =& -\frac{1}{n_\a \L|\frac{q_\a}{e}\R|e B}
  \L(-m_en_e\nu_{ei} \L[0.51\L(\ve{u}_{e,\|}-\ve{u}_{i,\|}\R) + \frac{\grad_\perp \L(p_e + p_i\R) \times \ve{b}}{enB} \R] \times\ve{b}\R)
  \note{$\ve{a}_\|\times\ve{b}=0$}
  \\
  =& \frac{m_en_e\nu_{ei} }{n_\a \L|\frac{q_\a}{e}\R|e B}
  \L( \frac{\grad_\perp \L[p_e + p_i\R] \times \ve{b}}{enB} \R) \times\ve{b}
  \note{Quasi-neutrality}
  \\
  =& \frac{m_e\nu_{ei} }{e B}
   \ve{b}\times\L( \ve{b}\times\frac{\grad_\perp \L[p_e + p_i\R]  }{enB} \R)
  \note{$-\ve{b}\times\ve{b}\times \ve{a}=\ve{a}_\perp$}
  \\
  =& - m_e\nu_{ei}  \frac{\grad_\perp \L(p_e + p_i\R)  }{n\L(eB\R)^2}
  \numberthis
  \label{eq:resDrift}
\end{align*}
%

There will also be a resistive force density on the electrons and ions from the collisions with neutrals.
To address this force drift, we start by assuming that
%
\begin{align*}
    \ve{R}_{n\to\a} =
    -
     m_\a n_\a \nu_{\a n}\L(
        \L[\ve{u}_{\a,\|}-\ve{u}_{n,\|} \R]
        +
        \L[\ve{u}_{\a,\perp}-\ve{u}_{n,\perp} \R]
        \R),
\end{align*}
%
where $\nu_{\a n}$ is given analytically in \cref{sec:nue,sec:nui}.
As we here want to model the neutrals as a static background, we get that
%
\begin{align}
    \ve{R}_{n\to\a} =
    -
     m_\a n_\a \nu_{\a n}\L(
        \ve{u}_{\a,\|}
        +
        \ve{u}_{\a,\perp}
        \R).
    \label{eq:neutRStat}
\end{align}
%
\Cref{eq:neutRStat} will be substituted into the $\ve{u}_{\a,R}$ drift of \cref{eq:first_order}.
As this is already a first order drift, we substitute only the zeroth order drifts into \cref{eq:neutRStat} the perpendicular velocities.
This gives
%
\begin{align}
    \ve{R}_{n\to\a} =
    - m_\a n_\a \nu_{\a n}\L( \ve{u}_{\a,\|}
    - \frac{\grad_\perp p_\a \times \ve{b}}{q_\a n_\a B}
    - \frac{\grad_\perp \phi \times \ve{b}}{B} \R).
    \label{eq:neut_res}
\end{align}
%
Inserting \cref{eq:neut_res} into the Pedersen drift of \cref{eq:first_order} yields
%
\begin{align*}
    \ve{u}_{\a,\text{Ped}}
    =&
    - \frac{ m_\a n_\a \nu_{\a n} }{n_\a q_\a B} \L( \ve{u}_{\a,\|}
    - \frac{\grad_\perp p_\a \times \ve{b}}{q_\a n_\a B}
    - \frac{\grad_\perp \phi \times \ve{b}}{B} \R) \times\ve{b}
    \note{Definition of perp. vectors}
    \\
    =&
    - \frac{\nu_{\a n} }{\om_{c\a}}
    \L(\frac{\grad_\perp p_\a}{q_\a n_\a B}
    + \frac{\grad_\perp \phi}{B} \R).
\end{align*}
%

\section{Zeroth order perpendicular terms}
%
We will now evaluate the individual terms of \cref{eq:cont_eq}, starting with the zeroth order terms.
By using the curvature operator from \cref{eq:curv_op}, we find that the $\ve{E}\times\ve{B}$ term gives
%
\begin{align*}
    \div \L(n_\a \ve{u}_{E}\R)
    &=
    n_\a \div \ve{u}_{E}
    + \ve{u}_{E} \cdot \grad n_\a
    \\
    &=
    n_\a \div \frac{-\grad_\perp \phi \times \ve{b}}{B}
    + \ve{u}_{E} \cdot \grad n_\a
    \\
    &=
    n_\a \mathcal{C}(\phi)
    + \ve{u}_{E} \cdot \grad n_\a.
    \numberthis
    \label{eq:div_ExB}
\end{align*}
%
and the diamagnetic term gives
%
\begin{align*}
    \div\L( n_\a \ve{u}_{\a,d} \R) =
    -\div\L( n_\a
    \frac{\grad_\perp p_\a \times\ve{b}}{q_\a n_\a  B}
    \R)
    %
    =
    -\div\L(
    \frac{\grad_\perp p_\a \times\ve{b}}{q_\a B}
    \R)
    =
    \mathcal{C}\L(\frac{p_\a}{q_\a}\R).
    \numberthis
 \label{eq:div_n_ud}
\end{align*}
%
In other words, we find that $n \div \ve{u}_{\a,d}$ cancels $\ve{u}_{\a,d} \cdot \grad n$ in the absence of magnetic field inhomogeneities.
This cancellation is referred to as \emph{diamagnetic cancellation} in the literature \cite{Garcia2005b}.

\section{First order perpendicular terms}
\label{sec:secondPerp}
%
We here discuss the resistivity terms and the source term of \cref{eq:cont_eq}.
The discussion of the polarization and viscous term will be given in \cref{sec:gyrovisc}.

%
From \cref{eq:resDrift} we get that
%
\begin{align}
 \div\L( n_\a \ve{u}_{e,R} \R)
 =
 - \div\L(\L|\frac{q_\a}{e}\R| m_e\nu_{ei}  \frac{\grad_\perp \L[p_e + p_i\R]  }{n\L[eB\R]^2}\R)
 =
 - \div\L(\L|\frac{q_\a}{e}\R|m_e\nu_{ei}  \frac{\grad_\perp \L[p_e + p_i\R]  }{\L[eB\R]^2}\R)
 \label{eq:div_nur}
\end{align}
%
We will from this point on assume that all collision frequencies are constant.
As seen from \cref{app:collisions} when assuming quasi-neutrality:
%
\begin{align*}
 \nu_{ei}\propto
 \frac{n}{T_e^{3/2}}\ln\Lambda \propto
 \frac{n}{T_e^{3/2}}\ln\L(n \L[\frac{T_e}{n}\R]^{3/2}\R)
\end{align*}
%
as we assume constant electron temperature, we get
%
\begin{align*}
 \partial_i \nu_{ei}\propto
 \partial_i n\ln\L(n^{-1/2}\R)\propto
 - \frac{1}{2}\L(\partial_i n\R)\L(\ln\L[n\R]+1\R)
 \appropto \frac{1}{2}\L(\partial_i n\R)\ln\L(n\R).
\end{align*}
%
As the logarithm of $n$ is slowly varying for high $n$, we see that the approximation is good as long as the gradients in $n$ are small.
Using this in \cref{eq:div_nur} gives
%
\begin{align*}
    \div(n_\a\ve{u}_{\a,R})
  =&
  - \L|\frac{q_\a}{e}\R| \frac{m_e\nu_{ei}}{e^2}
 \div\L(  \frac{\grad_\perp \L[p_e + p_i\R]  }{B^2}\R).
 \numberthis
 \label{eq:divRes}
\end{align*}
%
Note that although this term may be small as it is $\propto m_e$, we keep it as the term contributes to perpendicular diffusion of $n$ through the divergence of the pressure gradients.
This diffusion removes smalles scales from the system as it flattens the gradients.

Next, the Pedersen drift term yields
%
\begin{align*}
    \div\L(n_\a \ve{u}_{\a,\text{Ped}}\R)
    =
    \div\L(
    - n_\a \frac{\nu_{\a n} }{\om_{c\a}}
    \L[\frac{\grad_\perp p_\a}{q_\a n_\a B}
    + \frac{\grad_\perp \phi}{B} \R]
    \R)
    =
    \div\L(
    - n_\a \frac{\nu_{\a n} }{\om_{c\a}}
    \L[\frac{\grad_\perp p_\a}{q_\a n_\a B}
    + \frac{\grad_\perp \phi}{B} \R]
    \R)
    ,
\end{align*}
%
and the source term yields
%
\begin{align*}
    \div\L(n_\a \ve{u}_{\a,S}\R)
    =&
    \div\L(
    - n_\a
  \frac{ S_{\a,n}}{n_\a \om_{c\a}}
  \L[
  \frac{\grad_\perp p_\a}{n_\a  q_\a B}
  + \frac{\grad_\perp \phi}{B}
  \R]    \R)
  =
  -
  \div\L(
  \frac{ S_{\a,n}}{\om_{c\a}}
  \L[
  \frac{\grad_\perp p_\a}{n_\a  q_\a B}
  + \frac{\grad_\perp \phi}{B}
  \R]    \R).
\end{align*}
%

\subsection{Polarization and visocisty}
\label{sec:gyrovisc}
%
For further reference, it suffice to observe that $\ve{u}_{e,p}\propto\frac{1}{\om_{ce}}$, and that electron viscosity is of order $\mathcal{O}(\e^2)$ as seen from \cref{eq:DO}.

For the ions, we note that a material derivative appears in the polarization drift in \cref{eq:first_order}.
The material derivative must be treated with care.
Because of the advective term in the material derivative, we will for example have that
%
\begin{align*}
    \div \d_t \ve{v} \neq  \d_t \L(\div \ve{v}\R).
\end{align*}
%

One also have to take care about the advection term itself.
It appears that all the drifts can contribute to the advective part of the material derivative, but only the parallel velocity and the $\ve{E}\times\ve{B}$-drift contributes to the advection.
Although not trivial seen in the drift-fluid picture, it comes as a consequence of what is being referred to as \emph{gyroviscous cancellation} \cite{Chang1992,Smolyakov1998}.
Briefly explained, the cancellation comes as a consequence of the viscous part of the stress tensor cancels the diamagnetic drift.
This means that
% Remember that the u_e looking part in the polarization is not really u_e, but
% u_exb
%
% Previously tried:
% 1. Split n first (Volker)
% 2. Split parallel and put n into ddt
\begin{align*}
    &\div\L( n \ve{u}_{i,p} + n \ve{u}_{i,\nu}\R)
 \\
 \simeq&
 \div\L( n \frac{1}{\om_{ci}}
  \L[ \partial_t + (\ve{u}_{E} + \ve{u}_{i,\|})\cdot\nabla \R]
  \L[ - \frac{\grad_\perp \phi}{B} \R]
 \R)
 \\
 %
 =&
 - \div\L( \frac{n}{\om_{ci}}
  \L[ \d_t^E + \ve{u}_{i,\|}\cdot\nabla \R]
  \frac{\grad_\perp \phi}{B}
 \R)
 \numberthis
 \label{eq:start_of_boussinesq}
 \\
 %
 =&
 - \div\L( \frac{1}{\om_{ci}}\L[
 \L( \d_t^E + \ve{u}_{i,\|}\cdot\nabla \R)
 \L(\frac{\grad_\perp \phi}{B}n\R)
 -\frac{\grad_\perp \phi}{B}
 \L( \d_t^E + \ve{u}_{i,\|}\cdot\nabla \R)
 n
 \R]
 \R)
 \\
 %
 =&
 - \div\L( \frac{1}{\om_{ci}}
 \L[ \d_t^E + \ve{u}_{i,\|}\cdot\nabla \R]
 \L[\frac{\grad_\perp \phi}{B}n\R]
 \R)
 +
 \div\L( \frac{1}{\om_{ci}}
 \frac{\grad_\perp \phi}{B}
 \L[ \d_t^E + \ve{u}_{i,\|}\cdot\nabla \R]
 n
 \R).
\numberthis
\label{eq:from_gyroviscous_n_not_subs}
\end{align*}
%
We see that we can insert $Z$ times the ion continuity equation in the last term of \cref{eq:from_gyroviscous_n_not_subs}.
Since \cref{eq:from_gyroviscous_n_not_subs} is of order $\e$, which means that when inserted the order $\e$ terms from $Z$ times the ion continuity equation (that is $\ve{u}_{i,p}$, $\ve{u}_{i,R}$, $\ve{u}_{i,\text{Ped}}$, $\ve{u}_{i,\nu}$, $\ve{u}_{i,S}$) will be of order $\e^2$, and will therefore be neglected.
$Z$ times the ion continuity equation to order $\e^0$ reads
%
\begin{align*}
 \partial_t n + \div (n [ \ve{u}_{i,d} + \ve{u}_E + \ve{u}_{i,\|} ])
 &= S_{i, n}
 \\
 \partial_t n + \div (n \ve{u}_{i,d}) + \div (n\ve{u}_E) + \div (n\ve{u}_{i,\|} )
 &= S_{i, n}
 \note{\cref{eq:div_ExB,eq:div_n_ud}}
 \\
 \partial_t n
 + \frac{1}{e}\mathcal{C}\L(p_i\R)
 + n \mathcal{C}(\phi)
 + \ve{u}_E \cdot \grad n
 + n \div \ve{u}_{i,\|}
 + \ve{u}_{i,\|} \cdot \grad n
 &= S_{i, n}
 \\
 \partial_t n
 + \ve{u}_E \cdot \grad n
 + \ve{u}_{i,\|} \cdot \grad n
 &=
 S_{i, n}
 - \frac{1}{e}\mathcal{C}\L(p_i\R)
 - n \mathcal{C}(\phi)
 - n \div \ve{u}_{i,\|}
 \\
 \L(\d_t^E + \ve{u}_{i,\|} \cdot \grad\R) n
 &=
 S_{i, n}
 - \frac{1}{e}\mathcal{C}\L(p_i\R)
 - n \mathcal{C}(\phi)
 - n \div \ve{u}_{i,\|}.
 \numberthis
 \label{eq:i_cont}
\end{align*}
%
where we have defined
%
\begin{empheq}[box=\tcbhighmath]{align*}
    \d_t^E = \partial_t + \ve{u}_E\cdot\grad
\end{empheq}
%
Inserting \cref{eq:i_cont} into \cref{eq:from_gyroviscous_n_not_subs} yields
%
\begin{align*}
    &\div\L( n \ve{u}_{i,p} + n \ve{u}_{i,\nu}\R)
 \\
 \simeq&
 - \div\L( \frac{1}{\om_{ci}}
 \L[ \d_t^E + \ve{u}_{i,\|}\cdot\nabla \R]
 \L[\frac{\grad_\perp \phi}{B}n\R]
 \R)
 +
 \div\L( \frac{1}{\om_{ci}}
 \frac{\grad_\perp \phi}{B}
 \L[
 S_{i, n}
 - \frac{1}{e}\mathcal{C}\L(p_i\R)
 - n \mathcal{C}(\phi)
 - n \div \ve{u}_{i,\|}
 \R]
 \R).
\numberthis
\label{eq:from_gyroviscous}
\end{align*}
%
This equation has a central part in the current balance equation, as it ultimately will be responsible for the time derivative of the modified vorticity in \cref{chap:CELMA}.

\section{The electron density equation}
%
We can now use our results to derive an equation for the time derivative of the electron density.
Based on the quasi neutral assumption in \cref{sec:qn}, the evolution of the density can be described by both the electron continuity equation and the ionization number $Z$ times the ion continuity equation.
The two should differ only slightly.
We can therefore choose to use the electron continuity equation to calculate the evolution of the density.
This gives
%
\begin{align}
    \partial_t n_e + \div (n_e [
 \ve{u}_{e,d} + \ve{u}_E + \ve{u}_{e,p} + \ve{u}_{e,R}
 + \ve{u}_{e,\text{Ped}}
 + \ve{u}_{e,\nu}
 + \ve{u}_{e,S} + \ve{u}_{e,\|}
 ]) &= S_{n, e}.
 \label{eq:el_cont}
\end{align}
%
We neglect $\ve{u}_{e,\nu}$ as this drift is of $\mathcal{O}(\e^2)$.
Next, we observe that $\ve{u}_{e,p}$, $\ve{u}_{e,\text{Ped}}$ and $\ve{u}_{e,S}$ are small compared to the rest of the terms as they are propotional the electron mass through the $\om_{ce}^{-1}$ factor.
Using that $n_e\simeq n$, we get that
%
\begin{align*}
    \partial_t n &\simeq - \div\L(n\L[\ve{u}_E + \ve{u}_{e,D} + \ve{u}_{e,R} + \ve{u}_{e,\|}\R]\R) + S_{e,n}
    \\
%
    &=
    - \ve{u}_E\nabla\cdot n
    - n\div\ve{u}_E
    - \div\L(n \ve{u}_{e,D}\R)
    - \div\L(n \ve{u}_{e,R}\R)
    - \div\L(n \ve{u}_{e,\|}\R)
    + S_{e,n}\\
%
    \d_t^E n
    &=
    - n\mathcal{C}(\phi)
    + \frac{1}{e}\mathcal{C}(\phi)
    +\frac{1}{\mu}\frac{m_i\nu_{ei}}{e^2}
    \div\L( \frac{\grad_\perp \L[p_e + p_i\R]}{B^2} \R)
    - \div\L(n \ve{u}_{e,\|}\R)
    + S_{e,n}.
    \numberthis
    \label{eq:dens_evol_gen}
\end{align*}

\section{Current conservation equation}
\label{sec:curConserve}
%
\Cref{eq:dens_evol_gen} gives the equation to solve the density in time.
Similarly, we could make an equation that evolves $\ve{u}_{\a,\perp}$ in time, which would require one equation per sepcies for each perpendicular direction.
However, as we will see later, all the information we need for evolving $\ve{u}_{\a,\perp}$ in time are contained in the parallel part of the vorticity equation ($(\curl \ve{v}_E)\cdot \ve{b}$).
This equation can be derived from the current conservation, which we present in this section.
If we multiply the two continuity equations of \cref{eq:cont_eq} with $q_\a$, and add them, we get by applying quasi-neutrality that
%
\begin{align*}
    q_i\partial_t n_i + q_i\div (n_i \ve{u}_i)
    + q_e\partial_t n_e + q_e\div (n_e \ve{u}_e)
    =&
    q_iS_{i,n} + q_eS_{e,n}
    \\
    %
    Ze\partial_t n_i + Ze\div (n_i \ve{u}_i)
    - e\partial_t n_e - e\div (n_e \ve{u}_e)
    =&
    eZS_{i,n} - eS_{e,n}
    \note{Quasi-neutrality}
    \\
    %
    e\partial_t n + e\div (n \ve{u}_i)
    - e\partial_t n - e\div (n \ve{u}_e)
    =&
    eZS_{i,n} - eS_{e,n}
    \\
    %
    \div (en \ve{u}_i - en \ve{u}_e) =&
    e\L(ZS_{i,n} - S_{e,n}\R).
\end{align*}
%
If we use \cref{eq:kinSource} together with $\ve{j} = \sum_{\a}q_\a n_\a\ve{u}_\a$, we get that
%
\begin{align*}
    \\
    %
    \div \ve{j} =&
    e\L(S_{n} - S_{n}\R)
    \\
    %
    \div \L( \ve{j}_\perp + \ve{j}_\|\R)=& 0
    \\
    %
    \div \ve{j}_\perp =& -\div \ve{j}_\|.
\end{align*}
%
This gives
%
\begin{align*}
    \div \L( n [\ve{u}_{i,\perp} - \ve{u}_{e,\perp}] \R) =&
    -\frac{1}{e}\div \ve{j_\|}.
\end{align*}
%
Once again, we can neglect the electron drifts proportional to the electron mass as these terms will be small.
This gives
%
\begin{align}
    \div \L( n [
   \ve{u}_{E} - \ve{u}_{E}
  +\ve{u}_{i,d} - \ve{u}_{e,d}
  +\ve{u}_{i,p}
  +\ve{u}_{i,R} - \ve{u}_{e,R}
  +\ve{u}_{i,\text{Ped}}
  +\ve{u}_{i,\nu}
  +\ve{u}_{i,S}
  ] \R) =&
  -\frac{1}{e}\div \ve{j_\|}.
  \label{eq:current_drifts}
\end{align}
%
The $\ve{E}\times\ve{B}$ drift in \cref{eq:current_drifts} cancels as the drift is equal for electrons and ions.
The diamagnetic terms yield
%
\begin{align*}
 \div\L( n\ve{u}_{e,d} \R) - \div\L( n \ve{u}_{i,d} \R)
 &=
 \div\L( n \frac{-\grad_\perp p_e}{-n_ee B} \R) - \div\L( n \frac{-\grad_\perp p_i}{n_iZe B} \R)
 \note{Quasi-neutrality}
 \\
 %
 &=
 -\frac{1}{e}\L[\mathcal{C}(p_e) + \mathcal{C}\L(p_i\R)\R]
  \note{$\mathcal{C}(f) + \mathcal{C}(g) = \mathcal{C}(f+g)$\\
  }
 \\
 %
 &=
  -\frac{1}{e}\L[\mathcal{C}\L(p_e+p_i\R)\R].
\end{align*}
%
From \cref{eq:divRes} we can see that the resistivity terms cancels as the terms will have opposite sign for electrons and ions.
Next, the electron Pedersen drift is negligible due to the electron mass, whereas the ion Pedersen drift can be split into
%
\begin{align*}
    \div\L(n\ve{u}_{i,\text{Ped}} \R)
    =&
    -
    \div\L(\frac{n\nu_{in}}{\om_{ci}}
        \L[
            \frac{\grad_\perp p_i}{eZn_iB}
            +
            \frac{\grad_\perp \phi}{B}
        \R]
        \R)
        \note{$Zn_i \simeq n$}
        \\
    %
    =&
    -
    n
    \div\L(\frac{\nu_{in}}{\om_{ci}}
        \L[
            \frac{\grad_\perp p_i}{enB}
            +
            \frac{\grad_\perp \phi}{B}
        \R]
        \R)
    -
    \L(\frac{\nu_{in}}{\om_{ci}}
        \L[
            \frac{\grad_\perp p_i}{enB}
            +
            \frac{\grad_\perp \phi}{B}
        \R]
        \R)
        \cdot\grad
        n.
    \numberthis
    \label{eq:div_ped}
\end{align*}
%
By summing the different contributions, \cref{eq:current_drifts} can be now written as
%
\begin{align*}
  -\frac{1}{e}\div \ve{j_\|}
  =&
    \div \L( n [
   \ve{u}_{i,d} - \ve{u}_{e,d}
   +\ve{u}_{i,\text{Ped}}
  +\ve{u}_{i,p}
  +\ve{u}_{i,\nu}
  +\ve{u}_{i,S}
  ] \R)
  \\
%
%
=&
-\frac{1}{e}\L[\mathcal{C}\L(p_e+p_i\R)\R]
  \\
  &
  - n \div\L(\frac{\nu_{in}}{\om_{ci}}
        \L[ \frac{\grad_\perp p_i}{enB} + \frac{\grad_\perp \phi}{B} \R] \R)
  - \L(\frac{\nu_{in}}{\om_{ci}}
        \L[ \frac{\grad_\perp p_i}{enB} + \frac{\grad_\perp \phi}{B} \R] \R)
        \cdot\grad n
  \\
  %
  &
 - \div\L( \frac{1}{\om_{ci}}
 \L[ \d_t^E + \ve{u}_{i,\|}\cdot\nabla \R]
 \L[\frac{\grad_\perp \phi}{B}n\R] \R)
 \\&
 +
 \div\L( \frac{1}{\om_{ci}}
 \frac{\grad_\perp \phi}{B}
 \L[S_{i, n} - \frac{1}{e}\mathcal{C}\L(p_i\R) - n \mathcal{C}(\phi)
 - n \div \ve{u}_{i,\|} \R] \R)
 \\
%
 &
    - \div \L( \frac{ S_{i,n}}{\om_{ci}}
      \L[ \frac{\grad_\perp p_i}{n_i q_i B} + \frac{\grad_\perp \phi}{B} \R]
    \R)
  \\
%
%
\frac{1}{e}\div \ve{j}_\|
=&
\frac{1}{e}\L[\mathcal{C}\L(p_e+p_i\R)\R]
  \\
  &
  +  n \div\L(\frac{\nu_{in}}{\om_{ci}}
        \L[ \frac{\grad_\perp p_i}{enB} + \frac{\grad_\perp \phi}{B} \R] \R)
  + \L(\frac{\nu_{in}}{\om_{ci}}
        \L[ \frac{\grad_\perp p_i}{enB} + \frac{\grad_\perp \phi}{B} \R] \R)
        \cdot\grad n
  \\
  %
  &
 + \div\L( \frac{1}{\om_{ci}}
 \L[ \d_t^E + \ve{u}_{i,\|}\cdot\nabla \R]
 \L[\frac{\grad_\perp \phi}{B}n\R] \R)
 \\&
 -
 \div\L( \frac{1}{\om_{ci}}
 \frac{\grad_\perp \phi}{B}
 \L[S_{i, n} - \frac{1}{e}\mathcal{C}\L(p_i\R) - n \mathcal{C}(\phi)
 - n \div \ve{u}_{i,\|} \R] \R)
  \\
%
  &
    + \div \L( \frac{ S_{i,n}}{\om_{ci}}
      \L[ \frac{\grad_\perp p_i}{n_i q_i B} + \frac{\grad_\perp \phi}{B} \R]
    \R).
  \numberthis
  \label{eq:full_vort_eq}
\end{align*}
%
So far we have balanced the divergence of the parallel currents with the divergence of the perpendicular currents (using the drifts).
From this we have obtained a time derivative from the ion polarization drift (the third term in \cref{eq:full_vort_eq}).
In the next chapter, we will see how the ion polarization term turns into the time derivative of the parallel vorticity of the $\ve{E}\times\ve{B}$-drift under the assumption of a homogeneous $B$-field.
The homogeneous $B$-field will simplify our work of extracting the vorticity.
Therefore, we end our derivations in a slowly varying $B$-field here.
