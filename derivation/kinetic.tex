We will here derive the fluid equations used in the rest of this thesis from the Fokker-Planck%
\footnote{
    This equation can again be derived from the Klimontovich equation (see for example \cite{Klimontovich1983book}).
}
%
 equation by following the approach of \cite{Helander2002book}.
In addition we will include an inelastic source.
This source will either add or subtract particles to the distribution function depending on its sign.

We are looking at a distribution function $f_\a(\ve{r},\ve{v},t)$ of species $\a$, at point $\ve{z}(t)=(\ve{r}(t),\ve{v}(t))$ in the phase-space at time $t$.
The particles obey the conservation equation
%
\begin{align*}
    \partial_t f_\a + \partial_{\ve{z}} \L(\L[\d_t \ve{z}\R]f_\a\R) =& S'_\a,
\end{align*}
%
where $S'_\alpha$ is a source in the distribution function for species $\a$.
We will here use a source which fulfills the following:
%
\begin{align*}
    &\inde{S'_{\a}}{^3v} = S_{n,\a}&
    &\text{Particles are created.}&
    \\
    &\inde{m_\a \ve{v} S'_\a}{^3v} = 0&
    &\text{They are created with zero bulk velocity.}&
    \\
    &\inde{\frac{m_\a \ve{v}^2}{2} S'_\a}{^3v} = S_{E,\a}&
    &\text{They have a finite energy when created.}&
\end{align*}
%
Using that
%
\begin{align*}
    \d_t \ve{r}     =& \ve{v}\\
    \d_t \ve{v}     =& \frac{q_\a}{m_\a}\L(\ve{E}'+\ve{v}\times\ve{B}'\R)\\
    \partial_{\ve{v}} \L(\ve{E}'+\ve{v}\times\ve{B}'\R) =& 0,
\end{align*}
%
where $\ve{E}'$ and $\ve{B}'$ are the total fields, we get
%
\begin{align*}
      \partial_t f_\a
    + \ve{v}\cdot\nabla f_\a
    + \frac{q_\a}{m_\a}
      \L(\ve{E}'+\ve{v}\times\ve{B}'\R)\cdot\partial_{\ve{v}}f_\a
    =& S_\a.
\end{align*}

\section{Fokker-Planck equation}
We now introduce $\ve{E}$ and $\ve{B}$, which are the fields averaged over several Debye lengths.
To incorporate the microscopic fluctuations of the fields, we introduce the collision operator
%
\begin{align*}
    C_\a = \sum_\g C_{\a\g}(f_\a,f_\g)
\end{align*}
%
In the scope of this thesis, we will consider only elastic collisions between fully ionized, cold ions, electrons and cold neutrals (which only act as a static background).
In other words, we will let $\gamma \in \{e, i, n\}$ (denoting electrons, ions and neutrals), so that
%
\begin{align*}
    C_\a = C_{\a\b}(f_\a,f_\b) + C_{\a n}(f_\a,f_n)
\end{align*}
%
where $\a,\b \in \{e, i\}$ and $\a\neq \b$. Furthermore, the collision operator has the following properties:
%
\begin{align*}
    &\inde{C_{\a\g}}{^3v} = 0&
    &\text{Particle conservation}&
    \\
    &\inde{m_\a \ve{v} C_{\a\g}}{^3v} = -\inde{m_\a \ve{v} C_{\g\a}}{^3v}&
    &\text{Momentum conservation}&
    \\
    &\inde{\frac{m_\a \ve{v}^2}{2} C_{\a\g}}{^3v} =
    -\inde{\frac{m_\a \ve{v}^2}{2} C_{\g\a}}{^3v}&
    &\text{Energy conservation}&
\end{align*}
%
The resulting Fokker-Planck equation reads
%
\begin{empheq}[box=\tcbhighmath]{align}
      \partial_t f_\a
    + \ve{v}\cdot\nabla f_\a
    + \frac{q_\a}{m_\a}\L(\ve{E}+\ve{v}\times\ve{B}\R)\cdot\partial_{\ve{v}}f_\a
    =& S_\a + C_\a.
    \label{eq:fp}
\end{empheq}

\section{Moments of Fokker-Planck}
To follow the distribution functions in a 6-dimensional phase-space as a function of time is quite a daunting taks.
Instead, we will follow some averages of the distribution function.
This comes at the price of losing kinetic information in the system, like for example Landau damping.
We will use the notation
%
\begin{align*}
    \expt{A}_{f} \defined \frac{1}{n_\a}\inde{Af_\a}{^3v},
\end{align*}
%
where $\expt{1}_f=n_a$ is the density, to denote a weighted velocity average (a moment) of the Fokker-Planck equation.
From this, we define
%
\begin{align*}
    \expt{\ve{v}_\a}_f        \defined& \; \ve{u}_a,      \\
    \expt{m_\a n_\a \ve{v}_\a\ve{v}_\a}_f \defined&\; \te{\Pi}_\a.
\end{align*}
%
We call the difference between the velocity of a single particle and the macroscopic fluid velocity $\ve{w}$, so that
%
\begin{align*}
    \ve{w}_\a = \ve{v} - \ve{u}_\a.
\end{align*}
%
Thus, we define the scalar pressure $p_\a$, the pressure tensor $\te{P}_\a$ and the stress tensor $\te{\pi}_\a$ as
%
\begin{align*}
    \frac{n_\a m_\a\expt{w_\a^2}_f}{3} \defined& \; p_\a = n_\a T_\a\\
    n_\a m_\a\expt{\ve{w}_{\a}\ve{w}_{\a}}_f \defined& \; \te{P}_\a\\
    \te{\pi}_\a \defined& \; \te{P}_\a - \te{I}p_\a\\
\end{align*}
%
where $\te{I}$ is the identity tensor.
Finally we define
%
\begin{align*}
    \inde{m_\a\ve{v} C_{\a\g}(f)}{^3v} \defined& \ve{R}_{\g\to\a},
\end{align*}
%
where $\ve{R}_{\g\to\a}$ is the friction force on $\a$ given by $\g$.

\subsection{Zeroth moment}
We will now work through the zeroth moment term by term by applying $\expt{n_\a \;\cdot\;}_f=n_\a\expt{\;\cdot\;}_f$ on \cref{eq:fp}.
We have that
%
\begin{align*}
    \expt{\partial_t n_\a}_f
    &=
    \partial_t n_\a \expt{1}_f
    \\
%
%
    &=
    \partial_t n_\a,
\end{align*}
%
that
%
\begin{align*}
    \expt{n_\a\ve{v}\cdot\nabla}_f
    &=
    n_\a
    \div\expt{ \ve{v}}_f
    \note{$\ve{x}$ and $\ve{v}$ are independent coordinates.}
    \\
%
%
    &=
    n_\a
    \expt{\nabla\cdot\ve{v}}_f
    \\
    &=
    \frac{n_\a}{n_\a}
    \nabla\cdot\L(n_\a\ve{u}_\a\R)
    \note{$\nabla$ commutes with the integral}
%
%
    \\
    &=
    \nabla\cdot\L(n_\a\ve{u}_\a\R)
\end{align*}
%
and
%
\begin{align*}
    \expt{n_\a\frac{q_\a}{m_\a}\ve{E}\cdot\partial_{\ve{v}}}_f
    =&
    n_\a\frac{q_\a}{m_\a}\expt{\partial_{\ve{v}} \cdot\ve{E}}_f
    \note{$\ve{E}$ independent of $\ve{v}$}
    \\
%
    =&
    \frac{q_\a}{m_\a}\inde{\partial_{\ve{v}} \cdot\L(\ve{E}f_\a\R)}{^3v}
    \note{Gauss divergence theorem}
    \\
%
    =&
    \frac{q_\a}{m_\a}\defi{\partial_{\Om}}{}{\ve{E}f_\a}{S}
    \\
%
    =&
    0,
    \note{$f_\a\bigg|_{\pm \infty} =0 $ }
\end{align*}
%
and further that
%
\begin{align*}
    \expt{n_\a\frac{q_\a}{m_\a}\ve{v}\times\ve{B}\cdot\partial_{\ve{v}}}_f
    =&
    n_\a\frac{q_\a}{m_\a}\expt{\ve{v}\times\ve{B}\cdot\partial_{\ve{v}}}_f
    \\
%
%
    =&
    \frac{q_\a}{m_\a}
    \L(
      \inde{\partial_{\ve{v}}\cdot\L[\ve{v}\times\ve{B}f_\a\R]}{^3v}
      -
      \inde{f_\a\partial_{\ve{v}}\cdot\L[\ve{v}\times\ve{B}\R]}{^3v}
    \R)
    \\
%
%
    =&
    \frac{q_\a}{m_\a}
    \L(
      \defi{\partial_{\Om}}{}{\ve{v}\times\ve{B}f_\a}{S}
      -
      \inde{f_\a\partial_{\ve{v}}\cdot\L[\ve{v}\times\ve{B}\R]}{^3v}
    \R)
    \note{Vanishing surface integral, and
          $\ve{v}\times\ve{B}\perp\partial_{\ve{v}}$}
    \\
    =&
    0
\end{align*}
%
the source term gives
%
\begin{align*}
    \inde{S'_{\a}}{^3v} = S_{n,\a}
\end{align*}
%
We will now assume that the plasma consists of only one type of ions, in only one ionizational and excitational state.
Extension to this would require a distribution function for each set of atom at a specific ionizational and excitational state.
We will further assume that the creation of electrons and ions to this state happens instantaneously (that is, no intermediate ioniziations or excitations takes place).
If we take fully ionized helium as an example, this would mean that one ion, and two electrons would be created instantaneously.
More generally, $N_i$ ions and $ZN_i$ electrons will be created for the charge number $Z$.
In other words, we would have that
%
% % NOTE: This can be checked from continuity equations
%         ne > ni => ne simeq Zni
%         continuity eq ne simeq Z ni
%         From LHS we get that
\begin{align}
    S_{n,e}=ZS_{n,i}\defined S_n
    \label{eq:kinSource}
\end{align}
%
Finally the collision term gives
%
\begin{align*}
    \inde{C_{\a\b}}{^3v} = 0.
\end{align*}
%
Thus our continuity equation becomes
%
\begin{align}
    \partial_t n_\a + \div\L(n_\a \ve{u}_\a\R) = S_{n,\a} \label{kin:cont}.
\end{align}

\subsection{First moment}
We will now work through the first moment term by term by applying $\expt{m_\a n_\a \ve{v} \;\cdot\;}_f=m_\a n_\a\expt{ \ve{v} \;\cdot\;}_f$ on \cref{eq:fp}.
We have that
%
\begin{align*}
    \expt{\partial_t m_\a n_\a \ve{v}}_f
    &=
    \partial_t m_\a n_\a \expt{\ve{v}}_f
    \\
%
%
    &=
    \partial_t \L(m_\a n_\a \ve{u}_\a\R),
\end{align*}
%
that
%
\begin{align*}
    \expt{m_\a n_\a\ve{v}\ve{v}\cdot\nabla}_f
    &=
    \div\expt{m_\a n_\a\ve{v}\ve{v}}_f
    \note{$\ve{x}$ and $\ve{v}$ are independent coordinates}
    \\
%
%
    &=
    \div \te{\Pi}_\a
\end{align*}
%
and
%
\begin{align*}
    \expt{n_\a \ve{v} q_\a \ve{E}\cdot\partial_{\ve{v}}}_f
    =&
    n_\a q_\a \expt{ \ve{v}  \ve{E}\cdot\partial_{\ve{v}}}_f
    \note{$\ve{E}$ independent of $\ve{v}$}
    \\
%
    =&
    n_\a q_\a \expt{ \ve{v} \partial_{\ve{v}} \cdot \ve{E}}_f
    \\
%
    =&
    n_\a q_\a
    \L(
    \expt{ \partial_{\ve{v}} \cdot \L[\ve{v} \ve{E}\R]}_f
    -
    \expt{ \ve{E} \partial_{\ve{v}} \cdot \ve{v}}_f
    \R)
    \\
%
    =&
    n_\a q_\a
    \L(
    \frac{1}{n_\a}
    \inde{ \partial_{\ve{v}} \cdot \L[\ve{v} \ve{E} f_\a \R]}{^3v}
    -
    \ve{E}\expt{1}_f
    \R)
    \\
%
    =&
    n_\a q_\a
    \L(
    \frac{1}{n_\a}
    \defi{\partial_{\Om}}{}{ \partial_{\ve{v}} \cdot \L[\ve{v} \ve{E} f_\a
    \R]}{S}
    -
    \ve{E}
    \R)
    \note{$f_\a\bigg|_{\pm \infty} =0 $ }
    \\
%
    =&
    - n_\a q_\a \ve{E}
\end{align*}
%
further that
%
\begin{align*}
    \expt{n_\a q_\a \ve{v}\ve{v}\times\ve{B}\cdot\partial_{\ve{v}}}_f
    =&
    n_\a q_\a \expt{\ve{v}\ve{v}\times\ve{B}\cdot\partial_{\ve{v}}}_f
    \\
%
%
    =&
    q_\a
    \L(
      \inde{\partial_{\ve{v}}\cdot\L[\ve{v}\ve{v}\times\ve{B}f_\a\R]}{^3v}
      -
      \inde{f_\a\partial_{\ve{v}}\cdot\L[\ve{v}\ve{v}\times\ve{B}\R]}{^3v}
    \R)
    \\
%
%
    =&
    q_\a
    \L(
      \defi{\partial_{\Om}}{}{ \ve{v}\ve{v}\times\ve{B}f_\a }{S}
      -
      \inde{f_\a\partial_{\ve{v}}\cdot\L[\ve{v}\ve{v}\times\ve{B}\R]}{^3v}
    \R)
    \note{Vanishing surface integral}
    \\
%
%
    =&
    -
    q_\a
    \L(
     \inde{f_\a\ve{v}\partial_{\ve{v}}\cdot\L[\ve{v}\times\ve{B}\R]}{^3v}
     +
     \inde{f_\a\partial_{\ve{v}}\cdot\L[\ve{v}\R]\ve{v}\times\ve{B}}{^3v}
    \R)
    \note{$\ve{v}\times\ve{B}\perp\partial_{\ve{v}}$}
    \\
%
%
    =&
    -
    q_\a
     \inde{f_\a1\ve{v}}{^3v}\times\ve{B}
    \\
%
%
    =&
    -
    q_\a
    n_\a
    \ve{u_\a}\times\ve{B}.
\end{align*}
%
As we assume that the particles will be generated without any momentum, we have that
%
\begin{align*}
    \inde{m_a\ve{v}S'_{\a}}{^3v} = 0
\end{align*}
%
and finally, we use the following collision operator
%
\begin{align*}
    \inde{m_a\ve{v}C_{\a}}{^3v} =
    \sum_\g \inde{m_a\ve{v}C_{\a\g}}{^3v} =
    \inde{m_a\ve{v}C_{\a\b}}{^3v}
    +
    \inde{m_a\ve{v}C_{\a n}}{^3v}
    =
    \ve{R}_{\b\to\a}
    +
    \ve{R}_{n\to\a},
\end{align*}
%
where the subscript $n$ in $\ve{R}$ denotes neutrals.
Thus our momentum equation can now be written as
%
\begin{align}
      \partial_t \L(n_\a m_\a \ve{u}_{\a} \R)
    + \div\te{\Pi}_\a
    - q_\a n_\a\L(\ve{E}  + \ve{u_\a}\times\ve{B}\R)
    =
    \ve{R}_{\b\to\a}
    +
    \ve{R}_{n\to\a}
      \label{eq:mom_crude}
\end{align}
%
The first term can be written as
%
\begin{align*}
      \partial_t \L(n_\a m_\a \ve{u}_{\a} \R)
      =&
      n_\a m_\a \partial_t \L(\ve{u}_{\a} \R)
      +
      \ve{u}_{\a}m_\a\partial_t n_\a
\end{align*}
%
The second term can be expanded further by observing that
%
\begin{align*}
    \te{\Pi}_\a
    =&
    \expt{m_\a n_\a \ve{v}\ve{v}}_f
    \\
%
%
     =&
    m_\a n_\a
    \expt{\ve{v}\ve{v}}_f
    \\
%
%
     =&
    m_\a n_\a
    \expt{
    \L(
        \ve{w}_\a
        +
        \ve{u}_\a
    \R)
    \L(
        \ve{w}_\a
        +
        \ve{u}_\a
    \R)
        }_f
    \\
%
%
     =&
    m_\a n_\a
    \expt{
        \ve{w}_\a
        \ve{w}_\a
        +
        \ve{u}_\a
        \ve{w}_\a
        +
        \ve{w}_\a
        \ve{u}_\a
        +
        \ve{u}_\a
        \ve{u}_\a
        }_f
    \\
%
%
     =&
    m_\a n_\a
    \L(
    \expt{
        \ve{w}_\a
        \ve{w}_\a
        }_f
        +
        \ve{u}_\a
    \expt{
        \ve{w}_\a
        }_f
        +
    \expt{
        \ve{w}_\a
        }_f
        \ve{u}_\a
        +
        \ve{u}_\a
        \ve{u}_\a
    \expt{
        1
        }_f
    \R)
    \note{$\ve{w}_\a$ fluctuates around mean, so averages to $0$}
    \\
%
%
     =&
    m_\a n_\a
    \L(
    \expt{
        \ve{w}_\a
        \ve{w}_\a
        }_f
        +
        \ve{u}_\a
        \ve{u}_\a
    \R)
    \\
%
%
     =&
    \te{P}_\a
        +
    m_\a n_\a
        \ve{u}_\a
        \ve{u}_\a
    \\
%
%
     =&
    \te{\pi}_\a
    +
    \te{I}p_\a
        +
    m_\a n_\a
        \ve{u}_\a
        \ve{u}_\a
\end{align*}
%
The pressure tensor $\te{P}_\a$ would have been isotropic if the particles would be free to move in all directions.
However, charged particles gyrate whenever moving with a vector component perpendicular to the magnetic field.
Thus, the pressure along the magnetic field line is different from the pressure perpendicular to it.
Nevertheless, we can observe from the definitions that the tensors must be symmetric.

The stress tensor is assumed to be small compared to the other terms, and can be further written as
%
\begin{align*}
    \te{\pi} = \te{\pi}^S + \te{\pi}^C + \te{\pi}^G
\end{align*}
%
where $\te{\pi}^S$ is the viscosity which comes as a consequence of similar species velocity shear, $\te{\pi}^C$ comes from velocity compression along the magnetic field and $\te{\pi}^G$ is attributed by finite Larmor radius (FLR) effects.

The second term of \cref{eq:mom_crude} can now be written as
%
\begin{align*}
    \div\te{\Pi}_\a
      =&
    \div \te{\pi}_\a
    + \div \te{I}p_\a
        + m_\a \div \L( n_\a \ve{u}_\a \ve{u}_\a \R)
    \note{$\div\L(\ve{a}\ve{b}\R) =
    \ve{a} \cdot \grad \ve{b} + \L(\div\ve{a}\R)\ve{b}$}
    \\
%
%
      =&
    \div \te{\pi}_\a
    + \grad p_\a
    + m_\a n_\a \ve{u}_\a \cdot \nabla \ve{u}_\a
    + m_\a \div \L( n_\a \ve{u}_\a \R) \ve{u}_\a.
\end{align*}
%
Thus, \cref{eq:mom_crude} can be written
%
\begin{align*}
    &\quad
      n_\a m_\a \partial_t \L(\ve{u}_{\a} \R)
      + \ve{u}_{\a}m_\a\partial_t n_\a
    + m_\a n_\a \ve{u}_\a \cdot \nabla \ve{u}_\a
    + m_\a \div \L( n_\a \ve{u}_\a \R) \ve{u}_\a
    \\
    &=
    - \div \te{\pi}_\a
    - \grad p_\a
    + q_\a n_\a\L(\ve{E}  + \ve{u_\a}\times\ve{B}\R)
    + \ve{R}_{\b\to\a}
    + \ve{R}_{n\to\a}
    \\
    \\
%
%
    &\quad
      n_\a m_\a \L( \partial_t \ve{u}_{\a}
      + \ve{u}_\a \cdot \nabla \ve{u}_\a \R)
      + \L( \partial_t n_\a + \div \L[ n_\a \ve{u}_\a \R] \R) m_\a \ve{u}_\a
    \\
    &=
    - \div \te{\pi}_\a
    - \grad p_\a
    + q_\a n_\a\L(\ve{E}  + \ve{u_\a}\times\ve{B}\R)
    + \ve{R}_{\b\to\a}
    + \ve{R}_{n\to\a}
    \note{Insert continuity equation (\cref{kin:cont})}
    \\
    \\
%
%
    &\quad
      n_\a m_\a \L( \partial_t \ve{u}_{\a}
      + \ve{u}_\a \cdot \nabla \ve{u}_\a \R)
      \\
    &=
    - \div \te{\pi}_\a
    - \grad p_\a
    + q_\a n_\a\L(\ve{E}  + \ve{u_\a}\times\ve{B}\R)
    + \ve{R}_{\b\to\a}
    + \ve{R}_{n\to\a}
    - S_{\a,n}m_\a\ve{u}_\a.
    \numberthis
    \label{kin:mom_wo_full_deriv}
\end{align*}
%
If we now introduce $\d_{t,\a} = \partial_t + \ve{u}_{\a}\cdot\nabla$ and insert this into \cref{kin:mom_wo_full_deriv}, we get
%
\begin{align*}
    n_\a m_\a \d_{t,\a} \ve{u}_{\a}
    &=
    - \div \te{\pi}_\a
    - \grad p_\a
    + q_\a n_\a\L(\ve{E}  + \ve{u_\a}\times\ve{B}\R)
    + \ve{R}_{\b\to\a}
    + \ve{R}_{n\to\a}
    - S_{\a,n}m_\a\ve{u}_\a
    \numberthis
    \label{kin:mom}
\end{align*}

\section{Set of equations}
%
We have found that the two first moments of the Fokker-Planck equations (\cref{kin:cont,kin:mom}) can be written like
%
\begin{empheq}[box=\tcbhighmath]{align}
    \partial_t n_\a + \div\L(n_\a \ve{u}_\a\R) =& S_{n,\a}
    \label{fluideq:cont}
 \\
%
    n_\a m_\a \d_{t,\a} \ve{u}_{\a}
    =&
    - \div \te{\pi}_\a
    - \grad p_\a
    + q_\a n_\a\L(\ve{E}  + \ve{u_\a}\times\ve{B}\R)
    \nonumber
    \\
    &
    + \ve{R}_{\b\to\a}
    + \ve{R}_{n\to\a}
    - S_{\a,n}m_\a\ve{u}_\a
 \label{fluideq:mom}
\end{empheq}
%
where $\d_{t,\a} = \partial_t + \ve{u}_{\a}\cdot\nabla$ and $\ve{R}_{\g \to \a}$ denotes the force acting on species $\a$ by $\g$, i.e. the resistivity.
Both the resistivity and the viscosity tensor depend on the temperature.

We could go on with our derivation and include the second moment, namely the energy equation, which can be used to solve the temperature.
The energy equation will depend on a quantity belonging to the next moment, namely the heat flux $Q$, due to the $\ve{v}\cdot\nabla f_\a$ term in \cref{eq:fp}.
The heat flux equation would again depend on a quantity belonging to the next order, and so on.

In other words, we would need a way to properly "close" the system.
One often used method to close the set of equations, is the one suggested by Braginskii in his paper from $1965$ \cite{Braginskii1965}, which builds on the kinetic closure of gases derived by Chapman and Enskog \cite{Chapman1970book, Brush1972book}.
A brief overview of this closure method is given in \cite{Fitzpatrick2014book}.
The closure procedure uses that the mean free path is much larger than the ion Larmor radius (i.e. the plasma is magnetized), and uses this as an expansion parameter.
This gives an analytical expressions for $\ve{R}_{\b\to\a}$ (see \cref{app:collisions}) and the components of $\te{\pi}_\a$ (see \cref{app:piTensor}), and the set of equations (including the energy equations) are known as the Braginskii equations.

We will in this thesis close the system in a much simpler way.
If we assume that the electrons are isothermal, they can be described by a constant temperature, and hence there is no need for an equation to solve for $T_e$.
We will further assume that the ions are cold ($T_i$ is isothermal with a constant temperature equal $0$).
In this thesis we would like to study systems with the drift fluid approach (see \cref{chap:drift-order} for details), and we will therefore assume that the velocity component perpendicular to the magnetic field is much less than the ion sound speed.
This is in direct contradiction to our assumption of an isothermal system, which requires that the fluid velocity is much larger than the ion sound speed, so that any difference in temperature is quickly smoothed out.
Despite of this, we choose to use the assumption as it gives a much simpler system to solve and analyze.
Improvements to this crude approximation is therefore subject to future work.
