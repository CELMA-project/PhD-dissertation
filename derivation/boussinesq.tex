We will in this chapter modify the obtained equations by using the so called Boussinesq approximation.
This approximation is also sometimes refered to as the local approximation.
%FIXME: Is this true, and if yes, a reference should be there as well
The name Boussinesq approximation is a bit of a misnomer, as this actually refers to a series of approximations in fluid mechanics where the small variation in density is one of them \cite{Kundu2010book}.

The usage of this approximation in drift fluid equation is not always sound, as the criterion for the approximation is easily broken.
However, one argument to use the approximation is that one can make a set of equations which conserves energy without getting into the problem of having the polarization drift advecting itself (which leads to an infinite loop in the derivation).
% FIXME: Does it? Review why this is again

\section{The Boussinesq approximation in a slowly varying B-field}
\label{sec:boussinesq}
%
The goal of this approximation is to let $n$ commute with the gradient in \cref{eq:complVort}.
We start by splitting the density into a background profile $\overline{n}$ and the fluctuation $\delta_n$.
That is, we have
%
\begin{align*}
    n = \overline{n} + \delta_n
\end{align*}
%
% NOTE: When is it OK to work with complex plane waves
%       1. When summing and subtracting linear waves
%       2. When it is part of a plane wave solution
%       3. When the equations are Fourier transformed
%       When is it not OK to work with complex plane waves
%       1. When we are manipulating a non-linear expression (as Fourier modes
%          couple). That is: When the amplitude is not small
%
% NOTE: d/dx exp(ikx) = d/dx (cos(kx) + i sin(kx)) = k(sin(kx) - i cos(kx))
%
% NOTE: I cannot find any good reason to assume plane waves here. Seems like it
%       slightly complicates things a bit with the only gain that we get
%       something like k_perp
and we assume that $\overline{n} \gg \delta_n$.
%
Orders of magnitude estimates now gives
%
\begin{align*}
    &\grad \overline{n} \sim \frac{\overline{n}}{L_n}&
    %
    &\grad \delta_n
    =
    \grad_\perp \delta_n + \grad_\| \delta_n
    \sim \frac{\delta_n}{L_{\delta_n, \perp}} + \frac{\delta_n}{L_{\delta_n, \|}}
    \simeq \frac{\delta_n}{L_{\delta_n, \perp}}
    &
\end{align*}
%
where we have assumed that $L_{\delta_n, \perp} \ll L_{\delta_n, \|}$.

If we now normalize the densities with $n_0$ and the gradients with $\rho_s$, and write the normalized units with a breve ( $ \breve{ } $ ), we find that
%
\begin{align*}
    \breve{\grad} \breve{\overline{n}}
    \sim& \frac{\frac{\overline{n}}{n_0}}{\frac{L_n}{\rho_s}}
    = \frac{\overline{n}}{L_n}\frac{\rho_s}{n_0}
    \simeq \frac{\rho_s}{L_n}
    \note{$\frac{\overline{n}}{n_0} \simeq 1$}
\end{align*}
%
By assuming that $\frac{\rho_s}{L_n}\ll1$%
%
\footnote{Note that for the perpendicular velocities in \cref{app:DO}, we assumed that $\frac{\rho_s}{L_{\ve{u}_\perp}}\sim\sqrt{\e}$} %
%
(meaning that the density gradient length scales are much larger than $\rho_s$), we find that
%
\begin{align*}
    \breve{\grad} \breve{\overline{n}} \ll 1
\end{align*}
%
% FIXME: Consider to add how this relates to experimental findings. Could even compare them to computational! Just looking at plots, see that Ln > rho_s, but maybe not L_n >> rho_s
%
Further on, we have that
%
\begin{align*}
    \breve{\grad} \breve{\delta_n}
    \sim& \frac{\frac{\delta_n}{n_0}}{\frac{L_{\delta_n}}{\rho_s}}
    = \frac{\delta_n}{L_{\delta_n, \perp}}\frac{\rho_s}{n_0}
\end{align*}
%
From our assumption about small perturbations, we know that $\frac{\delta_n}{n_0}\ll 1$.
If we assume that $\frac{\rho_s}{L_{\delta_n, \perp}} \not\gg 1$.
Then
%
\begin{align*}
    \breve{\grad} \breve{\delta_n}
    \ll 1
\end{align*}
%
In other words, we have that
%
\begin{align*}
    \grad(fn) \simeq n\grad f
\end{align*}
%
Using the assumptions, we find that inserting this in \cref{eq:start_of_boussinesq} yields
%
\begin{align*}
 \div\L( n \frac{1}{\om_{ci}}
  \L[ \d_t^E + \ve{u}_{i,\|}\cdot\nabla \R]
  \L[ \frac{\grad_\perp \phi}{B} \R]
 \R)
 =&
 n\div\L(\frac{1}{\om_{ci}}
  \L[ \d_t^E + \ve{u}_{i,\|}\cdot\nabla \R]
  \L[ \frac{\grad_\perp \phi}{B} \R]
 \R)
 \\&+
 \L(\frac{1}{\om_{ci}}
  \L[ \d_t^E + \ve{u}_{i,\|}\cdot\nabla \R]
  \L[ \frac{\grad_\perp \phi}{B} \R]
 \R)
 \cdot\grad n
 \\
 %
 \simeq&
 \L(\overline{n} + \delta_n\R)
 \div\L(\frac{1}{\om_{ci}}
  \L[ \d_t^E + \ve{u}_{i,\|}\cdot\nabla \R]
  \L[ \frac{\grad_\perp \phi}{B} \R]
 \R)
 \note{$\overline{n} \gg \delta_n$}
 \\
  %
 \simeq&
 \overline{n}\div\L(
  \frac{1}{\om_{ci}}
  \L[ \d_t^E + \ve{u}_{i,\|}\cdot\nabla \R]
  \L[ \frac{\grad_\perp \phi}{B} \R]
 \R)
\end{align*}
%
In the last assumptions, we will assume a relatively flat background profiles, so that $n_0 \simeq \overline{n}$.
This means that \cref{eq:start_of_boussinesq} with the Boussinesq approximation yields
%
\begin{align*}
 \div\L( n
  \frac{1}{\om_{ci}}
  \L[ \d_t^E + \ve{u}_{i,\|})\cdot\nabla \R]
  \L[ - \frac{\grad_\perp \phi}{B} \R] \R)
 \simeq&
 n_0\div\L(
  \frac{1}{\om_{ci}}
  \L[ \d_t^E + \ve{u}_{i,\|}\cdot\nabla \R]
  \L[ \frac{\grad_\perp \phi}{B} \R]
 \R)
  \\
  %
  =&
 - n_0 \div\L(
    \frac{1}{\om_{ci}}
     \d_t^E \L[ \frac{\grad_\perp \phi}{B} \R] \R)
 - n_0 \div\L(
    \frac{1}{\om_{ci}}
     \ve{u}_{i,\|}\cdot\nabla
 \L[ \frac{\grad_\perp \phi}{B} \R]
 \R)
 %
 \\
 =&
 - n_0 \div\L(
    \frac{1}{\om_{ci}}
     \d_t^E \L[ \frac{\grad_\perp \phi}{B} \R] \R)
 - n_0 \div\L(
    \frac{1}{\om_{ci}}
     u_{i,\|}\partial_\|
 \L[ \frac{\grad_\perp \phi}{B} \R] \R)
\numberthis
\label{eq:boussinesq_vort}
\end{align*}
%
Let us now briefly comment on the approximations made
%
\begin{enumerate}[noitemsep]
    \item $\overline{n} \ll \delta_n$ : This approximation can be broken as the
        amplitudes can be considerably large.
        % FIXME: Citation needed
    \item $\frac{\rho_s}{L_n}\ll1$ : This approximation usually holds as gradients in the background density are sharp as compared to $\rho_s$, however it may be that it is not orders of magnitude larger.
        % FIXME: Citation needed
    \item $\overline{n} \simeq n_0$ : This assumption is maybe the hardest to fulfill, as the gradients usually are non-vanishing.
        % FIXME: Citation needed
\end{enumerate}
%
We see that although the assumptions can be met, it is not necessarily true.
As this approximation is done quite frequently in the literature
% FIXME: Citation needed
it is of interest to compare results with and without this approximation.

In the scope of this thesis, we only do the approximation here, and keeps gradients of $n$ elsewhere.
A more rigorous study could be to investigate the set of equations from a energy conservation point of view, and from that find out what $n$'s in the set of equations which should be transformed to $n_0$.

\section{The Boussinesq approximation in the CELMA model}
%
We will now see how the set of equations in the CELMA model is affected by the Boussinesq approximation.
Continuing from \cref{eq:boussinesq_vort} (using the assumption that $\grad B = 0$) yields
%
\begin{align*}
    &- n_0 \div\L(
       \frac{1}{\om_{ci}}
        \d_t^E \L[ \frac{\grad_\perp \phi}{B} \R] \R)
    - n_0 \div\L(
       \frac{1}{\om_{ci}}
        u_{i,\|}\partial_\|
    \L[ \frac{\grad_\perp \phi}{B} \R] \R)
    \\
    =&
    - \frac{n_0}{\om_{ci}}
    \L( \div\L[ \d_t^E \L( \frac{\grad_\perp \phi}{B} \R) \R]
     + \div\L[ u_{i,\|}\partial_\| \L( \frac{\grad_\perp \phi}{B} \R) \R] \R)
     \\
     %
     =&
     - \frac{n_0}{\om_{ci}}
     \L( \div\L[ \L(\partial_t
     + \ve{u}_E\cdot\grad\R) \L( \frac{\grad_\perp \phi}{B} \R) \R]
     + \div\L[ u_{i,\|}\partial_\| \L( \frac{\grad_\perp \phi}{B} \R) \R] \R)
     \note{Assume interchangibility of spatial and temporal derivatives}
     \\
     %
     =&
     - \frac{n_0}{\om_{ci}}
     \L( \div \partial_t\L( \frac{\grad_\perp \phi}{B} \R)
     + \div \L(\ve{u}_E\cdot\grad \frac{\grad_\perp \phi}{B} \R)
     + \div\L[ u_{i,\|}\partial_\| \L( \frac{\grad_\perp \phi}{B} \R) \R] \R)
     \\
     %
     =&
     - \frac{n_0}{\om_{ci}}\partial_t\Om
     - \frac{n_0}{\om_{ci}}
     \L(
     \div \L[\ve{u}_E\cdot\grad \frac{\grad_\perp \phi}{B} \R]
     + \div\L[ u_{i,\|}\partial_\| \L( \frac{\grad_\perp \phi}{B} \R) \R] \R)
     \note{$\partial_\|=0$}
     \\
     %
     =&
     - \frac{n_0}{\om_{ci}}\partial_t\Om
     - \frac{n_0}{\om_{ci}}
     \L(
     \div \L[\ve{u}_E\cdot\grad \frac{\grad_\perp \phi}{B} \R]
     + u_{i,\|}\partial_\|\Om
     + \frac{\grad_\perp\L( \partial_\|\phi\R)}{B}\cdot\grad u_{i,\|}
     \R)
\end{align*}
%
Following the derivation in \cref{app:vortDAdv} (but this time without $n$) we see that in cylinder geometry
%
\begin{align*}
    \div \L[\ve{u}_E\cdot\grad \frac{\grad_\perp \phi}{B} \R]
    = \frac{1}{B\rho}\{\phi, \Om\}
\end{align*}
%
so that
%
\begin{align*}
 \div\L( n
  \frac{1}{\om_{ci}}
  \L[ \d_t^E + \ve{u}_{i,\|})\cdot\nabla \R]
  \L[ - \frac{\grad_\perp \phi}{B} \R] \R)
  \simeq&
  - \frac{n_0}{\om_{ci}}\partial_t\Om
  - \frac{n_0}{\om_{ci}}
  \L(
  \frac{1}{B\rho}\{\phi, \Om\}
  + u_{i,\|}\partial_\|\Om
  + \frac{\grad_\perp\L( \partial_\|\phi\R)}{B}\cdot\grad u_{i,\|}
  \R)
\end{align*}
%
If we now collect the terms as we did in \cref{sec:CELMACollect}, we get
%
\begin{align*}
  %
  &
  \quad
 \frac{\nu_{in}}{\om_{ci}} \L(n\Om + \frac{\grad_\perp \phi}{B} \cdot \grad n\R)
  \\
 &
 + \frac{n_0}{\om_{ci}}\partial_t\Om
 + \frac{n_0}{\om_{ci}}
 \L(
 \frac{1}{B\rho}\{\phi, \Om\}
 + u_{i,\|}\partial_\|\Om
 + \frac{\grad_\perp\L( \partial_\|\phi\R)}{B}\cdot\grad u_{i,\|}
 \R)
 \\
 %
 &
 + \frac{1}{\om_{ci}}
    \div \L( S_n \L[ \frac{ \grad_\perp \phi }{ B } \R] \R)
 \\
 %
 =&
 \partial_\| j_\|
\end{align*}
%
which after rearranging yields
%
\begin{align*}
  %
 \frac{n_0}{\om_{ci}}\partial_t\Om
 %
 =&
 - \frac{\nu_{in}}{\om_{ci}} \L(n\Om + \frac{\grad_\perp \phi}{B} \cdot \grad n\R)
  \\
  %
 &- \frac{n_0}{\om_{ci}}
 \L( \frac{1}{B\rho}\{\phi, \Om\}
    + u_{i,\|}\partial_\|\Om
    + \frac{\grad_\perp\L( \partial_\|\phi\R)}{B}\cdot\grad u_{i,\|}
 \R)
 \\
 %
 &
 - \frac{1}{\om_{ci}}
    \div \L( S_n \L[ \frac{ \grad_\perp \phi }{ B } \R] \R)
 \\
 %
 &
 + \partial_\| j_\|
\end{align*}
%
We note that as we have no time derivative of $n$, the term arising from the density source does not cancel as it did without this approximation.
%

\section{Normalization}
Finally, normalization yields
%
\begin{align*}
 \frac{n_0}{\om_{ci}}\omega_{ci}^2\partial_t\Om
 %
 =&
 - \frac{\om_{ci}\nu_{in}}{\om_{ci}}
 \L(n_0\om_{ci}n\Om
 + \frac{c_s}{\rho_s}n_0\frac{\grad_\perp \phi}{B} \cdot \grad n\R)
  \\
  %
 &- \frac{n_0}{\om_{ci}}
 \L(\frac{1}{\rho_s} c_s \om_{ci}\frac{1}{B\rho}\{\phi, \Om\}
    + c_s \frac{1}{\rho_s}\omega_{ci}u_{i,\|}\partial_\|\Om
    + \frac{1}{\rho_s}c_s\frac{1}{\rho_s}c_s
    \frac{\grad_\perp\L( \partial_\|\phi\R)}{B}\cdot\grad u_{i,\|}
 \R)
 \\
 %
 &
 - \frac{1}{\om_{ci}}\frac{1}{\rho_s}n_0\omega_{ci}c_s
    \div \L( S_n \L[ \frac{ \grad_\perp \phi }{ B } \R] \R)
 \\
 %
 &
 + \frac{1}{\rho_s}n_0c_s\partial_\| j_\|
 \\
 %
 %
 %
 n_0\omega_{ci}\partial_t\Om
 %
 =&
 - \nu_{in}
 \L(n_0\om_{ci}n\Om
 + \om_{ci}n_0\frac{\grad_\perp \phi}{B} \cdot \grad n\R)
  \\
  %
 &- \frac{n_0}{\om_{ci}}
 \L(\om_{ci}^2\frac{1}{B\rho}\{\phi, \Om\}
    + \omega_{ci}^2u_{i,\|}\partial_\|\Om
    + \omega_{ci}^2
    \frac{\grad_\perp\L( \partial_\|\phi\R)}{B}\cdot\grad u_{i,\|}
 \R)
 \\
 %
 &
 - n_0\omega_{ci}
    \div \L( S_n \L[ \frac{ \grad_\perp \phi }{ B } \R] \R)
 \\
 %
 &
 + \omega_{ci}n_0\partial_\| j_\|
 \\
 %
 %
 %
 \partial_t\Om
 %
 =&
 - \nu_{in}
 \L(n\Om
 + \frac{\grad_\perp \phi}{B} \cdot \grad n\R)
  \\
  %
 &-
 \L(\frac{1}{B\rho}\{\phi, \Om\}
    + u_{i,\|}\partial_\|\Om
    +\frac{\grad_\perp\L( \partial_\|\phi\R)}{B}\cdot\grad u_{i,\|}
 \R)
 \\
 %
 &
 - \div \L( S_n \L[ \frac{ \grad_\perp \phi }{ B } \R] \R)
 \\
 %
 &
 + \partial_\| j_\|
\end{align*}

\section{The vorticity equation}
%
We now solve for the vorticity $\Om$ rather than the modified vorticity $\Om^D$, and get
%
\begin{empheq}[box={\tcbhighmath}]{align*}
 \partial_t\Om
 %
 =&
 - \nu_{in}
 \L(n\Om
 + \frac{\grad_\perp \phi}{B} \cdot \grad n\R)
  \\
  %
 &-
 \L(\frac{1}{B\rho}\{\phi, \Om\}
    + u_{i,\|}\partial_\|\Om
    +\frac{\grad_\perp\L( \partial_\|\phi\R)}{B}\cdot\grad u_{i,\|}
 \R)
 \\
 %
 &
 - \div \L( S_n \L[ \frac{ \grad_\perp \phi }{ B } \R] \R)
 \\
 %
 &
 + \partial_\| j_\|
  \\
  %
  &
  + D_{\Om, \|, \text{art}}    \partial_{\|}^2 \Om
  + D_{\Om, \perp, \text{art}} \grad_\perp^2 \Om
  \numberthis
  \label{eq:celma_vort_boussinesq}
\end{empheq}
%
To see how good or bad this approximation is (at least in our system), we will compare it with results not using this approximation in \cref{part:results}.
