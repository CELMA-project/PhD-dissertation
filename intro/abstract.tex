\begin{abstract}
    % Counter needs to be set within this environment
    \setcounter{page}{3}
    %
% FIXME: Check if somethings needs to be added or removed in the abstract
    Understanding the turbulent transport in the edge of plasma in fusion devices is of uttermost importance in order to make precise prediction of future fusion devices.
    Plasma turbulence observed in linear devices shares many important features with the turbulence observed in the edge of fusion devices, and are easier to diagnose due to lower temperatures and better access to the plasma.
    In order to gain greater insight in this complex turbulent behavior, simulation of plasma in a linear machine is done in this thesis.

    Here, a three dimensional drift-fluid model is derived from first principles for a plasma in a linear device.
    To account for the fluctuations at the same level as the background plasma, the traditional split between background and fluctuations has not been made.
    The model is implemented using the BOUT++ framework and is solved numerically.
    Special attention is given to the treatment the singularity at the cylinder axis, and at the inversion of the non-linear elliptic equation which is done to obtain the potential.
    The evolution of the plasma through the steady-state, linear phase and turbulent phase is investigated and compared between different $B$-field strengths.
    It is found that the drift-waves are responsible for the onset of turbulence, and that the turbulent radial flux is causing a flattening of the density profiles.
    Coherent structures from the intermittent radial flux in the turbulent state are investigated.

    Results of simulations using the Boussinesq approximation is compared to full simulations.
    It is found that the Boussinesq approximation leads to an unphysical increase of the potential as ions and electrons are lost at a different rate.
    \\

    \noindent
    \textbf{Keywords}: Cylindrical plasma, Drift-fluid equations, Numerical modeling, Drift-waves, Plasma turbulence, Sheath boundary condition
\end{abstract}
