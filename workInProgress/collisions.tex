% NOTE: The integrals here can be checked with symbolab.com
% NOTE: Derivation does not directly compare wit
%       \nu_{\a\b} = n_\b\expt{\sigma_{\a\b} v}_\a
%       as extra v^2 by considering drifting Maxwellians. Without this the
%       nu_ei frequency becomes infinite
FIXME: Normal ei collision not mentioned in main text, do so
FIXME: Add definition of vth

We will here follow the derivation of the electron-ion and the ion-ion
collision frequency \cite{Goldstone1995} in order to derive an estimate for the
elastic electron-neutral and ion-neutral collision frequency.

We do so by calculating the frictional force experienced by species $\a$ as it
is drifting thorugh species $\b$ which are stationary. We have
%
\begin{align*}
    \ve{F}_\a = - n_\a m_\a \expt{n_\b\sigma_{\a\b} v\ve{v}}_\a
\end{align*}
%
where $\expt{\cdot}_\a$ denotes the average over the drifting distribution
function of species $\a$, and $\sigma_{\a\b}$ is the cross section of the
process.
If we let the particles stream towards the stationary target along $z$, so that
the fuid velocity $\ve{u}_\a = u_z \ve{e}_z$, we get
%
\begin{align*}
    f_\a
    =&
    \frac{n_\a}{(2\pi)^{3/2}v_{th,\a}^3}
    \exp\L(-\frac{\L[\ve{v}-\ve{u}\R]^2}{2v_{th,\a}^2}\R)
    \\
    %
    %
    =&
    \frac{n_\a}{(2\pi)^{3/2}v_{th,\a}^3}
    \exp\L(-\frac{\ve{v}^2-2\ve{v}\ve{u}+\ve{u}^2}{2v_{th,\a}^2}\R)
    \\
    %
    %
    =&
    \frac{n_\a}{(2\pi)^{3/2}v_{th,\a}^3}
    \exp\L(-\frac{-2\ve{v}\cdot\ve{u}+\ve{u}^2}{2v_{th,\a}^2}\R)\exp\L(-\frac{\ve{v}^2}{2v_{th,\a}^2}\R)
    \note{Assume $\ve{u} \ll v_{th,\a}^2$}
    \\
    %
    %
    \simeq&
    \frac{n_\a}{(2\pi)^{3/2}v_{th,\a}^3}
    \L(\frac{2\ve{v}\cdot\ve{u}+1}{2v_{th,\a}^2}\R)\exp\L(-\frac{\ve{v}^2}{2v_{th,\a}^2}\R)
    \\
    =&
    \frac{2v_zu_z+1}{2v_{th,\a}^2}f_{\a,0}
\end{align*}
%
where $\ve{v}$ denotes the particle velocity, $f_{\a,0}$ denotes the
unshifted Maxwellian and
%
\begin{align*}
    v_{th,\a} \defined \sqrt{\frac{T_\a}{m_\a}} .
\end{align*}
%
Thus, the friction force in the direction of the drifting
is
%
\begin{align*}
    F_{\a,z} =& - n_\a m_\a \expt{n_\b\sigma_{\a\b} v v_z}_\a
    \\
    %
    %
    \simeq&
    - n_\a m_\a
    \frac{n_\b}{n_\a}\iiint\frac{2v_zu_z+1}{2v_{th,\a}^2}f_{\a,0}\sigma_{\a\b} v v_z\d^3v
    \\
    %
    %
    =&
    - n_\a m_\a
    \frac{n_\b}{n_\a}
    \L(
    u_z\iiint\frac{2v_z}{2v_{th,\a}^2}f_{\a,0}\sigma_{\a\b} v v_z\d^3v
    +\iiint\frac{1}{2v_{th,\a}^2}f_{\a,0}\sigma_{\a\b} v v_z\d^3v
    \R)
    \note{Integral over $v_z$ is $1/3$ of integral over $v$ due to sperical
        symmetry}
    \\
    %
    %
    =&
    - n_\a m_\a
    \frac{n_\b}{n_\a}
    \frac{1}{3}
    \L(
    u_z\iiint\frac{2}{2v_{th,\a}^2}f_{\a,0}\sigma_{\a\b} v^3\d^3v
    +\iiint\frac{1}{2v_{th,\a}^2}f_{\a,0}\sigma_{\a\b} v^2\d^3v
    \R)
    \note{FIXME: NO EVEN FUNC ARG}
    \\
    %
    %
    =&
    - n_\a m_\a
    u_z
    \frac{n_\b}{n_\a}
    \frac{1}{3}
    \frac{1}{v_{th,\a}^2}
    \iiint f_{\a,0}\sigma_{\a\b} v^3\d^3v
    \note{Spherical coordinates}
    \\
    %
    %
    =&
    - n_\a m_\a
    u_z
    \frac{n_\b}{n_\a}
    \frac{1}{3}
    \frac{1}{v_{th,\a}^2}
    \int_0^\infty\int_0^{2\pi}\int_0^\pi
    f_{\a,0}\sigma_{\a\b} v^5
    \sin\theta \d\theta \d \phi\d v
    \\
    %
    %
    =&
    - n_\a m_\a
    u_z
    \frac{n_\b}{n_\a}
    \frac{1}{3}
    \frac{1}{v_{th,\a}^2}
    4\pi
    \int_0^\infty
    f_{\a,0}\sigma_{\a\b} v^5
    \d v
     \\
    %
    %
    =&
    - n_\a m_\a
    u_z
    \nu_{\a\b, \text{stationary target}}
\end{align*}
%
where we here have defined the averaged collision frequency
%
\begin{align*}
    \nu_{\a\b, \text{stationary target}}
    \defined&
    \frac{n_\b}{n_\a}
    \frac{1}{3}
    \frac{1}{v_{th,\a}^2}
    4\pi
    \int_0^\infty
    f_{\a,0}\sigma_{\a\b} v^5
    \d v
    \\
    %
    %
    =&
    \frac{n_\b}{n_\a}
    \frac{4\pi}{3v_{th,\a}^2}
    \int_0^\infty
    \frac{n_\a}{(2\pi)^{3/2}v_{th,\a}^3}
    \exp\L(-\frac{\ve{v}^2}{2v_{th,\a}^2}\R)
    \sigma_{\a\b} v^5
    \d v
    \\
    %
    %
    =&
    \frac{n_\b4\pi}{3(2\pi)^{3/2}v_{th,\a}^5}
    \int_0^\infty
    \exp\L(-\frac{\ve{v}^2}{2v_{th,\a}^2}\R)
    \sigma_{\a\b} v^5
    \d v
\end{align*}
%
The subscript $_\text{stationary target}$ will be dropped from here on.

\section{Electron collisions}
Using the cross section for electron ion collisions
%
\begin{align*}
    \sigma_{ei} = \frac{e^4\ln\Lambda}{4\pi\e_0^2m_e^2v^4}
\end{align*}
%
yields
%
\begin{align*}
    \nu_{ei}
    =&
    \frac{n_i4\pi}{3(2\pi)^{3/2}v_{th,e}^5}
    \int_0^\infty
    \exp\L(-\frac{\ve{v}^2}{2v_{th,e}^2}\R)
    \frac{e^4\ln\Lambda}{4\pi\e_0^2m_e^2v^4} v^5
    \d v
    \\
    %
    %
    =&
    \frac{n_i}{2^{1/2}6\pi^{3/2}v_{th,e}^5}
    \frac{e^4\ln\Lambda}{\e_0^2m_e^2}
    \int_0^\infty
    \exp\L(-\frac{\ve{v}^2}{2v_{th,e}^2}\R)
    v \d v
    \\
    %
    %
    =&
    \frac{2}{2}
    \frac{n_i}{2^{1/2}6\pi^{3/2}v_{th,e}^5}
    \frac{e^4\ln\Lambda}{\e_0^2m_e^2}
    v_{th,e}^2
    \\
    %
    %
    =&
    \frac{2^{1/2}n_ie^4\ln\Lambda}{12\pi^{3/2}e_0^2m_e^2\L(\sqrt{\frac{T_e}{m_e}}\R)^3}
    \\
    %
    %
    =&
    \frac{2^{1/2}n_ie^4\ln\Lambda}{12\pi^{3/2}e_0^2m_e^{1/2}T_e^{3/2}}
\end{align*}
%
Using the cross section for electron neutral collision
%
\begin{align*}
    \sigma_{en} = \pi a_0^2
\end{align*}
%
yields
%
\begin{align*}
    \nu_{en}
    =&
    \frac{n_n4\pi}{3(2\pi)^{3/2}v_{th,e}^5}
    \int_0^\infty
    \exp\L(-\frac{\ve{v}^2}{2v_{th,e}^2}\R)
    \pi a_0^2 v^5
    \d v
    \\
    %
    %
    =&
    \frac{n_n4\pi^2 a_0^2 }{3(2\pi)^{3/2}v_{th,e}^5}
    \int_0^\infty
    \exp\L(-\frac{\ve{v}^2}{2v_{th,e}^2}\R)
    v^5
    \d v
    \\
    %
    %
    =&
    \frac{n_n4\pi^2 a_0^2 }{3(2\pi)^{3/2}v_{th,e}^5} 8v_{th,e}^6
    \\
    %
    %
    =&
    \frac{2}{2}
    \frac{32 n_n\pi^2 a_0^2 v_{th,e}}{3(2\pi)^{3/2}}
    \\
    %
    %
    =&
    \frac{2^{1/2}8 n_n\pi^{1/2} a_0^2 v_{th,e}}{3}
    \\
    %
    %
    =&
    \frac{2^{1/2}8 n_n\pi^{1/2} a_0^2 \sqrt{T_e}}{3\sqrt{m_e}}
\end{align*}
%
% NOTE: Corresponds rougly to what is reported in NRL plasma formulary in
%       weakly ionized plasmas


\section{Ion collisions}
The analysis done above was valid when the target was stationary with respect
to the colliding particles. This is a fairly good approximation when the
stationary particles are much heavier than the colliding particles. Hence, the
derivation of the average ion-ion collision frequency or the average
ion-neutral collision frequency is strictly not valid. However, according to
\cite{Goldstone1995}, the analysis yields the correct result within factors of
orders of unity.

One could do the analysis by going to the center of mass frame (which in the
end gives an additional factor $2^{-1/2}$) and use relative velocities one find
velocities
%
\begin{align*}
    \nu_{ii}
    %
    =&
    \frac{n_ie^4\ln\Lambda}{12\pi^{3/2}e_0^2m_i^{1/2}T_i^{3/2}}
\end{align*}
%
As the mass of the neutral atom is approximately the same as the ion mass (as
we are considering neutrals of the same species as the plasma), we get for the
ion-neutral collision
%
\begin{align*}
    \nu_{in}
    =&
    \frac{8 n_n\pi^{1/2} a_0^2 \sqrt{T_i}}{3\sqrt{m_i}}
\end{align*}
%







We can estimate the neutral density from the neutral gas pressure if we assume
%
\begin{align*}
    p_n = n_nT_{n}[\text{J}].
\end{align*}
%
If we further assume that the neutrals are at room temperature, we have that
%
\begin{align*}
    T_{n}[\text{K}] =& 293 \text{K}\\
    T_{n}[\text{J}] =& T_{n}[\text{K}]k_b = 293\text{K}\cdot 1.38064852\cdot 10^{-23}
    \text{J}\text{K}^{-1}\\
    T_{n}[\text{J}] =& 4.0453001636\cdot 10^{-21}\text{J}
\end{align*}
%
which means that
%
\begin{align*}
    n_n = \frac{p_n}{4.0453001636\cdot 10^{-21}\text{J}}
\end{align*}
%
FIXME: Schroeder scans from 0.16 Pa to 0.66Pa
with this calculation, she would have more neutrals than plasma particles...
