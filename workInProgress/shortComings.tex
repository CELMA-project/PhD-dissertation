In this chapter we aim to be as open as possible about the shortcomings of this thesis.
In this way, the reader will be aware that the alteration of these points may affect the obtained results.
The chapter is divided in two parts: Major and minor, where the major addresses issues which are believed to have svere impacts on the results, and the minor issues are believed to have a lesser impact.

\section{Major}
\begin{itemize}[noitemsep,nolistsep]
    \item The plasma is assumed to be isothermal.
        This assumption would be good if the heat flux were enough to equilibrate the temperature everywhere in the plasma for the time under consideration.
        However, the heat fluxes are not big enough, and temperature gradients has been found experimentally \cite{Schroder2003phd}.
    \item Not all the boundary conditions are not physically justified.
        All the boundary conditions at the opposite direction of the sheath (given that this location is in fact a stagnation point), the sheath boundary condition for the velocities $u_{e,\|}$ and $u_{i,\|}$ are somewhat valid given that the plasma in the parallel direction are almost in the steady state.
        The numerical boundary condition for $\phi$ at the sheath is valid given that the parallel boundary condition for $\Omega^D$ is valid.
        There are however, no reason for why the gradient of $\Omega^D$ should be $0$ at the sheath, and it might as well have a fixed gradient.
        Similarity, there is no reason for why the gradient of $n$ should be zero at the sheath.
        Perpendicularily we are forcing the system to have zero gradients on all fields.
        As stated before proper treatment of the boundary conditions should include the physical behavior of the surrounding chassis.
\end{itemize}

%FIXME: Move me to BC section
% Parallel direction: sheath almost steady state

% FIXME: Actually OK that the collisionality is assumed to be constant as
% d/dx nu ei is (under the assumption that Te is constant), 0.5 log(x), which for
% high number is almost constant. Just write it in

The collisionality is assumed to be constant.
The collisionality will vary across the domain.
The collisionality is proportional to $n\ln(n^{-1/2})=-\frac{1}{2}n\ln(n)$

\section{Minor}

\begin{itemize}[noitemsep,nolistsep]
    \item Treatment of the parallel resistivities.
        A proper treatment has resulted in oscillations at the edge of the domain which evolves to turbulence.
        The observed turbulence also has a different character as there are larger gradients, and more visible structures in the parallel direction.
    \item Adding electromagnetic effects.
        Although this has been reported to decrease the run time and increase the stability of the simulations.
        It is here found that the extra terms blows up in $A_\|$.
\end{itemize}
