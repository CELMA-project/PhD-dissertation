\begin{align*}
    n_\a m_\a \d_{t,\a} \ve{u}_{\a}
    =&
    - \div \te{\pi}_\a
    - \grad p_\a
    + q_\a n_\a\L(\ve{E}  + \ve{u_\a}\times\ve{B}\R)
    \nonumber
    \\
    &
    + \ve{R}_{\b\to\a}
    + \ve{R}_{n\to\a}
    - S_{\a,n}m_\a\ve{u}_\a
\end{align*}
%
Rearranging equation (\ref{fluideq:mom}), and multiplying it with
$\frac{q_\a}{m_\a}$ yields
%
\begin{align*}
    q_{\a} \partial_t (n_{\a} \ve{u}_{\a})
    =&
    - q_{\a} \ve{u}_{\a} \partial_t n_{\a}
    - \frac{q_\a}{m_\a}\ve{u}_{\a}\cdot\nabla\ve{u}_{\a}
    - \frac{q_\a}{m_\a}\div \te{\pi}_{\a}
    - \frac{q_\a}{m_\a}\grad p_{\a}
    + \frac{q_\a^2}{m_\a} n_\a\L(\ve{E}  + \ve{u_\a}\times\ve{B}\R)
    \\ &
    + \frac{q_\a}{m_\a}\ve{R}_{\b\to \a}
    + \frac{q_\a}{m_\a}\ve{R}_{n\to \a}
    - q_\a S_{\a,n}\ve{u}_{\a}
\end{align*}
%
Adding the equation for electrons and ions using quasi-neutrality yields
%
\begin{align*}
    \partial_t \ve{J}
    =&
     \frac{\ve{J}}{n} \partial_t n
     \\&
    + \frac{e}{m_e}\ve{u}_{e}\cdot\nabla\ve{u}_{e}
    - \frac{e}{m_i}\ve{u}_{i}\cdot\nabla\ve{u}_{i}
     \\&
    + \frac{e}{m_e}\div \te{\pi}_{e}
    - \frac{e}{m_i}\div \te{\pi}_{i}
     \\&
    + \frac{e}{m_e}\grad p_{e}
    - \frac{e}{m_i}\grad p_{i}
     \\&
     + \frac{e^2}{m_e} n\L(-\grad{\phi}-\partial_t \ve{A}+ \ve{u_e}\times\ve{B}\R)
     + \frac{e^2}{m_i} n\L(-\grad{\phi}-\partial_t \ve{A}+ \ve{u_i}\times\ve{B}\R)
    \\ &
    - \frac{e}{m_e}\ve{R}_{i\to e}
    + \frac{e}{m_i}\ve{R}_{e\to i}
     \\&
    - \frac{e}{m_e}\ve{R}_{n\to e}
    + \frac{e}{m_i}\ve{R}_{n\to i}
     \\&
     + S \frac{\ve{J}}{n}
     \numberthis
     \label{eq:current_eq}
\end{align*}
%
where $\curl\ve{A}=\ve{B}$. We then have that
%
\begin{align*}
    \curl\curl\ve{A}=&\curl\ve{B}
    \note{Low frequency}
    \\
    \grad^2\ve{A} - \grad(\div \ve{A})=&\mu_0\ve{J}
    \note{Coloumb gauge}
    % NOTE: See Griffiths
    \\
    \frac{\grad^2\ve{A}}{\mu_0}=&\ve{J}
    \\
    \ve{J} =& \frac{\div(\grad_\perp\ve{A}+\grad_\|\ve{A})}{\mu_0}
    \note{Assume $k_\perp \ll k_\|$}
    \\
    \ve{J} \simeq& \frac{\grad^2_\perp\ve{A}}{\mu_0}
\end{align*}
%
Thus the LHS can be written
%
\begin{align*}
    \partial_t \ve{J} \simeq \partial_t \frac{\grad^2_\perp\ve{A}}{\mu_0}
\end{align*}
%
We can now use order of magnitude arguments in order to get a feeling which of
the terms
$\simeq \partial_t \frac{\grad^2_\perp\ve{A}}{\mu_0}$ and
$\frac{e^2}{m_e}\partial_t \ve{A}$ which are dominating. If the latter
dominates the former, electromagnetic effects are important. That is if
%
\begin{align*}
    \partial_t \frac{\grad^2_\perp\ve{A}}{\mu_0}
    < &
    \frac{ne^2}{m_e}\partial_t \ve{A}
    \\
    \rightarrow &
    \\
    \frac{1}{\omega_{ci}} \frac{k_\perp^2\ve{A}}{\mu_0}
    < &
    \frac{ne^2}{m_e}\frac{1}{\omega_{ci}} \ve{A}
    \\
    \frac{k_\perp^2}{\mu_0}
    < &
    \frac{ne^2}{m_e}
    \\
    k_\perp^2
    < &
    \frac{\mu_0ne^2}{m_e}
    \\
    k_\perp^2
    < &
    \frac{2m_i}{2m_i}\frac{B^2}{T_e}\frac{T_e}{B^2}\frac{\mu_0ne^2}{m_e}
    \\
    k_\perp^2
    < &
    \frac{m_i}{2m_e}\frac{e^2B^2}{m_iT_e}\frac{2\mu_0nT_e}{B^2}
    \\
    k_\perp^2
    < &
    \frac{m_i}{2m_e}\frac{1}{\rho_s^2}\b
\end{align*}
%
where the plasma beta ($\b$) is the kinetic pressure over the magnetic
pressure. As
$ k_\perp^2\rho_s^2$ is typically in the order of unity for drift wave
turbulence, we find that electromagnetic effects becomes important whenever
%
\begin{align*}
    1
    < &
    \frac{m_i}{2m_e}\b
    \\
    \frac{2m_e}{m_i}
    < &
    \b
\end{align*}
%
We observe that we get the identical condition if we choose to compare with
$\frac{\ve{J}}{n}\partial_t n$ instead.
%FIXME; Consider to compare with other terms as well
