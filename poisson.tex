We will here derive the bracket operators used for perpendicular advection.
Under electrostatic conditions, we have that $\ve{u}_E =
-\frac{\nabla\phi\times\ve{b}}{B}$, which is similar to
$\ve{u}=\ve{k}\times\nabla\psi$ found in incompressible fluid flow
%
\begin{align*}
    \ve{u}_E =& -\frac{\nabla\phi\times\ve{b}}{B}\\
         %
         =&-\frac{\nabla\phi\times\ve{e}_2}{
               \sqrt{g_{22}}J^{-1}\sqrt{g_{22}}}\\
             %
         =&-\frac{J}{g_{22}}\nabla\phi\times\ve{e}_2\\
         %
         =&\frac{J}{g_{22}}\ve{e}_2\times\nabla\phi\\
         %
         =&\frac{J}{g_{22}}\ve{e}_2\times
           \L(\ve{e}^1\partial_1 + \ve{e}^3\partial_3\R)\phi\\
         %
         =&\frac{J}{g_{22}}
           \L(g_{21}\ve{e}^1 + g_{22}\ve{e}^2 + g_{23}\ve{e}^3\R)
           \times
           \L(\ve{e}^1\partial_1 +
              \ve{e}^2\partial_2 +
                  \ve{e}^3\partial_3\R)\phi\\
             %
         =&\frac{J}{g_{22}}
           \L(
             g_{21}\ve{e}^1\times\ve{e}^1\partial_1
           + g_{22}\ve{e}^2\times\ve{e}^1\partial_1
           + g_{23}\ve{e}^3\times\ve{e}^1\partial_1
           \R.
           \\
           &\quad\;
           + g_{21}\ve{e}^1\times\ve{e}^2\partial_2
           + g_{22}\ve{e}^2\times\ve{e}^2\partial_2
           + g_{23}\ve{e}^3\times\ve{e}^2\partial_2
           \\
           &\quad\;
           \L.
           + g_{21}\ve{e}^1\times\ve{e}^3\partial_3
           + g_{22}\ve{e}^2\times\ve{e}^3\partial_3
           + g_{23}\ve{e}^3\times\ve{e}^3\partial_3
           \R)
           \phi\\
         %
         =&\frac{J}{g_{22}}
           \L(
           - g_{22}\ve{e}^2\times\ve{e}^1\partial_1
           + g_{23}\ve{e}^3\times\ve{e}^1\partial_1
           \R.
           \\
           &\quad
           + g_{21}\ve{e}^1\times\ve{e}^2\partial_2
           - g_{23}\ve{e}^3\times\ve{e}^2\partial_2
           \\
           &\quad
           \L.
           - g_{21}\ve{e}^1\times\ve{e}^3\partial_3
           + g_{22}\ve{e}^2\times\ve{e}^3\partial_3
           \R)
           \phi\\
         %
         =&\frac{1}{g_{22}}
           \L(
           - g_{22}\ve{e}_3\partial_1
           + g_{23}\ve{e}_2\partial_1
           + g_{21}\ve{e}_3\partial_2
           - g_{23}\ve{e}_1\partial_2
           - g_{21}\ve{e}_2\partial_3
           + g_{22}\ve{e}_1\partial_3
           \R)
           \phi
           \numberthis
           \label{app:ExB}
\end{align*}
%
The electrostatic $E\times B$ advection operator thus becomes
%
\begin{align*}
    \ve{u}_E\cdot\nabla
    =& -\frac{\nabla\phi\times\ve{b}}{B}\cdot\nabla\\
    %
    %
    =&\frac{1}{g_{22}}
           \L(
           - g_{22}\ve{e}_3\partial_1
           + g_{23}\ve{e}_2\partial_1
           + g_{21}\ve{e}_3\partial_2
           - g_{23}\ve{e}_1\partial_2
           - g_{21}\ve{e}_2\partial_3
           + g_{22}\ve{e}_1\partial_3
           \R)
           \phi
       \cdot\L(\ve{e}^1\partial_1 + \ve{e}^2\partial_2 + \ve{e}^3\partial_3\R)\\
    %
    %
    =& \frac{1}{g_{22}}
           \L(
           - g_{22}\partial_1\phi\partial_3
           + g_{23}\partial_1\phi\partial_2
           + g_{21}\partial_2\phi\partial_3
           - g_{23}\partial_2\phi\partial_1
           - g_{21}\partial_3\phi\partial_2
           + g_{22}\partial_3\phi\partial_1
           \R)\\
    %
    %
    =& \frac{1}{g_{22}}
           \L(
             \L[
               g_{22}\partial_3\phi
             - g_{23}\partial_2\phi
             \R]\partial_1
           +
             \L[
               g_{23}\partial_1\phi
             - g_{21}\partial_3\phi
             \R]\partial_2
           +
             \L[
               g_{21}\partial_2\phi
             - g_{22}\partial_1\phi
             \R]\partial_3
           \R)\\
    %
    %
    =& \frac{1}{g_{22}}
               \L(
                 g_{21}\{\phi, \cdot\}_{2,3}
                 +
                 g_{22}\{\phi, \cdot\}_{3,1}
                 +
                 g_{23}\{\phi, \cdot\}_{1,2}
               \R)
\end{align*}
%
Where we have used the definition of the Poisson bracket
%
\begin{align*}
    \{a, b\}_{i,j} =
      \L(\partial_i a\R) \partial_j b
    - \L(\partial_j a\R) \partial_i b
\end{align*}
%
In a cylindrical geometry, all the off diagonal elements are zero, so
%
\begin{align}
    \ve{u}_E\cdot\nabla
    = \frac{1}{g_{22}} \L( g_{22}\{\phi, \cdot\}_{3,1} \R)
    = \{\phi, \cdot\}_{3,1}
    = \partial_3\phi\partial_1 - \partial_1\phi\partial_3
    = \partial_\theta\phi\partial_\rho - \partial_\rho\phi\partial_\theta
    \label{poi:defClebsh}
\end{align}
%
We note that equation (\ref{poi:defClebsh}) is only valid for a Clebsch system. If
the $B$-field is constant (as in our case), we are no longer in a Clebsch
system. However, the metric of the two systems coincide, and the derivation
could have been performed by removing the dependency of $B$ by multiplying the
result with $B'$, which is equal to the Clebsch system $B$-field, but without
any units.  Thus, for a constant $B$-field the $\ve{E}\times\ve{B}$ advection
can be written
%
\begin{align}
    B'\ve{u}_E\cdot\nabla
    = \frac{\sqrt{g_{\theta\theta}}}{J g_{\theta\theta}}
    (\partial_\theta\phi\partial_\rho - \partial_\rho\phi\partial_\theta)
    = \frac{1}{JB}
    (\partial_\theta\phi\partial_\rho - \partial_\rho\phi\partial_\theta)
    \label{poi:def}
\end{align}
%
where $B$ is the constant magnetic field.  This operator can be discretized
using the Arakawa bracket operator.

FIXME: Small b changes as well. Is the above valid?
grad phi times b (multiply with g22 J-1)
Get 1/J from the cross products
Conclusion: F3 looks wrong
should be no g thetas there
