% FIXME: NOT CALLED FLUTTERING
% FIXME: RELATIVE ERROR RATHER THAN ABSOULTE ERROR IN MES
We will now use the above derived equation in a homogeneous magnetic field,
assuming cold ions and constant electron temperature in a cylindrical geometry.
The resulting model will from here be referred to as the CELMA model, standing
for \textbf{C}onsistent \textbf{E}quations in a \textbf{L}inear \textbf{MA}chine

\section{The density equation}
%
We will now rewrite the electron density equation (equation
(\ref{eq:dens_evol_gen})) using that we are in a straight magnetic field (so
that $B=\text{const}$ and $\mathcal{C}(f)=0$) using the assumptions that the
electron temperature is constant and the ion temperature is negligible. This
yields
%
\begin{align*}
    \d_t n
    &=
    - n\mathcal{C}(\phi)
    + \frac{1}{e}\mathcal{C}(\phi)
    +\frac{1}{\mu}\frac{0.51m_i\nu_{ei}}{e^2}
    \div\L( \frac{\grad_\perp \L(p_e + p_i\R)}{B^2} \R)
    - \div\L(n \ve{u}_{e,\|}\R)
    + S_{n}
    \note{$p_i=0$\\$B=\text{const}$}
    \\
%
    &=
  \frac{1}{\mu}
  \frac{0.51m_iT_e\nu_{ei}}{B^2e^2}
   \grad_\perp^2 n
   - \div\L(n \ve{b}u_{e,\|}\R)
   + S_n
   \note{$\partial_i \ve{b} = 0$}
    \\
%
    &=
  \frac{1}{\mu}
  \frac{0.51m_iT_e\nu_{ei}}{B^2e^2}
   \grad_\perp^2 n
   - \ve{b}\cdot\grad \L(n u_{e,\|}\R)
   + S_n
    \\
%
    &=
  \frac{1}{\mu}
  \frac{0.51m_iT_e\nu_{ei}}{B^2e^2}
   \grad_\perp^2 n
   - \partial_\| \L(n u_{e,\|}\R)
   + S_n
\end{align*}
%
Using that
$\rho_s=\frac{c_s}{\om_{ci}}=\sqrt{\frac{T_e}{m_i}}\frac{m_i}{eB}
       =\sqrt{\frac{T_em_i}{e^2B^2}}$
we find that
%
\begin{align}
    \d_t n
    &=
  \frac{0.51\rho_s^2\nu_{ei}}{\mu}
   \grad_\perp^2 n
   - \partial_\|\L(n u_{e,\|} \R)
   + S_n
    \label{eq:non_norm_dens}
\end{align}



\section{Electron parallel momentum}
The parallel momentum equation for the electrons, under the assumption that $T_e=$
constant, can be written
%
\begin{align*}
 n_em_e\d_{t,e} \ve{u}_{e,\|}
 =&
 - \grad_\| \L(T_e n_e\R)
 - q_en_e \nabla_\|\phi
 - \L(\div\te{\pi}_e\R)_\|
 \\
 &
 + \ve{R}_{i\to e, \|}
 + \ve{R}_{n\to e, \|}
 - m_e\ve{u}_{e,\|}S_{n,e}
 \note{$n_i \simeq n_e$}
 \\
%
%
 n m_e\d_{t,e} \ve{u}_{e,\|}
 =&
 - T_e \grad_\| n
 + en \nabla_\|\phi
 \\
 &
 - 0.51 m_e n \nu_{ei}
 \L(
    \ve{u}_{e,\|}
    -
    \ve{u}_{i,\|}
 \R)
 - m_e n \nu_{en} \ve{u}_{e,\|}
 - m_e\ve{u}_{e,\|}S_{n,e}
 \\
%
%
n m_e\d_{t} \ve{u}_{e,\|}
 =&
 - n m_e \ve{u}_{e,\|} \cdot \nabla \ve{u}_{e,\|}
 - T_e \grad_\| n
 + en \nabla_\|\phi
 \\
 &
 - 0.51 m_e n \nu_{ei}
 \L(
    \ve{u}_{e,\|}
    -
    \ve{u}_{i,\|}
 \R)
 - m_e n \nu_{en}
    \ve{u}_{e,\|}
 - m_e\ve{u}_{e,\|}S_{n,e}
 \\
%
%
n m_e\d_{t} \L( \ve{b}u_{e,\|} \R)
 =&
 - n m_e u_{e,\|} \ve{b} \cdot \nabla \L(\ve{b}u_{e,\|}\R)
 - T_e \ve{b}\partial_\| n
 + en  \ve{b}\partial_\|\phi
 \\
 &
 - 0.51 m_e n \nu_{ei}
 \L(
    \ve{b}u_{e,\|}
    -
    \ve{b}u_{i,\|}
 \R)
 - m_e n \nu_{en}
    \ve{b}u_{e,\|}
 - m_e\ve{b}u_{e,\|}S_{n,e}
  \note{$\partial_t \ve{b} = \partial_i \ve{b} =0$}
 \\
%
%
\ve{b}n m_e\d_{t}  u_{e,\|}
 =&
 - \ve{b} n m_e u_{e,\|} \partial_\| u_{e,\|}
 - \ve{b}T_e \partial_\| n
 + \ve{b}en  \partial_\|\phi
 \\
 &
 - 0.51\ve{b} m_e n \nu_{ei}
 \L(
    u_{e,\|}
    -
    u_{i,\|}
 \R)
 - \ve{b}m_e n \nu_{en} u_{e,\|}
 - \ve{b}m_eu_{e,\|}S_{n,e}
 \\
%
%
n m_e\d_{t}  u_{e,\|}
 =&
 - n m_e u_{e,\|} \partial_\| u_{e,\|}
 - T_e \partial_\| n
 + en  \partial_\|\phi
 \\
 &
 - 0.51 m_e n \nu_{ei}
 \L(
    u_{e,\|}
    -
    u_{i,\|}
 \R)
 - m_e n \nu_{en} u_{e,\|}
 - m_eu_{e,\|}S_{n,e}
  \numberthis
  \label{eq:non_norm_e_mom}
\end{align*}
%

\section{Ion parallel momentum}
For the ions, using that $T_i = 0$, we have
%
\begin{align*}
 n_im_i\d_{t,i} \ve{u}_{i,\|}
 =&
 - \grad_\| \L(T_i n_i\R)
 - q_in_i\nabla_\|\phi
 - \L(\div\te{\pi}_i\R)_\|
 \note{Momentum conservation\\
       $n_e \simeq n_i$\\
       $T_i = 0$}
 \\
 &
 + \ve{R}_{e\to i,\|}
 + \ve{R}_{n\to i,\|}
 - m_i \ve{u}_{i,\|}S_{n,i}
 \\
%
%
nm_i\d_{t} \ve{u}_{i,\|}
 =&
 - nm_i \ve{u}_{i,\|} \cdot \grad \ve{u}_{i,\|}
 - nm_i\d_{t,i} \ve{u}_{i,\|}
 - en\nabla_\|\phi
 - \L(\div\te{\pi}_i\R)_\|
 \\
 &
 - \ve{R}_{i\to e,\|}
 + \ve{R}_{n\to i,\|}
 - m_i \ve{u}_{i,\|}S_{n,i}
 \\
%
%
nm_i\d_{t} \L(\ve{b}u_{i,\|}\R)
 =&
 - nm_i u_{i,\|} \ve{b}\cdot \grad\L( \ve{b}u_{i,\|}\R)
 - en\ve{b}\partial_\|\phi
 \\
 &
 + 0.51m_en \nu_{ei}\L(\ve{b}u_{e,\|}-\ve{b}u_{i,\|}\R)
 - m_in \nu_{in}\ve{b}u_{i,\|}
 - m_i \ve{b}u_{i,\|}S_{n,i}
  \note{$\partial_t \ve{b} = \partial_i \ve{b} =0$}
 \\
%
%
\ve{b}nm_i\d_{t} u_{i,\|}
 =&
 - \ve{b}nm_i u_{i,\|} \partial_\| u_{i,\|}
 - \ve{b}en\partial_\|\phi
 \\
 &
 + 0.51 \ve{b} m_en \nu_{ei}\L(u_{e,\|}-u_{i,\|}\R)
 - \ve{b}m_in \nu_{in}u_{i,\|}
 - \ve{b}m_i u_{i,\|}S_{n,i}
 \\
%
%
nm_i\d_{t} u_{i,\|}
 =&
 - nm_i u_{i,\|} \partial_\| u_{i,\|}
 - en\partial_\|\phi
 \\
 &
 + 0.51 m_en \nu_{ei}\L(u_{e,\|}-u_{i,\|}\R)
 - m_in \nu_{in}u_{i,\|}
 - m_i u_{i,\|}S_{n,i}
  \numberthis
  \label{eq:non_norm_i_mom}
\end{align*}
%

\section{The vorticity equation}
%
We will now define the vorticity
%
\begin{align*}
    \frac{\grad^2_\perp \phi}{B} \defined \Om
\end{align*}
%
This is in principle not the full vorticity, but rather the parallel part of
the advective vorticity, which has is analogue to what we find in fluid
mechanics
%
\begin{align*}
    \L(\curl \ve{u}_E\R)\cdot\ve{b}
    &=
    \L(-\curl \frac{\grad \phi \times \ve{b}}{B}\R)\cdot\ve{b}
    \note{$\grad \frac{1}{B}\simeq 0$}
    \\
%
    &=
    \frac{1}{B}\L(\curl \ve{b} \times \grad \phi \R)\cdot\ve{b}
    \note{$\curl (\ve{A}\times\ve{B}) = \ve{A}(\div\ve{B}) - \ve{B}(\div\ve{A})
                        + (\ve{B}\cdot\grad)\ve{A} - (\ve{A}\cdot\grad)\ve{B}$}
    \\
%
    &=
    \frac{1}{B}\L(   \ve{b}\L[\div\grad\phi\R]
                   - \grad\phi\L[\div\ve{b}\R]
                   + \L[\grad\phi\cdot\grad\R]\ve{b}
                   - \L[\ve{b}\cdot\grad\R]\grad\phi
               \R)\cdot\ve{b}
    \\
%
    &=
    \frac{1}{B}\L(   \ve{b}\L[\div\grad\phi\R]
                   - \L[\ve{b}\cdot\grad\R]\grad\phi
               \R)\cdot\ve{b}
    \note{$\div \ve{b} \simeq 0$ and $\grad \ve{b} \simeq 0$}
    \\
%
    &=
    \frac{1}{B}\L( \ve{b} \cdot \ve{b}\L[\div\grad\phi\R]
                   - \ve{b} \cdot \L[\ve{b}\cdot\grad\R]\grad\phi \R)
    \\
%
    &=
    \frac{1}{B}\L( \grad^2\phi
                   - \ve{b} \cdot \L[\grad\ve{b}\cdot\R]\grad\phi \R)
               \\
%
    &=
    \frac{1}{B}\L( \grad^2\phi
                   - \L[ \ve{b} \cdot \grad \R]\ve{b}\cdot\grad\phi \R)
               \\
%
    &=
    \frac{1}{B}\L( \grad^2\phi
                   - \div \L[ \ve{b} \ve{b}\cdot\grad\phi \R] \R)
               \\
%
    &=
    \frac{1}{B}\L(\grad_\perp^2\phi \R)
\end{align*}
%
Where we in the last line have used that
$
\grad^2_\perp = \grad^2 - \grad_\|^2 = \grad^2 -
                   \div \L[ \ve{b} \ve{b}\cdot\grad \R]
$
%
We will in light of this, using the current magnetic field topology,
investigate the left hand side of equation (\ref{eq:full_vort_eq}) term by term.


\subsection{The diamagnetic contribution}
%
\begin{align*}
    \frac{1}{e}\L[\mathcal{C}(p_e+p_i)\R]
\end{align*}
%
disappears as $\mathcal{C}(f)$ vanishes for a straight magnetic field.


\subsection{The neutral contribution}
%
The two next terms, origin from the Pedersen drift from (\ref{eq:div_ped}),
can be rewritten using that we are dealing with cold ions. We get
%
\begin{align*}
    \div\L(n\ve{u}_{i,\text{Ped}} \R)
    =&
    n \div\L(\frac{\nu_{in}}{\om_{ci}} \frac{\grad_\perp \phi}{B} \R)
    + \L(\frac{\nu_{in}}{\om_{ci}} \frac{\grad_\perp \phi}{B} \R) \cdot\grad n
    \note{Const $B$}
    \\
%
%
    =&
    n \frac{\nu_{in}}{\om_{ci}} \frac{\grad_\perp^2 \phi}{B}
    + \frac{\nu_{in}}{\om_{ci}} \frac{\grad_\perp \phi}{B} \cdot\grad n
    \\
%
%
  = &
 \frac{\nu_{in}}{\om_{ci}} \L(n\Om + \frac{\grad_\perp \phi}{B} \cdot \grad n\R)
\end{align*}
%


\subsection{The polarization contribution}
Further on, we will investigate the terms arising from the sum of the
divergence of the ion polarization drift and ion viscosity drift multiplied
with the density. The first term yields
%
\begin{align*}
 &
 \div\L( \frac{1}{\om_{ci}}
 \L[ \d_t + \ve{u}_{i,\|}\cdot\nabla \R]
 \L[\frac{\grad_\perp \phi}{B}n\R] \R)
    \\
    %
    =& \div\L( \frac{1}{\om_{ci}}\partial_t\L[\frac{\grad_\perp \phi}{B}n \R]\R)
    + \div\L(
    \frac{1}{\om_{ci}} \ve{u}_E\cdot\nabla \L[\frac{\grad_\perp \phi}{B} n\R]\R)
    + \div\L(
    \frac{1}{\om_{ci}} \ve{u}_{i,\|}\cdot\nabla\L[\frac{\grad_\perp \phi}{B}n\R]\R)
    \note{$B=\text{const}$}
    \\
    %
    =& \frac{1}{\om_{ci}}\div\L(\partial_t\L[\frac{\grad_\perp \phi}{B}n \R]\R)
    + \frac{1}{\om_{ci}} \div\L(
    \ve{u}_E\cdot\nabla \L[\frac{\grad_\perp \phi}{B}n \R]\R)
    + \frac{1}{\om_{ci}} \div\L(
    u_{i,\|}\ve{b}\cdot\nabla\L[\frac{\grad_\perp \phi}{B}n\R]\R)
    \note{Assume interchangibility of derivatives}
    \\
    %
    =& \frac{1}{\om_{ci}}\partial_t\L(\div\L[\frac{\grad_\perp \phi}{B}n \R]\R)
    + \frac{1}{\om_{ci}} \div\L(
    \ve{u}_E\cdot\nabla \L[\frac{\grad_\perp \phi}{B}n \R]\R)
    + \frac{1}{\om_{ci}} \div\L(
    u_{i,\|}\partial_\|\L[\frac{\grad_\perp \phi}{B}n\R]\R)
\end{align*}
%
We now define
%
\begin{align*}
    \Om^D \defined
    \div \L(n \frac{\grad_\perp \phi}{B} \R)
\end{align*}
%
which gives
%
\begin{align*}
    \frac{1}{\om_{ci}}\partial_t\Om^D
    + \frac{1}{\om_{ci}} \div\L(
    \ve{u}_E\cdot\nabla \L[\frac{\grad_\perp \phi}{B}n \R]\R)
    + \frac{1}{\om_{ci}} \div\L(
    u_{i,\|}\partial_\|\L[\frac{\grad_\perp \phi}{B}n\R]\R)
\end{align*}
%
As we have no curvature, the second term gives
%
\begin{align*}
    &
 - \div\L( \frac{1}{\om_{ci}}
 \frac{\grad_\perp \phi}{B}
 \L[S_{n} - \frac{1}{e}\mathcal{C}(p_i) - n \mathcal{C}(\phi)
 - n \div \ve{u}_{i,\|} \R] \R)
 \\
 =&
 - \frac{1}{\om_{ci}} \div\L(
 \frac{\grad_\perp \phi}{B}
 \L[S_{n} - n \div \L(\ve{b}u_{i,\|}\R) \R] \R)
 \note{$\partial_i \ve{b}=0$}
 \\
 =&
 - \frac{1}{\om_{ci}} \div\L(
 \frac{\grad_\perp \phi}{B}
 \L[S_{n} - n \ve{b}\cdot\grad u_{i,\|} \R] \R)
 \\
 =&
 - \frac{1}{\om_{ci}} \div\L(
 \frac{\grad_\perp \phi}{B}
 \L[S_{n} - n \partial_\| u_{i,\|} \R] \R)
 \\
 =&
 - \frac{1}{\om_{ci}} \div\L(
 \frac{\grad_\perp \phi}{B}
 S_{n} \R)
 + \frac{1}{\om_{ci}} \div\L(
 \frac{\grad_\perp \phi}{B}
 n \partial_\| u_{i,\|} \R)
\end{align*}


This means that the polarization contribution from equation
(\ref{eq:full_vort_eq}) can be written as
%
\begin{align*}
    &
    \frac{1}{\om_{ci}}\partial_t\Om^D
    + \frac{1}{\om_{ci}} \div\L(
    \ve{u}_E\cdot\nabla \L[\frac{\grad_\perp \phi}{B}n \R]\R)
    + \frac{1}{\om_{ci}} \div\L(
    u_{i,\|}\partial_\|\L[\frac{\grad_\perp \phi}{B}n\R]\R)
    \\
    -&
  \frac{1}{\om_{ci}} \div\L( \frac{\grad_\perp \phi}{B} S_{n} \R)
 + \frac{1}{\om_{ci}}
 \div\L( \frac{\grad_\perp \phi}{B} n \partial_\| u_{i,\|} \R)
\end{align*}

\subsection{The source contribution}
Further on, the contribution from the source can be rewritten.  Under the
assumption of cold ions ($T_i = 0$), we can simplify the source term in
equation (\ref{eq:full_vort_eq}) to be
%
\begin{align*}
    \div \L( \frac{ S_{i,n}}{\om_{ci}}
      \L[ \frac{\grad_\perp p_i}{n e B} + \frac{\grad_\perp \phi}{B} \R]
    \R)
    =&
    \div \L( \frac{ S_{i,n}}{\om_{ci}} \L[ \frac{\grad_\perp \phi}{B} \R] \R)
    \note{$T_i \simeq 0$\\ Const $B$\\ $S_{\a,n}=S_n$}
    \\
%
%
    =&
    \frac{1}{\om_{ci}} \div \L( S_n \L[ \frac{ \grad_\perp \phi }{ B } \R] \R)
\end{align*}
%

\subsection{The parallel contribution}
%
Next, the RHS of equation (\ref{eq:full_vort_eq}) reads
%
\begin{align*}
    \div \L(n [\ve{u}_{i,\|} - \ve{u}_{e,\|}]\R)
    &=
    \div \L(n [\ve{b}u_{i,\|} - \ve{b}u_{e,\|}]\R)
    \note{$\partial_i \ve{b} = 0$}
    \\
    %
    %
    &=
    \ve{b}\cdot\nabla \L(n [u_{i,\|} - u_{e,\|}]\R)
    \\
    %
    %
    &=
    \partial_\| \L(n [u_{i,\|} - u_{e,\|}]\R)
\end{align*}
%

\subsection{Collecting terms}
%
From the calculations above, equation (\ref{eq:full_vort_eq}) can be rewritten
to
%
\begin{align*}
  %
  &
  \quad
 \frac{\nu_{in}}{\om_{ci}} \L(n\Om + \frac{\grad_\perp \phi}{B} \cdot \grad n\R)
  \\
 &
 + \frac{1}{\om_{ci}} \partial_t \Om^D
 + \frac{1}{\om_{ci}} \div
 \L(
 \ve{u}_E\cdot\nabla \L[\frac{\grad_\perp \phi}{B}n \R]
 + u_{i,\|}\partial_\|\L[\frac{\grad_\perp \phi}{B}n\R]
 + \frac{\grad_\perp \phi}{B} n \partial_\| u_{i,\|}
 \R)
 - \frac{1}{\om_{ci}} \div\L( \frac{\grad_\perp \phi}{B} S_{n} \R)
 \\
 %
 &
 + \frac{1}{\om_{ci}}
    \div \L( S_n \L[ \frac{ \grad_\perp \phi }{ B } \R] \R)
 \\
 %
 =&
 \partial_\| \L(n [u_{i,\|} - u_{e,\|}]\R)
\end{align*}
%
Rearranging yields
%
\begin{align*}
  %
  \frac{1}{\om_{ci}}
  \partial_t \Om^D
  =&
  - \frac{\nu_{in}}{\om_{ci}} \L(n\Om + \frac{\grad_\perp \phi}{B} \cdot \grad n\R)
  \\
  %
  &
  - \frac{1}{\om_{ci}} \div
 \L(
 \ve{u}_E\cdot\nabla \L[\frac{\grad_\perp \phi}{B}n \R]
 + u_{i,\|}\partial_\|\L[\frac{\grad_\perp \phi}{B}n\R]
 + \frac{\grad_\perp \phi}{B} n \partial_\| u_{i,\|}
 \R)
 \\
 &
 %
 + \partial_\| \L(n [u_{i,\|} - u_{e,\|}]\R)
 \numberthis
 \label{eq:non_norm_vort_1}
\end{align*}
%
We note that $\frac{1}{\om_{ci}} \div\L( \frac{\grad_\perp \phi}{B} S_{n} \R)$
arising from the time derivative of $n_i$ in the polarization term of the
current conservation equation, cancels with the same term with opposite sign
arising from the source term of the current conservation equation.


\subsection{Vector advective terms}
%
We can simplify equation (\ref{eq:non_norm_vort_1}) even further by first
observing that
%
\begin{align*}
u_{i,\|}\partial_\|\L[\frac{\grad_\perp \phi}{B}n\R]
+ \frac{\grad_\perp \phi}{B} n \partial_\| u_{i,\|}
=
\partial_\| \L( u_{i,\|}\frac{\grad_\perp \phi}{B}n \R),
\end{align*}
%
and using the fact that in cylindrical coordinates we have that
$\partial_z \ve{e}_i = \partial_z \ve{e}^i = 0$. This yields
%
\begin{align*}
  %
  \frac{1}{\om_{ci}}
  \partial_t \Om^D
  =&
  - \frac{\nu_{in}}{\om_{ci}} \L(n\Om + \frac{\grad_\perp \phi}{B} \cdot \grad n\R)
  \\
  %
  &
  - \frac{1}{\om_{ci}} \div
 \L(
 \ve{u}_E\cdot\nabla \L[\frac{\grad_\perp \phi}{B}n \R]
 \R)
  - \frac{1}{\om_{ci}} \partial_\|\div
 \L( u_{i,\|}\frac{\grad_\perp \phi}{B}n \R)
 \\
 &
 %
 + \partial_\| \L(n [u_{i,\|} - u_{e,\|}]\R)
 \numberthis
 \label{eq:non_norm_vort_2}
\end{align*}
%
Note that the
%
$ - \frac{1}{\om_{ci}} \div
\L( u_{i,\|}\partial_\|\L[\frac{\grad_\perp \phi}{B}n\R] \R) $
%
term arises from the parallel advection in the polarization term, whereas the
%
$ - \frac{1}{\om_{ci}} \div
 \L( \frac{\grad_\perp \phi}{B} n \partial_\| u_{i,\|} \R) $
%
term arises from the ion continuity equation.

Next, it turns out that
%
\begin{align*}
 \frac{1}{\om_{ci}} \div
    \L( \ve{u}_E\cdot\nabla \L[\frac{\grad_\perp \phi}{B}n \R]\R)
\end{align*}
%
can be rewritten to
%
% \begin{align*}
%  \frac{1}{\om_{ci}} \div
%     \L( \ve{u}_E\cdot\nabla \L[\frac{\grad_\perp \phi}{B}n \R]\R)
%     lalala
%     \{\phi, \vort^D\}
%     +
%     \{\ve{u}_E \cdot \ve{u}_E, n\}
% \end{align*}
% F = qE
% F = ma = MLS^-2
% q = C
% => E = F/q = MLS^-2C^-1
% => phi = -div E = MS^-2C^-1
%
% u_E = - grad_perp phi x b /B = LS^-1
% => grad_perp phi /B = LS^-1
%
% om_ci = m/qB
% om_ci = T^-1
% n = L^-3
% div = L^-1
% partial_theta = 1
%
% FIXME
FIXME: Elaborate
The first term arises as $\div \ve{u}_E=0$ when the magnetic field is constant.
is expected.
A proof for this in normalized cylinder coordinates can be found in appendix
% FIXME
FIXME: Add appendix
.
A similar expression is expected to be found for other geometries as well, at
least as long as the $B$ field is constant. When this is not the case, terms
arising from fluttering is expected.
% FIXME
(FIXME add citation)



\subsection{The complete vorticity equation}


\section{Normalization}
\label{sec:norm}
We will now normalize using the ``standard'' gyro-Bohm normalization%
%
\footnote{Also known as: Bohm normalization}%
%

By looking at the Lorentz force, one can observe that $\|\ve{B}\|$ must have the
same dimension as $\frac{\|\ve{E}\|}{\|\ve{u}\|}$ which in the electrostatic
approximation can be written as $\frac{\|\grad \phi\|}{\|\ve{u}\|}$.
If we use the normalizations $\wt{\phi} = \frac{\phi e}{T_{e,0}}$ and
$\wt{\grad} = \frac{\grad}{\rho_s}$, we see that we
can use the normalization $\wt{B}=\frac{B}{\frac{T_{e,0}}{e \rho_s c_s}}$.

In addition we notice that $\frac{\grad_\perp \phi}{B}$ has the units of velocity.
Thus, we can normalize this with $c_s$. In other words we use the normalization
$\frac{\wt{\grad}_\perp \wt{\phi}}{\wt{B}} = \frac{\frac{\grad_\perp \phi}{B}}{c_s}$,
which gives $\frac{\grad_\perp \phi}{B} = \frac{\wt{\grad}_\perp \wt{\phi}}{\wt{B}}c_s$.
Hence, we can also use
$
 \wt{\Om} = \wt{\grad}\cdot\frac{\wt{\grad}_\perp \wt{\phi}}{\wt{B}}
 = \rho_s\div\frac{\frac{\grad_\perp \phi}{B}}{c_s}
= \rho_s\frac{\Om}{c_s}
 = \frac{\Om}{\om_{ci}}
$
, and similarly if we normalize $\wt{n} = \frac{n}{n_0}$, we get
$
\wt{\Om}^D = \wt{\grad}\cdot\L(\wt{n}\frac{\wt{\grad}_\perp \wt{\phi}}{\wt{B}}\R)
= \rho_s\div\L(\frac{\frac{n}{n_0}\frac{\grad_\perp \phi}{B}}{c_s}\R)
= \rho_sn_0\frac{\Om^D}{c_s}
= n_0\frac{\Om^D}{\om_{ci}}
$
. This yields
\\
\begin{empheq}[box={\tcbhighmath[colback=yellow!5!white]}]{align*}
    &\wt{\ve{x}}   =  \frac{\ve{x}}{\rho_s}&
    &\wt{t}        =  t\om_{ci}&
    &\wt{\ve{u}}   =  \frac{\ve{u}}{c_s}&
    &\wt{\phi}     =  \frac{e\phi}{T_{e,0}}&
    &\wt{\Om}      =  \frac{\Om}{\om_{ci}}&
    &\wt{\Om}^D    =  \frac{\Om^D}{n_0\om_{ci}}&
    \\
    &\wt{n}        =  \frac{n}{n_{0}}&
    &\wt{T_e}      =  \frac{T_e}{T_{e,0}}&
    &\wt{B}        =  \frac{B e \rho_s c_s}{T_{e,0}}&
    &\wt{\nu}_{x}  =  \frac{\nu_{x}}{\om_{ci}}&
    &\wt{S}        =  \frac{S}{n_0\om_{ci}}&
    &\frac{\wt{\grad}_\perp \wt{\phi}}{\wt{B}}
    = \frac{\grad_\perp \phi}{Bc_s}
\end{empheq}
\begin{empheq}[box={\tcbhighmath[colback=yellow!5!white]}]{align*}
    &\ve{x}       = \wt{\ve{x}}\rho_s&
    &    t        = \frac{\wt{t}}\om_{ci}&
    &\ve{u}       = \wt{\ve{u}}c_s&
    &    \phi     = \wt{\phi}\frac{T_{e,0}}{e}&
    &    \Om      = \wt{\Om}\om_{ci}&
    &    \Om^D    = \wt{\Om}^D\om_{ci}n_0&
    \\
    &    n        = \wt{n}n_{0}&
    &    B        = \wt{B}\frac{T_{e,0}}{ec_s\rho_s}&
    &    T_e      = \wt{T_e}T_{e,0}&
    &    \nu_{x}  = \wt{\nu}_{x}\om_{ci}&
    &    S        = \wt{S}n_0\om_{ci}&
    & \frac{\grad_\perp \phi}{B}
    = c_s\frac{\wt{\grad}_\perp \wt{\phi}}{\wt{B}}
\end{empheq}
\begin{empheq}[box={\tcbhighmath[colback=yellow!5!white]}]{align*}
    &\nabla       = \frac{1}{\rho_s} \wt{\nabla}&
    &\partial_t   = \om_{ci}\wt{\partial}_t&
    &\d_{t,\a}    = \wt{\partial_t}\om_{ci} +
    \frac{c_s (\ve{u}_\a)\cdot \wt{\grad}}{\rho_s}
    = \om_{ci}\wt{\d}_{t,\a}&
\end{empheq}
%

\noindent
where $\rho_s = \frac{c_s}{\om_{ci}}$ .

We will now normalize the equations derived above, and we will from here on
drop the tilde to simplify the writing.

\subsection{Normalization of \texorpdfstring{$n$}{the density}}
By normalizing equation (\ref{eq:non_norm_dens}) , we get
%
\begin{align*}
 n_{0} \om_{ci}
 \d_t n
 &=
 \frac{\rho_s^2\om_{ci}n_0}{\rho_s^2}
 \frac{0.51\nu_{ei}}{\mu} \grad_\perp^2 n
 - \frac{n_{0} c_s}{\rho_s}
 \partial_\|\L(n u_{e,\|}\R)
 + n_0 \om_{ci}
 S_n
 \\
%
%
 n_{0} \om_{ci}
 \d_t n
 &=
 \om_{ci}n_0 \frac{0.51\nu_{ei}}{\mu}
   \grad_\perp^2 n
 - n_{0} \om_{ci}
 \partial_\|\L(n u_{e,\|}\R)
 + n_0 \om_{ci}
 S_n
 \\
%
%
 \d_t n
 &=
 \frac{0.51\nu_{ei}}{\mu}
   \grad_\perp^2 n
   - \partial_\|\L(n u_{e,\|}\R)
 + S_n
\end{align*}
%
we will divide through by $1/n$ to ensure the non-negativity of the density when
solving the equations numerically. This gives
%
\begin{align*}
    \d_t \ln(n)
 &=
 \frac{0.51\nu_{ei}}{\mu} \frac{1}{n} \grad_\perp^2 n
 - \frac{1}{n} \partial_\|\L(n u_{e,\|}\R)
 + \frac{S_n}{n}
 \\
%
%
 &=
 \frac{0.51\nu_{ei}}{\mu}
 \L(
  \div \L[\frac{1}{n}\grad_\perp n\R]
   - \grad\frac{1}{n} \cdot \grad_\perp n
\R)
 - \partial_\| u_{e,\|}
 - u_{e,\|} \frac{1}{n} \partial_\| n
 + \frac{S_n}{n}
 \\
%
%
 &=
 \frac{0.51\nu_{ei}}{\mu}
 \L(
   \grad_\perp^2 \ln(n)
   - \frac{n}{n} \grad\frac{1}{n} \cdot \grad_\perp n
\R)
 - \partial_\| u_{e,\|}
 - u_{e,\|} \partial_\| \ln(n)
 + \frac{S_n}{n}
 \\
%
%
 &=
 \frac{0.51\nu_{ei}}{\mu}
 \L(
   \grad_\perp^2 \ln(n)
   - \grad\ln(n^{-1}) \cdot \grad_\perp \ln(n)
\R)
 - \partial_\| u_{e,\|}
 - u_{e,\|} \partial_\| \ln(n)
 + \frac{S_n}{n}
 \\
%
%
 &=
 \frac{0.51\nu_{ei}}{\mu}
 \L(
   \grad_\perp^2 \ln(n)
   + \grad\ln(n) \cdot \grad_\perp \ln(n)
 \R)
 - \partial_\| u_{e,\|}
 - u_{e,\|} \partial_\| \ln(n)
 +
 \frac{S_n}{n}
\end{align*}
%
We now have that
%
\begin{align*}
    \grad_\perp f \cdot \grad g
    =& \grad_\perp f \cdot \L(\grad_\perp + \grad_\| \R)g
    \\
    =& \grad_\perp f \cdot \grad_\perp g + \grad_\perp f \cdot \grad_\| g
    \\
    \note{$\ve{b}\perp\ve{e}^\rho$ and $\ve{b}\perp\ve{e}^\theta$ in
        cylindrical coordinate system}
    =& \grad_\perp f \cdot \grad_\perp g
    + \grad_\perp f \cdot \L(\ve{b}\ve{b}\cdot\R)\grad g
    \\
    =& \grad_\perp f \cdot \grad_\perp g
    + \L(\ve{b}\cdot\grad_\perp f \R)\L(\ve{b}\cdot\grad g\R)
    \\
    =& \grad_\perp f \cdot \grad_\perp g
    \numberthis
    \label{eq:gradperpGradperp}
\end{align*}
%
This yields
%
\begin{align*}
    \d_t \ln(n)
 &=
 \frac{0.51\nu_{ei}}{\mu}
 \L(
   \grad_\perp^2 \ln(n)
   + \grad_\perp\ln(n) \cdot \grad_\perp \ln(n)
 \R)
 - \partial_\| u_{e,\|}
 - u_{e,\|} \partial_\| \ln(n)
 +
 \frac{S_n}{n}
 \numberthis
 \label{eq:norm_ln_n}
\end{align*}

\subsection{Normalization of \texorpdfstring{$u_{e,\|}$
    }{parallel electron momentum}}
%
Normalization of equation (\ref{eq:non_norm_e_mom}) yields
%
\begin{align*}
 \om_{ci} c_s
 \d_t u_{e,\|}
 =&
 - \frac{c_s^2}{\rho_s}
  u_{e,\|} \partial_\| u_{e,\|}
 \\
 &
 - \frac{T_{e,0}n_0}{n_0\rho_s m_e}
 \frac{T_e}{n} \partial_\| n
 + \frac{e}{m_e}
 \frac{T_{e,0}}{e\rho_s} \partial_\|\phi
 \\
 &
 - \om_{ci} c_s
 0.51 \nu_{ei} \L(u_{e,\|}-u_{i,\|}\R)
 - \om_{ci} c_s
 \nu_{en} u_{e,\|}
 - \frac{c_s n_0 \om_{ci}}{n_0}
 \frac{u_{e,\|}S_{n}}{n}
 \\
%
%
 =&
 - \frac{c_s^2}{\rho_s}
 u_{e,\|} \partial_\| u_{e,\|}
 \\
 &
 - \frac{m_i}{m_i} \frac{T_{e,0}}{\rho_s m_e}
 \frac{T_e}{n} \partial_\| n
 + \frac{m_i}{m_i} \frac{1}{m_e} \frac{T_{e,0}}{\rho_s}
 \partial_\|\phi
 \\
 &
 - \om_{ci} c_s
 0.51 \nu_{ei} \L(u_{e,\|}-u_{i,\|}\R)
 - \om_{ci} c_s
 \nu_{en} u_{e,\|}
 - c_s \om_{ci}
 \frac{u_{e,\|}S_{n}}{n}
 \\
%
%
 =&
 - \frac{c_s^2}{\rho_s}
 u_{e,\|} \partial_\| u_{e,\|}
 \\
 &
 - \mu \frac{c_s^2}{\rho_s}
 \frac{T_e}{n} \partial_\| n
 + \mu \frac{c_s^2}{\rho_s}
 \partial_\|\phi
 \\
 &
 - \om_{ci} c_s
 0.51 \nu_{ei} \L(u_{e,\|}-u_{i,\|}\R)
 - \om_{ci} c_s
 \nu_{en} u_{e,\|}
 - c_s \om_{ci}
 \frac{u_{e,\|}S_{n}}{n}
 \\
%
%
 =&
 - \om_{ci} c_s
 u_{e,\|} \partial_\| u_{e,\|}
 \\
 &
 - \mu \om_{ci} c_s
 \frac{T_e}{n} \partial_\| n
 + \mu \om_{ci} c_s
 \partial_\|\phi
 \\
 -&
 \om_{ci} c_s
 0.51 \nu_{ei} \L(u_{e,\|}-u_{i,\|}\R)
 - \om_{ci} c_s
 \nu_{en} u_{e,\|}
 - c_s \om_{ci}
 \frac{u_{e,\|}S_{n}}{n}
 \\
%
%
 \d_t u_{e,\|}
 =&
 - u_{e,\|} \partial_\| u_{e,\|}
 - \mu T_e \partial_\| \ln(n)
 + \mu \partial_\|\phi
 - 0.51 \nu_{ei} \L(u_{e,\|}-u_{i,\|}\R)
 - \nu_{en} u_{e,\|}
 - \frac{u_{e,\|}S_{n}}{n}
 \note{$T_e$ is here the normalization constant}
 \\
%
%
 \d_t u_{e,\|}
 =&
 - u_{e,\|} \partial_\| u_{e,\|}
 + \mu \partial_\| \L(\phi - T_e  \ln(n)\R)
 - 0.51 \nu_{ei} \L(u_{e,\|}-u_{i,\|}\R)
 - \nu_{en} u_{e,\|}
 - \frac{u_{e,\|}S_{n}}{n}
\end{align*}
%
\subsection{Normalization of \texorpdfstring{$u_{i,\|}$}{parallel ion momentum}}
Further, normalization of equation (\ref{eq:non_norm_i_mom}) gives
%
\begin{align*}
 \om_{ci} c_s
 \d_t u_{i,\|}
 =&
 - \frac{c_s^2}{\rho_s}
 u_{i,\|} \partial_\| u_{i,\|}
 \\
 &
 - \frac{e}{m_i} \frac{T_{e,0}}{e\rho_s}
 \partial_\|\phi
 \\
 &
 - \omega_{ci} c_s
 \frac{ 0.51 \nu_{ei} }{ \mu } \L(u_{i,\|}-u_{e,\|}\R)
 - \om_{ci} c_s
 \nu_{in} u_{i,\|}
 - \frac{c_s n_0 \om_{ci}}{n_0}
 \frac{u_{i,\|}S_{n}}{n}
 \\
%
%
 =&
 - \frac{c_s^2}{\rho_s}
 u_{i,\|} \partial_\| u_{i,\|}
 \\
 &
 - \frac{c_s^2}{\rho_s}
 \partial_\|\phi
 \\
 &
 - \omega_{ci} c_s
 \frac{ 0.51 \nu_{ei} }{ \mu } \L(u_{i,\|}-u_{e,\|}\R)
 - \om_{ci} c_s
 \nu_{in} u_{i,\|}
 - c_s \om_{ci}
 \frac{u_{i,\|}S_{n}}{n}
 \\
%
%
 =&
 - \omega_{ci} c_s
 u_{i,\|} \partial_\| u_{i,\|}
 \\
 &
 - \omega_{ci} c_s
 \partial_\|\phi
 \\
 &
 - \omega_{ci} c_s
 \frac{ 0.51 \nu_{ei} }{ \mu } \L(u_{i,\|}-u_{e,\|}\R)
 - \om_{ci} c_s
 \nu_{in} u_{i,\|}
 - c_s \om_{ci}
 \frac{u_{i,\|}S_{n}}{n}
 \\
%
%
 \d_t u_{i,\|}
 =&
 - u_{i,\|} \partial_\| u_{i,\|}
 - \partial_\|\phi
 - \frac{ 0.51 \nu_{ei} }{ \mu } \L(u_{i,\|}-u_{e,\|}\R)
 - \nu_{in} u_{i,\|}
 - \frac{u_{i,\|}S_{n}}{n}
\end{align*}
%
\subsection{Normalization of \texorpdfstring{$\Omega^D$ and $\Omega$}{the
        vorticity}}
Finally, normalization of equation (\ref{eq:non_norm_vort_2}) yields
%
\begin{align*}
  %
  \frac{\om_{ci}\om_{ci}n_0}{\om_{ci}}
  \partial_t \Om^D
  =&
  - \frac{\nu_{in}\om_{ci}}{\om_{ci}} \L(\om_{ci}n_0n\Om
  + \frac{c_sn_0}{\rho_s}\frac{\grad_\perp \phi}{B} \cdot \grad n\R)
  \\
  %
  &
 - \frac{1}{\om_{ci}}
    \frac{c_sn_0c_s}{\rho_s\rho_s}
  \div
 \L(
 \ve{u}_E\cdot\nabla \L[\frac{\grad_\perp \phi}{B}n \R]
 \R)
 - \frac{1}{\om_{ci}}
    \frac{c_sn_0c_s}{\rho_s\rho_s}
\partial_\|\div \L( u_{i,\|}n \frac{\grad_\perp \phi}{B}\R)
 \\
 &
 %
 + \frac{n_0c_s}{\rho_s}
 \partial_\| \L(n [u_{i,\|} - u_{e,\|}]\R)
 \note{\ref{eq:gradperpGradperp}}
 \\
 %
 %
 \om_{ci}n_0
  \partial_t \Om^D
  =&
  - \nu_{in} \L(\om_{ci}n_0n\Om
  + \om_{ci}n_0\frac{\grad_\perp \phi}{B} \cdot \grad_\perp n\R)
  \\
  %
  &
 - \frac{1}{\om_{ci}} \om_{ci}n_0\om_{ci}
  \div
 \L(
 \ve{u}_E\cdot\nabla \L[\frac{\grad_\perp \phi}{B}n \R]
 \R)
 - \frac{1}{\om_{ci}} \om_{ci}n_0\om_{ci}
\partial_\|\div \L( u_{i,\|}n \frac{\grad_\perp \phi}{B}\R)
 \\
 &
 %
 + n_0\om_{ci} \partial_\| \L(n [u_{i,\|} - u_{e,\|}]\R)
 \\
 %
 %
  \partial_t \Om^D
  =&
  - \nu_{in} \L(n\Om + \frac{\grad_\perp \phi}{B} \cdot \grad_\perp n\R)
  \\
  %
  &
  -\div
 \L(
 \ve{u}_E\cdot\nabla \L[\frac{\grad_\perp \phi}{B}n \R]
 \R)
 -\partial_\|\div \L( u_{i,\|}n \frac{\grad_\perp \phi}{B}\R)
 \\
 &
 %
 + \partial_\| \L(n [u_{i,\|} - u_{e,\|}]\R)
 \\
 %
 %
  \partial_t \Om^D
  =&
  - \nu_{in} n\Om - \nu_{in} \frac{\grad_\perp \phi}{B} \cdot \grad_\perp n
  \\
  %
  &
  - \div \L( \ve{u}_E\cdot\nabla \L[\frac{\grad_\perp \phi}{B}n \R] \R)
  - \partial_\|\div \L( u_{i,\|}n \frac{\grad_\perp \phi}{B}\R)
 \\
 &
 %
 + n \partial_\| \L( u_{i,\|} - u_{e,\|} \R)
 + (u_{i,\|} - u_{e,\|})\partial_\| n
 \numberthis
 \label{eq:normalized_non_boussinesq}
\end{align*}
%
and
%
\begin{align*}
    \om_{ci}n_0\Om^D &= \frac{n_0 c_s}{\rho_s}\div\L(n\frac{\grad_\perp\phi}{B}\R)\\
    \Om^D &= \div\L(n\frac{\grad_\perp\phi}{B}\R)
\end{align*}
%
If we instead use the Boussinesq approximation (equation
(\ref{eq:boussinesq_non_norm})), equation (\ref{eq:normalized_non_boussinesq})
would take the form
%
\begin{align*}
  %
  \partial_t \Om
  =&
  - \nu_{in} n\Om - \nu_{in} \frac{\grad_\perp \phi}{B} \cdot \grad_\perp n
  \\
  %
  &
  - \div\L(\ve{u}_E\cdot \nabla \frac{\grad_\perp \phi}{B} \R)
  - u_{i,\|}\partial_\|\Om
  - \frac{\grad_\perp\L( \partial_\|\phi\R)}{B}\cdot\grad u_{i,\|}
 \\
 &
 %
 - \div\L(S_n \frac{ \grad_\perp \phi }{B}\R)
 + n \partial_\| \L(u_{i,\|} - u_{e,\|}]\R)
 + \L(u_{i,\|} - u_{e,\|}]\R) \partial_\| n
  \numberthis
  \label{eq:normalized_boussinesq}
\end{align*}
%
and
%
\begin{align*}
    \om_{ci}\Om &= \frac{ c_s}{\rho_s}\div\L(\frac{\grad_\perp\phi}{B}\R)\\
    \Om &= \div\L(\frac{\grad_\perp\phi}{B}\R)
\end{align*}
%

\section{Boundary conditions}
%
Setting correct boundary condition when calculating PDEs of plasma quantities
turns out to be a challenge. One of the reasons for this is the formation of
plasma sheaths between the plasma and the material which is due to the
difference in mobility between the ion species and the electron species. This
leads to a potential build-up on the material surface which affects the plasma
upstream.

Unfortunately the fluid description of the plasma breaks down at the sheath as
mentioned in \cite{Loizu2012a}. Hence, a proper description in this area can
only be accounted for with kinetic codes, solving for example the Fokker-Planck
equation. In other words, the fluid models are only valid up until the sheath
entrance. One should notice though that in reality it is hard, and maybe even
impossible to tell where the bulk plasma end, and where does the sheath start.

%
\subsection{Boundary conditions at SE}
% NOTE: Consider to add calculation of sheath velocities
Following the calculations of Choudora and Bohm (see for example
\cite{Stangeby2000book}), it is common practice to define the sheath entrance
to be the place where ion velocity reaches the ion sound speed (that is where
$u_{i,\|}=c_s$), and we will adopt this practice in this thesis. Considering a
quiescent plasma, one can calculate the equilibrium profiles which the plasma
obtains in contact with materials, and use this to set the boundaries for
$u_{i, \|}$ and $u_{e, \|}$ at the sheath entrance (SE). One should note
though, that even these boundary conditions are valid only in steady state.

\subsubsection{Ion velocity at SE}
For the ion velocity, we have from steady state calculations defined the sheath
entrance to be the point where
%
\begin{align*}
    u_{i,\|} \bigg|_{L_z} = c_s
\end{align*}
%
Normalization gives
%
\begin{align*}
    u_{i,\|} \bigg|_{L_z} = 1
\end{align*}
%

\subsubsection{Electron velocity at SE}
%
Further, the steady state gives the following condition on the parallel
electron velocity
%
\begin{align*}
    u_{e,\|} \bigg|_{L_z} = c_s \exp\L(\Lambda - \frac{e[\phi_0 + \phi]}{T_e}\R),
\end{align*}
%
where $\phi_0$ is an arbitrary potential%
%
\footnote{Due to the inversion algorithm $\phi_0$ has been set to $\Lambda$ for
    better numerical stability}%
%
we have defined $\phi_{\text{sheath entrance}} = 0$, and
$\Lambda=\ln\L(\sqrt{\frac{\mu}{2\pi}}\R)$. Normalization gives
%
\begin{align*}
    u_{e,\|} \bigg|_{L_z} = \exp\L(\Lambda - \phi\R)
\end{align*}
%


\subsubsection{Density BC at SE}
%
% NOTE: Tried
%   1. From steady state electron continuity
%      This means that n cannot change at the boundary
%   2. Neumann = 0
%      Could be that this it has a gradient
%      Problem that neumann both at SE and non-SE site
% NOTE: Idea
%   1. Could try robin BC at non-SE site
%
% FIXME: Find out what this should be
%
Although the "standard" calcuation for the sheath gives us conditions for
setting the velocities in steady state, non such conditions exists for the
density. Instead, one would have to rely on other arguments. In
\cite{Loizu2012a}, Loizu et al. presents a boundary condition for the density
in a field line geometry, where the field lines are allowed to be tilted with
respect to the end-plate. For a field line geometry where the field lines are
perpendicular to the end-plate (which is the case in this thesis), the boundary
condition reduces to
%
\begin{align*}
    \L. \partial_\| n \R|_{L_z}&= -\L.\frac{n}{c_s}\partial_\| u_{i.\|}\R|_{L_z}\\
    \note{$\L. u_{i.\|}\R|_{L_z} = c_s$}
    \L.u_{i.\|} \partial_\| n \R|_{L_z}&= -\L.n \partial_\| u_{i.\|}\R|_{L_z}\\
    \L.u_{i.\|} \partial_\| n \R|_{L_z}+\L.n \partial_\| u_{i.\|}\R|_{L_z}&= 0\\
    \L.\partial_\|\L(u_{i.\|}  n \R)\R|_{L_z}&= 0\\
    \L.\ve{b}\cdot\grad\L(u_{i.\|}  n \R)\R|_{L_z}&= 0
    \note{$\partial_i \ve{b} = 0$}
    \\
    \L.\div\L(\ve{b}u_{i,\|}  n \R)\R|_{L_z}&= 0\\
    \L.\div\L(\ve{u}_{i,\|}  n \R)\R|_{L_z}&= 0\\
    \L.\L(\grad_\perp + \grad_\|\R)\cdot\L(\ve{u}_{i,\|} n \R)\R|_{L_z}&= 0\\
    \L.\grad_\|\cdot\L(\ve{u}_{i,\|} n \R)\R|_{L_z}&= 0
    \note{$\grad_\| \cdot \ve{u}_{i,\perp} = 0$}
    \\
    \L.\div_\|\L(\ve{u}_{i} n \R)\R|_{L_z}&= 0
    \note{Continuity equation}
    \\
    \L.- \partial_tn \R|_{L_z}- \L.\grad_\perp\cdot\L(\ve{u}_{i} n \R)\R|_{L_z}&= 0\\
    \L.\partial_tn \R|_{L_z}&=  - \L.\grad_\perp\cdot\L(\ve{u}_{i} n \R)\R|_{L_z}
    \numberthis
    \label{eq:nBC}
\end{align*}
%
In other words, it says that the only change in the density at the SE can come
from perpendicular outflux. Another way to look at it is to consider
%
\begin{align*}
    \L.\L[\inde{\partial_\|\L(u_{i.\|}  n \R)}{z}\R]\R|_{L_z}&= 0\\
    \L.u_{i.\|}  n \R|_{L_z}&= C
\end{align*}
%
which means that the flux through the SE is constant in time, which may be a
too big constraint on the system.

One alternative is to set
%
\begin{align*}
    \L. \partial_\| n \R|_{L_z} &= 0
\end{align*}
%
If we follow the derivation of equation \ref{eq:nBC} and use the ion continuity
equation, this can be written as
%
\begin{align*}
    \L.- \partial_tn \R|_{L_z}- \L.\grad_\perp\cdot\L(\ve{u}_{i} n \R)\R|_{L_z}
    -\L.n \partial_\| u_{i.\|}\R|_{L_z}
    &= 0\\
    \L.\partial_tn \R|_{L_z}&=  - \L.\grad_\perp\cdot\L(\ve{u}_{i} n \R)\R|_{L_z}-
    \L.n \partial_\| u_{i.\|}\R|_{L_z}
\end{align*}
%
which means that the flux through the SE is no longer fixed. One should note
that if one has a Neumann condition in both ends of the machine, the PDE is
formally ill-posed, and the solutions found are unique only up to some
constant. When numerically solving the system the solution found will be
specified by the initial condition. Although it is important to be aware of, it
is not believed to change the physical behavior of the system as the dynamics
are driven by its source and its sinks.

A third approach, which does not have the same problem, is to let the density
be completely free at the sheath enterance and instead fix the value and the
gradient at the stagnation point (where the velocities are $0$) using a Cauchy
boundary condition (not to be confused with a Robin or "mixed" boundary
condition). This kind of boundary condition specifies the parallel dynamics
fully. The derivatives of $n$ at the last physical point before the SE can then
be calculated numerically by either a one-sided stencil, or to exrapolate the
solution to a ghost point and use this is a \emph{artificial boundary
condition} \cite{Leveque2007book}.

A final possibility is to let the derived equation for the evolution of the
density be valid at the boundary.

In the scope of this thesis the differences between $\L. \partial_\| n \R|_{L_z} = 0$
and the Cauchy BC has been investigated.

% FIXME: Add reference to where stuff is compared

\subsubsection{Potential BC at SE}
%
% NOTE: Tried
%   1. Fixed value
%      Bad idea, as difference in parallel velocities are not regulated by the
%      potential
%   2. Neumann 0
%      Bad idea as this may make an artificial constraint on the system
% FIXME: Maybe flawed?
The potential in our equations are not being evolved in time directly, but is
calculated by inverting either $\Om$ or $\Om^D$ for each drift plane (that is
for each perpendicular plane). This inversion takes into account the
outer radial boundary condition in $\phi$. This outer radial boundary condition
is set from the material properties of the wall. On the other hand, we do not
have any physical material constraint at the SE. The only thing we impose it
that $\phi$ on the plate very close to the SE is set to an arbitrary constant.
It is not clear how this is reflected at the SE in a non-steady state plasma.

However, parallel derivatives of the potential is being taken in our set of
equations. In order to calculate the parallel derivative of $\phi$ in the last
point before the SE, we need either to make a one sided stencil for this very
point (as the value of the boundary, and thereby the value of the ghost point
is unknown), or we can extrapolate the value of $\phi$ to the ghost point (as
we anyway assume that $\phi$ is a smooth function in order to discretize the
differential operators working on $\phi$). The latter has been chosen in the
current implementation.

\subsubsection{Vorticity BC at SE}
% NOTE: Tried
%   1. Neumann = 0
%      Could be that this it has a gradient
%      Problem that neumann both at SE and non-SE site
% IDEA:
%   1. Should be 0, as the wall is slowing down? Maybe not as not necessarily
%      have a non-slip BC on the wall
%
% FIXME: Find out what this should be
%
FIXME
No good physical argument yet. ATM implemented as neumann zero.

\subsection{BCs at non-SE drift plane}
% FIXME:
% Check that this is what you really want to do
The boundary conditions at the opposite site of the SE in the cylinder varies
from experiment to experiment (as does also the source term in the distribution
function itself). In the following model, we want to model a cylinder which is
mirrored at the opposite side of SE. In other words, the model does not reflect
any experimental linear devices, but can help to shed light on some of the
features found in real world experiments. The mirroring condition gives the
following set of boundary conditions
%
\begin{align*}
    &\partial_\| n \bigg|_0    =0 &
    &u_{e,\|} \bigg|_0         =0 &
    &u_{i,\|} \bigg|_0         =0 &
    &\partial_\| \phi \bigg|_0 =0 &
    &\partial_\|\Om \bigg|_0   =0 &
\end{align*}


\subsection{Radial boundary conditions}
%
The radial boundary conditions can in principle determine the results of the
experiments entirely. This have been seen in experiments on for example the
Mirabelle machine \cite{Schroder2001}. However, to correctly asset the physics,
one would have to take into account the neutral interaction and the material
properties of the wall. We will therefore in this thesis focus on a much
simplified approximation of the radial boundary conditions. In fact we will let
the radial boundary condition be approximately where the neutrals are
dominating, and thus dampening out the plasma dynamics.

\subsubsection{BC at the center}
%
The cylindrical coordinate system has a discontinuity at the center where
$\rho=0$. One way to avoid this problem is to put the grid points close to, but
not in the very center. In a cylinder, there is no real boundary condition in
the center, but we anyway have to address $\rho$ derivatives for the grid
points which lays close to the center. A workaround for this problem is to put
the boundary condition half way between a inner ghost point and the first inner
grid point (both located $\frac{\Delta \rho}{2}$ from the center, diametrically
opposite of each other). This method was first suggested by Naulin, V. et. al.
in \cite{Naulin2008}.

% FIXME: Describe how this is done in laplace inversion

\subsubsection{Density BC at outer radius}
The density will go toward $0$ as we are approaching the wall, thus we set
%
\begin{align*}
    n \bigg|_{L_\rho} = C,
\end{align*}
%
where $C$ is a small constant. Setting this too small, however gives numerical
problems due to loss of precision. In normalized units, $C = \frac{1}{20}$ of
the maximum value in the equilibrium.

\subsubsection{Velocity BCs at outer radius}
% FIXME:
% Not really physically explained
% If we let u_|_ =\= 0 we have no physical explanation why this should be 0
FIXME: Not really physically explained

There should be no parallel velocity towards the edge at the outer boundary,
thus
%
\begin{align*}
    u_{e,\|} \bigg|_{L_\rho} = u_{i,\|} \bigg|_{L_\rho} = 0
\end{align*}

\subsubsection{Potential BC at outer radius}
%
As there are no real sheath where the magnetic field lines are perpendicular to
the material, thus the potential at outer wall should be the at the floating
potential in an ideal conductor which we have assumed here. Of course, there
will be a small deviation from this as ions are lost a bit more rapidly
radially due a larger Larmor radius of the ions as compared to the electrons.
This is neglected here as finite Larmor radius (FLR) effects are not taken
into account here. For that reason we have that
%
\begin{align*}
    \phi \bigg|_{L_\rho} = 0
\end{align*}


\subsubsection{Vorticity velocity BC at outer radius}
%
% NOTE: Tried
%       1. Om = 0
%       This will give rise to a boundary layer which will try to oppose the
%       eventual solid body rotation in the center
% NOTE: The boundary here is set from either the non-slip or the free-slip
%       condition of v_theta
% FIXME: Write me, see blackboard photo: BC-rho-on-vort
%
FIXME
%
\begin{align*}
    \Om \bigg|_{L_\rho} = 0
\end{align*}


\section{Implementation}
\label{sec:implementation}
%FIXME:
FIXME: Rewrite this as implementation has changed

%
We will solve the derived equations using the BOUT++ framework, using the field
aligned discretization operators. We note that we are not working in a field
aligned coordinate system, but in a cylindrical coordinate system described in
appendix \ref{app:cylcoord}. The difference between the field aligned and our
cylindrical coordinate system lays in the value of $B$, where $B_\text{Field
    aligned}=\frac{1}{\rho}$, whereas $B_\text{Our system}=\text{Constant}$.
The only place one needs to take care about this is when using the Poisson
brackets to calculate the $\ve{E}\times \ve{B}$ advection, which will be
discussed in the next section.

\subsection{\texorpdfstring{$\ve{E}\times \ve{B}$}{ExB} advection}
\label{sec:ExBadv}
% FIXME:
FIXME: See Derivation of pure solenoid field in the coordinates manual: We are
interested in the pure solenoidal field. What about the strength of the field
when normalizing? Metric of Clebsch and cylinder coincides, but BOUT++ return
Arakawa in Clebsch. Note that our field strength is only appearing in units
like $\rho_s$.

FIXME: Must also be fixed in equations above. NOTE: Volker's implementation
also works like that

As described in appendix \ref{app:poisson}, we can write the
$\ve{E}\times\ve{B}$ advection in a field aligned coordinate system as
%
\begin{align}
    \ve{u}_E\cdot\nabla
    = -\frac{\nabla\phi\times\ve{b}}{B}\cdot\nabla
    = \{\phi, \cdot\}_{\theta,\rho}
    = \partial_\theta\phi\partial_\rho - \partial_\rho\phi\partial_\theta
    \label{eq:non_norm_adv}
\end{align}
%
Nevertheless, we are using a cylindrical coordinate system, where $B$ is
constant.  Since $B$ is constant,
we can choose $B_0$ in the normalization in such a way that $\widetilde{B}=1$.
Accordingly, the derivation of the normalized $\ve{E}\times \ve{B}$ advection
follows that of appendix \ref{app:poisson} with the only difference that we
multiply the equations with
$B_\text{Field aligned}=\frac{1}{J}\sqrt{g_{zz}}=\frac{1}{J}=\frac{1}{\rho}$.
This means that in normalized units (dropping the subscript on the Poisson
brackets from now in) gives
%
\begin{align*}
    \ve{u}_E\cdot\nabla = \frac{1}{J}\{\phi, \cdot\}
\end{align*}


\subsection{The divergence terms}
%
The four divergence terms in equation (\ref{eq:normalized_non_boussinesq})
cannot be implemented directly into BOUT++. The reason for this is that there
are not implemented own finite difference (FD) operators for these terms in
full. Neither can one put together the different FD operators composing the
full terms in order to build these full term FD operators, as some of the
boundary conditions will remain unspecified.

To exemplify this, consider a FD operator such as $\partial^{\text{FD}}_i$
(where the $^\text{FD}$ denotes a discretized FD operator) working on $f$ to
yield $g$ (that is $g = \partial^{\text{FD}}_if$). $g$ will be calculated in
all inner points by using the information of the boundary condition of $f$.
However, the information of the boundary conditions of $g$ will remain
unspecified after this operation.  This means that operations such as
$\partial^{\text{FD}}_j g = \partial_j^{\text{FD}}\L(\partial_i^{\text{FD}}
f\R)$ is not possible to perform as the boundary condition of $g$ is
unspecified.

There are at least two ways to work around this problem. Firstly, one could use
combined operators such as $\L(\parti{^2}{_j\partial_i}\R)^\text{FD}f$.  Given
that we are using the appropriate finite difference approximation (FDA), we
would not run into the trouble of the "missing" boundary condition as it does
not go through the immediate step of calculating $g$.  However, rewriting the
divergence terms in such a way that we will only make use of FDAs which does
not give troubles of "missing" boundaries yields many terms, and implementing
all the resulting terms can be error prone as there are so many of them. Also,
one should not neglect the possibility of making an calculation mistake along
the way.

Secondly, one could calculate the boundary condition for $g$ by evaluating the
derivative of the $f$ on the boundary, and apply it to $g$ before calculating
$\partial^{\text{FD}}_j$. As this method is less error prone than the previous,
we will pursue this method.

\subsubsection{Writing out the vectors in the divergence terms}
% FIXME: Rewrite what BC's you need. Should be properly written in the code
Before we proceed any further, we will write out the vectors in the divergence
terms in equation (\ref{eq:normalized_non_boussinesq}). First we consider
%
\begin{align*}
    \ve{u}_E\cdot\grad\L[n\frac{\grad_\perp\phi}{B}\R]
    =&
    \frac{1}{J}
    \L\{\phi, n\frac{\grad_\perp\phi}{B}\R\}
    \\
    =&
    \frac{1}{J}\L(
    \partial_\theta \phi \partial_\rho \L[n\frac{\grad_\perp\phi}{B}\R]
    -
    \partial_\rho \phi \partial_\theta \L[n\frac{\grad_\perp\phi}{B}\R]
    \R)
    \\
    =&
    \frac{1}{J}\L(
    \partial_\theta \phi \partial_\rho
    \L[\ve{e}^\rho n\frac{\partial_\rho \phi}{B}
    + \ve{e}^\theta n\frac{\partial_\theta \phi}{B}\R]
    -
    \partial_\rho \phi \partial_\theta
    \L[\ve{e}^\rho n\frac{\partial_\rho \phi}{B}
    + \ve{e}^\theta n\frac{\partial_\theta \phi}{B}\R]
    \R)
    \\
    =&
    \frac{1}{J}\L(
    \partial_\theta \phi \partial_\rho
    \L[ \ve{e}^\rho n\frac{\partial_\rho \phi}{B} \R]
    +
    \partial_\theta \phi \partial_\rho
    \L[ \ve{e}^\theta n\frac{\partial_\theta \phi}{B} \R]
    -
    \partial_\rho \phi \partial_\theta
    \L[ \ve{e}^\rho n\frac{\partial_\rho \phi}{B} \R]
    -
    \partial_\rho \phi \partial_\theta
    \L[ \ve{e}^\theta n\frac{\partial_\theta \phi}{B} \R]
    \R)
    \\
    =&
    \frac{1}{J}\L(
    \ve{e}^\rho \partial_\theta \phi \partial_\rho
    \L[ n\frac{\partial_\rho \phi}{B} \R]
    +
    n\frac{\partial_\rho \phi}{B}
    \partial_\theta \phi \partial_\rho \ve{e}^\rho
    +
    \ve{e}^\theta \partial_\theta \phi \partial_\rho
    \L[ n\frac{\partial_\theta \phi}{B} \R]
    +
    n\frac{\partial_\theta \phi}{B}
    \partial_\theta \phi \partial_\rho \ve{e}^\theta
    \R.
    \\&
    \L.
    -
    \ve{e}^\rho \partial_\rho \phi \partial_\theta
    \L[ n\frac{\partial_\rho \phi}{B} \R]
    -
    n\frac{\partial_\rho \phi}{B}
    \partial_\rho \phi \partial_\theta \ve{e}^\rho
    -
    \ve{e}^\theta \partial_\rho \phi \partial_\theta
    \L[ n\frac{\partial_\theta \phi}{B} \R]
    -
    n\frac{\partial_\theta \phi}{B}
    \partial_\rho \phi \partial_\theta \ve{e}^\theta
    \R)
    \\
    =&
    \frac{1}{J}\L(
    \ve{e}^\rho \partial_\theta \phi \partial_\rho
    \L[ n\frac{\partial_\rho \phi}{B} \R]
    +
    n\frac{\partial_\rho \phi}{B}
    \partial_\theta \phi \L[0\R]
    +
    \ve{e}^\theta \partial_\theta \phi \partial_\rho
    \L[ n\frac{\partial_\theta \phi}{B} \R]
    +
    n\frac{\partial_\theta \phi}{B}
    \partial_\theta \phi
    \L[ -\frac{1}{\rho} \ve{e}^\theta \R]
    \R.
    \\&
    \L.
    -
    \ve{e}^\rho \partial_\rho \phi \partial_\theta
    \L[ n\frac{\partial_\rho \phi}{B} \R]
    -
    n\frac{\partial_\rho \phi}{B}
    \partial_\rho \phi
    \L[\rho \ve{e}^\theta\R]
    -
    \ve{e}^\theta \partial_\rho \phi \partial_\theta
    \L[ n\frac{\partial_\theta \phi}{B} \R]
    -
    n\frac{\partial_\theta \phi}{B}
    \partial_\rho \phi
    \L[ -\frac{1}{\rho} \ve{e}^\rho \R]
    \R)
    \\
    =&
    \ve{e}^\rho
    \frac{1}{J}\L(
    \partial_\theta \phi \partial_\rho  \L[ n\frac{\partial_\rho \phi}{B} \R]
    -\partial_\rho \phi \partial_\theta \L[ n\frac{\partial_\rho \phi}{B} \R]
    + \frac{1}{\rho}n\frac{\partial_\theta \phi}{B} \partial_\rho \phi
    \R)
    \\&
    +
    \ve{e}^\theta
    \frac{1}{J}\L(
    \partial_\theta \phi \partial_\rho   \L[ n\frac{\partial_\theta \phi}{B} \R]
    - \partial_\rho \phi \partial_\theta \L[ n\frac{\partial_\theta \phi}{B} \R]
    - \frac{1}{\rho} n\frac{\partial_\theta \phi}{B} \partial_\theta \phi
    - \rho n\frac{\partial_\rho \phi}{B} \partial_\rho \phi
    \R)
    \\
    =&
    \ve{e}^\rho
    \frac{1}{J} \L(
    \L\{ \phi, n\frac{\partial_\rho \phi}{B} \R\}
    + \frac{1}{J} n\frac{\partial_\theta \phi}{B} \partial_\rho \phi
    \R)
    +
    \ve{e}^\theta
    \frac{1}{J}\L(
    \L\{ \phi, n\frac{\partial_\theta \phi}{B} \R\}
    - \frac{1}{J} n\frac{\partial_\theta \phi}{B} \partial_\theta \phi
    - Jn\frac{\partial_\rho \phi}{B} \partial_\rho \phi
    \R)
    \numberthis
    \label{eq:vec_adv_expanded}
\end{align*}
%
In order to implement equation (\ref{eq:vec_adv_expanded}) using FDAs as
described above, we see that we need to specify the $\rho$ boundaries (see next
section) of $n \partial_\rho \phi$ and $n \partial_\theta \phi$ in order to use
the Poisson bracket. Note that we do not need to calculate any $\theta$
boundaries, as the $\theta$-direction is periodical.

We also need to figure out what boundaries we need to specify on the terms of
equation (\ref{eq:vec_adv_expanded}) in order to take the divergence of this
equation, using a FDA. The divergence in general coordinates is given in
equation (2.6.39) in
% FIXME: Fix this cite
\cite{Dhaeseleer1991book}, and which reads
%
\begin{align}
    \div\ve{A}=\frac{1}{J}\partial_i\L(JA^i\R)
    \label{eq:div}
\end{align}
%
The orthogonality of the cylindrical coordinate system ensures that
$A_i\ve{e}^i=A_i g^{ii}\ve{e}_i$. Thus, from equation (\ref{eq:div}) we see that the
divergence operator will take the $i$ derivatives of the $i$ terms in equation
(\ref{eq:vec_adv_expanded}). From this we can conclude that we need to
specify the $\rho$ boundaries of $ \frac{1}{J} \L\{ \phi, n\frac{\partial_\rho
    \phi}{B} \R\} $ and $ \frac{1}{J^2} n\frac{\partial_\theta \phi}{B}
\partial_\rho \phi $ before taking the divergence.

Again, we need not to specify any $\theta$ boundaries as $\theta$ is
periodical. Further on, as the $\ve{e}^z$ component of equation
(\ref{eq:vec_adv_expanded}) is $0$, the divergence of equation
(\ref{eq:vec_adv_expanded}) will yield no parallel derivatives, and no $z$
boundary conditions need to be set.

Secondly we have
%
\begin{align*}
    u_{i,\|}\partial_\|\L(\frac{\grad_\perp \phi}{B}n\R)
    =&
    u_{i,\|}\partial_\|
    \L( \ve{e}^\rho\frac{\partial_\rho \phi}{B}n
    + \ve{e}^\theta\frac{\partial_\theta \phi}{B}n \R)
    \note{$\partial_z \ve{e}^i=0$}
    \\
    =&
    \ve{e}^\rho u_{i,\|}\partial_\|\L( \frac{\partial_\rho \phi}{B}n \R)
    + \ve{e}^\theta u_{i,\|}\partial_\|\L( \frac{\partial_\theta \phi}{B}n\R)
    \numberthis
    \label{eq:par_vec_adv_expanded}
\end{align*}
%
where we have to set the $z$ boundaries of $\frac{\partial_\rho \phi}{B}n$
and $\frac{\partial_\theta \phi}{B}n$ in order to calculate the parallel
derivatives of equation (\ref{eq:par_vec_adv_expanded}) using a FDA.

Following the discussion above, we need to specify the $\rho$ boundaries of
$ u_{i,\|}\partial_\| \frac{\partial_\rho \phi}{B}n $ in order to take the
divergence of equation (\ref{eq:par_vec_adv_expanded}) using a FDA.

Thirdly, we have
%
\begin{align*}
    \frac{\grad_\perp \phi}{B}\L(S_n - n\partial_\|u_{i,\|}\R)
    =&
    \ve{e}^\rho\frac{\partial_\rho \phi}{B}\L(S_n - n\partial_\|u_{i,\|}\R)
    +\ve{e}^\theta\frac{\partial_\theta \phi}{B}\L(S_n - n\partial_\|u_{i,\|}\R)
    \numberthis
    \label{eq:i_cont_expanded}
\end{align*}
%
This means that we need to find the $\rho$ boundaries of
$\frac{\partial_\rho \phi}{B}S_n$
and
$\frac{\partial_\rho \phi}{B}n\partial_\|u_{i,\|}$
before taking the divergence of equation (\ref{eq:i_cont_expanded}) using a FDA.

Finally we have
%
\begin{align}
 S_n \frac{ \grad_\perp \phi }{B}
 &=
 \ve{e}^\rho S_n \frac{ \partial_\rho \phi }{B}
 +\ve{e}^\theta S_n \frac{ \partial_\theta \phi }{B}
 \label{eq:div_S}
\end{align}
%
and we see that we need to specify the $\rho$ boundary condition of
$ \ve{e}^\rho S_n \frac{ \partial_\rho \phi }{B}$ in order to find the
divergence of equation (\ref{eq:div_S}) using a FDA.

\subsubsection{Specification of BC of the intermediate fields}
%
% FIXME: Rewrite this as it is complicatedly written
In our implementation we will place the boundaries half between the grid
points. Assume now that we have our original field $f$ which the know the
boundary condition for, and we wish to evaluate the derivative at the boundary
so that $g_{\text{B}}=\partial^{\text{FD}}f\bigg|_\text{boundary}$, where
$g_{\text{B}}$ denotes the value on the boundary of $g$. Approximating
$g_{\text{B}}$ with a second order centred FDA yields
%
\begin{align*}
    g_{\text{B}} \simeq \frac{f_\text{GP} - f_{\text{LIP}}}{h}
\end{align*}
%
where $f_\text{GP}$ denotes the ghost point of $f$ located an additional grid
space distance $h$ from $f_{\text{LIP}}$, which is the last inner grid point of
$f$.

% See
% BOUT-projects/cylinder_tests/own_diffusion_python/boundary_polynomial
From $g_{\text{B}}$ we can now use a Newton polynomial to extrapolate the
value of the rightmost ghost point of $g$. The Newton polynomial reads
%
\begin{align}
    p_N(x)=a_0+\sum_{i=1}^Na_i\prod_{j=0}^{i-1}(x-x_j)
    \label{eq:new_pol}
\end{align}
%
the coefficients of this polynomial can be found by solving the following set
of equation up to order $N$ for the unknown coefficients
%
\begin{equation}
    \begin{aligned}
        g(x_0)=&a_0\\
        g(x_1)=&a_0+a_1(x_1-x_0)\\
        g(x_2)=&a_0+a_1(x_2-x_0) + a_2(x_2-x_0)(x_2-x_1)\\
        \vdots&
        \label{eq:divdiffsys}
    \end{aligned}
\end{equation}
%
which can be solved easily using Newton's divided differences.
We would like to use a fourth order polynomial (standard order used when
setting boundaries in BOUT++) to determine the value in at the ghost point,
using information from the three preceding grid points and the value at the
boundary.

In other words, we let
\begin{align*}
x_i =
\{x_{\text{GP}-3}, x_{\text{GP}-2}, x_{\text{GP}-1}, x_\text{GP}\}=
\{x_{\text{GP}-3}, x_{\text{GP}-3}+h, x_{\text{GP}-3}+2h, x_{\text{GP}-3}+3h\}
\end{align*}
%
be the four rightmost grid points (including the ghost point $\text{GP}$). We
use the function values in these points to solve the equation system
(\ref{eq:divdiffsys}) (by, for example, using the recursive divided differences
formula), and insert them into the fourth order Newton polynomial of equation
(\ref{eq:new_pol}). If we evaluate this in
$x=x_\text{B}=x_{\text{GP}-1} +\frac{h}{2}$, we find that
%
\begin{align*}
    g_\text{GP} =
    -\frac{1}{5}g(x_{\text{GP}-3})
    + g(x_{\text{GP}-2})
    -3g(x_{\text{GP}-1})
    +\frac{16}{5}g_{\text{B}}
\end{align*}
%
Notice how $x_{\text{GP}-3}$ got cancelled in the equation system
(\ref{eq:divdiffsys}), and how $h$ got cancelled through the divided differences.

Needless to say, if we need the ghost point of a composite field, let's $e\cdot
g$, in order to calculate $\partial_i^{\text{FD}} \L(e\cdot g\R)$, we can
simply multiply the two ghost points together in order to find the composite
ghost point, that is $\L(e \cdot g\R)_\text{GP} = e_\text{GP}\cdot g_\text{GP}$.


\subsection{Finding the potential}
The observant reader may have noticed that we do not evolve the potential $\phi$
in time, and thus need another way of knowing $\phi$ at each time step. As we
know that
%
\begin{align*}
    \Om^D = \div\L(n\frac{\grad_\perp\phi}{B}\R)
    = n\div\L(\frac{\grad_\perp\phi}{B}\R) +
    \frac{\grad_\perp\phi}{B}\cdot\grad n
    = n\frac{\grad_\perp^2\phi}{B} +
    \grad n\cdot\frac{\grad_\perp\phi}{B}
    = n\frac{\grad_\perp^2\phi}{B} +
    \grad_\perp n\cdot\frac{\grad_\perp\phi}{B}
\end{align*}
%
the current time step $\Om^D$ can be solved to find $\phi$. From that we find
that
%
\begin{align*}
    \Om^D =& n\frac{\grad_\perp^2\phi}{B} +
    \grad_\perp n\cdot\frac{\grad_\perp\phi}{B}
    \\
    \frac{\Om^D}{n} =& \Om +
    \frac{1}{n}\grad_\perp n\cdot\frac{\grad_\perp\phi}{B}
    \\
    \Om =& \frac{\Om^D}{n} -
    \frac{1}{n}\grad_\perp n\cdot\frac{\grad_\perp\phi}{B}
\end{align*}


\subsection{Artificial viscosity}\label{sec:art_visc}
%
In the derivation we have neglected terms which is of order lower than first
order, as these terms are believed to have negligible contribution on the
overall set of equation. One of the drawbacks is, however, that we also have
neglected viscous terms which will dampen small scales in the system. Thus,
if we no not re-introduce some dissipation for numerical purposes, energy is
going to build-up on small scales, and would in the end make the simulation
crash.
%FIXME: Add cite to Phillips instability if applicable

Therefore, we add a dissipation on the form
%
\begin{align*}
    D_{f, \|, \text{art}} \nabla_{\|}^2 f
    + D_{f, \perp, \text{art}} \grad_\perp^2 f
    &=
    D_{f, \|, \text{art}} \div \L(\ve{b}\ve{b}\cdot\grad\R) f
    + D_{f, \perp, \text{art}} \grad_\perp^2 f
    \note{$\partial_i \ve{b} = 0$}
    \\
    %
    &=
    D_{f, \|, \text{art}} \ve{b}\cdot\grad \L(\ve{b}\cdot\grad\R) f
    + D_{f, \perp, \text{art}} \grad_\perp^2 f
    \\
    %
    &=
    D_{f, \|, \text{art}} \partial_\|^2 f
    + D_{f, \perp, \text{art}} \grad_\perp^2 f
    \numberthis
    \label{eq:art_vort}
\end{align*}
%
by exchanging $\div \te{\pi}$ in equation (\ref{fluideq:mom}) with equation
(\ref{eq:art_vort}). This is a somewhat crude approximation, but serves as a
good first approximation. In the non-normalized set of equations the $D$
coefficients would have the units of dynamical viscosity, and would be
normalized by
\\
%
\begin{minipage}{0.4\textwidth}
\begin{empheq}[box={\tcbhighmath[colback=yellow!5!white]}]{align*}
    &    D_{f, \text{art}}  = \wt{D}_{f, \text{art}}m_\a n_0 \rho_s c_s&
\end{empheq}
\end{minipage}
\hfill
\begin{minipage}{0.4\textwidth}
\begin{empheq}[box={\tcbhighmath[colback=yellow!5!white]}]{align*}
    &\wt{D}_{f, \text{art}}  =  \frac{D_{f, \text{art}}}{m_\a n_0 \rho_s c_s}&
\end{empheq}
\end{minipage}
\vspace{0.5cm}
\\
%
We notice that when using equation (\ref{fluideq:mom}) in the derivations,
division by $n$ on the RHS of the equations occurs in the density equation and
the parallel momentum equations, but not in the vorticity equation. The
artificial viscosity in the density is not divided by $n$ as $\frac{1}{n}\grad
n = \ln(n)$.



\subsection{BC for Laplace inversion using FFT}
%
% FIXME:
Write me



\section{The set of equations}
% FIXME:
FIXME: Arakawa brackets multiplied with 1/J

By using what we found in section \ref{sec:ExBadv} and \ref{sec:art_visc}, the
full set of equations reads
%
\begin{empheq}[box={\tcbhighmath}]{align*}
    \Om^D =& n\frac{\grad_\perp^2\phi}{B} +
    \grad_\perp n\cdot\frac{\grad_\perp\phi}{B}
 \numberthis
 \label{eq:cyto_vortD}
 \\
%
%
%
\Om =& \frac{\grad_\perp^2\phi }{B}
 \numberthis
 \label{eq:cyto_vort}
 \\
%
%
%
\partial_t \ln(n)
=&
 -\L\{\phi,\ln(n)\R\}
 +\frac{0.51\nu_{ei}}{\mu}
 \L(
   \grad_\perp^2 \ln(n)
   + \L[\grad_\perp \ln(n)\R]^2
\R)
  \\
  %
  &
- \partial_\|u_{e,\|}
- u_{e,\|} \partial_\| \ln(n)
 + \frac{S_n}{n}
  \\
  %
  &
 + D_{\ln(n), \|, \text{art}} \partial_{\|}^2  \ln(n)
 + D_{\ln(n), \perp, \text{art}} \grad_\perp^2 \ln(n)
 \numberthis
 \label{eq:cyto_dens}
 \\
%
%
%
\partial_t u_{e,\|}
 =&
 -\L\{\phi,u_{e,\|}\R\}
 - u_{e,\|} \partial_\| u_{e,\|}
 + \mu \partial_\| \L(\phi - T_e  \ln(n)\R)
  \\
  %
  &
 - 0.51 \nu_{ei} \L(u_{e,\|}-u_{i,\|}\R)
 - \nu_{en} u_{e,\|}
 - \frac{u_{e,\|}S_{n}}{n}
  \\
  %
  &
 + \frac{D_{u_{e,\|}, \|, \text{art}   }}{n}  \partial_{\|}^2  u_{e,\|}
 + \frac{D_{u_{e,\|}, \perp, \text{art}}}{n} \grad_\perp^2 u_{e,\|}
 \numberthis
 \label{eq:cyto_e_mom}
 \\
%
%
%
\partial_t u_{i,\|}
 =&
 -\L\{\phi,u_{i,\|}\R\}
 - u_{i,\|} \partial_\| u_{i,\|}
 - \partial_\|\phi
  \\
  %
  &
 - \frac{ 0.51 \nu_{ei} }{ \mu } \L(u_{i,\|}-u_{e,\|}\R)
 - \nu_{in} u_{i,\|}
 - \frac{u_{i,\|}S_{n}}{n}
  \\
  %
  &
 +\frac{ D_{u_{i,\|}, \|, \text{art}   }}{n} \partial_{\|}^2  u_{i,\|}
 +\frac{ D_{u_{i,\|}, \perp, \text{art}}}{n} \grad_\perp^2 u_{i,\|}
 \numberthis
 \label{eq:cyto_i_mom}
 \\
%
%
%
  \partial_t \Om^D
  =&
  - \nu_{in} n\Om - \nu_{in} \frac{\grad_\perp \phi}{B} \cdot \grad_\perp n
  \\
  %
  &
  - \div \L( \ve{u}_E\cdot\nabla \L[\frac{\grad_\perp \phi}{B}n \R] \R)
  - \partial_\|\div \L( u_{i,\|}n \frac{\grad_\perp \phi}{B}\R)
 \\
 &
 %
 + n \partial_\| \L( u_{i,\|} - u_{e,\|} \R)
 + (u_{i,\|} - u_{e,\|})\partial_\| n
  \\
  %
  &
  + D_{\Om^D, \|, \text{art}}    \partial_{\|}^2 \Om^D
  + D_{\Om^D, \perp, \text{art}} \grad_\perp^2 \Om^D
 \numberthis
 \label{eq:cyto_vortD_evolution}
\end{empheq}
