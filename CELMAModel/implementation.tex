\section{Implementation}
\label{sec:implementation}
This section describes the details in the numerical implementation of the CELMA
code

\subsection{Obtaining \texorpdfstring{$\phi$}{the potential}}
%
We observe the dependency of $\phi$ in equation (\ref{eq:cyto_dens}) -
(\ref{eq:cyto_vortD_evolution}), but that $\phi$ is not described by a
initial boundary value problem equation. Instead, we must find alternative ways
of extracting $\phi$. We will in the following discuss two ways of doing so.


\subsubsection{As a matrix inversion problem}
%
The problem of obtaining $\phi$ can be posed as a matrix problem
$A\ve{x}=\ve{b}$, where $\ve{x}$ is an array of all the spatial values of
$\phi$ ordered in some way, and $\ve{b}$ is an array of all the spatial values
of $\Omega^D$ ordered in the same way. Since we are working in an orthogonal
coordinate system, we have that $\Om^D = \div\L(n\grad_\perp\phi\R) =
\grad_\perp\cdot\L(n\grad_\perp\phi\R)$, as no basis vector parallel to the
magnetic field can be obtained from taking the derivative of the vector
$n\grad_\perp\phi$, which has only perpendicular components. Thus, in our case,
we can solve the $A\ve{x}=\ve{b}$ system for each plane perpendicular to the
magnetic field. That is, our matrix $A$ would be a $n_x \times n_y$ matrix,
where $n_x$ and $n_y$ is the number of points for the two perpendicular
directions
%
\footnote{
Note that in the BOUT++ implementation, $y$ is chosen as the direction parallel
to the magnetic field.  due to historical reasons, and $n_y$ would be named
$n_z$ in BOUT++ convention.  In order not to confuse readers unfamiliar with
BOUT++, $z$ is chosen as the coordinate along the magnetic field unless other
is specified.
}%
%
. We note that if $\div\L(n\grad_\perp\phi\R)$ were not purely
perpendicular, we would have to solve $A\ve{x}=\ve{b}$ for the whole domain. In
other words, the size of matrix $A$ would be $n_x \times n_y \times n_z$, and
would be considerably harder to solve numerically.

As noted in \cite{Wiesenberger2014Phd} (in the case where $P=1$, that is, in
the finite difference case), the discretization of the elliptic equation
$\div\L(n\grad_\perp\phi\R)=\Om^D$ can be formulated in a symmetric manner,
when special care is taken at the boundary.

Solving for the ghost-point, meaning that the ghost-point would be one of the
unknown in $A\ve{x}=\ve{b}$, would break the symmetry. Instead, one must
reformulate the boundary condition in a way such that it becomes an equation
for the ghost point. The equation of the ghost point can then be substituted
into the equations for the first/last inner point (the point just before the
boundary) and thus effictively eliminating the ghost point from the set of
equations.

To exemplify, consider a second order dirichlet boundary condition with the
boundary half between grid points for the equation $\partial_x^2 f = b$, where
$f_{-1}$ denotes the value at the ghost point, $f_{\text{BC}}$ denotes the
value at the boundary and $f_{1}$ denotes the value at the first inner ghost
point. The boundary condition can now be written
$\frac{f_{-1}+f_{1}}{2}=f_{\text{BC}}$, and the equation for the first inner
point could be written $\frac{f_{-1}+2f{1}+f_{2}}{(\Delta x)^2}=b_1$. This
would lead to the equation system
%
\begin{align*}
    A\cdot\ve{f}=&\ve{b}\\
    %
    \frac{1}{(\Delta x)^2}
    \begin{bmatrix}
        (\Delta x)^2\frac{1}{2} & (\Delta x)^2\frac{1}{2} & 0 & 0 & \ldots\\
        1                       & 2                       & 1 & 0 & \ldots\\
        0                       & 1                       & 2 & 1 & \ldots\\
        \vdots                  & \vdots              &\vdots&\vdots&\ddots\\
    \end{bmatrix}
    \cdot
    \begin{bmatrix}
        f_{-1}\\
        f_{1}\\
        f_{2}\\
        f_{3}\\
        \vdots
    \end{bmatrix}
    =&
    \begin{bmatrix}
        f_{\text{BC}}\\
        b_{1}\\
        b_{2}\\
        b_{3}\\
        \vdots
    \end{bmatrix}
\end{align*}
%
which is clearly non-symmetric.

The symmetric way to implement this would be to write
$f_{-1}=2f_{\text{BC}}-f_{1}$ for the boundary condition, and substitute this
into the 2nd order finite difference equation for the first inner ghost point.
This gives
%
\begin{align*}
    \frac{f_{-1}+2f{1}+f_{2}}{(\Delta x)^2}&=b_1\\
    \frac{2f_{\text{BC}}-f_{1}+2f{1}+f_{2}}{(\Delta x)^2}&=b_1\\
    \frac{f{1}+f_{2}}{(\Delta x)^2}&=b_1 - \frac{2f_{\text{BC}}}{(\Delta x)^2}
\end{align*}
%
This would give
%
\begin{align*}
    A\cdot\ve{f}=&\ve{b}\\
    %
    \frac{1}{(\Delta x)^2}
    \begin{bmatrix}
        1                       & 1                       & 0 & 0 & \ldots\\
        1                       & 2                       & 1 & 0 & \ldots\\
        0                       & 1                       & 2 & 1 & \ldots\\
        \vdots                  & \vdots              &\vdots&\vdots&\ddots\\
    \end{bmatrix}
    \cdot
    \begin{bmatrix}
        f_{1}\\
        f_{2}\\
        f_{3}\\
        \vdots
    \end{bmatrix}
    =&
    \begin{bmatrix}
        b_{1} - \frac{2f_{\text{BC}}}{(\Delta x)^2}\\
        b_{2}\\
        b_{3}\\
        \vdots
    \end{bmatrix}
\end{align*}
%
which is symmetric. Although difficult, one can show that the non-linear
elliptic equation in its symmetric form can be singular positive definite and
thus be solved using the conjugate gradient method.
% FIXME: Add reference

\subsubsection{The Naulin solver}
%
The potential can also be found in an iterative way. The method first used by
Naulin in \cite{Naulin2008} will be presented here, and will be refered to as
the Naulin solver.

The method can be used as long as
%
\begin{enumerate}[noitemsep,nolistsep]
    \item $\div\L(\frac{\grad_\perp\phi}{B}\R) = \frac{\grad_\perp^2\phi}{B}$
    \item $\grad f \cdot \grad_\perp g = \grad_\perp f \cdot \grad_\perp g$
\end{enumerate}
%
Point 1. is satisfied in our system as $B$ is constant, and because derivatives
of the perpendicular basis vectors does not yield parallel components in our
system. Point 2. is satisfied as the dot product of the perpendicular basis
vectors and the parallel basis vector is zero. We then get that
%
\begin{align*}
    \Om^D =& \div\L(n\frac{\grad_\perp\phi}{B}\R)\\
    %
    =& n\div\L(\frac{\grad_\perp\phi}{B}\R) +
    \frac{\grad_\perp\phi}{B}\cdot\grad n
    \\
    %
    =& n\frac{\grad_\perp^2\phi}{B} +
    \grad n\cdot\frac{\grad_\perp\phi}{B}
    \\
    %
    =& n\frac{\grad_\perp^2\phi}{B} +
    \grad_\perp n\cdot\frac{\grad_\perp\phi}{B}
    \\
    %
    \Om^D =& n\frac{\grad_\perp^2\phi}{B} +
    \grad_\perp n\cdot\frac{\grad_\perp\phi}{B}
    \\
    %
    \frac{\Om^D}{n} =& \Om +
    \frac{1}{n}\grad_\perp n\cdot\frac{\grad_\perp\phi}{B}
    \\
    %
    \Om =& \frac{\Om^D}{n} -
    \grad_\perp \ln(n) \cdot\frac{\grad_\perp\phi}{B}
\end{align*}
%
Using square bracket superscript as iteration number, the algorithm can be
stated in the following way:
%
\begin{algorithm}
\begin{enumerate}[noitemsep,nolistsep]
    \item Calculate
        $ \Om^{[i]} = \frac{\Om^D}{n} -
        \grad_\perp \ln(n) \cdot\frac{\grad_\perp\phi^{[i]}}{B}
        $
    \item Invert $\grad_\perp^2 \frac{\phi^{[i+1]}}{B} = \Om^{[i]}$ by the method
        described in \ref{app:lapInv}.
    \item Calculate
        $E_{\text{abs}, L_\infty} = \max \L|\phi^{[i]} - \phi^{[i+1]}\R|$
        and
        $E_{\text{rel}, L_\infty} = \max \L|\frac{\phi^{[i]} - \phi^{[i+1]}}{\phi^{[i]}}\R|$
    \item Check whether $E_{\text{abs}, L_\infty} > \text{Tolerance}_\text{abs}$
    \begin{itemize}[noitemsep,nolistsep]
        \item If yes: Check $E_{\text{abs}, L_\infty} > \text{Tolerance}_\text{rel}$
            \begin{itemize}[noitemsep,nolistsep]
                \item If yes: Assign $\phi^{[i+1]} \to \phi^{[i]}$, increase the
                    iteration number, throw an error if the iteration number is
                    above a predefined max iteration number, if not repeat from
                    step 1.
                \item Else, if no: Stop. Function returns
            \end{itemize}
        \item Else, if no: Stop. Function returns
    \end{itemize}
\end{enumerate}
\end{algorithm}
%


%FIXME: MAKE APPENDIX
FIXME: MAKE APPENDIX
\subsubsection{LAPLACE INVERSION}


\subsection{Advective terms}
\label{sec:ExBadv}
As already noted in section \ref{sec:vecAdvTerm}, we can write
$\ve{E}\times\ve{B}$-advective terms as Poisson brackets. The proof is found in
appendix \ref{app:poisson}. The benefits of writing terms on Poisson brackets
are presented in Arakawa's paper from 1966 \cite{Arakwa1966}. In short, the
paper shows that a na\"ive finite difference discretization of the Poisson
bracket does not conserve energy and enstrophy, and gives an alternative
way of discretize in orthogonal curvilinear coordinates in order to keep these
quantities conserved. If the energy and enstrophy is not conserved, fake
generation of these quantities occur, which eventually will lead to a blow up
of the simulation (in a way described by Phillips in \cite{Phillips1959}).

It is also possible to discretize the term $\{\ve{u}_E^2, n\}$
using Arakawa's method. We observe that in cylindrical coordinates, we have
%
\begin{align*}
    \{\ve{u}_E^2, n\} &= \L\{\L(\frac{\grad_\perp \phi}{B}\R)^2, n\R\}
    \note{Constant $B$}
    \\
    %
    &= \L(\frac{1}{B}\R)^2
    \L\{\L(\L[\ve{e}^{\rho}\partial_{\rho} + \ve{e}^{\theta}\partial_{\theta}\R] \phi\R)
        \cdot
        \L(\L[\ve{e}^{\rho}\partial_{\rho} + \ve{e}^{\theta}\partial_{\theta}\R] \phi\R)
        , n\R\}
    \note{Orthogonality}
    \\
    %
    &= \L(\frac{1}{B}\R)^2
    \L\{g^{\rho\rho}\L(\partial_{\rho} \phi\R)^2+
        g^{\theta\theta}\L(\partial_{\theta} \phi\R)^2
        , n\R\}
    \\
    %
    &= \L(\frac{1}{B}\R)^2
    \L\{\L(\partial_{\rho} \phi\R)^2+ \frac{1}{\rho^2}\L(\partial_{\theta} \phi\R)^2
        , n\R\}
\end{align*}
%
Here, we must take care when we treat the ghost points. No ghost points is
needed in the $\theta$ direction, as this direction is periodic. Thus, for
$\partial_{\theta} \phi$, we only need to make sure that we take the $\theta$
derivatives at the ghost points in $\rho$.

For $\partial_{\rho} \phi$, we must re-apply the values in the $\rho$ ghost
points as the derivative is not calculated there. For the inner ghost point,
the same procedure as used in
% FIXME: Add section
FIXME: Add section
can be used. For the outer ghost point, we use a fourth order Newton polynomial
of the four previous points in the $\rho$ direction, evaluated in the ghost
point. If we say that $f=\partial_{\rho} \phi$, this extrapolation reads
%
\begin{align*}
    f_{0} = 4f_{-1} - 6f_{-2} + 4f_{-3} - f_{-4}
\end{align*}
%
where $f_{0}$ is the value of $f$ at the position of the ghost point and
$f_{-i}$ is the value of $f$ in a position $-i\Delta \rho$ away from the ghost
point (where $\Delta \rho$ is the grid spacing in $\rho$).

This way of discretizing is second order accurate, as indicated in appendix
%FIXME: Add to appendix MES
FIXME: Add to appendix MES
FIXME: ADD HOW PARALLEL DERIVATIVE OF RHO IS IMPLEMENTED
YOU ARE HERE




THIS IS FROM DELETED STUFF
Secondly we have
%
\begin{align*}
    u_{i,\|}\partial_\|\L(\frac{\grad_\perp \phi}{B}n\R)
    =&
    u_{i,\|}\partial_\|
    \L( \ve{e}^\rho\frac{\partial_\rho \phi}{B}n
    + \ve{e}^\theta\frac{\partial_\theta \phi}{B}n \R)
    \note{$\partial_z \ve{e}^i=0$}
    \\
    =&
    \ve{e}^\rho u_{i,\|}\partial_\|\L( \frac{\partial_\rho \phi}{B}n \R)
    + \ve{e}^\theta u_{i,\|}\partial_\|\L( \frac{\partial_\theta \phi}{B}n\R)
    \numberthis
    \label{eq:par_vec_adv_expanded}
\end{align*}
%
where we have to set the $z$ boundaries of $\frac{\partial_\rho \phi}{B}n$
and $\frac{\partial_\theta \phi}{B}n$ in order to calculate the parallel
derivatives of equation (\ref{eq:par_vec_adv_expanded}) using a FDA.





\subsection{Artificial viscosity}\label{sec:art_visc}
%
In the derivation we have neglected terms which is of order lower than first
order, as these terms are believed to have negligible contribution on the
overall set of equation. One of the drawbacks is, however, that we also have
neglected viscous terms which will dampen small scales in the system. Thus,
if we no not re-introduce some dissipation for numerical purposes, energy is
going to build-up on small scales, and would in the end make the simulation
crash.
%FIXME: Add cite to Phillips instability if applicable

Therefore, we add a dissipation on the form
%
\begin{align*}
    D_{f, \|, \text{art}} \nabla_{\|}^2 f
    + D_{f, \perp, \text{art}} \grad_\perp^2 f
    &=
    D_{f, \|, \text{art}} \div \L(\ve{b}\ve{b}\cdot\grad\R) f
    + D_{f, \perp, \text{art}} \grad_\perp^2 f
    \note{$\partial_i \ve{b} = 0$}
    \\
    %
    &=
    D_{f, \|, \text{art}} \ve{b}\cdot\grad \L(\ve{b}\cdot\grad\R) f
    + D_{f, \perp, \text{art}} \grad_\perp^2 f
    \\
    %
    &=
    D_{f, \|, \text{art}} \partial_\|^2 f
    + D_{f, \perp, \text{art}} \grad_\perp^2 f
    \numberthis
    \label{eq:art_vort}
\end{align*}
%
by exchanging $\div \te{\pi}$ in equation (\ref{fluideq:mom}) with equation
(\ref{eq:art_vort}). This is a somewhat crude approximation, but serves as a
good first approximation. In the non-normalized set of equations the $D$
coefficients would have the units of dynamical viscosity, and would be
normalized by
\\
%
\begin{minipage}{0.4\textwidth}
\begin{empheq}[box={\tcbhighmath[colback=yellow!5!white]}]{align*}
    &    D_{f, \text{art}}  = \wt{D}_{f, \text{art}}m_\a n_0 \rho_s c_s&
\end{empheq}
\end{minipage}
\hfill
\begin{minipage}{0.4\textwidth}
\begin{empheq}[box={\tcbhighmath[colback=yellow!5!white]}]{align*}
    &\wt{D}_{f, \text{art}}  =  \frac{D_{f, \text{art}}}{m_\a n_0 \rho_s c_s}&
\end{empheq}
\end{minipage}
\vspace{0.5cm}
\\
%
We notice that when using equation (\ref{fluideq:mom}) in the derivations,
division by $n$ on the RHS of the equations occurs in the density equation and
the parallel momentum equations, but not in the vorticity equation. The
artificial viscosity in the density is not divided by $n$ as $\frac{1}{n}\grad
n = \ln(n)$.



\subsection{BC for Laplace inversion using FFT}
%
% FIXME:
Write me
