\section{Implementation}
\label{sec:implementation}
%FIXME:
FIXME: Rewrite this as implementation has changed

%
We will solve the derived equations using the BOUT++ framework, using the field
aligned discretization operators. We note that we are not working in a field
aligned coordinate system, but in a cylindrical coordinate system described in
appendix \ref{app:cylcoord}. The difference between the field aligned and our
cylindrical coordinate system lays in the value of $B$, where $B_\text{Field
    aligned}=\frac{1}{\rho}$, whereas $B_\text{Our system}=\text{Constant}$.
The only place one needs to take care about this is when using the Poisson
brackets to calculate the $\ve{E}\times \ve{B}$ advection, which will be
discussed in the next section.

\subsection{\texorpdfstring{$\ve{E}\times \ve{B}$}{ExB} advection}
\label{sec:ExBadv}
% FIXME:
FIXME: See Derivation of pure solenoid field in the coordinates manual: We are
interested in the pure solenoidal field. What about the strength of the field
when normalizing? Metric of Clebsch and cylinder coincides, but BOUT++ return
Arakawa in Clebsch. Note that our field strength is only appearing in units
like $\rho_s$.

FIXME: Must also be fixed in equations above. NOTE: Volker's implementation
also works like that

As described in appendix \ref{app:poisson}, we can write the
$\ve{E}\times\ve{B}$ advection in a field aligned coordinate system as
%
\begin{align}
    \ve{u}_E\cdot\nabla
    = -\frac{\nabla\phi\times\ve{b}}{B}\cdot\nabla
    = \{\phi, \cdot\}_{\theta,\rho}
    = \partial_\theta\phi\partial_\rho - \partial_\rho\phi\partial_\theta
    \label{eq:non_norm_adv}
\end{align}
%
Nevertheless, we are using a cylindrical coordinate system, where $B$ is
constant.  Since $B$ is constant,
we can choose $B_0$ in the normalization in such a way that $\widetilde{B}=1$.
Accordingly, the derivation of the normalized $\ve{E}\times \ve{B}$ advection
follows that of appendix \ref{app:poisson} with the only difference that we
multiply the equations with
$B_\text{Field aligned}=\frac{1}{J}\sqrt{g_{zz}}=\frac{1}{J}=\frac{1}{\rho}$.
This means that in normalized units (dropping the subscript on the Poisson
brackets from now in) gives
%
\begin{align*}
    \ve{u}_E\cdot\nabla = \frac{1}{J}\{\phi, \cdot\}
\end{align*}


\subsection{The divergence terms}
%
The four divergence terms in equation (\ref{eq:normalized_non_boussinesq})
cannot be implemented directly into BOUT++. The reason for this is that there
are not implemented own finite difference (FD) operators for these terms in
full. Neither can one put together the different FD operators composing the
full terms in order to build these full term FD operators, as some of the
boundary conditions will remain unspecified.

To exemplify this, consider a FD operator such as $\partial^{\text{FD}}_i$
(where the $^\text{FD}$ denotes a discretized FD operator) working on $f$ to
yield $g$ (that is $g = \partial^{\text{FD}}_if$). $g$ will be calculated in
all inner points by using the information of the boundary condition of $f$.
However, the information of the boundary conditions of $g$ will remain
unspecified after this operation.  This means that operations such as
$\partial^{\text{FD}}_j g = \partial_j^{\text{FD}}\L(\partial_i^{\text{FD}}
f\R)$ is not possible to perform as the boundary condition of $g$ is
unspecified.

There are at least two ways to work around this problem. Firstly, one could use
combined operators such as $\L(\parti{^2}{_j\partial_i}\R)^\text{FD}f$.  Given
that we are using the appropriate finite difference approximation (FDA), we
would not run into the trouble of the "missing" boundary condition as it does
not go through the immediate step of calculating $g$.  However, rewriting the
divergence terms in such a way that we will only make use of FDAs which does
not give troubles of "missing" boundaries yields many terms, and implementing
all the resulting terms can be error prone as there are so many of them. Also,
one should not neglect the possibility of making an calculation mistake along
the way.

Secondly, one could calculate the boundary condition for $g$ by evaluating the
derivative of the $f$ on the boundary, and apply it to $g$ before calculating
$\partial^{\text{FD}}_j$. As this method is less error prone than the previous,
we will pursue this method.

\subsubsection{Writing out the vectors in the divergence terms}
% FIXME: Rewrite what BC's you need. Should be properly written in the code
Before we proceed any further, we will write out the vectors in the divergence
terms in equation (\ref{eq:normalized_non_boussinesq}). First we consider
%
\begin{align*}
    \ve{u}_E\cdot\grad\L[n\frac{\grad_\perp\phi}{B}\R]
    =&
    \frac{1}{J}
    \L\{\phi, n\frac{\grad_\perp\phi}{B}\R\}
    \\
    =&
    \frac{1}{J}\L(
    \partial_\theta \phi \partial_\rho \L[n\frac{\grad_\perp\phi}{B}\R]
    -
    \partial_\rho \phi \partial_\theta \L[n\frac{\grad_\perp\phi}{B}\R]
    \R)
    \\
    =&
    \frac{1}{J}\L(
    \partial_\theta \phi \partial_\rho
    \L[\ve{e}^\rho n\frac{\partial_\rho \phi}{B}
    + \ve{e}^\theta n\frac{\partial_\theta \phi}{B}\R]
    -
    \partial_\rho \phi \partial_\theta
    \L[\ve{e}^\rho n\frac{\partial_\rho \phi}{B}
    + \ve{e}^\theta n\frac{\partial_\theta \phi}{B}\R]
    \R)
    \\
    =&
    \frac{1}{J}\L(
    \partial_\theta \phi \partial_\rho
    \L[ \ve{e}^\rho n\frac{\partial_\rho \phi}{B} \R]
    +
    \partial_\theta \phi \partial_\rho
    \L[ \ve{e}^\theta n\frac{\partial_\theta \phi}{B} \R]
    -
    \partial_\rho \phi \partial_\theta
    \L[ \ve{e}^\rho n\frac{\partial_\rho \phi}{B} \R]
    -
    \partial_\rho \phi \partial_\theta
    \L[ \ve{e}^\theta n\frac{\partial_\theta \phi}{B} \R]
    \R)
    \\
    =&
    \frac{1}{J}\L(
    \ve{e}^\rho \partial_\theta \phi \partial_\rho
    \L[ n\frac{\partial_\rho \phi}{B} \R]
    +
    n\frac{\partial_\rho \phi}{B}
    \partial_\theta \phi \partial_\rho \ve{e}^\rho
    +
    \ve{e}^\theta \partial_\theta \phi \partial_\rho
    \L[ n\frac{\partial_\theta \phi}{B} \R]
    +
    n\frac{\partial_\theta \phi}{B}
    \partial_\theta \phi \partial_\rho \ve{e}^\theta
    \R.
    \\&
    \L.
    -
    \ve{e}^\rho \partial_\rho \phi \partial_\theta
    \L[ n\frac{\partial_\rho \phi}{B} \R]
    -
    n\frac{\partial_\rho \phi}{B}
    \partial_\rho \phi \partial_\theta \ve{e}^\rho
    -
    \ve{e}^\theta \partial_\rho \phi \partial_\theta
    \L[ n\frac{\partial_\theta \phi}{B} \R]
    -
    n\frac{\partial_\theta \phi}{B}
    \partial_\rho \phi \partial_\theta \ve{e}^\theta
    \R)
    \\
    =&
    \frac{1}{J}\L(
    \ve{e}^\rho \partial_\theta \phi \partial_\rho
    \L[ n\frac{\partial_\rho \phi}{B} \R]
    +
    n\frac{\partial_\rho \phi}{B}
    \partial_\theta \phi \L[0\R]
    +
    \ve{e}^\theta \partial_\theta \phi \partial_\rho
    \L[ n\frac{\partial_\theta \phi}{B} \R]
    +
    n\frac{\partial_\theta \phi}{B}
    \partial_\theta \phi
    \L[ -\frac{1}{\rho} \ve{e}^\theta \R]
    \R.
    \\&
    \L.
    -
    \ve{e}^\rho \partial_\rho \phi \partial_\theta
    \L[ n\frac{\partial_\rho \phi}{B} \R]
    -
    n\frac{\partial_\rho \phi}{B}
    \partial_\rho \phi
    \L[\rho \ve{e}^\theta\R]
    -
    \ve{e}^\theta \partial_\rho \phi \partial_\theta
    \L[ n\frac{\partial_\theta \phi}{B} \R]
    -
    n\frac{\partial_\theta \phi}{B}
    \partial_\rho \phi
    \L[ -\frac{1}{\rho} \ve{e}^\rho \R]
    \R)
    \\
    =&
    \ve{e}^\rho
    \frac{1}{J}\L(
    \partial_\theta \phi \partial_\rho  \L[ n\frac{\partial_\rho \phi}{B} \R]
    -\partial_\rho \phi \partial_\theta \L[ n\frac{\partial_\rho \phi}{B} \R]
    + \frac{1}{\rho}n\frac{\partial_\theta \phi}{B} \partial_\rho \phi
    \R)
    \\&
    +
    \ve{e}^\theta
    \frac{1}{J}\L(
    \partial_\theta \phi \partial_\rho   \L[ n\frac{\partial_\theta \phi}{B} \R]
    - \partial_\rho \phi \partial_\theta \L[ n\frac{\partial_\theta \phi}{B} \R]
    - \frac{1}{\rho} n\frac{\partial_\theta \phi}{B} \partial_\theta \phi
    - \rho n\frac{\partial_\rho \phi}{B} \partial_\rho \phi
    \R)
    \\
    =&
    \ve{e}^\rho
    \frac{1}{J} \L(
    \L\{ \phi, n\frac{\partial_\rho \phi}{B} \R\}
    + \frac{1}{J} n\frac{\partial_\theta \phi}{B} \partial_\rho \phi
    \R)
    +
    \ve{e}^\theta
    \frac{1}{J}\L(
    \L\{ \phi, n\frac{\partial_\theta \phi}{B} \R\}
    - \frac{1}{J} n\frac{\partial_\theta \phi}{B} \partial_\theta \phi
    - Jn\frac{\partial_\rho \phi}{B} \partial_\rho \phi
    \R)
    \numberthis
    \label{eq:vec_adv_expanded}
\end{align*}
%
In order to implement equation (\ref{eq:vec_adv_expanded}) using FDAs as
described above, we see that we need to specify the $\rho$ boundaries (see next
section) of $n \partial_\rho \phi$ and $n \partial_\theta \phi$ in order to use
the Poisson bracket. Note that we do not need to calculate any $\theta$
boundaries, as the $\theta$-direction is periodical.

We also need to figure out what boundaries we need to specify on the terms of
equation (\ref{eq:vec_adv_expanded}) in order to take the divergence of this
equation, using a FDA. The divergence in general coordinates is given in
equation (2.6.39) in
% FIXME: Fix this cite
\cite{Dhaeseleer1991book}, and which reads
%
\begin{align}
    \div\ve{A}=\frac{1}{J}\partial_i\L(JA^i\R)
    \label{eq:div}
\end{align}
%
The orthogonality of the cylindrical coordinate system ensures that
$A_i\ve{e}^i=A_i g^{ii}\ve{e}_i$. Thus, from equation (\ref{eq:div}) we see that the
divergence operator will take the $i$ derivatives of the $i$ terms in equation
(\ref{eq:vec_adv_expanded}). From this we can conclude that we need to
specify the $\rho$ boundaries of $ \frac{1}{J} \L\{ \phi, n\frac{\partial_\rho
    \phi}{B} \R\} $ and $ \frac{1}{J^2} n\frac{\partial_\theta \phi}{B}
\partial_\rho \phi $ before taking the divergence.

Again, we need not to specify any $\theta$ boundaries as $\theta$ is
periodical. Further on, as the $\ve{e}^z$ component of equation
(\ref{eq:vec_adv_expanded}) is $0$, the divergence of equation
(\ref{eq:vec_adv_expanded}) will yield no parallel derivatives, and no $z$
boundary conditions need to be set.

Secondly we have
%
\begin{align*}
    u_{i,\|}\partial_\|\L(\frac{\grad_\perp \phi}{B}n\R)
    =&
    u_{i,\|}\partial_\|
    \L( \ve{e}^\rho\frac{\partial_\rho \phi}{B}n
    + \ve{e}^\theta\frac{\partial_\theta \phi}{B}n \R)
    \note{$\partial_z \ve{e}^i=0$}
    \\
    =&
    \ve{e}^\rho u_{i,\|}\partial_\|\L( \frac{\partial_\rho \phi}{B}n \R)
    + \ve{e}^\theta u_{i,\|}\partial_\|\L( \frac{\partial_\theta \phi}{B}n\R)
    \numberthis
    \label{eq:par_vec_adv_expanded}
\end{align*}
%
where we have to set the $z$ boundaries of $\frac{\partial_\rho \phi}{B}n$
and $\frac{\partial_\theta \phi}{B}n$ in order to calculate the parallel
derivatives of equation (\ref{eq:par_vec_adv_expanded}) using a FDA.

Following the discussion above, we need to specify the $\rho$ boundaries of
$ u_{i,\|}\partial_\| \frac{\partial_\rho \phi}{B}n $ in order to take the
divergence of equation (\ref{eq:par_vec_adv_expanded}) using a FDA.

Thirdly, we have
%
\begin{align*}
    \frac{\grad_\perp \phi}{B}\L(S_n - n\partial_\|u_{i,\|}\R)
    =&
    \ve{e}^\rho\frac{\partial_\rho \phi}{B}\L(S_n - n\partial_\|u_{i,\|}\R)
    +\ve{e}^\theta\frac{\partial_\theta \phi}{B}\L(S_n - n\partial_\|u_{i,\|}\R)
    \numberthis
    \label{eq:i_cont_expanded}
\end{align*}
%
This means that we need to find the $\rho$ boundaries of
$\frac{\partial_\rho \phi}{B}S_n$
and
$\frac{\partial_\rho \phi}{B}n\partial_\|u_{i,\|}$
before taking the divergence of equation (\ref{eq:i_cont_expanded}) using a FDA.

Finally we have
%
\begin{align}
 S_n \frac{ \grad_\perp \phi }{B}
 &=
 \ve{e}^\rho S_n \frac{ \partial_\rho \phi }{B}
 +\ve{e}^\theta S_n \frac{ \partial_\theta \phi }{B}
 \label{eq:div_S}
\end{align}
%
and we see that we need to specify the $\rho$ boundary condition of
$ \ve{e}^\rho S_n \frac{ \partial_\rho \phi }{B}$ in order to find the
divergence of equation (\ref{eq:div_S}) using a FDA.

\subsubsection{Specification of BC of the intermediate fields}
%
% FIXME: Rewrite this as it is complicatedly written
In our implementation we will place the boundaries half between the grid
points. Assume now that we have our original field $f$ which the know the
boundary condition for, and we wish to evaluate the derivative at the boundary
so that $g_{\text{B}}=\partial^{\text{FD}}f\bigg|_\text{boundary}$, where
$g_{\text{B}}$ denotes the value on the boundary of $g$. Approximating
$g_{\text{B}}$ with a second order centred FDA yields
%
\begin{align*}
    g_{\text{B}} \simeq \frac{f_\text{GP} - f_{\text{LIP}}}{h}
\end{align*}
%
where $f_\text{GP}$ denotes the ghost point of $f$ located an additional grid
space distance $h$ from $f_{\text{LIP}}$, which is the last inner grid point of
$f$.

% See
% BOUT-projects/cylinder_tests/own_diffusion_python/boundary_polynomial
From $g_{\text{B}}$ we can now use a Newton polynomial to extrapolate the
value of the rightmost ghost point of $g$. The Newton polynomial reads
%
\begin{align}
    p_N(x)=a_0+\sum_{i=1}^Na_i\prod_{j=0}^{i-1}(x-x_j)
    \label{eq:new_pol}
\end{align}
%
the coefficients of this polynomial can be found by solving the following set
of equation up to order $N$ for the unknown coefficients
%
\begin{equation}
    \begin{aligned}
        g(x_0)=&a_0\\
        g(x_1)=&a_0+a_1(x_1-x_0)\\
        g(x_2)=&a_0+a_1(x_2-x_0) + a_2(x_2-x_0)(x_2-x_1)\\
        \vdots&
        \label{eq:divdiffsys}
    \end{aligned}
\end{equation}
%
which can be solved easily using Newton's divided differences.
We would like to use a fourth order polynomial (standard order used when
setting boundaries in BOUT++) to determine the value in at the ghost point,
using information from the three preceding grid points and the value at the
boundary.

In other words, we let
\begin{align*}
x_i =
\{x_{\text{GP}-3}, x_{\text{GP}-2}, x_{\text{GP}-1}, x_\text{GP}\}=
\{x_{\text{GP}-3}, x_{\text{GP}-3}+h, x_{\text{GP}-3}+2h, x_{\text{GP}-3}+3h\}
\end{align*}
%
be the four rightmost grid points (including the ghost point $\text{GP}$). We
use the function values in these points to solve the equation system
(\ref{eq:divdiffsys}) (by, for example, using the recursive divided differences
formula), and insert them into the fourth order Newton polynomial of equation
(\ref{eq:new_pol}). If we evaluate this in
$x=x_\text{B}=x_{\text{GP}-1} +\frac{h}{2}$, we find that
%
\begin{align*}
    g_\text{GP} =
    -\frac{1}{5}g(x_{\text{GP}-3})
    + g(x_{\text{GP}-2})
    -3g(x_{\text{GP}-1})
    +\frac{16}{5}g_{\text{B}}
\end{align*}
%
Notice how $x_{\text{GP}-3}$ got cancelled in the equation system
(\ref{eq:divdiffsys}), and how $h$ got cancelled through the divided differences.

Needless to say, if we need the ghost point of a composite field, let's $e\cdot
g$, in order to calculate $\partial_i^{\text{FD}} \L(e\cdot g\R)$, we can
simply multiply the two ghost points together in order to find the composite
ghost point, that is $\L(e \cdot g\R)_\text{GP} = e_\text{GP}\cdot g_\text{GP}$.


\subsection{Finding the potential}
The observant reader may have noticed that we do not evolve the potential $\phi$
in time, and thus need another way of knowing $\phi$ at each time step. As we
know that
%
\begin{align*}
    \Om^D = \div\L(n\frac{\grad_\perp\phi}{B}\R)
    = n\div\L(\frac{\grad_\perp\phi}{B}\R) +
    \frac{\grad_\perp\phi}{B}\cdot\grad n
    = n\frac{\grad_\perp^2\phi}{B} +
    \grad n\cdot\frac{\grad_\perp\phi}{B}
    = n\frac{\grad_\perp^2\phi}{B} +
    \grad_\perp n\cdot\frac{\grad_\perp\phi}{B}
\end{align*}
%
the current time step $\Om^D$ can be solved to find $\phi$. From that we find
that
%
\begin{align*}
    \Om^D =& n\frac{\grad_\perp^2\phi}{B} +
    \grad_\perp n\cdot\frac{\grad_\perp\phi}{B}
    \\
    \frac{\Om^D}{n} =& \Om +
    \frac{1}{n}\grad_\perp n\cdot\frac{\grad_\perp\phi}{B}
    \\
    \Om =& \frac{\Om^D}{n} -
    \frac{1}{n}\grad_\perp n\cdot\frac{\grad_\perp\phi}{B}
\end{align*}


\subsection{Artificial viscosity}\label{sec:art_visc}
%
In the derivation we have neglected terms which is of order lower than first
order, as these terms are believed to have negligible contribution on the
overall set of equation. One of the drawbacks is, however, that we also have
neglected viscous terms which will dampen small scales in the system. Thus,
if we no not re-introduce some dissipation for numerical purposes, energy is
going to build-up on small scales, and would in the end make the simulation
crash.
%FIXME: Add cite to Phillips instability if applicable

Therefore, we add a dissipation on the form
%
\begin{align*}
    D_{f, \|, \text{art}} \nabla_{\|}^2 f
    + D_{f, \perp, \text{art}} \grad_\perp^2 f
    &=
    D_{f, \|, \text{art}} \div \L(\ve{b}\ve{b}\cdot\grad\R) f
    + D_{f, \perp, \text{art}} \grad_\perp^2 f
    \note{$\partial_i \ve{b} = 0$}
    \\
    %
    &=
    D_{f, \|, \text{art}} \ve{b}\cdot\grad \L(\ve{b}\cdot\grad\R) f
    + D_{f, \perp, \text{art}} \grad_\perp^2 f
    \\
    %
    &=
    D_{f, \|, \text{art}} \partial_\|^2 f
    + D_{f, \perp, \text{art}} \grad_\perp^2 f
    \numberthis
    \label{eq:art_vort}
\end{align*}
%
by exchanging $\div \te{\pi}$ in equation (\ref{fluideq:mom}) with equation
(\ref{eq:art_vort}). This is a somewhat crude approximation, but serves as a
good first approximation. In the non-normalized set of equations the $D$
coefficients would have the units of dynamical viscosity, and would be
normalized by
\\
%
\begin{minipage}{0.4\textwidth}
\begin{empheq}[box={\tcbhighmath[colback=yellow!5!white]}]{align*}
    &    D_{f, \text{art}}  = \wt{D}_{f, \text{art}}m_\a n_0 \rho_s c_s&
\end{empheq}
\end{minipage}
\hfill
\begin{minipage}{0.4\textwidth}
\begin{empheq}[box={\tcbhighmath[colback=yellow!5!white]}]{align*}
    &\wt{D}_{f, \text{art}}  =  \frac{D_{f, \text{art}}}{m_\a n_0 \rho_s c_s}&
\end{empheq}
\end{minipage}
\vspace{0.5cm}
\\
%
We notice that when using equation (\ref{fluideq:mom}) in the derivations,
division by $n$ on the RHS of the equations occurs in the density equation and
the parallel momentum equations, but not in the vorticity equation. The
artificial viscosity in the density is not divided by $n$ as $\frac{1}{n}\grad
n = \ln(n)$.



\subsection{BC for Laplace inversion using FFT}
%
% FIXME:
Write me
